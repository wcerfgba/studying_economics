\documentclass[landscape]{slides}

\usepackage[landscape]{geometry}

\usepackage{pdfpages}
\usepackage{eurosym}
\usepackage{amssymb}
\usepackage{amsmath}
\usepackage{float}

\usepackage{hyperref}

\def\mathbi#1{\textbf{\em #1}}

\topmargin=-1.8cm \textheight=17cm \oddsidemargin=0cm
\evensidemargin=0cm \textwidth=22cm

\author{Emmanuel Saez}

\date{Berkeley}

\title{230B: Public Economics \\
Capital Taxation} \onlyslides{1-300}

\newenvironment{outline}{\renewcommand{\itemsep}{}}

\begin{document}

\begin{slide}
\maketitle
\end{slide}

\begin{slide}
\begin{center}
{\bf MOTIVATION}
\end{center}
1) Capital income is about 25\% of national income (labor income
is 75\%) but distribution of capital income is much more unequal
than labor income

Capital income inequality is due to differences in savings
behavior but also inheritances received

$\Rightarrow$ Equity suggests it should be taxed more than labor

2) Capital Accumulation correlated strongly with growth [although
causality link is not obvious] and capital accumulation might be
sensitive to the net-of-tax return.

$\Rightarrow$ Efficiency cost of capital taxation might be high.

\end{slide}

\begin{slide}
\begin{center}
{\bf MOTIVATION}
\end{center}

3) Capital more mobile internationally than labor 
%$\Rightarrow$ Incidence of capital taxation might fall on workers:

Key distinction is \textbf{residence} vs. \textbf{source} base capital taxation:

\textbf{Residence: } Capital income tax based on residence of owner of capital.

Most individual income tax systems are residence based (with credits for taxes
paid abroad)

Incidence falls on owner $\Rightarrow$ can only escape
tax through evasion (tax heavens) or changing residence (mobility of persons)

Tax evasion of capital income through tax heavens is a very serious concern (Zucman QJE'13, '15)

\textbf{Source:} Capital income tax based on location of capital (most corporate income tax systems
are source based)

Incidence is then partly shifted to labor if capital is mobile.

Example: Open economy with fully mobile capital and source taxation: 
Local GDP: $wL+rK=F(K,L)=L \cdot F(K/L,1)=L \cdot f(k)$ where $k=K/L$ is capital stock per worker

Net-of-tax rate of return
is fixed by the international rate of return $r^*$ so that
$(1-\tau_c)F_K(K,L)=(1-\tau_c)f'(k)=r^*$ where $k=K/L$ is capital stock per worker and $\tau_c$ corp tax rate

As $wL+r^*K=F(K,L)$, wage $w=F_L(K,L)=f(k)-r^* \cdot k$ falls with $\tau_c$

4) Capital taxation is extremely complex and provides many tax
avoidance opportunities
\end{slide}

\begin{slide}
\begin{center}
{\bf MACRO FRAMEWORK}
\end{center}

Constant return to scale aggregate production:

$Y = F(K,L) = r K + w L$ = output = income

$K$ = capital stock (wealth), $L$ = labor input

$r$ = rate of return on capital, $w$ is wage rate

$rK$ = capital income, $wL$ = labor income

$\alpha=rK/Y$ = capital income share (constant $\alpha$ when
$F(K,L)=K^{\alpha} L^{1-\alpha}$ Cobb-Douglas), $\alpha \simeq
30\%$

$\beta=K/Y$ = wealth to annual income ratio, $\beta \simeq 4-6$

$r= (rK/Y) \cdot (Y/K) = \alpha/\beta$, $r=5-6\%$

\end{slide}



%\begin{slide}
%\begin{center}
%{\bf SAVING FLOWS}
%\end{center}
%Saving is a flow and wealth or net worth is a stock
%
%Three saving flows:
%
%1) {\bf Personal saving:} individual income less individual
%consumption [fell dramatically in the US since 1980s, recent
%$\uparrow$ since 2008]
%
%2) {\bf Corporate Saving:} retained earnings =  after tax profits
%- distributions to shareholders
%
%3) {\bf Government Saving:} Taxes - Expenditures [federal, state
%and local]
%
%Taxes on savings might affect different savings flows differently:
%savings subsidy through a tax credit can $\uparrow$ individual
%savings but $\downarrow$ govt saving [if govt spending stays
%constant]
%\end{slide}




\begin{slide}
\includepdf[pages={15, 7-8,10,11,9}]{capitalincometax_attach.pdf}
\end{slide}


\begin{slide}
\begin{center}
{\bf Piketty (2014) book: Capital in the 21st Century}
\end{center}
Analyzes income, wealth, inheritance data over the long-run:

1) Growth rate $n+g$ = population growth + growth per capita. Population growth will converge
to zero, growth per capita for frontier economies is modest (1\%) $\Rightarrow$ long-run $g \simeq 1\%, n \simeq 0\%$

2) Long-run steady-state Wealth to income ratio ($\beta$) = savings rate ($s$) / annual growth ($n+g$): $\beta=s/(n+g)$

Proof: $K_{t+1}=(1+n+g) \cdot K_t = K_t + s\cdot Y_t \Rightarrow K_t/Y_t = s /(n+g)$

With $s=8\%$ and $n+g=2\%$, $\beta=400\%$ but with $s=8\%$ and $n+g=1\%$, $\beta=800\%$
$\Rightarrow$ Wealth will become important

\end{slide}


\begin{slide}
\begin{center}
{\bf Piketty (2014) book: Capital in the 21st Century}
\end{center}

3) After-tax rate of return on wealth $\bar{r}=r(1-\tau_K)=4-5\%$ significantly larger than $n+g$ [except exceptional period of 1930--1970]

With $\bar{r} > n+g$, role of inheritance in wealth and wealth concentration become large [past swallows the future]

Explanation: Rentier who saves all his return on wealth accumulates wealth at rate $\bar{r}$ bigger than $n+g$ and hence
his wealth grows relative to the size of the economy. The bigger $\bar{r}-(n+g)$, the easier it is for wealth to ``snowball''

$\Rightarrow$ Capital taxation reduces $r$ to $\bar{r}=r \cdot (1-\tau_K)$ $\Rightarrow$ This can reduce wealth concentration


\end{slide}

\begin{slide}
\includepdf[pages={18}]{capitalincometax_attach.pdf}
\end{slide}



\begin{slide}
\begin{center} {\bf WEALTH AND CAPITAL INCOME IN AGGREGATE} \end{center}
{\bf Definition:} Capital Income = Returns from Wealth Holdings

Aggregate US {\bf Personal} Wealth $\simeq$ 4*GDP $\simeq$ \$60 Tr

{\bf Tangible assets:} residential real estate (land+buildings)
[income = rents] and unincorporated business + farm assets [income
= profits]

{\bf Financial assets:} corporate stock [income = dividends +
retained earnings], fixed claim assets (corporate and govt bonds,
bank accounts) [income = interest]

{\bf Liabilities:} Mortgage debt, Student loans, Consumer credit debt

Substantial amount of financial wealth is held indirectly through:
pension funds [DB+DC], mutual funds, insurance reserves

\end{slide}

\begin{slide}
\includepdf[pages={22, 21}]{capitalincometax_attach.pdf}
\end{slide}


%\begin{slide}
%\begin{center} {\bf CAPITAL INCOME IN NATIONAL ACCOUNTS} \end{center}
%
%Gross capital income (before depreciation) is about 40\% of GDP
%
%Net capital income (after depreciation) is about 25\% of personal
%income
%
%The capital income share in total income is relatively stable in
%the long-run (but with some short term fluctuations)
%
%Average real rate of return of capital around 5-6\%, varies
%greatly from year to year
%\end{slide}

\begin{slide}
\begin{center} {\bf INDIVIDUAL WEALTH AND CAPITAL INCOME} \end{center}

Wealth = $W$, Return = $r$, Capital Income = $rW$

$W_t=W_{t-1}+r_t W_{t-1}+ E_t + I_t - C_t $

where $W_t$ is wealth at age $t$, $C_t$ is consumption, $E_t$
labor income earnings (net of taxes), $r_t$ is the average (net)
rate of return on investments and $I_t$ net inheritances (gifts
received and bequests minus gifts given).

Replacing $W_{t-1}$ and so on, we obtain the following expression
(assuming initial wealth $W_0$ is zero):
$$W_t= \sum_{k=1}^t (E_k-C_k+I_k) \prod_{j=k+1}^{t}(1+r_j)$$
\end{slide}

\begin{slide}
\begin{center} {\bf INDIVIDUAL WEALTH AND CAPITAL INCOME} \end{center}
$$W_t= \sum_{k=1}^t (E_k-C_k) \prod_{j=k+1}^{t}(1+r_j)+\sum_{k=1}^t I_k\prod_{j=k+1}^{t}(1+r_j)$$

1st term is \textbf{life-cycle} wealth, 2nd term is \textbf{inheritance} wealth

Differences in Wealth and Capital income due to:

1) Age

2) past earnings, and past saving behavior $E_t-C_t$ [life
cycle wealth]

3) Net Inheritances received $I_t$ [transfer wealth]

4) Rates of return $r_t$

[details in Davies-Shorrocks '00, Handbook chapter]

\end{slide}


\begin{slide}
\begin{center} {\bf WEALTH DISTRIBUTION} \end{center}
Wealth inequality is very large (much larger than labor income)

US Household Wealth is divided 1/3,1/3,1/3 for the top 1\%, the
next 9\%, and the bottom 90\% [bottom 1/2 households hold almost
no wealth]

Financial wealth is more unequally distributed than (net) real
estate wealth

Share of real estate wealth falls at the top of the wealth
distribution

Growth of private pensions [such as 401(k) plans] has
``democratized'' stock ownership in the US

US public underestimates extent of wealth inequality and thinks the
ideal wealth distribution should be a lot less unequal [Norton-Ariely '11] 
\end{slide}


\begin{slide}
\includepdf[pages={6}]{capitalincometax_attach.pdf}
\end{slide}

%\begin{slide}
%\begin{center} {\bf WEALTH DISTRIBUTION} \end{center}
%Wealth is more unequally distributed than income [true in all
%countries]
%
%Top 1\% income share in the US is around 20\%
%
%Top 1\% labor income share in the US (among workers) is around
%15\%
%
%US Income concentration has increased sharply since 1970:
%
%Top 1\% income share was 9\% in 1970 and 20\%+ in recent years
%[Piketty-Saez QJE'03 updated]
%
%US Wealth concentration has also increased but likely less than income
%
%%Top 1\% wealth share has grown ``only'' from 31\% in 1962 to 34\%
%%in 2007 based on the Survey of Consumer Finances [Scholz '03,
%%Kennickell '09]
%\end{slide}

\begin{slide}
\begin{center} {\bf WEALTH MEASUREMENT} \end{center}
In the US, wealth distribution much less well measured than income distribution
because no systematic administrative source (no wealth tax). 3 methods
to estimate wealth distribution:

\textbf{1) Surveys:} US Survey of Consumer Finances (SCF)

Top 10\% wealth share has grown from 67\% in 1989 to 75\% in 2010

Top 1\% wealth share has grown ``only'' from 30\% in 1989 to 35\%
in 2010 [Kennickell '09, '12]

Problems: small sample size, measurement error, only every 3 years, starts in 1989

\pagebreak

\textbf{2) Estate multiplier method:} use annual estate tax statistics and re-weights individual 
estates by inverse of death probability [based on age$\times$gender$\times$social class]

Kopczuk-Saez NTJ'04 create series 1916-2000 and find fairly small increases in wealth
concentration in recent decades

Problems:  social class effect on mortality not well known, significant estate tax avoidance, noisy
measure of ``young wealth'', estates cover only the super rich (top .1\% in recent years)

\textbf{3) Capitalization method:} use capital income from individuals tax statistics and estimates rates of returns
by asset class to infer wealth: shows big increase
in wealth concentration [Saez-Zucman '14 in progress]

%Piketty (2014) compiles US (and Europe) wealth concentration series
\end{slide}


\begin{slide}
\includepdf[pages={23,24, 33, 34}]{capitalincometax_attach.pdf}
\end{slide}

\begin{slide}
\includepdf[pages={25}]{capitalincometax_attach.pdf}
\end{slide}

\begin{slide}
\includepdf[pages={29, 26-28}]{capitalincometax_attach.pdf}
\end{slide}

\begin{slide}
\begin{center} {\bf CAPITAL TAXATION IN THE US} \end{center}

Good US references: Gravelle '94 book, Slemrod-Bakija '04 book

1) {\bf Corporate Income Tax} (fed+state): 35\% Federal tax rate
on profits of corporations [complex rules with many industry
specific provisions]: effective tax rate much lower and incidence
depends on mobility of capital

%[Full expensing means deducting investment costs instead of
%depreciation, amounts to exempting normal rate of return from tax]

2) {\bf Individual Income Tax} (fed+state): taxes many forms of
capital income

Realized capital gains and dividends (dividends since '03 only)
receive preferential treatment

Imputed rent of home owners, returns on pension funds, state+local
government bonds interest are exempt

\end{slide}

\begin{slide}
\begin{center} {\bf FACTS OF US CAPITAL INCOME TAXATION} \end{center}

3) {\bf Estate and gift taxes:}

Fed taxes estates above \$5.5M exemption (only .1\% of deceased
liable), tax rate is 40\% above exemption (2013+)

Charitable and spousal giving is exempt

Substantial tax avoidance activity through tax accountants

Step-up of realized capital gains at death (lock-in effect)

4) {\bf Property taxes} (local) on real estate (old tax):

Tax varies across jurisdictions. About 0.5\% of market value on
average, like a 10\% tax on imputed rent if return is 5\%

Lock-in effect in states that use purchase price base such as
California

%Note that Value Added Tax (VAT) [does not exist in the US] exempts
%normal return on capital but taxes excess returns

\end{slide}

%\begin{slide}
%\begin{center}
%{\bf LIFE CYCLE MODEL OF WEALTH (MODIGLIANI)}
%\end{center}
%Individuals work for $R$ years and live for $T$ years: $T-R$ is
%retirement duration
%
%Individuals earn income $w_t$ from period $0$ to $R$ and earn zero
%afterwards
%
%Individuals have additive separable utility $\sum_{t=1}^T
%u(c_t)/(1+\delta)^t$ with concave $u(.)$
%
%subject to inter-temporal budget constraint: $\sum_{t=1}^T
%c_t/(1+r)^t \leq \sum_{t=1}^R w_t/(1+r)^t$ (multiplier $\lambda$)
%
%FOC: $u'(c_t)/(1+\delta)^t=\lambda / (1+r)^t$
%
%Euler equation: $u'(c_{t+1})/u'(c_t)=(1+\delta)/(1+r)$
%
%
%\end{slide}
%
%\begin{slide}
%\begin{center}
%{\bf LIFE CYCLE MODEL OF WEALTH (MODIGLIANI)}
%\end{center}
%Euler equation: $u'(c_{t+1})/u'(c_t)=(1+\delta)/(1+r)$
%
%If $\delta<r$, $c_{t+1}>c_t$ $\Rightarrow$ Individuals save to
%consume more later on
%
%If $\delta>r$, $c_{t+1}<c_t$ $\Rightarrow$ Individuals want to
%consume more earlier on
%
%If $\delta=r$ then $c_t$ is constant with $t$:
%
%$\Rightarrow$ Individuals want to smooth consumption by saving
%while working and consuming saving while retired $\Rightarrow$
%Wealth $W_t$ is inversely U-shaped during life-cycle
%
%$\Rightarrow$ Wealth inequality only slightly higher than labor
%income inequality [does not fit facts]
%\end{slide}

%\begin{slide}
%\begin{center}
%{\bf OTHER FACTORS AFFECTING WEALTH DISPERSION}
%\end{center}
%1) Heterogeneity in tastes for saving:
%
%$\bullet$ traditional discount rate
%
%$\bullet$ self-control problems [hyperbolic discount rate] and
%financial education
%
%2) Rates of returns received on assets: traditional risk aversion,
%luck, but also financial education
%
%3) Net inheritances and gifts received [in general from parents]
%
%\end{slide}

\begin{slide}
\begin{center}
{\bf LIFE CYCLE VS. INHERITED WEALTH}
\end{center}
Old view: Tobin and Modigliani: life cycle wealth accounts for the
bulk of the wealth held in the US. Kotlikoff-Summers JPE'81
challenged the old view (debate Kotlikoff vs. Modigliani in JEP'88) 

Why is this question important? 

1) Economic Modeling: what accounts for wealth accumulation and
inequality? Is widely used life-cycle model with no bequests a
good approximation? %[Causality between growth and savings]

2) Policy Implications: taxation of capital income and estates.
Role of pay-as-you-go vs. funded retirement programs

Key problem is that the definition of life-cycle vs. inherited wealth
is not conceptually clean (Modigliani does not capitalize inherited wealth
while Kotlikoff-Summers do) 

\end{slide}


\begin{slide}
\begin{center} {\bf LIFE CYCLE VS. INHERITED WEALTH} \end{center}
Piketty-Postel-Vinay-Rosenthal EEH'14 (PPVR) propose better definition to resolve Modigliani
vs. Kotlikoff-Summers controversy (see Piketty-Zucman Handbook chapter '14)

Individual wealth accumulation:
\[ W_t= \sum_{k=1}^t (E_k-C_k) \cdot (1+r)^{t-k}+\sum_{k=1}^t I_k \cdot (1+r)^{t-k} \]
\small
If $W_t > \sum_{k=1}^t I_k \cdot (1+r)^{t-k}$ then individual also saves out of labor income $E_k$
and inherited wealth is $\sum_{k=1}^t I_k \cdot (1+r)^{t-k}$

If $W_t \leq \sum_{k=1}^t I_k \cdot (1+r)^{t-k}$ then individual consumes part of inheritances (in addition to labor income)
and inherited wealth is $W_t$

PPVR requires micro-data for implementation. If we assume uniform saving rate $s$, there is a simplified
formula for share of inherited wealth $b_y/[b_y+(1-\alpha) \cdot s]$ with $b_y$ bequest flow/national income
and $\alpha$ capital share

\end{slide}


%\begin{slide}
%\begin{center}
%{\bf LIFE CYCLE VS. INHERITED WEALTH}
%\end{center}
%
%$W$ total wealth in the economy, $LCW$ is life cycle wealth, and
%$T$ wealth due to transfers
%
%Two components in the individual wealth equation $W_t$:
%$$LCW_t= \sum_{k=1}^t (E_k-C_k) (1+r)^{t-k}$$
%$$T_t= \sum_{k=1}^t I_k (1+r)^{t-k}$$
%Aggregate this over all individuals or households in the economy
%to estimate $T/W$ and $LCW/W$. Two methods:
%
%(1) Compute $T_t$ (flow of bequests method)
%
%(2) Compute $LCW_t$ (comparison of earnings and consumption)
%
%\end{slide}

%\begin{slide}
%\includepdf[pages={1}]{capitalincometax_attach.pdf}
%\end{slide}

%\begin{slide}
%\begin{center}
%{\bf LIFE CYCLE VS. INHERITED WEALTH}
%\end{center}
%(a) Modigliani JEP'88 claims that over 2/3 of wealth is due to
%life-cycle
%
%(b) Kotlikoff-Summers JPE'81, JEP'88 claim that over 2/3 of wealth
%is due to transfers
%
%Differences due primarily in methodology (Gale and Scholtz
%JEP'94):
%
%(a) how to capitalize past transfers
%
%(b) whether to count college tuition paid by parents as transfers
%
%Transfer wealth is probably quite important, especially at the top
%of the wealth distribution
%\end{slide}
%
%\begin{slide}
%\includepdf[pages={2}]{capitalincometax_attach.pdf}
%\end{slide}

\begin{slide}
\begin{center}
{\bf LIFE CYCLE VS. INHERITED WEALTH}
\end{center}
How do the shares of inheritance vs.
life-cycle evolve over time? First measure is inheritance flow to national income

Inheritance share likely huge in the distant past: class society
with rentiers vs. workers [Delong '03]

Inheritance share $\downarrow$ in 20th century but 
has $\uparrow$ recently in France (Piketty QJE'11, Piketty-Zucman '14 handbook chapter)

Post-war period was a time of fast population growth and fast
economic growth $\Rightarrow$  If $n+g$ (growth) large relative to
$r$ (rate of return on wealth) $\Rightarrow$ Inheritances play a
minor role in life-time wealth

In general $r>n+g$ in which case inheritances play a large role in aggregate
wealth and wealth concentration is going back (Western countries moving in that
direction, Piketty '14)

\end{slide}

\begin{slide}
\includepdf[pages={5, 31, 30, 32}]{capitalincometax_attach.pdf}
\end{slide}



%\begin{slide}
%\begin{center} {\bf KEY ELEMENTS OF DEBATE ON CAPITAL INCOME
%TAXATION}\end{center}
%
%Economic debate:
%
%1) Distributional concerns: capital income accrues
%disproportionately to higher income families
%
%2) Efficiency concerns: capital tax distorts savings, business
%creation, capital mobility across countries
%
%Public policy debate:
%
%3) Should we tax income vs. consumption? [Fundamental tax reform
%debate]
%
%4) Should we encourage savings by cutting tax on capital income or
%with tax favored savings vehicles?
%\end{slide}

\begin{slide}
\begin{center} {\bf TAXES IN OLG LIFE-CYCLE MODEL} \end{center}
$\max \:\: U=u(c_1,l_1)+ \delta u(c_2,l_2)$

No tax situation: earn $w_1l_1$ in period 1, $w_2 l_2$ in period 2

Savings $s=w_1 l_1 - c_1$, $c_2=w_2 l_2 + (1+r)s$

Capital income $r s$

Intertemporal budget with no taxes:

$c_1+c_2/(1+r) \leq w_1 l_1 + w_2 l_2 /(1+r)$

This model has uniform rate of return and does not capture excess
returns
\end{slide}

\begin{slide}
\begin{center} {\bf TAXES IN OLG MODEL} \end{center}
Budget with consumption tax $t_c$: $$(1+t_c) [ c_1+c_2/(1+r)] \leq
w_1 l_1 + w_2 l_2/(1+r)$$ Budget with labor income tax $\tau_L$:
$$c_1+c_2/(1+r)  \leq (1-\tau_L)[w_1 l_1 + w_2 l_2/(1+r)]$$
Consumption and labor income tax are equivalent if
$$1+t_c=1/(1-\tau_L)$$ Both taxes distort only labor-leisure choice
\end{slide}

\begin{slide}
\begin{center} {\bf TAXES IN OLG MODEL} \end{center}
Budget with capital income tax $\tau_K$: $$c_1+c_2/(1+r(1-\tau_K))
\leq w_1 l_1 + w_2 l_1/(1+r(1-\tau_K))$$ $\tau_K$ distorts only
inter-temporal consumption choice

Budget with comprehensive income tax $\tau$:
$$c_1+c_2/(1+r(1-\tau)) \leq (1-\tau)[w_1 l_1 + w_2
l_2/(1+r(1-\tau))]$$ $\tau$ distorts both labor-leisure and
inter-temporal consumption choices

$\tau$ imposes ``double'' tax: (1) tax on earnings, (2) tax on
savings
\end{slide}

\begin{slide}
\begin{center} {\bf EFFECT OF $r$ ON SAVINGS} \end{center}
Assume that labor supply is fixed. Draw graph.
Suppose $r$ $\uparrow$:

1) Substitution effect: price of $c_2$ $\downarrow$ $\Rightarrow$
$c_2 \uparrow$, $c_1$ $\downarrow$ $\Rightarrow$  savings
$s=w_1l_1-c_1$ $\uparrow$.

2) Wealth effect: Price of $c_2$ $\downarrow$ $\Rightarrow$ both
$c_1$ and $c_2$ $\uparrow$ $\Rightarrow$ save less

3) Human wealth effect: present discounted value of labor income
$\downarrow$ $\Rightarrow$ both $c_1$ and $c_2$ $\downarrow$
$\Rightarrow$ save more

Note: If $w_2 l_2 < c_2$ (ie $s>0$), 2)+3) $\Rightarrow$ save less

Total net effect is theoretically ambiguous $\Rightarrow$ $\tau_K$
has ambiguous effects on $s$
\end{slide}

\begin{slide}
\begin{center} {\bf SHIFT FROM LABOR TO CONSUMPTION TAX} \end{center}
Labor and consumption are equivalent for the individual if
$1+t_c=1/(1-\tau_L)$ but savings pattern is different

Assume $w_2=0$ and $l_1=1$

$(1+t_c)[c_1 + c_2/(1+r)] = w_1$ with consumption tax

$c_1 + c_2/(1+r) = (1-t_L) w_1$ with labor tax

1) Consumption tax $t_c$: $c^c_1=(w_1-s_c)/(1+t_c)$,
$c^c_2=(1+r)s_c/(1+t_c)$

2) Labor income tax $\tau_L$: $c^L_1=w_1(1-\tau_L)-s_L$,
$c^L_2=(1+r)s_L$

Same consumption in both cases so $s_L=s_c/(1+t_c)$ $\Rightarrow$
Save more with a consumption tax
\end{slide}

\begin{slide}
\begin{center} {\bf TRANSITION FROM LABOR TO C TAX} \end{center}
In OLG model and closed economy, capital stock is due to
life-cycle savings $s$

Start with labor tax $\tau_L$ and switch to a
consumption tax $t_c$

The old [at time of transition] would have paid nothing in labor
tax regime but now have to pay tax on $c_2$

For the young [and future generations], the two regimes look
equivalent so they now save more and increase the capital stock

However, this increase in capital stock comes at the price of
hurting the old who are taxed twice
\end{slide}

\begin{slide}
\begin{center} {\bf TRANSITION FROM LABOR TO C TAX} \end{center}
Suppose the government keeps the old as well off as in previous
system by exempting them from consumption tax

This creates a deficit in government budget equal to

$d=\tau_L w_1 - t_c c_1 = t_c w_1/(1+t_c) -t_c c_1 = t_c s_L $

Extra saving by the young is $s_c-s_L=t_c s_L$ exactly equal to
government deficit.

{\bf Full neutrality result:} Extra savings of young is equal to
old capital stock + new government deficit $\Rightarrow$ no change
in the aggregate capital stock

Full neutrality depends crucially on same $r$ for govt debt and
aggregate $r$ [in practice: equity premium puzzle]

[Same result for Social Security privatization]
\end{slide}

%\begin{slide}
%\begin{center} {\bf AUERBACH-KOTLIKOFF '87 MODEL} \end{center}
%
%Develop an inter-temporal Computational General Equilibrium (CGE)
%model:
%
%1) Life cycle model, no bequests, people live for 55 years (born
%at age 21). Work for 45 years, and retire for 10 years.
%
%2) Only one individual per cohort, representative agent model [
%Useful for redistribution analysis across cohorts but not within
%cohorts]
%
%3) Stock of wealth = life cycle savings [Classical Modigliani
%graph]
%
%4) Labor income tax distorts labor supply, capital income tax
%distorts savings choice.
%
%CES utility, discount rate, path of earnings with life cycle.
%\end{slide}
%
%\begin{slide}
%\begin{center} {\bf AUERBACH-KOTLIKOFF '87: RESULTS} \end{center}
%
%Tax reform experiments: shift from comprehensive income tax to
%either
%
%(a) pure consumption tax
%
%(b) pure wage income tax
%
%(c) pure capital income tax
%
%[budget neutral but no transitional compensation]
%
%\end{slide}
%
%\begin{slide}
%\includepdf[pages={3,4}]{capitalincometax_attach.pdf}
%\end{slide}
%
%\begin{slide}
%\begin{center} {\bf AUERBACH-KOTLIKOFF '87: RESULTS} \end{center}
%1) Effect on capital stock:
%
%(a) Consumption tax is best (because no compensation of the old)
%
%(b) Wage income tax has limited impact on capital stock
%
%(c) Capital income tax is worst (significant elasticity of savings
%wrt to $r$).
%
%\end{slide}
%
%\begin{slide}
%\begin{center} {\bf AUERBACH-KOTLIKOFF '87: RESULTS} \end{center}
%
%2) Effect on welfare measured in percentage increase of
%consumption for each generation:
%
%Consumption tax hurts current old and benefits the young and
%future generations [no transitional relief]
%
%Wage income tax benefits the old but hurts the young
%
%Capital income tax hurts current generation (double tax), benefits
%next generation (implicit levy of previously accumulated capital)
%but hurts future generations (inefficient)
%
%Key lessons: Transitional reliefs rules and anticipated vs. not
%tax changes has large impact on results
%\end{slide}

\begin{slide}
\begin{center} {\bf OPTIMAL CAPITAL INCOME TAXATION}
\end{center}
Complex problem with many sub-literatures: Banks and Diamond
Mirrlees Review '09 provide recent survey

1) Life-cycle models [linear and non-linear earnings tax]

2) Models with bequests [many models including the infinite
horizon model]

3) Models with future earnings uncertainty: New Dynamic Public
Finance [Kocherlakota '09 book]

Bigger gap between theory and policy practice than in the case of
static labor income taxation
\end{slide}

%\begin{slide}
%\begin{center} {\bf RAMSEY TAX IN LIFE-CYCLE MODEL}
%\end{center}
%
%Best reference is King (1980) [volume Heal-Hughes]. Also
%Atkinson-Sandmo EJ '80, Sandmo '85 PE Handbook Chapter,
%Atkinson-Stiglitz '80, Chap 14-4.
%
%Ramsey model with {\bf representative} agent and {\bf linear}
%taxes on labor and savings to raise exogenous amount of revenue
%
%Individual maximization problem:
%\[V(q,w(1-\tau_L))= max_{c_1,c_2,l} \,\, u(c_1,c_2,l) \]
%\[\mathrm{st} \quad c_1+c_2/(1+r(1-\tau_K))=wl(1-\tau_L) \]
%where $q=1/(1+r(1-\tau_K))$ and $p=1/(1+r)$ are post-tax and pre-tax
%prices of $c_2$
%
%%Output is given by constant return to scale production function
%%$Y=F(K,l)=l f(K/l)=l f(k)$, $r=F_K=f'(k)$ and $w=F_l$
%
%\end{slide}
%
%\begin{slide}
%\begin{center} {\bf RAMSEY CAPITAL INCOME TAX} \end{center}
%
%%{\bf Implication:}
%
%Optimal tax rates can be obtained by solving the standard Ramsey
%problem:
%\[\max_{\tau_K,\tau_L} \:\: V(q,w(1-\tau_L)) \quad \mathrm{st} \quad
%w l \tau_L + (q-p) c_2 \geq g \quad (\lambda) \]
%where $g$ is exogenous tax revenue requirement
%
%Can apply the results from the 3 good Ramsey model
%
%Derive FOC for $\tau_K$ and $\tau_L$
%
%Can express them in terms of compensated elasticities
%
%\end{slide}
%
%\begin{slide}
%\begin{center} {\bf RAMSEY CAPITAL INCOME TAX} \end{center}
%Combining the two FOC to get rid of $\lambda$, you get:
%$$\frac{r \tau_K}{1+r} (\sigma_{L2}-\sigma_{22}) = \frac{\tau_L}{1-\tau_L}
%(\sigma_{LL}-\sigma_{2L})$$ where
%$\sigma_{LL}=(w(1-\tau_L)/l)\partial l^c / \partial
%(w(1-\tau_L))>0$ is the compensated elasticity of labor supply
%wrt net wage rate
%
%$\sigma_{22}=(q/c_2)\partial c_2^c / \partial q<0$
%
%$\sigma_{L2}=(q/l)\partial l^c / \partial q$
%
%$\sigma_{2L}=(w(1-\tau_L)/c_2)\partial c^c_2 / \partial
%(w(1-\tau_L))$
%
%Formula defines relative optimal rates of taxation on labor and
%capital (absolute levels depend on $g$)
%\end{slide}
%
%\begin{slide}
%\begin{center} {\bf RAMSEY CAPITAL INCOME TAX: DISCUSSION} \end{center}
%Little known about cross elasticities so we might as well assume
%that they are zero [symmetric by Slutsky] $\Rightarrow$ Optimal
%formula simplifies to:
%$$-\frac{r \tau_K}{1+r} \sigma_{22} = \frac{\tau_L}{1-\tau_L} \sigma_{LL} $$
%{\bf Inverse elasticity rule} as in standard Ramsey model: If $
%\sigma_{LL}  << | \sigma_{22} | $ then $\tau_K$ should be small
%relative to $\tau_L$
%
%{\bf Key lesson:} What matters is the relative size of
%elasticities, not the number of distortions
%\end{slide}
%
%\begin{slide}
%\begin{center} {\bf FELDSTEIN JPE'78} \end{center}
%
%Feldstein JPE'78 makes famous theoretical argument why $
%\sigma_{22}$ can be large even if
%$\varepsilon^u_{sq}=(q/s)\partial s/\partial q$ [uncompensated
%savings elasticity] is zero: Budget $c_1+q c_2 = w(1-\tau_L)l+y$
%
%Slutsky equation [$y$ is endowment =0 in equilibrium]: $\partial
%c^c_2 /
%\partial q = \partial c_2 /
%\partial q + c_2 \partial c_2 / \partial y$ $\Rightarrow$
%
%$\sigma_{22}=\varepsilon^u_{2q} + q \partial c_2 / \partial y$
%
%$c_2=s/q$ so $\varepsilon^u_{2q}= (q/c_2)\partial c_2/\partial q =
%\varepsilon^u_{sq}-1$ $\Rightarrow$
%
%$\sigma_{22}=\varepsilon^u_{sq} -1 + q \partial c_2 / \partial y$
%
%$c_1+ q c_2 = w(1-\tau_L)l+y$ $\Rightarrow$ $\partial c_1 /
%\partial y + q \partial c_2 / \partial y =  w(1-\tau_L)\partial l /
%\partial y +1 \simeq 1$ (small income effects on labor supply)
%
%$\sigma_{22} \simeq \varepsilon^u_{sq} - \partial c_1 / \partial y
%\simeq -\partial c_1 / \partial y \leq -0.75$ [as saving rate
%modest]
%
%\end{slide}

%\begin{slide}
%\begin{center} {\bf CORLETT-HAGUE RESTUD'53-54} \end{center}
%Obtain optimal tax formulas in three good model $c_1,c_2,l$
%
%$\sigma_{1L})=\sigma_{2L}=\sigma_{LL}
%
%Tax rate on $c_1$ equal to tax rate on $c_2$ $\Leftrightarrow$ Tax
%only on labor is sufficient $\Leftrightarrow$
%$$\sigma_{1L})=\sigma_{2L}=\sigma_{LL}$$
%
%Compensated change in the wage rate leads to equal proportionate
%changes in $c_1$, $c_2$, and $l$ $\Leftrightarrow$
%
%[$\Leftrightarrow$ Ratio of money spent on $c_1$ and $c_2$ is
%independent of wage rate]
%
%Tax $t_r>0$ $\Leftrightarrow$ Compensated change in the wage rate
%has larger effect on $c_1$ than on $c_2$
%
%\end{slide}

%\begin{slide}
%\begin{center} {\bf RAMSEY TAX: ENDOGENOUS CAPITAL STOCK} \end{center}
%Full dynamic model:
%
%Govt maximizes $SW = \sum_t V_t/(1+\delta)^t$
%
%subject to $\sum_t Tax_t/(1+r)^t \geq \sum_t g_t/(1+r)^t$
%
%$\Rightarrow$ Optimal dynamic capital stock $k$ is given by
%Modified Golden rule $r=f'(k)=\delta$
%
%Optimal $k$ can be attained in steady state using debt policy
%[implicit in budget constraint]
%
%In that case, optimal $\tau_K,\tau_L$ given by same static Ramsey rule
%
%Problems of dynamic efficiency (optimal $K$ stock) and efficiency within
%a generation ($\tau_L,\tau_K$) are orthogonal
%\end{slide}
%
%\begin{slide}
%\begin{center} {\bf RAMSEY TAX: ENDOGENOUS CAPITAL STOCK} \end{center}
%
%If the govt cannot use debt policy then optimal dynamic capital
%level may not be attained because savings equal capital $s_t=K_t$
%$\Rightarrow$ tax formulas need to be modified: optimal tax rates
%reflect
%
%(a) the trade-off between conventional [intra-generational]
%efficiency losses [static Ramsey]
%
%(b) the failure to achieve the dynamic optimality condition on
%capital stock [inter-generational efficiency trade-off]
%
%$\Rightarrow$ Effect on capital tax rate level is actually
%ambiguous
%
%\end{slide}

%\begin{slide}
%\begin{center} {\bf RAMSEY CAPITAL INCOME TAX} \end{center}
%
%Government maximizes:
%
%$SW = \sum_t V_t/(1+\delta)^t$ subject to $\sum_t Tax_t/(1+r)^t
%\geq \sum_t g_t/(1+r)^t$ [$g_t$ exogenous govt expenditures in
%period $t$]
%
%Implicit in this budget constraint: the government can use debt
%paying the same return $r$ as physical capital.
%
%Use of debt allows govt to reach dynamically efficient capital
%stock level $r=F_K = \delta$ [return = discount rate: modified
%golden rule] in the long-run steady state
%
%Proof: If $r> \delta$ in long-run: govt increases taxes by \$1 in
%period $t$ and decreases taxes by \$1+$r$ in period $t+1$: budget
%neutral and
%$dSW=-V^t_R/(1+\delta)^t+(1+r)V_R^{t+1}/(1+\delta)^{t+1}=V_t^R/(1+\delta)^t
%\cdot [-1 + (1+r)/(1+\delta)]>0$
%
%\end{slide}

%\begin{slide}
%\begin{center} {\bf RAMSEY CAPITAL INCOME TAX: REMARKS} \end{center}
%
%1) No redistributive concerns: Can extend model to the
%multi-person case $\Rightarrow$ Higher rate $\tau_K$ if capital
%income concentrated among the rich (Park JPubE, 1991).
%
%2) No bequests so this model does not capture an important aspect
%of wealth accumulation and justification for redistribution.
%
%3) Only a two period model, if more periods are introduced (as in
%the Auerbach-Kotlikoff simulation model), then optimal tax formula
%would be more complex.
%\end{slide}

\begin{slide}
\begin{center} {\bf Life-Cycle model: Atkinson-Stiglitz JpubE '76} \end{center}

Heterogeneous individuals and government uses nonlinear tax on
earnings. Should the govt also use tax on savings?

$V^h= \max U^h(v(c_1,c_2),l)$ st
$c_1+c_2/(1+r(1-\tau_K))=wl-T_L(wl)$

If utility is weakly separable and $v(c_1,c_2)$ is the same for
all individuals, then the government should use only labor income
tax and should not use tax on savings

Recent proof by Laroque EJ '05 or Kaplow JpubE '06.

Tax on savings justified if: 

(1) High skill people have higher
taste for saving (e.g, high skill people have lower discount rate) [Saez, JpubE '02]

(2) $c_2$ is complementary with
leisure
\end{slide}

\begin{slide}
\begin{center} {\bf Life-cycle model: linear labor income tax} \end{center}

Suppose the government can only use linear earnings tax: $wl \cdot (1-\tau_L) + E$

If sub-utility $v(c_1,c_2)$ is also homothetic of degree one [i.e., $v(\lambda c_1,\lambda c_2)=\lambda v(c_1,c_2)$ for all
$\lambda$]
then $\tau_K=0$ is again optimal [linear tax counter-part of Atkinson-Stiglitz, see Deaton, 1979]

In the general case $V^h(c_1,c_2,l)$, optimal $\tau_K$ is not always zero

Old literature considered the Ramsey one-person model of linear taxation and expressed optimal
$\tau_K$ as a function of compensated price and cross-price elastiticities [Corlett-Hague REstud'54,
King, 1980, and Atkinson-Sandmo EJ'80]

\end{slide}

%\begin{slide}
%\begin{center} {\bf DIAMOND-SPINNEWIJN '09} \end{center}
%Heterogeneity of individuals in ability (wage rate) and discount
%rate. Discrete earnings choice model (high vs. low) and discrete
%discount (high vs. low) [Four type model]
%
%Govt can tax both earnings and savings non-linearly:
%bi-dimensional tax function with bi-dimensional heterogeneity
%
%Start from no savings tax and optimal earnings tax
%
%{\bf Result:} introducing a small savings tax on high earners or a
%small savings subsidy on low earners increases welfare
%
%{\bf Intuition:} Those valuing the future more (relative to the
%disutility of work) are more willing to work than those valuing
%the future less $\Rightarrow$ work IC constraint binds for high
%wage/low savers but not for high wage/high saver $\Rightarrow$
%Scope for taxing savings
%%
%%Interesting work in bi-dimensional pb (related to
%%Kleven-Kreiner-Saez EMA '08 on couples)
%
%\end{slide}

\begin{slide}
\begin{center} {\bf LIMITS OF LIFE-CYCLE MODEL} \end{center}
Atkinson-Stiglitz shows that life-time savings should not be
taxed, tax only labor income

From justice view: seems fair to not discriminate against savers
if labor earnings is the only source of inequality and is taxed
non-linearly

In reality, capital income inequality also due

(1) difference in rates of returns

(2) shifting of labor income into capital income

(3) inheritances

(1) is not relevant if individuals handle risky assets rationally
(as in CAPM model), probably not a very good assumption
$\Rightarrow$ Tax on lucky returns might be desirable

\end{slide}

\begin{slide}
\begin{center} {\bf SHIFTING OF LABOR / CAPITAL INCOME} \end{center}
In practice, difficult to distinguish between capital and labor
income [e.g., small business profits, professional traders]

Differential tax treatment can induce shifting:

(1) US C-corporations vs S-corporations: shift from corporate
income (and subsequent realized capital gains) toward individual business
income [Gordon and Slemrod '00]

(2) Carried interest in the US: hedge fund and private equity fund
managers receive fraction of profits of assets they manage for
clients. Those profits are really labor income but are taxed as
realized capital gains

(3) Finnish Dual income tax system: taxes separately capital
income at preferred rates since 1993: Pirttila and Selin SJE'11
show that it induced shifting from labor to capital income
especially among self-employed
\end{slide}


\begin{slide}
\begin{center} {\bf Theory: Shifting of Labor / Capital Income} \end{center}

Extreme case where government cannot distinguish at all between
labor and capital income $\Rightarrow$ Govt observes only $wl+rK$
$\Rightarrow$ Only option is to have identical MTRs at individual
level $\Rightarrow$ General income tax $Tax=T(wl+rK)$

With a finite shifting elasticity, differential MTRs for labor and
capital income taxation induce an additional shifting distortion

The higher the shifting elasticity, the closer the tax rates on
labor and capital income should be [Christiansen and Tuomala
ITAX'08, see also Piketty-Saez Handbook chapter '13]

In practice, this seems to be a very important consideration when
designing income tax systems [especially for top incomes]
$\Rightarrow$ Strong reason for having $\tau_L=\tau_K$ at the top
\end{slide}


%\begin{slide}
%\begin{center}
%{\bf MODELS OF BEQUESTS AND GIFTS}
%\end{center}
%Individuals receive inheritances and inter-vivos gifts. Those
%arise because of:
%
%1) Accidental bequests: uncertain life-time with imperfect annuity
%markets $\Rightarrow$ people die with positive wealth
%
%2) Wealth loving utility: individuals enjoy owning wealth above
%consumption utility [Carroll '00]
%
%3) Altruistic (or warm glow) bequests/gifts: people enjoy making
%transfers to children [Barro-Becker dynastic model is most famous
%example]
%
%4) Manipulative / social norms bequests: bequests used to extract
%services from heirs or pressure from heirs to leave bequests
%[equal estate division Wilhelm AER'96, spousal pensions Aura
%JpubE'05]
%
%\end{slide}


\begin{slide}
\begin{center} {\bf Taxation of Inheritances: Welfare Effects} \end{center}
Definitions: donor is the person giving, donee is the person
receiving

Inheritances and inter-vivos transfers raise difficult issues:

(1) Inequality in inheritances contributes to economic inequality:
seems fair to redistribute from those who received inheritances to
those who did not

(2) However, it seems unfair to double tax the donors who worked
hard to pass on wealth to children

$\Rightarrow$ Double welfare effect: inheritance tax hurts donor
(if donor altruistic to donee) and donee (which receives less) [Kaplow, '01]
\end{slide}


\begin{slide}
\begin{center} {\bf Estate Taxation in the United States} \end{center}

Estate federal tax imposes a tax on estates above \$5.5M exemption (only about .1\% of deceased
liable), tax rate is 40\% above exemption (2013+)

Charitable and spousal giving are fully exempt from the tax

E.g.: if Bill Gates / Warren Buffet give all their wealth to charity, they won't pay estate tax

Support for estate tax is pretty weak (``death tax'') but public does not know that estate tax affects
only richest

Support for estate tax increase shots up from 17\% to 53\% when survey respondents are informed that only richest pay it
(Kuziemko-Norton-Saez-Stantcheva '13 do an online Mturk survey experiment)


\end{slide}

\begin{slide}
\includepdf[pages={20}]{capitalincometax_attach.pdf}
\end{slide}



\begin{slide}
\begin{center} {\bf Taxation of Inheritances: Behavioral Responses} \end{center}

Potential behavioral response effects of inheritance tax:

(1) reduces wealth accumulation of altruistic donors (and hence
tax base) [no very good empirical evidence, Slemrod-Kopczuk 01]

(2) reduces labor supply of altruistic donors (less motivated to
work if cannot pass wealth to kids) [no good evidence]

(3) induces donees to work more through income effects (Carnegie
effect, decent evidence from Holtz-Eakin,Joulfaian,Rosen QJE'93)

Critical to understand why there are inheritances to decide on
optimal inheritance tax policy. 4 main models of bequests: (a)
accidental, (b) bequests in the utility, (c) manipulative bequest motive, (d)
dynastic
\end{slide}

\begin{slide}
\begin{center} {\bf ACCIDENTAL BEQUESTS} \end{center}

People die with a stock of wealth they intended to spend on
themselves: Such bequests arise because of imperfect annuity
markets

Annuity is an insurance contract converting lumpsum amount into a
stream of payments till end of life [insurance against risk of
living too long]

Annuity markets are imperfect because of adverse selection
[Finkelstein-Poterba EJ'02, JPE'04] or behavioral reasons
[inertia, lack of planning]

Public retirement programs [and defined benefit private
pensions] are in general mandatory annuities

Newer defined contribution private pensions [401(k)s in the US]
are in general not annuitized
\end{slide}

\begin{slide}
\begin{center} {\bf ACCIDENTAL BEQUESTS} \end{center}

Bequest taxation has no distortionary effect on behavior of donor
and can only increase labor supply of donees (through income
effects) $\Rightarrow$ strong case for taxing bequests heavily

Kopczuk JPE '03 makes the point that estate tax plays the role of
a ``second-best'' annuity:

Estate tax paid by those who die early, and not by those who die
late $\Rightarrow$ Implicit insurance against risk of living too
long

\textbf{Wealth loving:} Same tax policy conclusion arises if donors have wealth in their
utility function [social status or power, Carroll '98]

Kopczuk-Lupton REStud'07 shows that only 1/2 of people accumulate wealth for
bequest motives
\end{slide}

\begin{slide}
\begin{center} {\bf Bequests in the Utility: Warm Glow Or Altruistic} \end{center}

$u(c)-h(l)+ \delta v(b)$ where $c$ is own consumption, $l$ is
labor supply, and $b$ is net-of-tax bequests left to next
generation and $v(b)$ is warm glow utility of bequests

Budget with no estate tax: $c+b/(1+r) = wl-T_L(wl)$

Budget with bequest tax at rate $\tau_B$: $c+b/[(1+r)(1-\tau_B)] =
wl-T_L(wl)$

Suppose first that $b$ is not bequeathed but used for
``after-life'' consumption [e.g., funerary monument of no value to
next generation]

$\Rightarrow$ Atkinson-Stiglitz implies that $b$ should not be
taxed [$\tau_B=0$] and that nonlinear tax on $wl$ is enough for
redistribution

\end{slide}

\begin{slide}
\begin{center} {\bf Bequests in the Utility: Warm Glow Or Altruistic} \end{center}

Suppose now that $b$ is given to a heir who derives utility
$v^{heir}(b)$ $\Rightarrow$ $b$ creates a positive externality (to donee) and
hence should be subsidized $\Rightarrow$ $\tau_B<0$ is optimal

Kaplow '01 makes this point informally

Farhi-Werning QJE'10 develop formal model of non-linear Pigouvian
subsidization of bequests with 2 generations and social Welfare:
$$SWF=\int[u(c)-h(l)+\delta v(b) +v^{heir}(b)]f(w)dw$$
The marginal external effect of bequests is $dv^{heir}/db$ and
hence should be smaller for large $b$

$\Rightarrow$ Optimal subsidy rate is smaller for large estates
$\Rightarrow$ progressive estate subsidy
\end{slide}

%\begin{slide}
%\begin{center} {\bf WARM GLOW BEQUESTS: ISSUES} \end{center}
%
%(a) If past inheritances come from untaxed labor income, then it
%is desirable to tax inheritances [important when income tax
%starts]
%
%(b) Double counting issue: should social welfare double count
%bequests? [both for donor and donee]
%
%{\bf Yes} under utilitarian framework [Kaplow '01]
%
%{\bf No:} utilitarian framework with double counting generates
%predictions that conflict with horizontal equity:
%
%$\bullet$ Govt should tax less those well loved by other people
%
%$\bullet$ Govt should care more about kids with parents than
%orphans
%\end{slide}


\begin{slide}
\begin{center}
{\bf A-S Fails with Inheritances In General Equilibrium (Piketty-Saez ECMA'13)}
\end{center}

Atkinson-Stiglitz applies when sole source of lifetime income is labor:
\[ c+ b(left)/(1+r)=w l-T(w l) \quad (w = \text{productivity, }
 l = \text{labor supply}) \]

In GE, bequests provide an additional source of life-income:
\[c+b(left)/(1+r)=w l-T(w l)+b(received) \]

$\Rightarrow$ conditional on $w l$, high $b(left)$ is a signal of high $b(received)$
$\Rightarrow$ $b(left)$ should be taxed even with optimal $T(w l)$

$\Rightarrow$ Two-dim. inequality requires two-dim. tax policy tool

\textbf{Extreme example:} no heterogeneity in productivity $w$
but pure heterogeneity in bequests motives $\Rightarrow$
bequest taxation is desirable for redistribution

%Note: bequests generate positive externality on donors and hence should be taxed less (but still >0)

\end{slide}


\begin{slide}
\begin{center}
{\bf Piketty-Saez Simplified Optimal Inheritance Tax Model}
\end{center}
Measure one of individuals, who are both bequests receivers and bequest leavers (in ergodic
general equilibrium)

Linear tax $\tau_B$ on bequests funds lumsump grant $E$

Life-time budget constraint: $c_i+b_i = R (1-\tau_B) b^r_i + y_{Li} + E$

with $c_i$ consumption, $b_i$ bequests left, $y_{Li}$ inelastic labor income,
$b^r_i$ pre-tax bequests received,
$R=1+r$ generational rate of return on bequests

Individual $i$ has utility $V^i(c,\underline{b})$ with $\underline{b}=R(1-\tau_B)b$ net-of-tax bequests left
and solves
\[ \max_{b_i} V^{i}(y_{Li}+E + R (1-\tau_B) b^r_i -
b_{i} ,Rb_{i} (1-\tau_{B})) \Rightarrow V^i_c = R(1-\tau_B) V^i_{\underline{b}}  \]

\end{slide}

\begin{slide}
\begin{center}
{\bf Piketty-Saez ECMA'13 Optimal Inheritance Tax}
\end{center}
Government budget constraint is $E=\tau_B b$ with $b$ aggregate (=average) bequests.
Govt  solves:
\[ \max_{\tau_B} \int_i \omega_i V^{i}(y_{Li}+\tau_B b + R (1-\tau_B) b^r_i -
b_{i} ,Rb_{i} (1-\tau_{B})) \]
with $\omega_i \geq 0$ Pareto weights

Meritocratic Rawlsian criterion: maximize welfare of those receiving no inheritances
with uniform social marginal welfare weight $\omega_i V^i_c$ among zero-receivers

(e.g., people not responsible for $b_i^r$ but responsible for $y_{Li}$) $\Rightarrow$
\[ \textbf{Optimal inheritance tax rate: } \tau _{B}= \frac{1- \bar{b}} {1+e_B} \] 
with $e_B=\frac{1-\tau_B}{b}\frac{db}{d(1-\tau_B)}$ elasticity of aggregate bequests and   $\bar{b} =\frac{E[b_i | b_i^r=0]}{b}$ relative
bequest left by zero-receivers
\end{slide}


\begin{slide}
\begin{center}
{\bf Piketty-Saez ECMA'13 Optimal Inheritance Tax: Proof}
\end{center}
\[ SWF= \int_i \omega_i V^{i}(y_{Li}+\tau_B b -
b_{i} ,Rb_{i} (1-\tau_{B})) \]
[NB: removed term $R (1-\tau_B) b^r_i$ because $\omega_i=0$ when $b^r_i=0$]
\[ 0=\frac{dSWF}{d\tau_B} = \int_i \omega_i   \cdot \left ( V_c^{i} \left [ b - \tau_B \frac{db}{d(1-\tau_B)} \right ] - R b_i V^i_{\underline{b}}
\right )  \Rightarrow \]
\[ 0 = \int_i \omega_i  \cdot \left ( V_c^{i} \cdot b \left [ 1 - \frac{\tau_B}{1-\tau_B} e_B \right ] - \frac{b_i}{1-\tau_B} V^i_{c}
\right )  \Rightarrow \]
\[ 0 = b \left [ 1 - \frac{\tau_B}{1-\tau_B} e_B \right ] -
 \frac{1}{1-\tau_B} \cdot \frac{\int_i \omega_i V^i_{c} \cdot b_i} {\int_i  \omega_i V^i_{c}}
  \Rightarrow \] as $\omega_i V^i_c \equiv 0$ for $b_i^r>0$ and $\omega_i V^i_c\equiv 1$ for $b_i^r=0$ $ \Rightarrow$
\[ 0 = 1 -\tau_B - \tau_B \cdot e_B - \frac{E[b_i | b_i^r=0]}{b} \Rightarrow \tau _{B}= \frac{1- \bar{b}} {1+e_B}  \]
\end{slide}

\begin{slide}
\begin{center}
{\bf Piketty-Saez ECMA'13 Optimal Inheritance Tax}
\end{center}
\[ \textbf{Optimal inheritance tax rate: } \tau _{B}= \frac{1- \bar{b}} {1+e_B} \] 
1) Optimal $\tau_B < 1/(1+e_B)$ revenue maximizing rate because zero-receivers care about bequests they leave

2) $\tau_B=0$ if $\bar{b}=1$ (i.e, zero-receivers leave as much bequest as average)

3) If bequests are quantitatively important, highly concentrated, and low wealth mobility then $\bar{b}<<1$

4) Empirically $e_B$ small (Kopcuzk-Slemrod '01) but poorly known, $\bar{b} = 2/3$ in US (SCF data) but poorly
measured

5) Formula can be extended to other social criteria, elastic labor supply, wealth loving preferences, altruistic preferences
[see Piketty-Saez ECMA'13]

\end{slide}


\begin{slide}
\begin{center}
{\bf Optimal Capital Stock and Modified Golden Rule}
\end{center}
\textbf{Modified Golden Rule:} $r=\delta + \gamma \cdot g$ 
\\ with $r=F_K(K,L)=f'(k)$ rate of return,
$\delta$ discount rate, $\gamma$ curvature of $u'(c)=c^{-\gamma}$, $g$ growth rate (per capita).
%Not clear at all that MGR holds in practice.

Proof: \$1 extra in period $t$ gives social welfare $u'(c_t)$

\$$1+r$ extra in period $t+1$ gives social welfare $\frac{(1+r)u'(c_{t+1})}{1+\delta}=\frac{(1+r)u'(c_{t})}{1+\delta} \frac{u'(c_{t+1})}{u'(c_t)}
= \frac{1+r}{(1+\delta)(1+g)^{\gamma}} u'(c_t) $  $\Rightarrow$ $1+r=(1+\delta)(1+g)^{\gamma}$

This is equivalent to $r=\delta + \gamma \cdot g$ when the period is small. QED.

Normatively $\delta = 0$ seems justified. Small capital stock and $r>g$ desirable if $\gamma$ is high
[controversy Stern vs. Nordhaus]
\end{slide}

\begin{slide}
\begin{center}
{\bf Optimal Capital Stock and Modified Golden Rule}
\end{center}
\textbf{Modified Golden Rule:} $r=\delta + \gamma \cdot g$ 

Bequest and capital taxes affect capital stock

However, if govt can use debt, it can control capital  stock
 
If debt used to set optimal capital stock at the Modified Golden Rule) then
effects of taxes on $K$ stock can be ignored $\Rightarrow$
Optimal $K$ stock and optimal redistribution are \textbf{orthogonal}

If K stock is not at Modified Golden Rule, then optimal K tax formulas include
a corrective term

[see King 1980 and Atkinson-Sandmo 1980
in OLG life-cycle model, Piketty-Saez ECMA'13 for models with bequests]

In practice: no reason for MGR to hold, govts do not actively target K stock 
\end{slide}



\begin{slide}
\begin{center} {\bf MANIPULATIVE BEQUESTS} \end{center}

Parents use potential bequest to extract favors from children

Empirical Evidence: Bernheim-Shleifer-Summers JPE '85 show that
number of visits of children to parents is correlated with
bequeathable wealth but not annuitized wealth of parents

$\Rightarrow$ Bequest becomes one additional form of labor income
for donee and one consumption good for donor

$\Rightarrow$ Inheritances should be counted and taxed as labor
income for donees

\end{slide}

\begin{slide}
\begin{center} {\bf SOCIAL-FAMILY PRESSURE BEQUESTS} \end{center}

Parents may not want to leave bequests but feel compelled to by
pressure of heirs or society: bargaining between parents and
children

With estate tax, parents do not feel like they need to give as
much $\Rightarrow$ parents are made better-off by the estate tax
$\Rightarrow$ Case for estate taxation stronger [Atkinson-Stiglitz
does not apply and no double counting of bequests]

Empirical evidence:

Aura JpubE'05: reform of private pension annuities in the US in
1984 requiring both spouses signatures when retiring worker decides to get
a single annuity or couple annuity: reform $\uparrow$ sharply
couple annuities choice

Equal division of estates [Wilhelm AER'96, Light-McGarry '04]: estates are
very often divided equally but gifts are not
\end{slide}



\begin{slide}
\begin{center} {\bf DYNASTIC MODEL OR INFINITE HORIZON} \end{center}
Special case of warm glow: $V_t=u(c_t,l_t) + V_{t+1}/(1+\delta)$
implies $$V_0 = \sum_{t \geq 0}  u(c_t,l_t)/(1+\delta)^t$$ $$\text{subject to } \sum_{t \geq 0}
c_t/(1+r)^t \leq \sum_{t \geq 0} w_t l_t/(1+r)^t$$
Dynasty with \textbf{Ricardian equivalence:} consumption depends only on
PDV of earnings of dynasty

Poor empirical fit:

1) Altonji-Hayashi-Kotlikoff AER'92, JPE'97 show that income
shocks to parents have bigger effect on parents consumption than
on kids consumption (and conversely)

2) Temporary tax cut debt financed [fiscal stimulus] should have
no impact on consumption but actually do
\end{slide}

%\begin{slide}
%\begin{center} {\bf DYNASTIC MODEL OR INFINITE HORIZON} \end{center}
%Debate on what should the govt objective be:
%
%First generation $V_0$ or steady-state $V_t$ (for $t$ large)
%
%With uncertainty: $V_0$ objective leads to crazy asymptotic
%results: {\bf immiseration} results of Akteson-Lucas REStud '92
%
%$\Rightarrow$ Cross sectional inequality explodes and $EV_t
%\rightarrow -\infty$ (optimal to keep motivating future
%generations, and does not hurt too much from $V_0$ perspective)
%
%Violates sense that govt should not exacerbate inequality due to
%inter-generational linkage
%
%{\bf Steady state} $EV_t$ seems better objective but then the govt
%would want to push capital into the future (at the expense of past
%generations) [Phelan REStud'06, Farhi-Werning JPE'08 solve models
%with no capital with this objective]
%
%%No very good solution to this conundrum
%\end{slide}

\begin{slide}
\begin{center} {\bf INFINITE HORIZON MODEL: CHAMLEY-JUDD} \end{center}
Govt can collect taxes using linear
labor income tax or capital income taxes that vary period by period $\tau_L^t$, $\tau_K^t$

Goal of the government is to maximize utility of the dynasty
\[V_0 = \sum_t u(c_t,l_t)/(1+\delta)^t \text{ st } \quad \sum_t q_t c_t \leq
\sum_t q_t w_t(1-\tau_L^t) l_t+A_0 \quad (\lambda) \]
$q_0=1$, ...,
$q_t=1/\prod_{s=1}^t(1+r_s(1-\tau^s_K))$, ...

%FOC in $l_t$ and $c_t$ $\Rightarrow$ $w_t(1-\tau_L^t)
%u^t_c-u^t_l=0$, $u_c^{t+1}/u_c^t=(1+\delta)/(1+r(1-\tau^t_K))$

With constant tax rate $\tau_K$ and constant $r$: Before tax
price: $p_t=1/(1+r)^t$ and after-tax price
$q_t=1/(1+r(1-\tau_K))^t$ $\Rightarrow$

Price distortion $q_t/p_t$ grows exponentially with time
\end{slide}

%\begin{slide}
%\begin{center} {\bf INFINITE HORIZON MODEL: CHAMLEY-JUDD} \end{center}
%Let us assume representative agent with fixed labor supply and no
%labor income tax for notational simplicity [not relevant for key
%results]
%
%$V_0$ objective: govt choose path of $\tau^t_K$ (or equivalently
%$q_t$) that maximizes $V_0$ st govt budget constraint
%$$\sum_t (q_t-p_t)(c_t-w_t) \geq \sum_t p_t g_t$$ with $g_t$
%exogenous govt spending requirement
%\end{slide}

\begin{slide}
\begin{center} {\bf CHAMLEY-JUDD: RESULTS} \end{center}

Chamley-Judd show that the capital income tax rate always tends to
zero asymptotically: no capital tax in the long-run: 

Two equivalent ways to understand this result:

(1) A constant
tax on capital income creates an exponentially growing
distortion which is inefficient

(2) The PDV of the capital income tax base is infinitely elastic
with respect to ant increase in $\tau_K$ in the distant future 
[Piketty-Saez '13]

Intuition: $u_c(c_{t+1})/u_c(c_t)=(1+\delta)/(1+r(1-\tau_K)) \Rightarrow$
savings decisions infinitely elastic to $r(1-\tau_K) -\delta$

If $r (1-\tau_K) >\delta$, accumulate
forever. If $r (1-\tau_K) <\delta$, get in debt as much as possible.


\end{slide}


%\begin{slide}
%\begin{center} {\bf CHAMLEY-JUDD, INFINITE ELASTICITY} \end{center}
%
%Two classes: capitalists save as in infinite horizon model,
%workers do not save (consume wages $w$ with no labor supply
%effects) [Piketty '00 handbook chapter]
%
%Can capital tax at rate $\tau$ be desirable for workers in
%steady-state?
%
%$r=f'(k)$ and $w=f(k)-k f'(k)$, tax $\tau$, net return is
%$(1-\tau)f'(k)$
%
%Infinite horizon: modified Golden rule: discount rate $\delta =
%(1-\tau)f'(k)$ (if $>$, save more and $k$ increases, if $<$, save
%less and $k$ decreases).
%
%Workers get $w+f'(k) \tau k = f(k) - (1-\tau) k f'(k) = f(k) -
%\delta k$, maximized when $f'(k)=\delta$, ie $\tau=0$
%
%Intuition: Supply of $k$ is infinitely elastic: taxing an
%infinitely elastic good cannot be desirable [even in the steady
%state]
%\end{slide}
%
%\begin{slide}
%\begin{center} {\bf CHAMLEY-JUDD, $V_0$ objective} \end{center}
%
%It is possible to build a model with endogenous discount rate
%$\delta(c)$ so that elasticity of $k$ stock with respect to
%long-term return $r$ is finite
%
%Judd JpubE '85 shows that:
%
%If workers have the same discount rate as capitalists
%(asymptotically) then long-run zero capital income tax result
%carries over
%
%This is because the elasticity of the PDV of the tax base with respect 
%to long-distant tax change remains infinite [Piketty-Saez ECMA'13]
%
%% about inter-temporal distortions: a constant capital
%%income tax rate $\tau_K$ produces a growing distortion overtime
%%while Ramsey recommends to spread taxes across goods
%
%% and has nothing to do
%%with intra-generational redistribution
%\end{slide}

%\begin{slide}
%\begin{center} {\bf CHAMLEY-JUDD, $V_0$ objective} \end{center}
%Proof:
%
%Consider steady state (large $t$): everything has converged:
%$u_c^t=u_c$, $r_t=r$, $\tau^t_K=\tau_K$
%
%If $\tau_K>0$ then $r(1-\tau_K)=\delta$ $\Rightarrow$ $r> \delta$
%
%Suppose govt increases taxes by \$1 in period $t$ and decreases
%taxes by \$1+$r$ in period $t+1$
%
%(1) budget neutral for govt
%
%(2) Social welfare effect:
%$dSW=-u^t_c/(1+\delta)^t+(1+r)u_c^{t+1}/(1+\delta)^{t+1}=u_c/(1+\delta)^t
%\cdot [-1 + (1+r)/(1+\delta)]>0$
%
%$\Rightarrow$ Cannot be an optimum: intertemporal distortion
%cannot help in the long-run
%\end{slide}

\begin{slide}
\begin{center} {\bf ISSUES IN INFINITE HORIZON MODEL} \end{center}
1) Taxing initial wealth is most efficient [as this is
lumpsum taxation] $\Rightarrow$ solutions typically bang-bang: tax capital as much
as possible early, then zero

2) Chamley-Judd tax is not time consistent: the government would
like to renege and start taxing capital again

3) Zero-long run tax result is not robust to using progressive
income taxation [Piketty, '01, Saez JpubE'13]

4) Dynastic model requires strong homogeneity assumptions (in
discount rates) to generate reasonable steady states [unlikely to
hold in practice]

5) Introducing stochastic shocks in labor/preferences in dynastic
model leads to finite elasticities (and reasonable optimal tax rates)
[Piketty-Saez ECMA'13]

%{\bf Bottom line:} Not very useful model for thinking about
%capital income taxation
\end{slide}

%\begin{slide}
%\begin{center} {\bf PROGRESSIVE TAX IN $\infty$ HORIZON: PIKETTY '01} \end{center}
%Dynastic utilities with inelastic labor supply
%$$W = \sum_t u(c_t)/(1+\delta)^t$$
%$r_t=f'(k_t)$, $w_t=f(k_t)-r_t k_t$
%
%Distribution of wealth: $a_t$ with density $f_t(a_t)$ so that $k_t
%= \int a g_t(a)da$
%
%Golden rule capital stock $k^*$:  $f'(k^*)=\delta$
%
%With no taxes: In steady state: $f'(k_{\infty})=\delta$ (i.e
%$k_{\infty}=k^*$), any $g_{\infty}(a)$ possible as long as $k^* =
%\int a g_{\infty} (a)da$
%
%Proof: suppose $r_{\infty}=f'(k_{\infty})>\delta$ $\Rightarrow$
%$u'(c_{t+1})/u(c_{t})=(1+\delta)/(1+r_t)<1$ i.e., $c_{t+1}>c_{t}$
%$\Rightarrow$ Individuals want to shift consumption toward future
%$\Rightarrow$ Save more and accumulate capital indefinitely [not a
%steady state]
%\end{slide}
%
%
%
%\begin{slide}
%\begin{center} {\bf PROGRESSIVE TAX IN $\infty$ HORIZON: PIKETTY '01} \end{center}
%
%Suppose a progressive capital income tax is introduced: $\tau_K=0$
%when $a \leq \bar{k}$ and $\tau_K=\tau>0$ when $a \geq \bar{k}$
%
%Assume $\bar{k}>k^*$
%
%In the steady state:
%
%1) Golden rule capital stock: $k_{\infty}=\int a g_{\infty}
%(a)da=k^*$
%
%2) Truncated wealth distribution: $a\leq \bar{k}$ for all
%individuals
%
%Proof: In steady state, all individuals must face same net-of-tax
%rate $r_{\infty}(1-\tau_K)$ $\Rightarrow$ All individuals in same
%tax bracket $[0,\bar{k}]$ or $(\bar{k},\infty)$. But
%$(\bar{k},\infty)$ is impossible because $k_{\infty}=\int a g(a)da
%\geq  \bar{k} > k^*$ and hence $f'(k_{\infty})<\delta$
%
%\end{slide}
%
%\begin{slide}
%\begin{center} {\bf PROGRESSIVE TAX IN $\infty$ HORIZON: SAEZ '13} \end{center}
%Piketty '01 shows that progressive capital income tax with
%exemption up to $k^*$ equalizes wealth without affecting long-run
%capital stock
%
%Seems desirable from steady-state perspective
%
%Saez '13 shows that such progressive taxation is desirable from
%period 0 perspective if
%
%$a \cdot \sigma <1$ where $a$ is Pareto parameter of initial
%wealth distribution and $\sigma$ is inter-temporal elasticity of
%substitution $u(c_t)=[c_t^{1-1/\sigma}-1]/[1-1/\sigma]$
%
%Long-run wealth distribution will then be truncated
%\end{slide}

%\begin{slide}
%\begin{center} {\bf PROGRESS IN OPTIMAL CAPITAL INCOME THEORY} \end{center}
%
%Research plan of Piketty and Saez '12
%
%1) Develop tax formulas that are based on sufficient statistics
%that can be estimated empirically [behavioral responses and
%distributive factors]
%
%2) Formulas should be robust to heterogeneity in preferences
%[accidental, warm glow, altruistic]
%
%3) Predictions from theory should be somewhat aligned to actual
%practice [taxing only earnings and not at all capital does not fit
%with actual practice]
%\end{slide}
%
%\begin{slide}
%\begin{center} {\bf PIKETTY-SAEZ '12 SIMPLIFIED MODEL} \end{center}
%Agent $i$ in cohort $t$ (1 cohort =1 period =H years)
%
%Receives bequest $b_{ti}=z_{it} b_t$ at beginning of period $t$ where $b_t$ average
%bequest and $z_{i}$ (normalized) bequest received
%
%At the end of period $t$, individual receives (inelastic)
%labor income $y_{Lti}=\theta_{it} y_{Lt}$, consumes $c_{ti}$,
%and leaves bequest $b_{t+1i}$ to unique child so as to maximize:
%\[U_{it}=(1-s_{it}) \log c_{ti} + s_{it} \log [(1-\tau_B)e^{rH}b_{t+1 i}] \quad \mathrm{s.c.}\]
%\[ c_{ti} + b_{t+1 i} \leq
% (1-\tau_B)b_{t}z_{it} e^{rH} +(1-\tau_L)y_{Lt}\theta_i \]
%$\tau_B=$ bequest tax rate, $\tau_L=$ labor income tax rate
%\[\Rightarrow b_{t+1i}=s_{ti} \cdot [ (1-\tau_B)b_{t}z_{it} e^{rH} +(1-\tau_L)y_{Lt}\theta_{it} ] \]
%
%$y_{Lt}=y_{L0}e^{gHt}$ (exogenous growth at rate $g$)
%\end{slide}
%
%\begin{slide}
%\begin{center}
%{\bf INHERITANCE FLOWS}
%\end{center}
%Assume $(s_{it},\theta_{it})$ iid within and across cohorts
%\[b_{t+1i}=s_{ti} \cdot [ (1-\tau_B)b_{t}z_{it} e^{rH} +(1-\tau_L)y_{Lt}\theta_{it} ] \]
%Aggregates to:
%\[b_{t+1}=s \cdot [ (1-\tau_B)b_{t}e^{rH} +(1-\tau_L)y_{Lt} ] \]
%Steady-state convergence $b_{t+1}=b_t e^{gH}$:
%\[\Rightarrow b_{yL}=\frac{b_t e^{rH}}{y_{Lt}} = \frac{s(1-\tau_L)e^{(r-g)H}}{1-s(1-\tau_B)e^{(r-g)H}} \]
%$b_{yL}$ increases with $r-g$ (capitalization effect, Piketty QJE'11)
%
%$r-g=3\%, \tau_L=\tau_B=10\%, H=30, s=10\% \Rightarrow b_{yL}=28\%$
%
%$r-g=1\%, \tau_L=\tau_B=30\%, H=30, s=10\% \Rightarrow b_{yL}=10\%$
%
%Shocks $(s_{it},\theta_{it})$ generate steady-state distribution $(z_{it},\theta_{it})$
%\end{slide}
%
%
%\begin{slide}
%\begin{center}
%{\bf MODEL: GOVERNMENT OBJECTIVE}
%\end{center}
%Govt chooses $\tau_B, \tau_L$ to maximize \textbf{steady-state} social welfare
%\[SWF = \int \omega^i U^i d\Psi(z) dF(\theta) \]
%with $\Psi(z)$ cdf of (normalized) inheritance $z$ and $F(\theta)$ cdf of
%labor productivity $\theta$, $\omega^i=$ welfare weights concentrated on those with zero
%inheritance $(z_{it}=0)$ (meritocratic).
%
%subject to budget balance constraint
%\[ \tau _{L} y_{Lt}+\tau _{B}b_{t}e^{rH}=\tau y_t \quad \mathrm{with} \quad \tau \quad  \mathrm{given} \]
%Consider small $d\tau_B>0$, can cut $d\tau_L<0$ by:
%\[-y_{Lt} d\tau_L = d\tau_B b_t e^{rH} \left ( 1 - e_B \frac{\tau_B}{1-\tau_B} \right ) \]
%$e_B =$ long-run elasticity of $b_{yL}$ with respect to $1-\tau_B$
%\[ e_B =\frac{1-\tau _{B}}{b_{yL}} \frac{db_{yL}}{d(1-\tau _{B})} \]
%
%\end{slide}
%
%\begin{slide}
%\begin{center}
%{\bf OPTIMAL INHERITANCE TAX RATE}
%\end{center}
%\textbf{Meritocratic Rawlsian Optimum:} maximize welfare of those receiving no inheritance
%\[  \tau_B=\frac{1+s - e^{-(r-g)H}} {1+s+e_B} \]
%$\tau_B \downarrow$ with $e_B$. Even if $e_B=0$, we have $\tau_B<1$. $e_B=\infty \Rightarrow
%\tau_B=0$
%%(trade-off between taxing  inheritors from my cohort vs. my children)
%
%$\tau_B \uparrow$ with $r-g$: Taxing bequests raises $\tau_B b_t e^{rH}$ from
%inheritors in my cohort
%but costs $\tau_B b_{t+1}=\tau_B b_t e^{gH}$ to what I leave to my child
%
%If $r-g=3\%, H=30, s=10\%, e_B=.25 \Rightarrow \tau_B=51\%$
%
%If $r-g=1\%, H=30, s=10\%, e_B=.25 \Rightarrow \tau_B=27\%$
%\end{slide}
%
%\begin{slide}
%\begin{center}
%{\bf OPTIMAL INHERITANCE TAX RATE: PROOF}
%\end{center}
%Preferred tax $(\tau_L,\tau_B)$ for zero inheritor with savings taste $s_{ti}$:
%\[U_{it}=(1-s_{ti}) \log c_{ti} + s_{ti} \log[b_{t+1i}e^{rH}(1-\tau_B)]\]
%\[c_{ti} = (1-s_{ti}) y_{Lt}\theta_{ti} (1-\tau_L) \quad \mathrm{and} \quad b_{t+1i} = s_{ti} y_{Lt}\theta_{ti} (1-\tau_L) \]
%\[U_{it}=\log(1-\tau_L) +s_{ti}\log(1-\tau_B) + constant \]
%\[\mathrm{FOC} \quad \frac{d\tau_L}{1-\tau_L} + s_{ti} \frac{d\tau_B}{1-\tau_B}=0 \]
%Govt budget $d\tau_L = -d\tau_B b_{yL} \left ( 1 - e_B \tau_B/(1-\tau_B) \right )$ implies:
%\[ \frac{d\tau_L}{1-\tau_L} =  -
%\frac{d\tau_B}{1-\tau_B}  \cdot  \frac{s e^{(r-g)H}}{1-s(1-\tau_B)e^{(r-g)H}} \cdot [ 1 -\tau_B (e_B+1)]  \]
%Using FOC, we obtain preferred $\tau_B$ of $s_{ti}$ person
%\[ \Rightarrow  \tau_B=\frac{1+s_{ti} - (s_{ti}/s) e^{-(r-g)H}} {1+s_{ti}+e_B} \] 
%\end{slide}

\begin{slide}
\begin{center} {\bf NEW DYNAMIC PUBLIC FINANCE: REFERENCES} \end{center}
Dynamic taxation in the presence of future earnings uncertainty 

Recent series of papers following upon on Golosov, Kocherlakota,
Tsyvinski REStud '03 (GKT)

Principle can be understood in 2 period model: Diamond-Mirrlees
JpubE '78 and Cremer-Gahvari EJ '95

Generalized to many periods by GKT and subsequent papers

Simple exposition is Kocherlakota AER-PP '04

Two comprehensive surveys: Golosov-Tsyvinski-Werning '06 and
Kocherlakota '10 book
\end{slide}


\begin{slide}
\begin{center} {\bf NEW DYNAMIC PUBLIC FINANCE (NDPF)} \end{center}

Key ingredient is uncertainty in future ability $w$

2 period simple model:

(0) Everybody is identical in period 0: no work and consume $c_0$,
period 0 utility is $u(c_0)$

(1) Ability $w$ revealed in period 1, work $l$ and earn $z=wl$,
consume $c_1$, period 1 utility $u(c_1)-h(l)$

Total utility $u(c_0)+ \beta [ u(c_1) -h(l)]$

Rate of return $r$, gross return $R=1+r$

Discount rate $\beta<1$

\end{slide}


\begin{slide}
\begin{center} {\bf STANDARD EULER EQUATION} \end{center}
No govt intervention: $c_0+c_1/R = wl/R$

Solve model backward (assume $c_0$ given):

Period 1: $c_1=wl - R c_0$, choose $l$ to maximize
$u(wl-Rc_0)-h(l)$

$\Rightarrow$ FOC $w u'(wl-Rc_0)=h'(l)$ $\Rightarrow$
$l^*=l(w,c_0)$

Period 0: Choose $c_0$ to maximize: 
\[u(c_0)+ \beta \int [ u(w
l^*-Rc_0) -h(l^*)]f(w)dw \]

FOC for $c_0$ (using envelope condition for $l^*$) $$u'(c_0) =
\beta R \int u'(c_1)f(w)dw$$ This is called the {\bf Euler equation}

\end{slide}

\begin{slide}
\begin{center} {\bf MECHANISM DESIGN} \end{center}

Government would like to redistribute from high $w$ to low $w$.
Government does not observe $w$ but can observe $c_0,c_1,z=wl$ and
can set taxes as a function of $c_0,c_1,z$

Equivalently (using revelation principle), govt can offer menu
$(c_0,c_1(w),z(w))_w$ and let individuals truthfully reveal their
$w$

Govt program: choose menu $(c_0,c_1(w),z(w))_w$ to maximize:
\[ SW=u(c_0) + \beta \int [u(c_1(w))-h(z(w)/w)]f(w)dw \text{ st} \]
1) Budget: $c_0 + \int c_1(w)f(w)dw/R \leq \int z(w)f(w)dw/R$

2) Incentive Compatibility (IC): individual $w$ prefers
$c_0,c_1(w),z(w)$ to any other $c_0,c_1(w'),z(w')$

\end{slide}



\begin{slide}
\begin{center} {\bf INVERSE EULER EQUATION} \end{center}
\textbf{Inverse Euler equation} holds at the govt optimum:
$$\frac{1}{u'(c_0)} = \frac{1}{\beta R} \cdot \int \frac{1}{u'(c_1(w))}
f(w)dw$$ \textbf{Proof:} small deviation in menus offered: $\Delta c_0 =
-\varepsilon/u'(c_0)$ and $\Delta c_1(w) = \varepsilon/ [\beta
u'(c_1(w))]$ with small $\varepsilon>0$

Does not affect individual utilities in any state:
\[ u(c_0+\Delta c_0) + \beta u(c_1(w)+\Delta c_1(w))=u(c_0) + \beta
u(c_1(w))\] \[ + \Delta c_0 u'(c_0) + \Delta c_1(w) \beta u'(c_1(w)) =
u(c_0) + \beta u(c_1(w))\]

%$u(c_0) + \beta u(c_1(w)) - \varepsilon + \varepsilon =u(c_0) +
%\beta u(c_1(w))$

$\Rightarrow$ (IC) continues to hold and $SW$ unchanged

Deviation must be budget neutral at optimum
$$\Rightarrow -\frac{\varepsilon}{u'(c_0)}+\frac{1}{R} \int
 \frac{\varepsilon f(w)dw}{\beta u'(c_1(w))} =0 $$
\end{slide}


\begin{slide}
\begin{center} {\bf INTERTEMPORAL WEDGE} \end{center}
Jensen Inequality:   for $K(.)$ convex $$\Rightarrow K \left (\int x(w) dF(w) \right ) < \int K(x(w))
dF(w)$$

Apply this to $K(x)=1/x$ and $x(w)=u'(c_1(w))$ $\Rightarrow$
$$\frac{1}{\int u'(c_1(w)) f(w)dw}< \int \frac{f(w)dw}{u'(c_1(w))}=
\frac{\beta R}{u'(c_0)} $$ $$\Rightarrow u'(c_0) < \beta R \int
u'(c_1(w)) f(w)dw$$ $\Rightarrow$ Optimal govt redistribution
imposes a positive tax wedge on intertemporal choice

\end{slide}


\begin{slide}
\begin{center} {\bf NDPF DECENTRALIZATION AND INTUITION} \end{center}

{\bf Decentralization:} Optimum can be decentralized with a tax on
capital income [which depends on current labor income] along with
a nonlinear tax on wage income [Kocherlakota EMA'06]

{\bf Economic intuition:} If high skill person works less (to
imitate lower skill person), person would also like to reduce
$c_0$ and hence save more, so tax on savings is a good way to
discourage imitation

Result depends crucially on rationality in inter-temporal choices + income effects
on labor: not clear yet how applicable this is in practice

Would be valuable to explore empirically for example whether DI (disability insurance) 
cheaters were saving more than non cheaters [would require merging SSA data and
tax/wealth data, hard to do]

\end{slide}

\begin{slide}
\begin{center} {\bf NDPF NUMERICAL SIMULATIONS} \end{center}

Farhi-Werning '11 propose numerical calibration and show that, for
realistic parameters, the welfare gain of using full nonlinear optimal capital/labor taxation
is very small ($0.1\%$ in aggregate welfare) relative to using only optimal labor taxation

Golosov-Troshkin-Tsyvinski '11 also find on average small welfare gains and small
optimal capital tax rates

$\Rightarrow$ Suggests that the mechanism is not quantitatively important even assuming
the theory is right 

$\Rightarrow$ Policy relevance of the NDPF for capital taxation likely to be limited 

DI/retirement application of NDPF might be quantitatively more important


\end{slide}

\begin{slide}
\begin{center}
{\bf REFERENCES}
\end{center}
{\small

Altonji, J., F. Hayashi, and L. Kotlikoff ``Is the Extended Family Altruistically Linked? Direct Tests Using Micro Data'', American Economic Review, Vol. 82, 1992, 1177-98. \href{http://links.jstor.org/stable/pdfplus/2117473.pdf} {(web)}

Altonji, J., F. Hayashi and L. Kotlikoff ``Parental Altruism and Inter Vivos Transfers: Theory and Evidence'', Journal of Political Economy, Vol. 105, 1997, 1121-66. \href{http://links.jstor.org/stable/pdfplus/2138891.pdf} {(web)}

Atkinson, A.B. and A. Sandmo ``Welfare Implications of the Taxation of Savings'', Economic Journal, Vol. 90, 1980, 529-49. \href{http://links.jstor.org/stable/pdfplus/2231925.pdf} {(web)}

Atkinson, A.B. and J. Stiglitz ``The design of tax structure: Direct versus indirect taxation'', Journal of Public Economics, Vol. 6, 1976, 55-75. \href{http://elsa.berkeley.edu/~saez/course/AtkinsonStiglitz_JPubE(1976).pdf} {(web)}

Atkinson, A.B. and J. Stiglitz Lectures on Public Economics, Chap 14-4 New York: McGraw Hill, 1980. \href{http://elsa.berkeley.edu/~saez/course/Atkinson,Stiglitz(1980)book_Chapter14_4.pdf} {(web)}

Auerbach, A. and L. Kotlikoff ``Evaluating Fiscal Policy with a Dynamic Simulation Model'', American Economic Review, May 1987, 49�55. \href{http://links.jstor.org/stable/pdfplus/1805428.pdf} {(web)}

Aura, S. ``Does the Balance of Power Within a Family Matter? The Case of the Retirement Equity Act'', Journal of Public Economics, Vol. 89, 2005, 1699-1717. \href{http://elsa.berkeley.edu/~saez/course/Aura_JPubE(2005).pdf} {(web)}

Banks J. and P. Diamond ``The Base for Direct Taxation'', IFS Working Paper, The Mirrlees Review: Reforming the Tax System for the 21st Century, Oxford University Press, 2009. \href{http://elsa.berkeley.edu/~saez/course/Banks and Diamond_IFS(2009).pdf} {(web)}

Bernheim, B. D., A. Shleifer, and L. Summers ``The Strategic Bequest Motive'', Journal of Political Economy, Vol. 93, 1985, 1045-76. \href{http://links.jstor.org/stable/pdfplus/1833175.pdf} {(web)}

Carroll, C. ``Why Do the Rich Save So Much?'', NBER Working Paper No. 6549, 1998. \href{http://www.nber.org/papers/w6549} {(web)}

Chamley, C. ``Optimal Taxation of Capital Income in General Equilibrium with Infinite Lives'', Econometrica, Vol. 54, 1986, 607-622. \href{http://links.jstor.org/stable/pdfplus/1911310.pdf} {(web)}

Christiansen, Vidar and Matti Tuomala ``On taxing capital income with income shifting'',
International Tax and Public Finance, Vol. 15, 2008, 527-545.
\href{http://elsa.berkeley.edu/~saez/course/christiansen-tuomalaITAX08capitalincomeshifting.pdf} {(web)}

Cremer, H. and F. Gahvari ``Uncertainty and Optimal Taxation: In defense of Commodity Taxes'', Journal of Public Economics, Vol. 56, 1995, 291-310. \href{http://elsa.berkeley.edu/~saez/course/Cremer and Ghavari_JPubE(1995).pdf} {(web)}

Davies, J. and A. Shorrocks, Chapter 11 The distribution of wealth, In: Anthony B. Atkinson and Francois Bourguignon, Editor(s), Handbook of Income Distribution, Elsevier, 2000, Vol. 1, 605-675. \href{http://elsa.berkeley.edu/~saez/course/Davies,Shorrocks(2000).pdf} {(web)}

DeLong, J.B. ``Bequests: An Historical Perspective,'' in A. Munnell, ed., \emph{The Role and Impact
of Gifts and Estates}, Brookings Institution, 2003 \href{http://elsa.berkeley.edu/~saez/course/DeLong(2003)_MunnellBook.pdf} {(web)}

Diamond, P. and J. Mirrlees ``A Model of Social Insurance with Variable Retirement'', Journal of Public Economics, Vol. 10, 1978, 295-336. \href{http://elsa.berkeley.edu/~saez/course/Diamond and Mirrlees_JPubE(1978).pdf} {(web)}

\textbf{Diamond, Peter and Emmanuel Saez ``The Case for a Progressive Tax: From Basic Research to Policy Recommendations'', Journal of Economic Perspectives, 25(4), Fall 2011, 165-190.
\href{http://elsa.berkeley.edu/~saez/diamond-saezJEP11full.pdf} {(web)} }

Diamond, P. and J. Spinnewijn ``Capital Income Taxes with Heterogeneous Discount Rates'', NBER Working Paper, No. 15115, 2009. \href{http://www.nber.org/papers/w15115.pdf} {(web)}

Farhi E. and I. Werning ``Progressive Estate Taxation'', Quarterly Journal of Economics, Vol. 125, 2010, 635-673. \href{http://elsa.berkeley.edu/~saez/course/Fahri and Werning_QJE(2010).pdf} {(web)}

Farhi E. and I. Werning ``Capital Taxation: Quantitative Explorations
of the Inverse Euler Equation,'' forthcoming \emph{Journal of Political Economy} 2011.
\href{http://elsa.berkeley.edu/~saez/course/FahriWerningJPE11.pdf} {(web)}

Feldstein, M. ``The Welfare Cost of Capital Income Taxation'', Journal of Political Economy, Vol. 86, 1978, 29-52. \href{http://links.jstor.org/stable/pdfplus/1829755.pdf} {(web)}

Finkelstein A. and J. Poterba, ``Adverse Selection in Insurance Markets: Policyholder Evidence from the U.K. Annuity Market'', Journal of Political Economy, Vol. 112, 2004, 183-208. \href{http://links.jstor.org/stable/pdfplus/3555197.pdf} {(web)}

Finkelstein A. and J. Poterba, ``Selection Effects in the United Kingdom Individual Annuities Market'', The Economic Journal, Vol. 112, 2002, 28-50. \href{http://www.jstor.org/stable/pdfplus/798430.pdf} {(web})

Gale, William G. and John Karl Scholz, ``Intergenerational
Transfers and the Accumulation of Wealth'', Journal of Economic Perspectives, Vol. 8(4), 1994
145-160. \href{http://www.jstor.org/stable/pdfplus/2138343.pdf} {(web)} 

Golosov, M., N. Kocherlakota and A. Tsyvinski ``Optimal Indirect and Capital Taxation'', Review of Economic Studies, Vol. 70, 2003, 569-587. \href{http://links.jstor.org/stable/pdfplus/3648601.pdf} {(web)}

Golosov, M. and A. Tsyvinski ``Designing Optimal Disability Insurance: A Case for Asset Testing'', Journal of Political Economy, Vol. 114, 2006, 257-279. \href{http://www.journals.uchicago.edu/doi/abs/10.1086/500549} {(web)}

Golosov, Mikhail, Maxim Troshkin, and Aleh Tsyvinski 2011. ``Optimal Dynamic Taxes.'' 
Princeton Working Paper
\href{http://elsa.berkeley.edu/~saez/course/Golosovetal11.pdf} {(web)}

Golosov, M., M. Troshkin, A. Tsyvinski and M. Weinzierl ``Preference Heterogeneity and Optimal Commodity Taxation'', MIMEO Yale, July 2011. \href{http://elsa.berkeley.edu/~saez/course/Golosovetal(2011).pdf} {(web)}

Golosov, M., A. Tsyvinski and I. Werning ``New Dynamic Public Finance: a User's Guide'' NBER Macro Annual 2006. \href{http://www.nber.org/chapters/c11181.pdf} {(web)}

Gordon, R.H. and J. Slemrod ``Are ``Real'' Responses to Taxes Simply Income Shifting Between Corporate and Personal Tax Bases?,'' NBER Working Paper, No. 6576, 1998. \href{http://www.nber.org/papers/w6576} {(web)}

Holtz-Eakin, D., D. Joulfaian and H.S. Rosen ``The Carnegie Conjecture: Some Empirical Evidence'', Quarterly Journal of Economics Vol. 108, May 1993, pp.288-307. \href{http://links.jstor.org/stable/pdfplus/2118337.pdf} {(web)}

Judd, K. ``Redistributive Taxation in a Simple Perfect Foresight Model'', Journal of Public Economics, Vol. 28, 1985, 59-83. \href{http://elsa.berkeley.edu/~saez/course/Judd_JPubE(1985).pdf} {(web)}

Kaplow, L. ``A Framework for Assessing Estate and Gift Taxation'', in
Gale, William G., James R. Hines Jr., and Joel Slemrod (eds.) {\em Rethinking estate and gift taxation} Washington, D.C.: Brookings Institution Press, 2001.
\href{http://www.nber.org/papers/w7775.pdf} {(web)}

Kaplow, L. ``On the undesirability of commodity taxation even when income taxation is not optimal'', Journal of Public Economics, Vol.90, 2006, 1235-1260. \href{http://elsa.berkeley.edu/~saez/course/Kaplow_JPubE(2006).pdf} {(web)}

Kennickell, A. ``Ponds and streams: wealth and income in the U.S., 1989 to 2007'', Board of Governors of the Federal Reserve System (U.S.), Finance and Economics Discussion Series: 2009-13, 2009. \href{http://elsa.berkeley.edu/~saez/course/Kennickell(2009).pdf} {(web)}

King, M. ``Savings and Taxation'', in G. Hughes and G. Heal, eds., Public Policy and the Tax System (London: George Allen Unwin, 1980), 1-36. \href{http://elsa.berkeley.edu/~saez/course/King(1980).pdf} {(web)} 

\textbf{Kocherlakota, N. ``Wedges and Taxes'', American Economic Review, Vol. 94, 2004, 109-113.  \href{http://links.jstor.org/stable/pdf/3592866.pdf} {(web)} }

Kocherlakota, N. ``Zero Expected Wealth Taxes: A Mirrlees Approach to Dynamic Optimal Taxation'', Econometrica, Vol.73, 2005. \href{http://links.jstor.org/stable/pdfplus/3598884.pdf} {(web)}

Kocherlakota, N. \emph{New Dynamic Public Finance}, Princeton University Press: Princeton, 2010.

Kopczuk, W. ``The Trick Is to Live: Is the Estate Tax Social Security for the Rich?'', The Journal of Political Economy, Vol. 111, 2003, 1318-1341. \href{http://www.jstor.org/stable/pdfplus/10.1086/378529.pdf} {(web)}

Kopczuk, Wojciech ``Taxation of Intergenerational Transfers and Wealth'', 
in A. Auerbach, R. Chetty, M. Feldstein, and E. Saez (eds.), Handbook of Public Economics, Vol. 5 (Amsterdam: North-Holland, 2013). \href{http://www.nber.org/papers/w18584.pdf} {(web)}

Kopczuk, Wojciech and Joseph Lupton 2007. ``To Leave or Not to Leave: The Distribution
of Bequest Motives,'' Review of Economic Studies, 74(1), 207-235.
\href{http://www.jstor.org/stable/pdfplus/4123242.pdf} {(web)}

Kopczuk, Wojciech and Emmanuel Saez ``Top Wealth Shares in the United States, 1916-2000: Evidence from Estate Tax Returns'', National Tax Journal, 57(2), Part 2, June 2004, 445-487. 
\href{http://elsa.berkeley.edu/~saez/course/kopczuksaez04.pdf} {(web)} 

Kopczuk, Wojciech and Joel Slemrod, ``The Impact of the Estate Tax on the Wealth Accumulation and Avoidance Behavior of Donors'', in William G. Gale, James R. Hines Jr., and Joel B. Slemrod (eds.), \emph{Rethinking Estate and Gift Taxation,} Washington, DC: Brookings Institution Press, 2001, 299-343.
\href{http://www.nber.org/papers/w7960.pdf} {(web)}

Kotlikoff, L. ``Intergenerational Transfers and Savings'', Journal of Economic Perspectives, Vol. 2, 1988, 41-58. \href{http://links.jstor.org/stable/pdfplus/1942848.pdf} {(web)}

Kotlikoff, L. and L. Summers ``The Role of Intergenerational Transfers in Aggregate Capital Accumulation'', Journal of Political Economy, Vol. 89, 1981, 706-732. \href{http://links.jstor.org/stable/pdfplus/1833031.pdf} {(web)}

Kuziemko, Ilyana, Michael I. Norton, Emmanuel Saez, and Stefanie Stantcheva ``How Elastic are Preferences for Redistribution? Evidence from Randomized Survey Experiments,'' NBER Working Paper No. 18865, 2013.
\href{http://www.nber.org/papers/w18865.pdf} {(web)}

Laroque, G. ``Indirect Taxation is Superfluous under Separability and Taste Homogeneity: A Simple Proof'', Economic Letters, Vol. 87, 2005, 141-144. \href{http://elsa.berkeley.edu/~saez/course/Laroque(2005).pdf} {(web)}

Light, Audrey and Kathleen McGarry. ``Why Parents Play Favorites: Explanations For Unequal Bequests,'' American Economic Review, 2004, v94(5,Dec), 1669-1681.
\href{http://www.jstor.org/stable/pdfplus/3592839.pdf} {(web)}

Modigliani, F. ``The Role of Intergenerational Transfers and Lifecyle Savings in the Accumulation of Wealth'', Journal of Economic Perspectives, Vol. 2, 1988, 15-40. \href{http://links.jstor.org/stable/pdfplus/1942847.pdf} {(web)}

Norton, M. and D. Ariely ``Building a Better America--One Wealth Quintile at a Time'',
Perspectives on Psychological Science 2011 6(9).
\href{http://elsa.berkeley.edu/~saez/course/norton-ariely11.pdf} {(web)}

Park, N. ``Steady-state solutions of optimal tax mixes in an overlapping-generations model'', Journal of Public Economics, Vol. 46, 1991, 227-246. \href{http://elsa.berkeley.edu/~saez/course/Park_JPubE1991.pdf} {(web)}

Piketty, Thomas 2000. ``Theories of Persistent Inequality
and Intergenerational Mobility'', In Atkinson A.B., Bourguignon F., eds. 
\emph{Handbook of Income Distribution}, (North-Holland). 
\href{http://elsa.berkeley.edu/~saez/course/PikettyHID00.pdf} {(web)}

Piketty, T. ``Income Inequality in France, 1901-1998'', CEPR Discussion Paper No. 2876, 2001 (appendix with
optimal capital tax point). \href{http://elsa.berkeley.edu/~saez/course/PikettyCEPR01.pdf} {(web)}

Piketty, T. ``Income Inequality in France, 1901-1998'', Journal of Political Economy, Vol. 111, 2003, 1004-1042. \href{http://links.jstor.org/stable/pdfplus/3555130.pdf} {(web)}

Piketty, T. ``On the Long-Run Evolution of Inheritance: France 1820-2050'', Quarterly Journal of
Economics, 126(3), 2011, 1071-1131. \href{http://elsa.berkeley.edu/~saez/course/PikettyQJE11.pdf} {(web)}


Piketty, Thomas, \emph{Capital in the 21st Century},  Cambridge: Harvard University Press, 2014,  
\href{http://piketty.pse.ens.fr/en/capital21c2}{(web)}

Piketty, Thomas, Gilles Postel-Vinay and Jean-Laurent Rosenthal, ``Inherited vs.
Self-Made Wealth: Theory and Evidence from a Rentier Society (1872-1927),''  Explorations in
Economic History, 2014.
\href{http://elsa.berkeley.edu/~saez/course/ppvr13.pdf} {(web)}

Piketty, T. and E. Saez ``Income Inequality in the United States, 1913-1998'', Quarterly Journal of Economics, Vol. 118, 2003, 1-39. \href{http://links.jstor.org/stable/pdfplus/25053897.pdf} {(web)}

Piketty, T. and E. Saez ``A Theory of Optimal Capital Taxation'', NBER Working Paper No. 17989, 2012. 
\href{http://www.nber.org/papers/w17989.pdf} {(web)}

\textbf{Piketty, T. and E. Saez ``A Theory of Optimal Inheritance Taxation'', Econometrica, 81(5), 2013, 1851-1886.
\href{http://elsa.berkeley.edu/users/saez/piketty-saezECMA13.pdf} {(web)} }

Piketty, T. and G. Zucman ``Capital is Back: Wealth-Income Ratios in Rich Countries, 1700-2010'', Quarterly Journal of Economics,  2014 \href{http://elsa.berkeley.edu/~saez/course131/Piketty-Zucman13.pdf}{(web)}

Piketty, T. and G. Zucman ``Wealth and Inheritance in the Long-Run'', Handbook of Income Distribution, Volume 2, Elsevier-North Holland, 2014
\href{http://elsa.berkeley.edu/~saez/course/PikettyZucman2014HIDRevised.pdf} {(web)}


Pirttila, Jukka and Hakan Selin, ``Income shifting within a dual income tax system: evidence from the Finnish tax reform,'' Scandinavian Journal of Economics, 113(1), 120-144, 2011.
\href{http://elsa.berkeley.edu/~saez/course/pirttila-selinSJE11.pdf} {(web)}

Saez, E. ``Optimal Capital Income Taxes in the Infinite Horizon Model'', Journal of Public
Economics, 97(1), 2013, 61-74. \href{http://elsa.berkeley.edu/~saez/saezJpubE12optKtax.pdf} {(web)}

Saez, E. ``The Desirability of Commodity Taxation under Nonlinear Income Taxation and Heterogeneous Tastes'', Journal of Public Economics, Vol. 83, 2002, 217-230. \href{http://elsa.berkeley.edu/~saez/course/Saez_JPubE(2002).pdf} {(web)}

Saez, E. and G. Zucman ``The Distribution of US Wealth, Capital Income and Returns since 1913'', Working Paper 2014 (in progress), preliminary slides
\href{http://elsa.berkeley.edu/~saez/course/SaezZucman2014Slides.pdf} {(web)}


Sandmo, A. ``The Effects of Taxation on Savings and Risk-Taking'', in A. Auerbach  and M. Feldstein, Handbook of Public Economics vol. 1, chapter 5, 1985, 265-311. \href{http://elsa.berkeley.edu/~saez/course/Sandmo_Handbook.pdf} {(web)}

Scholz, J. ``Wealth Inequality and the Wealth of Cohorts'', University of Wisconsin, mimeo,2003. \href{http://elsa.berkeley.edu/~saez/course/Scholz(2003).pdf} {(web)}

Slemrod, J. and J. Bakija, Taxing Ourselves: A Citizen's Guide to the Debate over Taxes. The MIT Press, 2004.

Wilhelm, Mark O.  ``Bequest Behavior and the Effect of Heirs' Earnings: Testing the Altruistic Model of Bequests,''
American Economic Review, 86(4), 1996, 874-892. \href{http://www.jstor.org/stable/pdfplus/2118309.pdf} {(web)}

Zucman, G. ``The Missing Wealth of Nations: Are Europe and the US Net
Debtors or Net Creditors'', Quarterly Journal of Economics, 2013, 1321-1364. \href{http://elsa.berkeley.edu/~saez/course/zucmanQJE13.pdf} {(web)}

Zucman, G. \emph{La Richesse Cach\'ee des Nations}, Paris: Seuil, 2012 [translation in english in 2015],
summary in Journal of Economic Perspectives, 2014
\href{http://gabriel-zucman.eu/richesse-cachee/} {(web)}

Zucman, Gabriel, ``Taxing Offshore Wealth and Profits,'' Journal of Economic Perspectives 28(3),
2014.


}



\end{slide}



\end{document}



