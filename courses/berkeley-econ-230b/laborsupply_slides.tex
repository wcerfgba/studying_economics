\documentclass[landscape]{slides}

\usepackage[landscape]{geometry}

\usepackage{pdfpages}
\usepackage{amsmath}
\usepackage{{hyperref}}

\def\mathbi#1{\textbf{\em #1}}

\topmargin=-1.8cm \textheight=17cm \oddsidemargin=0cm
\evensidemargin=0cm \textwidth=22cm

\author{Emmanuel Saez}

\date{Berkeley}

\title{230B: Public Economics \\
Labor Supply Responses to Taxes and Transfers} \onlyslides{1-300}

\newenvironment{outline}{\renewcommand{\itemsep}{}}

\begin{document}

\begin{slide}
\maketitle
\end{slide}

\begin{slide}
\begin{center}
{\bf MOTIVATION}
\end{center}
1) Labor supply responses to taxation are of fundamental
importance for income tax policy [efficiency costs and optimal tax
formulas]

2) Labor supply responses along many dimensions:

(a) Intensive: hours of work on the job, intensity of work,
occupational choice [including education]

(b) Extensive: whether to work or not [e.g., single parent who needs child care, retirement and
migration decisions]

3) Reported earnings for tax purposes can also vary due to (a) tax
avoidance [legal tax minimization], (b) tax evasion [illegal
under-reporting]

4) Different responses in short-run and long-run: long-run
response most important for policy but hardest to estimate
\end{slide}

%\begin{slide}
%\begin{center}
%{\bf OUTLINE}
%\end{center}
%
%1) Labor Supply Elasticity Estimation: Methodological Issues
%
%2) Estimates of hours/participation elasticities
%
%3) Responses to low-income transfer programs (EITC)
%
%4) Inter-temporal Models and Macro Estimates 
%
%5) Elasticity of\ Taxable Income (separate slides set)
%
%%7) Implications for Preference Parameters
%\end{slide}

%\begin{slide}
%\begin{center}
%{\bf REFERENCES}
%\end{center}
%{\bf 1. Surveys in labor economics:}
%
%a) Pencavel (1986) \textit{Handbook of Labor Economics} vol 1
%
%b) Heckman and Killingsworth (1986) \textit{Handbook of Labor
%Econ} vol 1
%
%c) Blundell and MaCurdy (1999) \textit{Handbook of Labor
%Economics} vol 3
%
%d) Keane JEL'2011 (structural)
%
%{\bf 2. Surveys in public economics:}
%
%a) Hausman (1985) \textit{Handbook of Public Economics} vol 1
%
%b) Moffitt (2003)\textit{\ Handbook of Public Economics} vol 4
%
%c) Saez, Slemrod, and Giertz JEL (2012) (reduced form)
%
%\end{slide}


\begin{slide}
\begin{center}
{\bf STATIC MODEL: SETUP}
\end{center}
Baseline model:\ (a) static, (b) linearized tax system, (c) pure
intensive margin choice, (d) single hours choice, (e) no frictions

Let $c$ denote consumption and $l$ hours worked, utility $u(c,l)$
increases in  $c$, and decreases in $l$

Individual earns wage $w$ per hour (net of taxes) and has $y$ in
non-labor income

Key example: pre-tax wage rate $w^p$ and linear tax system with
tax rate $\tau$ and demogrant $G$ $\Rightarrow$  $c=w^p (1-\tau)l +
G$

Individual solves \[ \max_{c,l} u(c,l) \quad \text{subject to} \quad c=wl+y \]
\end{slide}


\begin{slide}
\begin{center}
{\bf LABOR SUPPLY BEHAVIOR}
\end{center}
FOC: $w u_c + u_l =0$ defines uncompensated (Marshallian)  labor
supply function $l^u(w,y)$

Uncompensated elasticity of labor supply: $\varepsilon^u = (w/l)
\partial l^u/ \partial w$ [\% change in hours when net wage $w
\uparrow$ by 1\%]

Income effect parameter: $\eta = w \partial l / \partial y \leq
0$: \$ increase in earnings if person receives \$1 extra in
non-labor income

Compensated (Hicksian) labor supply function $l^c(w,u)$ which
minimizes cost $ c - wl$ st to constraint $u(c,l) \geq u$.

Compensated elasticity of labor supply: $\varepsilon^c = (w/l)
\partial l^c/ \partial w>0$

Slutsky equation: $\partial l/ \partial w = \partial l^c/ \partial
w + l  \partial l /\partial y$ $\Rightarrow$ $\varepsilon^u =
\varepsilon^c + \eta $

\end{slide}


\begin{slide}
\includepdf[pages={108-112}]{laborsupply_attach.pdf}
\end{slide}


%\begin{slide}
%\begin{center}
%{\bf IMPORTANT SPECIAL CASE: NO INCOME EFFECTS}
%\end{center}
%Quasi-linear utility function $u(c,l)=c - h(l)$
%
%$\max_l wl+y -h(l)$ $\Rightarrow$ $h'(l)=w$
%
%$\Rightarrow$ Marshallian $l^u(w,y)=l(w)$ labor supply independent
%of $y$
%
%$\Rightarrow$ Hicksian  $l^c(w,u)=l(w)$ labor supply independent
%of $y$ [parallel indifference curves]
%
%$\Rightarrow$ Identical uncompensated and compensated labor supply
%
%$\Rightarrow$  $\eta=0$ and $\epsilon^c = \epsilon^u >0$
%
%Iso-elastic utility function: $u(c,l)=c-a\frac{l^{1+1/\varepsilon }}{%
%^{1+1/\varepsilon }}$ $\Rightarrow$ $w =C \cdot l^{\varepsilon}$
%\end{slide}

\begin{slide}
\begin{center}
{\bf BASIC CROSS SECTION ESTIMATION}
\end{center}
Data on hours or work, wage rates, non-labor income started
becoming available in the 1960s when first micro surveys and
computers appeared:

Simple OLS regression:
$$l_i = \alpha + \beta w_i + \gamma y_i + X_i \delta +
\epsilon_i$$

$w_i$ is the net-of-tax wage rate

$y_i$ measures non-labor income [including spousal earnings for
couples]

$X_i$ are demographic controls [age, experience, education, etc.]

$\beta$ measures uncompensated wage effects, and $\gamma$ income
effects [can be converted to $\varepsilon^u$, $\eta$]
\end{slide}

\begin{slide}
\begin{center}
{\bf BASIC CROSS SECTION RESULTS}
\end{center}

{\bf 1. Male workers} [primary earners when married] (Pencavel,
1986 survey):

a) Small effects $\varepsilon^u=0$, $\eta=-0.1$, $\varepsilon^c=0.1$
with some variation across estimates (sometimes $\varepsilon^c<0$).

{\bf 2. Female workers} [secondary earners when married]
(Killingsworth and Heckman, 1986):

Much larger elasticities on average, with larger variations across
studies. Elasticities go from zero to over one. Average around
0.5. Significant income effects as well

Female labor supply elasticities have declined overtime as women
become more attached to labor market (Blau-Kahn JOLE'07)
\end{slide}

%\begin{slide}%
%\begin{center}
%{\bf PROBLEMS WITH OLS ESTIMATION OF LABOR SUPPLY EQUATION}
%\end{center}
%
%1) Econometric issues
%
%a) Unobserved heterogeneity [tax instruments]
%
%b) Measurement error in wages and division bias [tax instruments]
%
%c) Selection into labor force [selection models]
%
%d) Endogenous tax rates [non-linear budget set methods]
%
%2) Extensive vs. intensive margin responses [participation models]
%
%3) Non-hours responses [taxable income]
%
%\end{slide}




\begin{slide}
\begin{center}
{\bf KEY ISSUE: $w$ correlated with tastes for work}
\end{center}
\[l_i = \alpha + \beta w_i + \gamma  y_i + \epsilon_i\]
Identification is based on cross-sectional variation in $w_i$:
comparing hours of work of highly skilled individuals (high $w_i$)
to hours of work of low skilled individuals (low $w_i$)

If highly skilled workers have more taste for work (independent of
the wage effect), then $\epsilon_i$ is positively correlated with
$w_i$ leading to an upward bias in OLS

Plausible scenario: hard workers acquire better education and
hence have higher wages

Controlling for $X_i$ can help but can never be sure that we have
controlled for all the factors correlated with $w_i$ and tastes
for work: {\bf Omitted variable bias} 

$\Rightarrow$ Tax changes
provide more compelling identification
\end{slide}

%\begin{slide}
%\begin{center}
%{\bf ISSUE 2: Measurement error in hours}
%\end{center}
%In general $w$ computed as earnings / hours $\Rightarrow$ Can
%create division bias
%
%Let $l^{\ast }$ denote true hours, $l$ observed hours
%
%Compute $w=z/l$ where $z$ is observed earnings%
%\begin{eqnarray*}
%&\Rightarrow &\log l=\log l^{\ast }+\mu \quad  \mathrm{measurement} \quad \mathrm{error} \\
%&\Rightarrow &\log w=\log z-\log l=\log z-\log l^{\ast }-\mu =\log
%w^{\ast }-\mu
%\end{eqnarray*}%
%Spurious negative correlation between $\log l$ and $\log w$ [e.g,
%workers with high misreported hours also have low imputed wages]
%biasing elasticity estimate downward
%
%Solution:\ tax instruments again
%\end{slide}


%\begin{slide}
%\begin{center}
%{\bf ISSUE 3: Non-participation}
%\end{center}
%Consider model with fixed costs of working, where some individuals
%choose not to work
%
%Wages are unobserved for non-labor force participants
%
%Thus, OLS regression on workers only includes observations with $%
%l_{i}>0$
%
%This can bias OLS\ estimates: low wage earners must have very high
%unobserved propensity to work to find it worthwhile
%
%Requires a selection correction pioneered by Heckman in 1970s
%(e.g. Heckit, Tobit, or ML estimation): problem is that
%identification is based on strong functional form assumptions [See
%Killingsworth and Heckman (1986) for implementation]
%
%Current approach: use tax instruments and look directly at participation
%margin
%\end{slide}


%\begin{slide}%
%\begin{center}
%{\bf Extensive vs. Intensive Margin}
%\end{center}
%
%Related issue:\ want to understand effect of taxes on labor force
%participation decision
%
%With fixed costs of work, individuals may jump from
%non-participation to part time or full time work (non-convex
%budget set)
%
%This can be handled using a discrete choice model:%
%\[
%P=\phi (\alpha +\varepsilon \log (1-\tau )-\eta y)
%\]
%where $P\in \{0,1\}$ is an indicator for whether the individual
%works
%
%Function $\phi $ typically specified as logit, probit, or linear
%prob model
%
%Note:\ here it is critical to have tax variation; regression
%cannot be run with wage variation
%
%\end{slide}

%\begin{slide}
%\begin{center}
%{\bf  ISSUE 4: Non-hours responses}
%\end{center}
%Traditional literature focused purely on hours of work and labor
%force participation
%
%Problem: income taxes distort many margins beyond hours of work
%
%a) Non-hours margins may be more important quantitatively
%
%b) Hours very hard to measure (most ppl report 40 hours per week)
%
%Two solutions in modern literature:
%
%a) Focus on total earnings ($z=wl$) [or taxable income]  as a
%broader measure of labor supply
%
%b) Focus on subgroups of workers for whom hours are better
%measured, e.g. taxi drivers
%\end{slide}

%\begin{slide}
%\begin{center}
%{\bf ISSUE 5: NON-LINEAR BUDGET SETS}
%\end{center}
%Actual tax system is not linear but piece-wise linear with varying
%marginal tax rate $\tau$ due to (a) means-tested transfer
%programs, (b) progressive individual income tax
%
%%Individual maximization problem:
%%
%%$\max u(w^pl-T(w^pl),l)$ $\Rightarrow$ FOC $u_c w^p(1-T')+u_l=0$
%
%Same theory applies when considering the linearized tax system
%$c=wl+y$ with $w=w^p(1-T')$ and $y$ defined as virtual income
%(intercept of budget with x-axis when setting $l=0$)
%
%Main complications: 
%
%(a) $w$ [and $y$] become endogenous to choice
%of $l$
%
%(b) FOC may not hold if individual bunches at a kink
%
%(c) FOC may not characterize the optimum choice
%\end{slide}
%
%\begin{slide}
%\includepdf[pages={1}]{laborsupply_attach.pdf}
%\end{slide}
%
%
%\begin{slide}
%\begin{center}
%{\bf ISSUE 5: NON-LINEAR BUDGET SETS}
%\end{center}
%
%Non-linear budget set creates two econometric problems:
%
%1) Model mis-specification:\ OLS regression no longer recovers
%structural elasticity parameter of interest
%
%2) Econometric bias: $\tau _{i}=T'(w_il_i)$ and $y_i$ depends on income $w_{i}l_{i}$
%and hence on $l_{i}$ 
%
%Tastes for work are positively correlated with $\tau _{i}$ (due to progressive tax system) $%
%\rightarrow $\ downward bias in OLS\ regression of hours worked on
%net-of-tax rates
%
%\end{slide}
%
%\begin{slide}
%\begin{center}
%{\bf OLD NON-LINEAR BUDGET SET METHOD}
%\end{center}
%Issue addressed by non linear budget set studies pioneered by
%Hausman in late 1970s (Hausman, 1985 PE handbook chapter)
%
%Method uses a structural model of labor supply to derive and estimate
%labor supply function fully consistent with theory
%
%Key point: the method still uses the standard cross-sectional variation
%in pre-tax wages $w^p$ for identification. Taxes are seen as a
%problem to deal with rather than an opportunity for
%identification.
%
%New literature identifying labor supply elasticities using tax
%changes has a totally different perspective: taxes are seen as an
%{\bf opportunity} to identify labor supply
%\end{slide}
%
%



%\begin{slide}
%\begin{center}
%{\bf ISSUE 4: NON-LINEAR BUDGET SETS}
%\end{center}
%Actual tax system is not linear but piece-wise linear with varying
%marginal tax rate $\tau$ due to (a) means-tested transfer
%programs, (b) progressive individual income tax, (c) ceiling in
%payroll tax
%
%Individual maximization problem:
%
%$\max u(w^pl-T(w^pl),l)$ $\Rightarrow$ FOC $u_c w^p(1-T')+u_l=0$
%
%Same theory applies when considering the linearized tax system
%$c=wl+y$ with $w=w^p(1-T')$ and $y$ defined as virtual income
%(intercept of budget with x-axis when setting $l=0$)
%
%Main complications: (a) $w$ [and $y$] become endogenous to choice
%of $l$, (b) FOC may not hold if individual bunches at a kink, (c)
%FOC may not characterize the optimum choice
%\end{slide}
%
%
%\begin{slide}
%\includepdf[pages={1}]{laborsupply_attach.pdf}
%\end{slide}
%
%
%\begin{slide}
%\begin{center}
%{\bf ISSUE 4: NON-LINEAR BUDGET SETS}
%\end{center}
%
%Non-linear budget set creates two problems:
%
%1) Model mis-specification:\ OLS regression no longer recovers
%structural elasticity parameter $\varepsilon $ of interest
%
%Two reasons:\ (a)\ underestimate response because people pile up
%at kink and (b) mis-estimate income effects
%
%2) Econometric bias: $\tau _{i}$ depends on income $w_{i}l_{i}$
%and hence on $l_{i}$
%
%Tastes for work are positively correlated with $\tau _{i}$ $%
%\rightarrow $\ downward bias in OLS\ regression of hours worked on
%net-of-tax rates
%
%Solution to problem \#2:\ only use reform-based variation in tax
%rates. But problem \#1 requires fundamentally different estimation method
%
%\end{slide}
%
%\begin{slide}
%\begin{center}
%{\bf NON-LINEAR BUDGET SET METHOD}
%\end{center}
%Issue addressed by non linear budget set studies pioneered by
%Hausman in late 1970s (Hausman, 1985 PE handbook chapter)
%
%Method uses a structural model of labor supply
%
%Key point: the method uses the standard cross-sectional variation
%in pre-tax wages $w^p$ for identification. Taxes are seen as a
%problem to deal with rather than an opportunity for
%identification.
%
%New literature identifying labor supply elasticities using tax
%changes has a totally different perspective: taxes are seen as an
%{\bf opportunity} to identify labor supply
%\end{slide}
%
%
%\begin{slide}
%\begin{center}
%{\bf NON-LINEAR BUDGET SET METHOD}
%\end{center}
%
%1) Assume an\ uncompensated labor supply equation:%
%\[
%l=\alpha +\beta w(1-\tau )+\gamma y+\epsilon
%\]
%
%2) Error term $\epsilon $ is normally distributed with variance
%$\sigma ^{2}$
%
%3) Observed variables: $w_{i}$, $\tau _{i}$, $y_{i}$, and $l_{i}$
%
%4) Technique: (a) construct likelihood function given observed
%labor supply choices on NLBS, (b) find parameters ($\alpha ,\beta
%,\gamma $) that maximize likelihood
%
%5) Important insight:\ need to use \textquotedblleft virtual
%incomes\textquotedblright\ in lieu of actual unearned income with
%NLBS
%
%\end{slide}
%
%
%\begin{slide}%
%\begin{center}
%{\bf NLBS Likelihood Function (2 brackets)}
%\end{center}
%Individual can locate on first bracket, on second bracket, or at
%the kink $l_{K}$
%
%Likelihood = probability that we see individual $i$ at labor supply $%
%l_{i}$ given a parameter vector
%
%Decompose likelihood into three components
%
%Component 1:\ individual $i$ on first bracket: $0<l_{i}<l_{K}$%
%$$l_{i}=\alpha +\beta w_{i}(1-\tau ^{1})+\gamma y^{1}+\epsilon
%_{i}$$
%
%Error $\epsilon _{i}=l_{i}-(\alpha +\beta w_{i}(1-\tau
%^{1})+\gamma y^{1})$. Likelihood:$ L_{i}=\phi ((l_{i}-(\alpha
%+\beta w_{i}(1-\tau ^{1})+\gamma y^{1})/\sigma )$
%
%Component 2: individual $i$ on second bracket: $l_{K}<l_{i}$:
%$L_{i}=\phi ((l_{i}-(\alpha +\beta w_{i}(1-\tau ^{2})+\gamma
%y^{2})/\sigma )$
%\end{slide}
%
%
%\begin{slide}%
%\begin{center}
%{\bf NLBS\ Likelihood Function}
%\end{center}
%Now consider individual $i$ located at the kink point
%
%1) If tax rate is $\tau ^{1}$ and virtual income $y^{1}$
%individual wants to work $l>l_{K}$
%
%2) If tax is $\tau ^{2}$ and virtual income $y^{2}$ individual
%wants to work $l<l_{K}$
%
%
%3) These inequalities imply:%
%\begin{eqnarray*}
%\alpha +\beta w_{i}(1-\tau ^{1})+\gamma y^{1}+\epsilon _{i} &>&l_{K}>\alpha
%+\beta w_{i}(1-\tau ^{2})+\gamma y^{2}+\epsilon _{i} \\
%l_{K}-(\alpha +\beta w_{i}(1-\tau ^{1})+\gamma y^{1} &<&\epsilon
%_{i}<l_{K}-(\alpha +\beta w_{i}(1-\tau ^{2})+\gamma y^{2})
%\end{eqnarray*}
%
%4) Contribution to likelihood is probability that error lies in
%this
%range:%
%\begin{eqnarray*}
%L_{i} &=&\Psi \lbrack (l_{K}-(\alpha +\beta w_{i}(1-\tau ^{2})+\gamma
%y^{2}))/\sigma ] \\
%&&-\Psi \lbrack (l_{K}-(\alpha +\beta w_{i}(1-\tau ^{1})+\gamma
%y^{1}))/\sigma ]
%\end{eqnarray*}
%
%\end{slide}
%
%
%\begin{slide}%
%\begin{center}
%{\bf Maximum Likelihood Estimation}
%\end{center}
%1) Log likelihood function is $L=\sum_{i}\log L_{i}$
%
%2) Final step is solving%
%\[
%\max L(\alpha ,\beta ,\gamma ,\sigma )
%\]
%
%3) In practice, likelihood function much more complicated because
%of more kinks, non-convexities, and covariates
%
%4) But basic technique remains the same
%\end{slide}
%
%\begin{slide}
%\begin{center}
%{\bf Hausman (1981)\ Application}
%\end{center}
%1) Hausman applies method to 1975 PSID cross-section
%
%a) Finds significant compensated elasticities and large income
%effects
%
%b) Elasticities larger for women than for men
%
%2) Shortcomings of this implementation
%
%a) Sensitivity to functional form choices, which is a larger issue
%with structural estimation
%
%b) No tax reforms, so does not solve fundamental econometric
%problem that\ tastes for work may be correlated with $w$
%\end{slide}
%
%\begin{slide}
%\begin{center}
%{\bf NLBS and Bunching at Kinks}
%\end{center}
%
%Subsequent studies obtain different estimates (MaCurdy, Green, and
%Paarsh 1990, Blomquist 1995)
%
%a) Several studies find \textbf{negative} compensated wage
%elasticity estimates
%
%b) Debate:\ impose requirement that compensated elasticity is
%positive or conclude that data rejects model?
%
%Fundamental source of problem: labor supply model predicts that
%individuals should bunch at the kink points of the tax schedule
%
%a) But we observe very little bunching at kinks (Heckman vs.
%Hausman), so model is rejected by the data
%
%b) Interest in NLBS\ models diminished despite their conceptual
%advantages over OLS\ methods
%\end{slide}
%


%\begin{slide}
%\begin{center}
%{\bf Natural Experiment Labor Supply Literature}
%\end{center}
%Literature exploits variation in taxes/transfers to estimate Hours
%and Participation Elasticities
%
%%1) Return to simple model where we ignore non-linear budget set
%%issues
%
%1) Large literature in labor/Public economics estimates effects of
%taxes and wages on hours worked and participation
%
%2) Now discuss some estimates from this older literature
%\end{slide}

\begin{slide}
\begin{center}
{\bf Negative Income Tax (NIT) Experiments}
\end{center}

1) Best way to resolve identification problems: exogenously
change taxes/transfers with a {\bf randomized experiment}

2)  NIT experiment conducted in 1960s/70s in Denver, Seattle, and
other cities

3) First major social experiment in U.S. designed to test proposed
transfer policy reform

4) Provided lump-sum welfare grants $G$ combined with a steep
phaseout rate $\tau $ (50\%-80\%) [based on family earnings]

5) Analysis by Rees (1974), Munnell (1986) book, Ashenfelter and\
Plant JOLE'90, and others

6) Several groups, with randomization within each; approx. N = 75
households in each group
\end{slide}

\begin{slide}
\includepdf[pages={43}]{laborsupply_attach.pdf}
\end{slide}

\begin{slide}
\includepdf[pages={158}]{laborsupply_attach.pdf}
\end{slide}

%\begin{slide}
%\begin{center}
%{\bf NIT Experiments: Ashenfelter and Plant JHR' 90}
%\end{center}
%1) Present non-parametric evidence of labor supply effects
%
%2) Compare actual benefit payments to treated household vs. hypothetical
%benefit payments to control households
%
%3) Difference in benefit payments reflects aggregates hours and
%participation responses
%
%4) This is the relevant parameter for expenditure calculations and
%for welfare analysis
%
%5) Shortcoming: approach does not decompose estimates into income
%and substitution effects and intensive vs. extensive margin
%
%$\Rightarrow$ Hard to identify the key elasticity relevant for policy
%purposes and predict labor supply effect of other programs
%\end{slide}
%
%\begin{slide}
%\includepdf[pages={44-45}]{laborsupply_attach.pdf}
%\end{slide}
%

\begin{slide}
\begin{center}
{\bf NIT Experiments: Findings}
\end{center}
See Ashenfelter and Plant JHR' 90 for non-parametric evidence. 
More parametric evidence in earlier work. Key results:

1) Significant labor supply response but small overall

2) Implied earnings elasticity for males around 0.1

3) Implied earnings elasticity for women around 0.5

4) Academic literature not careful to decompose response along
intensive and extensive margin

5) Response of women is concentrated along the extensive margin
(can only be seen in official govt. report)

6) Earnings of treated women who were working before the
experiment did not change much

\end{slide}

%\begin{slide}
%\begin{center}
%{\bf Problems with NIT Experimental Design}
%\end{center}
%Estimates from NIT\ not considered fully credible due to several
%shortcomings:
%
%1) {\bf Self reported earnings:} Treatments had financial
%incentives to under-report earnings $\Rightarrow$ Lesson: need to
%match with administrative records [Greenberg and Halsey JOLE'83]
%
%2) {\bf Selective attrition:}
%
%After initial year, data collected based on voluntary income
%reports by families $\Rightarrow$ Those in less generous
%groups/far above break-even point had much less incentive to
%report $\Rightarrow$ Attrition rates higher in these groups
%$\Rightarrow$ No longer a random sample of treatment + controls
%[Ashenfelter-Plant JOLE'90]
%
%3) Response might be smaller than real reform b/c of {\bf General
%Equilibrium} effects
%\end{slide}


%\begin{slide}%
%\begin{center}
%{\bf Social Experiments: Costs/Benefits}
%\end{center}
%1) Cost of NIT\ experiments: around \$1 billion (in today's
%dollars)
%
%2) Huge cost for a social experiment but trivial relative to
%budget of the US federal government (\$3 trillion)
%
%3) Should the government do more experimentation? Potential
%benefits:
%
%a) Narrow the standard error around estimates
%
%b) Allow implementation of better tax and redistribution policy
%
%[Literature on optimal experimenting in engineering and
%agriculture but never applied to economics, pb. is instability of
%parameters]
%\end{slide}

\begin{slide}
\begin{center}
{\bf From true experiment to ``natural experiments'': \\ income effects on lottery winners}
\end{center}
True experiments are costly to implement and hence rare

Real economic world provides variation that can
be exploited to estimate behavioral responses $\Rightarrow$
\textbf{Natural Experiments}

Natural experiments can come very close to true experiments:

Imbens, Rubin, Sacerdote AER '01 did a survey of lottery winners
and non-winners matched to Social Security administrative data to
estimate income effects

Lottery generates random assignment conditional on playing

Find significant but relatively small income effects: $\eta = w
\partial l/\partial y$ between -0.05 and -0.10

Identification threat: differential response-rate among groups
\end{slide}

\begin{slide}
\includepdf[pages={46,47}]{laborsupply_attach.pdf}
\end{slide}


\begin{slide}
\begin{center}
{\bf Difference-in-Difference (DD) methodology}
\end{center}

Two groups: Treatment group (T) which faces a change [lottery winners] and control group (C) which does not [non winners]

Compare the evolution of T group (before and after change) to the evolution of the C group (before and after change)

DD identifies the \textbf{treatment effect} if the \textbf{parallel trend assumption} holds:

Absent the change, $T$ and $C$ would have evolved in parallel 

DD most convincing when groups are very similar to start with

Should always test DD using data from more periods and plot the two time
series to check parallel trend assumption

\end{slide}


\begin{slide}
\begin{center}
{\bf Labor supply and lotteries in Sweden}
\end{center}
Cesarini et al. (2017) use Swedish population wide administrative data with more
compelling setting: (1) bank accounts with random prizes (PLS), (2) monthly lottery subscription (Kombi), 
and (3) TV show participants (Triss)

\textbf{Key results:}

1) Effects on both extensive and intensive labor supply margin, time persistent

2) Significant but small income effects: $\eta = w
\partial l/\partial y \simeq -0.1$

3) Effects on spouse but not as large as on winner 

$\Rightarrow$ Rejects
the \textbf{unitary} model of household labor supply: \\
$\max u(c_1,c_2,l_1,l_2)$ st $c_1+c_2 \leq w_1 l_1 + w_2 l_2 +R$
$\Rightarrow$ only household $R$ matters 

\end{slide}

\begin{slide}
\includepdf[pages={113}]{laborsupply_attach.pdf}
\end{slide}

\begin{slide}
\includepdf[pages={114,115}]{laborsupply_attach.pdf}
\end{slide}

%\begin{slide}
%\begin{center}
%{\bf Instrumental Variable Methods}
%\end{center}
%1) Another strategy to overcome endogeneity is instrumenting for
%wage rate
%
%2) Mroz (1987):\ often-cited survey/meta-analysis of earlier
%studies
%
%3) Uses PSID to test widely-used IV's for married women's wage%
%\begin{eqnarray*}
%l_{i} &=&\alpha +\beta w+\gamma X+\varepsilon \\
%w &=&\theta Z+\mu
%\end{eqnarray*}
%
%4) Uses Hausman specification/overidentification test to show that
%many instruments violate $EZ\varepsilon =0$
%
%\end{slide}
%
%\begin{slide}
%\begin{center}
%{\bf Hausman Test}
%\end{center}
%
%1) Suppose you can divide instrument set into those that are
%credibly exogenous ($Z$) and those that are questionable ($Z^{\ast
%}$)
%
%2) Null hypothesis:\ both are exogenous
%
%3) Alternative hypothesis:\ $Z^{\ast }$ is endogenous
%
%4) Compute IV\ estimate of $\beta $ with small and large
%instrument set and test for equality of the coefficients
%
%5) Note that is often a very low power test (accept validity if
%instruments are weak)
%\end{slide}
%
%
%
%\begin{slide}
%\begin{center}
%{\bf Mroz 1987: Setup and Results}
%\end{center}
%1) Uses background variables as ``credibly exogenous'' instruments
%[Parents' education, age, education polynomials]
%
%2) Tests validity of labor market experience, average hourly
%earnings, and previous reported wages
%
%3) Rejects validity of all three
%
%4) Shows that earlier estimates are highly fragile and unreliable
%
%5) Contributed to emerging view that policy variation (e.g.,
%taxes)\ was necessary to really identify these elasticities
%properly
%
%\end{slide}

%\begin{slide}
%\begin{center}
%{\bf Tax Reform Variation (Eissa 1995)}
%\end{center}
%1) Modern studies use tax changes as \textquotedblleft natural
%experiments\textquotedblright
%
%2) Representative pedagogical example:\ Eissa (1995)
%
%3) Uses the Tax Reform Act (TRA) of 1986 to identify the effect of MTRs
%on labor force participation and hours of married women
%
%4) TRA 1986 cut top income MTR from 50\% to 28\% from 1986 to 1988
%but did not significantly change tax rates for the middle class
%
%5) Substantially increased incentives to work of wives of high
%income husbands relative wives of middle income husbands
%
%\end{slide}
%
%\begin{slide}
%\includepdf[pages={48}]{laborsupply_attach.pdf}
%\end{slide}



%\begin{slide}
%\includepdf[pages={49-51}]{laborsupply_attach.pdf}
%\end{slide}
%
%
%\begin{slide}%
%\begin{center}
%{\bf Eissa 1995: Results}
%\end{center}
%1) Participation elasticity around 0.4 but large standard errors
%
%2) Hours elasticity of 0.6
%
%3) Total elasticity (unconditional hours) is $0.4+0.6=1$
%
%\end{slide}
%
%\begin{slide}
%\begin{center}
%{\bf Eissa 1995: Caveats}
%\end{center}
%1) Does the common trends assumption hold? Potential story biasing
%the result:
%
%Trend toward \textquotedblleft power couples\textquotedblright\
%could bias DD upward: In 1983-1985, high income husbands
%had non-working spouses, In 1989-1991, high income husbands married to
%professionals [and no change for middle class]
%
%2) $LFP$ before the reform is very different across $T$ and $C$
%groups $\Rightarrow$ DD sensitive to functional form assumption
%[such as levels vs logs]
%
%3) Liebman and Saez (2006) plot full time-series CPS plot and show
%that Eissa's results are not robust using admin data (SSA\ matched
%to\ SIPP) [unfortunately, IRS public tax data does not break down
%earnings within couples]
%
%\end{slide}
%
%\begin{slide}
%\includepdf[pages={52}]{laborsupply_attach.pdf}
%\end{slide}


\begin{slide}
\begin{center}
{\bf Labor Supply Substitution Effects: \\ Tax Free Second Jobs in Germany}
\end{center}

In 2003, Germany made secondary jobs (paying less than 400 Euros/month) tax free:
amounts to a 20-60\% subsidy on second job earnings (depending on family marginal tax rate) $\Rightarrow$ almost pure substitution labor supply effect

Tazhitdinova '20 uses social security admin monthly earnings data

Fraction of population holding second jobs increased sharply (from 2.5\% to 6-7\%)
with bigger response overtime

Finds no offsetting effect on primary earnings $\Rightarrow$ People did work more

Looks like a big labor supply response but likely happened because employers 
willing to create lots of mini-jobs to accommodate supply

\end{slide}

\begin{slide}
\includepdf[pages={150-151}]{laborsupply_attach.pdf}
\end{slide}





\begin{slide}%
\begin{center}
{\bf Married Women Elasticities:\ Blau and Kahn '07}
\end{center}

1) Identify elasticities from 1980-00 using grouping instrument

a) Define cells (year$\times$age$\times$education) and compute mean wages

b) Instrument for actual wage with mean wage in cell

2) Identify purely from group-level variation, which is less
contaminated by individual endogenous choice

3) Results: (a) total hours elasticity for {\bf married women}
(including intensive +\ extensive margin) shrank from 0.4 in 1980 to 0.2
in early 2000s, (b) effect of husband earnings $\downarrow$ overtime

4) Interpretation: elasticities shrink as women become more
attached to the labor force

\end{slide}


\begin{slide}%
\begin{center}
{\bf Summary of Static Labor Supply Literature (SKIP)}
\end{center}
1) Small elasticities for prime-age males

Probably institutional restrictions, need for at least one income,
etc. prevent a short-run response

2) Larger responses for workers who are less attached to labor
force: Married women, low income earners, retirees

3) Responses driven primarily by extensive margin

a) Extensive margin (participation) elasticity around 0.2-0.5

b) Intensive margin (hours)\ elasticity smaller
\end{slide}




\begin{slide}%
\begin{center}
{\bf Responses to Low-Income Transfer Programs}
\end{center}

1) Particular interest in treatment of low incomes in a
progressive tax system: are they responsive to incentives?

2) Complicated set of transfer programs in\ US

a) In-kind: food stamps, Medicaid, public housing, job training,
education subsidies

b) Cash:\ TANF, EITC, SSI

3) See Gruber undergrad textbook for details on institutions

\end{slide}


%\begin{slide}
%\begin{center}
%{\bf Overall Costs of Anti Poverty Programs}
%\end{center}
%1) US government (fed+state and local) spent \$800bn in 2013 on
%income-tested programs
%
%a) About 4\% of GDP but 15\% of \$5 Trillion govt budget
%(fed+state+local).
%
%b) About 50\% is health care (Medicaid)
%
%2) Only \$200 billion in cash (1\% of GDP, or 25\% of transfer
%spending)
%
%\end{slide}




\begin{slide}%
\begin{center}
{\bf 1996 US Welfare Reform}
\end{center}
%1) Largest change in welfare policy

1) Reform modified AFDC\ cash welfare program to provide more
incentives to work (renamed TANF)

a) Requiring recipients to go to job training or work

b) Limiting the duration for which families able to receive
welfare

c) Reducing phase-out rate of benefits

2) Fed govt provided incentives for states to experiment
with reforms in 1992-1995 (state waivers). Some did randomize
experiments.
%Variation across states because fed govt. gave block grants
%with guidelines

4) EITC\ also expanded during this period:\ general shift from
welfare to \textquotedblleft workfare\textquotedblright

Did welfare reform and EITC increase labor supply?

\end{slide}

\begin{slide}
\includepdf[pages={98}]{laborsupply_attach.pdf}
\end{slide}

%\begin{slide}
%\begin{center}
%{\bf Welfare Reform:\ Two Empirical Questions}
%\end{center}
%1) Incentives:\ did welfare reform actually increase labor supply?
%
%a) Test whether EITC\ expansions affect labor supply
%
%b) Use state welfare  randomized  experiments implemented before reform
%to assess effects of switch from AFDC to TANF. Canada also experimented with
%welfare reform (Card and Hyslop '05)
%
%2) Benefits:\ did removing many people from transfer system reduce
%their welfare? How did consumption change?
%
%
%Focus on single mothers, who were most impacted by reform
%
%\end{slide}


\begin{slide}
\begin{center}
{\bf Randomized welfare experiment: \\SSP Welfare Demonstration in Canada}
\end{center}

Canadian Self Sufficiency Project (SSP):
randomized experiment that gave welfare recipients an earnings subsidy for 36 months in 1990s
(but need to start working by month 12 to get it)

3 year temporary participation tax rate cut from average rate of 74.3\% to 16.7\%
[get to keep 83 cents for each \$ earned instead of 26 cents]

Card and Hyslop (EMA 2005) provide classic analysis. Two results:

1) Strong effect on employment rate during experiment (peaks at 14 points)

2) Effect quickly vanishes when the subsidy stops after 36 months (entirely gone by month 52)

\end{slide}

\begin{slide}
\includepdf[pages={84}]{laborsupply_attach.pdf}
\end{slide}



\begin{slide}
\begin{center}
{\bf Earned Income Tax Credit (EITC) program}
\end{center}
Kleven (2019) provides comprehensive EITC re-analysis
using women aged 20-50 and CPS data


1) EITC started small in the 1970s but was expanded in 1986-88,
1994-96, 2008-09: today, largest means-tested cash transfer
program [\$75bn in 2019, 30m families recipients]

2) Eligibility: families with kids and low earnings.

3) Refundable Tax credit: administered as
annual tax refund received in Feb-April, year $t+1$ (for earnings
in year $t$)

4) EITC has flat pyramid structure with phase-in (negative MTR),
plateau, (0 MTR), and phase-out (positive MTR)

5) States have added EITC components to their income taxes [in
general a percentage of the Fed EITC, great source of natural
experiments, understudied bc CPS too small]


\end{slide}

%\begin{slide}
%\includepdf[pages={54}]{laborsupply_attach.pdf}
%\end{slide}

\begin{slide}
\includepdf[scale=1.0,pages={122-133}]{laborsupply_attach.pdf}
\end{slide}

\begin{slide}
\begin{center}
{\bf Welfare Reform and EITC Expansion: Labor supply}
\end{center}

Incredible increase in labor force participation of single mothers during the 1990s
when welfare reform and EITC expansion happened

Unlikely that the EITC can explain it because other EITC changes haven't generated
such large effects

Sociological evidence shows that welfare reform ``scared'' single mothers into working

Single moms in the US were suddenly expected to work

Kleven (2019): Maybe a unique combination of EITC reform, welfare reform,
economic upturn, and changing social norms lead to this shift


\end{slide}





%\begin{slide}
%\begin{center}
%{\bf Theoretical Behavioral Responses to the EITC}
%\end{center}
%{\bf Extensive margin:} positive effect on Labor Force
%Participation as EITC makes work more attractive
%
%%(Meyer-Rosebaum 2001 find that 60\% of the increase
%%of LFP of single mothers in 1990s due to EITC expansion]
%
%{\bf Intensive margin:} earnings conditional on working, mixed
%effects
%
%1) Phase in: (a) Substitution effect: work more due to wage
%subsidy, (b) Income effect: work less $\Rightarrow$ Net
%effect: ambiguous; probably work more
%
%2) Plateau: Pure income effect (no change in net wage)
%$\Rightarrow$ Net effect: work less
%
%3) Phase out: (a) Substitution effect: work less, (b) Income
%effect: also work less $\Rightarrow$ Net effect: work less
%
%Should expect bunching at the EITC kink points
%\end{slide}

%\begin{slide}%
%\begin{center}
%{\bf Eissa and Liebman 1996}
%\end{center}
%1) Pioneering study of labor force participation of single mothers
%before/after 1986-7 EITC expansion using CPS data
%
%2) Limitation: this expansion was relatively small
%
%3) Diff-in-Diff strategy:
%
%a) Treatment group: single women with kids
%
%b) Control group: single women without kids
%
%c) Comparison periods: 1984-1986 vs. 1988-1990
%\end{slide}
%
%
%\begin{slide}
%\includepdf[pages={55}]{laborsupply_attach.pdf}
%\end{slide}
%
%
%\begin{slide}
%\begin{center}
%{\bf Diff-in-Diff (DD) Methodology:}
%\end{center}
%{\bf Step 1: Simple Difference}
%
%Outcome: $LFP$ (labor force participation)
%
%Two groups: Treatment group (T) which faces a change [single women
%with kids] and control group (C) which does
%not [single women without kids]
%
%Simple Difference estimate: $D=LFP^T-LFP^C$ captures treatment
%effect if absent the treatment, $LFP$ equal across 2 groups
%
%Note: this assumption always holds when $T$ and $C$ status is randomly
%assigned
%
%Test for this assumption: Compare $LFP$ before treatment happened
%$D_B=LFP_B^T-LFP_B^C$
%
%\end{slide}
%
%\begin{slide}
%\begin{center}
%{\bf Diff-in-Diff (DD) Methodology:}
%\end{center}
%{\bf Step 2: Diff-in-Difference (DD)}
%
%If $D_B \neq 0$, can estimate DD: $$DD=D_A-D_B=LFP_A^T-LFP_A^C -
%[LFP_B^T-LFP_B^C]$$ (A = after reform, B = before reform)
%
%DD is unbiased if {\bf parallel trend} assumption holds:
%
%Absent the change, difference across $T$ and $C$ would have stayed
%the same before and after
%
%OLS Regression estimation of DD:
%$$LFP_{it}=\beta_0 AFTER + \beta_1 TREAT + \gamma AFTER \cdot
%TREAT + \varepsilon$$
%\[ \hat{\gamma}_{OLS}=LFP_A^T-LFP_A^C - [LFP_B^T-LFP_B^C] \]
%\end{slide}
%
%\begin{slide}
%\includepdf[pages={56}]{laborsupply_attach.pdf}
%\end{slide}
%
%\begin{slide}
%\begin{center}
%{\bf Diff-in-Diff (DD) Methodology}
%\end{center}
%DD most convincing when groups are very similar to start with
%[closer to randomized experiment]
%
%Should always test DD using data from more periods and plot the two time
%series to check parallel trend assumption
%
%Use alternative control groups [not as convincing as potential
%control groups are many]
%
%In principle, can create a DDD as the difference between actual DD
%and $DD^{Placebo}$ (DD between 2 control groups). However, DDD of
%limited interest in practice because
%
%(a) if $DD^{Placebo}\neq 0$, DD test fails, hard to believe DDD
%removes bias
%
%(b) if $DD^{Placebo}= 0$, then DD=DDD but DDD has higher s.e.
%\end{slide}
%
%
%\begin{slide}
%\includepdf[pages={57-58}]{laborsupply_attach.pdf}
%\end{slide}
%
%
%
%
%
%
%\begin{slide}
%\begin{center}
%{\bf Diff-in-Diff (DD) Methodology}
%\end{center}
%1) DD sensitive to functional form (e.g. log vs levels) when $D_{before} \neq 0$.
%
%\small
%Example: $T$ $\uparrow$ from 40\% to 50\% and $C$ $\uparrow$ from 15\% to 20\%:
%$DD_{level}=[50-40]-[20-15]=5$ but $DD_{log}=\log[50/40] - \log [20/15] = - .06$
%\normalsize
%
%2) To obtain elasticity estimate, need to take ratio of $DD_{outcome}$ to $DD_{policy \:\: change}$ to
%form the \textbf{Wald estimate}:
%\[\hat{e} = \frac{ [ \log LFP_A^T- \log LFP_A^C ] - [\log LFP_B^T-\log LFP_B^C ] }{\log (1-\tau_A^T)- \log (1-\tau_A^C) ] - [\log (1-\tau_B^T)-\log (1-\tau_B^C) ] } \]
%$DD_{policy \:\: change}$ is the \textbf{1st stage}, $DD_{outcome}$ is the \textbf{reduced form} effect, the ratio is the \textbf{2nd stage} estimate
%
%\small
%Wald estimated with 2SLS regression:
%\[LFP_{it}=\beta_0 AFTER + \beta_1 TREAT +e \cdot \log(1-\tau) + \varepsilon\]
%where $\log(1-\tau)$ is instrumented with interaction
%$AFTER \cdot TREAT$
%
%\end{slide}
%
%
%\begin{slide}%
%\begin{center}
%{\bf Eissa and Liebman 1996: Results}
%\end{center}
%1) Find a small but significant DD effect: 2.4\% (larger DD effect
%4\% among women with low education) $\Rightarrow$ Translates into
%substantial participation elasticities above 0.5
%
%2) Note the labor force participation for women with/without
%children are not great comparison groups (70\% LFP vs. +90\%):
%time series evidence is only moderately convincing
%
%3) Subsequent studies have used bigger 1990s EITC expansions and also find positive effects on labor force
%participation of single women/single mothers (Meyer-Rosenbaum 2001) but contaminated
%by AFDC/TANF transition
%
%4) Conventional standard errors probably overstate precision
%\end{slide}
%
%\begin{slide}%
%\begin{center}
%{\bf Bertrand-Duflo-Mullainathan QJE'04}
%\end{center}
%Show that conventional standard errors in fixed effects
%regressions with state reform variation are too low
%
%Randomly generated placebo state laws: half the states pass law at
%random date. $I_{st}$ is one if state $s$ has law in place at time
%$t$.
%
%Use female wages $w_{ist}$ in CPS data and run OLS:
%$$\log w_{ist} = A_s + B_t + b I_{st} + \varepsilon_{ist}$$
%
%$\hat{b}$ significant (at 5\% level) in 65\% of cases $\Rightarrow$
%$\varepsilon_{ist}$ are not iid
%
%Clustering by state*year cells is not enough (significant 45\% of
%the time)
%
%Need to cluster at state level to obtain reasonable s.e. because
%of strong serial correlation within states
%\end{slide}
%

%\begin{slide}
%\begin{center}
%{\bf Meyer and Rosenbaum 2001}
%\end{center}
%1) Exploit the much bigger 1990s expansion in EITC
%
%2) Document dramatic (6 pp, 10\%) increase in LFP\ for single
%women with children around EITC expansion
%
%3) No change for women without children
%
%4) Problem: expansion took place at same time as welfare reform
%
%5) Try to disentangle effects of welfare waivers, changes in AFDC
%and state taxes, etc. using state-level variation
%
%Bottom line: elasticity of participation w.r.t. tax/transfer
%incentives is significant
%\end{slide}
%
%\begin{slide}
%\includepdf[pages={59,60}]{laborsupply_attach.pdf}
%\end{slide}
%
%\begin{slide}
%\includepdf[pages={61}]{laborsupply_attach.pdf}
%\end{slide}
%
%
%\begin{slide}
%\begin{center}
%{\bf Meyer and Rosenbaum 2001}
%\end{center}
%
%1)Analyze the introduction of EITC and Welfare waivers for the
%period 1984-1996 using CPS data
%
%2) Identification strategy: compare single mothers to single women
%without kids
%
%3) Key covariates in regression model: (a) EITC, (b) AFDC
%benefits, (c) Medicaid, (d) Waivers, (e) Training, (f) Child Care
%\end{slide}
%
%
%\begin{slide}%
%\begin{center}
%{\bf Meyer and Rosenbaum 2001}
%\end{center}
%From 1984-1996, the extra increase in single mom's relative to
%single women without kids is explained by:
%
%a) EITC expansion (60\%)
%
%b) Welfare max benefit reduction (AFDC and food stamps) (25\%)
%
%c) Medicaid if work (-10\%) (insignificant and wrong sign)
%
%d) Welfare waivers (time limits) 15\%
%
%e) Child care and training: 15\%
%
%\end{slide}%
%
%\begin{slide}%
%\begin{center}
%{\bf Eissa and Hoynes 2004}
%\end{center}
%
%1) EITC\ based on family rather than individual income
%
%2) Study married couples with low earnings, recognizing that EITC\
%\textit{reduces} their incentive to work
%
%3) Married women with husband earning \$10-15K are in the
%phase-out range and face high MTR's
%
%a) Payroll tax 15\%
%
%b) EITC phase-out 20\%
%
%c) State and federal income tax 0-20\%
%
%3) Similar identification strategy:\ compare those with and
%without kids
%
%\end{slide}
%
%
%\begin{slide}%
%\begin{center}
%{\bf Eissa and Hoynes: Results}
%\end{center}
%1) Conclude that EITC expansions between 1984 and 1996:
%
%a) Increased married men's labor force participation by 0.2\%
%
%b) Reduced married women's labor force participation by about 1\%
%
%2) Implies that the EITC is effectively subsidizing married
%mothers to stay at home and \textit{reducing} total labor supply
%for married households
%\end{slide}%

%\begin{slide}%
%\begin{center}
%{\bf Meyer and\ Sullivan 2004}
%\end{center}
%
%1) Examine the consumption patterns of single mothers and their
%families from 1984--2000 using CEX\ data
%
%2) Question:\ did single mothers' consumption fall because they
%lost welfare benefits and were forced to work?
%\end{slide}
%
%
%\begin{slide}
%\includepdf[pages={62, 63}]{laborsupply_attach.pdf}
%\end{slide}
%
%\begin{slide}%
%\begin{center}
%{\bf Welfare Reform Effects on Consumption}
%\end{center}
%Meyer and Sullivan '04 examine consumption of single mothers and their
%families from 1984--2000 using CEX\ data
%
%1) Material conditions of single mothers did not decline in 1990s,
%either in absolute terms or relative to single childless
%women or married mothers
%
%2) In most cases, evidence suggests that the material conditions
%of single mothers have improved slightly
%
%3) Question: is this because economy was booming in 1990s?
%
%4) Is workfare approach more problematic in current economy?
%[SNAP households surged from 12M in '07 to 20M in '10 while TANF
%households increased slightly from 1.7M in '07 to 1.85M in '10]
%
%\end{slide}%
%

\begin{slide}%
\begin{center}
{\bf Bunching at Kinks (Saez AEJ-EP'10)}
\end{center}
Key prediction of standard labor supply model: individuals should bunch
at (convex) kink points of the budget set

1) The only non-parametric source of identification for intensive
elasticity in a single cross-section of earnings is amount of
bunching at kinks creating by tax/transfer system

%2) All other tax variation is contaminated by heterogeneity in
%tastes

2) Saez '10 develops method of using bunching at kinks to estimate the
compensated income elasticity

%4) Idea: if this non-parametric method does not recover
%positive compensated elasticities, then little value in additional
%structure of nonlinear budget set models

Formula for elasticity: 
$ \varepsilon ^{c}=\frac{dz/z^{\ast }}{dt/(1-t)}=$ excess mass at
kink / change in NTR

$\Rightarrow$ Amount of bunching proportional to compensated elasticity

Blomquist et al. JPE'21: Bunching method requires making assumptions on counterfactual density 
(but testable using tax changes see Londono-Avila '18 below)
\end{slide}

\begin{slide}
\includepdf[pages={2,3}, scale=1.1]{laborsupply_attach.pdf}
\end{slide}

\begin{slide}
\includepdf[pages={134}, scale=1.1]{laborsupply_attach.pdf}
\end{slide}


\begin{slide}
\begin{center}
{\bf Bunching at Kinks (Saez AEJ-EP'10)}
\end{center}

1) Uses individual tax return micro
data (IRS\ public use files) from 1960 to 2004

2) Advantage of dataset over survey data: very little measurement error

3) Finds bunching around:

a) First kink point of the Earned Income Tax Credit (EITC),
especially for self-employed

b) At threshold of the first tax bracket where tax liability
starts, especially in the 1960s when this point was very stable

4) However, no bunching observed around all other kink points
\end{slide}

\begin{slide}
\includepdf[pages={4}]{laborsupply_attach.pdf}
\end{slide}

%\begin{slide}
%\includepdf[pages={19}]{pdf/chetty-harvard.pdf}
%\end{slide}

\begin{slide}
\includepdf[pages={5}, scale=1.4]{laborsupply_attach.pdf}
\end{slide}

\begin{slide}
\includepdf[pages={6}, scale=1.4]{laborsupply_attach.pdf}
\end{slide}

\begin{slide}
\includepdf[pages={7-8}, scale=1.4]{laborsupply_attach.pdf}
\end{slide}

%\begin{slide}
%\includepdf[pages={10}]{laborsupply_attach.pdf}
%\end{slide}

%\begin{slide}%
%\begin{center}
%{\bf Friedberg 2000: Social Security Earnings Test}
%\end{center}
%1) Uses CPS data on labor supply of retirees receiving Social
%Security benefits
%
%2) Studies bunching based on responses to Social Security earnings
%test
%
%3) Earnings test:\ phaseout of SS\ benefits with earnings above an
%exempt amount around \$14K/year
%
%4) Today: Phaseout rate varies by age group: 50\% (below 66), 33\%
%(age 66), 0 (above 66)
%
%5) Friedberg exploits 1983 reform (CPS age = age + 1):
%
%(a) Before: test up to age 71, no test at age 72+
%
%(b) After: test up to age 69, no test at age 70+
%
%\end{slide}
%
%\begin{slide}
%\includepdf[pages={11}]{laborsupply_attach.pdf}
%\end{slide}
%
%\begin{slide}
%\includepdf[pages={12}]{laborsupply_attach.pdf}
%\end{slide}
%
%\begin{slide}
%\begin{center}
%{\bf Friedberg 2000: Estimates}
%\end{center}
%1) Estimates elasticities using Hausman method, finds relatively
%large compensated and uncompensated elasticities
%
%2) Ironically, lost social security benefits are considered
%delayed retirement with an actuarial adjustment of future benefits
%$\Rightarrow$ (a) No kink if person has average life expectancy,
%(b) kink if person has less than average life expectancy
%
%3) So the one kink where we do find real bunching is actually not
%real! (people may not understand rules, or have myopia)
%
%\end{slide}

\begin{slide}
\begin{center}
{\bf Why not more bunching at kinks?}
\end{center}
1) True intensive elasticity of response may be small

2) Randomness in income generation process: Saez (1999) shows that
year-to-year income variation too small to erase bunching if
elasticity is large

3) Frictions: Adjustment costs and institutional constraints (Chetty, Friedman, Olsen, and Pistaferri QJE'11)

4) Information and salience

%Chetty-Friedman-Saez AER'13 show how information about EITC affects bunching at kink point


\end{slide}



%\begin{slide}%
%\begin{center}
%{\bf Chetty, Friedman, Olsen, and Pistaferri QJE'11}
%\end{center}
%1) If workers face adjustment costs, may not reoptimize in
%response to tax changes of small size and scope in short run
%
%a) Search costs, costs of acquiring information about taxes
%
%b) Institutional constraints imposed by firms (e.g. 40 hour week)
%that does not apply to the self-employed or workers with more flexibility (e.g.
%secondary earners)
%
%%2) Could explain why long-term macro studies find larger elasticities (see end of lecture)
%
%2) Question: How much are elasticity estimates affected by
%frictions?
%\end{slide}

%\begin{slide}%
%\begin{center}
%{\bf Chetty et al. 2011: Model}
%\end{center}
%1) Firms post jobs with different hours offers
%
%2) Workers draw from this distribution and must pay search cost to
%reoptimize
%
%3) Therefore not all workers locate at optimal choice
%
%4) Bunching at kink and observed responses to tax reforms
%attenuated
%
%\end{slide}

%\begin{slide}
%\begin{center}
%{\bf Chetty et al. 2011: Testable Predictions}
%\end{center}
%Model generates three predictions:
%
%1) \textbf{[Size]} Larger tax changes generate larger observed
%elasticities
%
%Large tax changes are more likely to induce workers to search for
%a different job
%
%2) \textbf{[Scope]} Tax changes that apply to a larger group of
%workers generate larger observed elasticities
%
%Firms tailor jobs to preferences of common workers
%
%3) \textbf{[Search Costs]} Workers with lower search costs exhibit
%larger elasticities from individual bunching
%
%\end{slide}

%\begin{slide}
%\includepdf[pages={13}]{laborsupply_attach.pdf}
%\end{slide}

%\begin{slide}%
%\begin{center}
%{\bf Chetty et al. 2011: Administrative data}
%\end{center}
%
%Matched employer-employee panel data with admin tax records for
%full population of Denmark matching employee-employer information
%
%Sample restriction: Wage-earners aged 15-70, 1994-2001
%
%Approximately 2.42 million people per year
%
%Important development in empirical micro in recent years: shift from survey
%data to administrative data (Card-Chetty-Feldstein-Saez '10 and Einav and Levin NBER'13]
%
%\end{slide}
%
%

%
%
%\begin{slide}
%\includepdf[pages={14, 17,19,18,30,33,39-40}]{laborsupply_attach.pdf}
%\end{slide}
%
%\begin{slide}%
%\begin{center}
%{\bf Chetty et al. 2011: Results}
%\end{center}
%
%1) Search costs attenuate observed behavioral responses
%substantially: find larger elasticities around large kink points
%
%2) Groups with more flexibility respond more (secondary earners,
%self-employed)
%
%3) Overall elasticities estimated from bunching are small in magnitude
%perhaps because frictions prevent full response)
%
%$\Rightarrow$ Bunching methods are good to detect behavioral responses but
%not necessarily to pin down magnitude of a long-run response to a large tax reform
%
%%3) Firm responses and coordination critical for understanding
%%behavior: individual and group elasticities may differ
%%significantly
%
%\end{slide}



\begin{slide}%
\begin{center}
{\bf EITC Behavioral Studies}
\end{center}
Evidence of response along extensive margin but little
evidence of response along intensive margin (except for
self-employed) $\Rightarrow$ Possibly due to lack of understanding
of the program

Qualitative surveys show that:

Low income families know about EITC and understand that they get a
tax refund if they work

However very few families know whether tax refund $\uparrow$ or
$\downarrow$ with earnings

Such confusion might be good for the government as the EITC
induces work along participation margin without discouraging work
along intensive margin (Liebman-Zeckhauser '04, Rees-Jones and Taubinsky '16)
\end{slide}


%\begin{slide}%
%\begin{center}
%{\bf Chetty and Saez AEJ-AP'13 EITC INFO}
%\end{center}
%
%1) Randomized experiment with tax preparer H\&R Block: tax pros
%[H\&R Block employees] provide EITC information to half of 43,000
%EITC filers in 2008 tax season
%
%2) Analyze whether earnings the following year are affected by the
%information treatment
%
%{\bf Key results:}
%
%1) Half of the tax pros induce treated clients to increase their
%EITC refunds by choosing an earnings level closer to the peak of
%the EITC schedule
%
%2) Rest of tax professionals seem to increase earnings of their
%treated clients across the board [possible
%
%3) Treatment effects are larger for the self-employed
%\end{slide}
%
%\begin{slide}
%\includepdf[pages={65,66,67,68,69}]{laborsupply_attach.pdf}
%\end{slide}
%

\begin{slide}%
\begin{center}
{\bf Chetty, Friedman, Saez AER'13 EITC heterogeneity}
\end{center}
Use US population wide tax return data since 1996 (through IRS special contract)
%\href{http://www.irs.gov/taxstats/article/0,,id=245237,00.html} {(weblink)}

1) Substantial heterogeneity in fraction of EITC recipients bunching (using self-employment)
across \textbf{geographical
areas}

$\Rightarrow$ Information on EITC varies across areas and grows overtime

2) Places with high self-employment EITC bunching display
{\bf wage earnings} distribution more concentrated around plateau

3) Omitted variable test: use birth of first child to test
causal eff`EITC on wage earnings

$\Rightarrow$ Evidence of wage earnings response to EITC along intensive
margin


\end{slide}

\begin{slide}
\includepdf[pages={91, 86-90, 92-95}]{laborsupply_attach.pdf}
\end{slide}

\begin{slide}%
\begin{center}
{\bf IMPLICATIONS OF ROLE OF INFORMATION}
\end{center}
{\bf Empirical work:}

Information should be a key explanatory variable in estimation of
behavioral responses to govt programs

When doing empirical project, always ask the question: did people affected
understand incentives?

Cannot identify structural parameters of preferences without
modeling information and salience

{\bf Normative analysis:}

Information is a powerful and inexpensive policy tool to affect
behavior

Should be incorporated into optimal policy design problems 
%(see e.g., Gabaix-Farhi '15)

\end{slide}


\begin{slide}%
\begin{center}
{\bf Value of Administrative data}
\end{center}

Key advantages of admin data (in most advanced countries such as Scandinavia):

1) Size (often full population available)

2) Longitudinal structure (can follow individual across years)

3) Ability to match wide variety of data (tax records, earnings records, family records, health records, education records)

US is lagging behind in terms of admin data access [hard to match across agencies]

Private sector also generates valuable \textbf{big data} (Google, Credit Bureaus, personnel/health data from large companies) 

\end{slide}


%\begin{slide}
%\begin{center}
%{\bf ADVANCE EITC}
%\end{center}
%Recipients get EITC with tax refund in a single annual refund in
%Feb year $t+1$ which seems suboptimal: (a) free interest loan to
%govt and (b) harder to smooth consumption [surveys show that
%primary use of tax refund is to pay overdue bills]
%
%Tax filers have option to use Advance EITC to get part of EITC in
%the paycheck by filing a W5 form with employer [reverse of tax
%withholding]: take up extremely low ($<$2\%)
%
%Possible explanation: (a) Information, (b) Lack of employer
%cooperation, (c) Risk of owing taxes if not EITC eligible, (d) Tax
%filers like big refunds, (e) Inertia (default is no Advance EITC)
%
%\end{slide}
%
%\begin{slide}
%\begin{center}
%{\bf ADVANCE EITC}
%\end{center}
%Jones AEJ-AP'10 carries a randomized experiment with large
%employer to encourage take-up and gets significant but very small
%take-up effect suggesting that (a) [Information] and (b) [Employer
%cooperation] cannot explain low take-up
%
%(d) [Love of refunds] seems plausible but (1) not supplied by
%market absent refunds [employers could also pay part of wages as
%annual lumpsum], (2) A-EITC use has not $\uparrow$ with EITC
%expansions
%
%(c) [Risk of owing taxes] and (e) [Inertia] are likely part of the
%explanation
%
%Interesting research topic: Have big tax refunds fuelled low
%income credit [tax refund loans, payday loans, etc.]? Are big
%refunds useful forced saving mechanisms?
%\end{slide}
%
%\begin{slide}%
%\begin{center}
%{\bf Other Behavioral Responses to Transfer Programs}
%\end{center}
%
%1)  Bitler, Gelbach, and Hoynes (2005), Kline and Tartari (2013) use
%randomized welfare experiments (pre-welfare reform) where
%control group had AFDC and treatment group had a TANF type welfare
%
%$\Rightarrow$ Shift in budget set has different effects on different people
%(some induced to work more, some less) consistent with standard labor
%supply model
%
%2) Other studies have examined effects of low-income assistance
%programs on other margins such as family structure (divorce rate,
%number of kids) and find limited effects
%
%3) Empirical work on tagging and in-kind programs is more limited
%and is an important area for further research
%
%\end{slide}




\begin{slide}
\begin{center}
{\bf Bunching at Notches: Kleven and Waseem '13}
\end{center}
Taxes and transfers sometimes also generate \textbf{notches} (=discontinuities) in the budget set

Such discontinuities should create bunching (and gaps) in the resulting distributions

Kleven and Waseem QJE'13 pioneered tax notch analysis in the case of the Pakistani
income tax where \textbf{average} tax rate jumps

$\Rightarrow$ Bunching below the notch and gap in density just above the notch

Recently Londono-Velez and Avila (2018) use notch analysis to study wealth tax in Columbia

They show clean prior-year counterfactual overcoming the Blomquist et al. '21 critique

%Example: Pakistani income tax creates notches because \textbf{average} tax rate jumps
%$\Rightarrow$ Bunching below the notch and gap in density just above the notch
%
%\textbf{Empirically:} Kleven and Waseem QJE'13 find
%evidence of bunching (primarily among self-employed)
%but size of the response is quantitatively small
%
%Large fraction of taxpayers are unresponsive to notch likely  due to lack of information

\end{slide}




\begin{slide}
\includepdf[pages={103,104}]{laborsupply_attach.pdf}
\end{slide}

\begin{slide}
\includepdf[pages={118, 116, 117}]{laborsupply_attach.pdf}
\end{slide}

\begin{slide}
\begin{center}
{\bf Bunching at notches: elasticity estimation}
\end{center}
With optimization frictions (lack of information, costs of adjustment),
a fraction of individuals fail to respond to notch

Kleven-Waseem use empirical density in the theoretical gap area to measure the fraction
of unresponsive individuals

This allows them to back up the frictionless elasticity (i.e. the elasticity
among responsive individuals)

The frictionless elasticity is much higher than the reduced form elasticity
but remains still relatively modest

%Additional notch studies: Best and Kleven '14 on UK housing purchase tax (stamp duty), Kopczuk-Munroe AEJ'15
%on NY-NJ Mansion tax [also find evidence of bunching responses]

\end{slide}



\begin{slide}%
\begin{center}
{\bf Many Recent Bunching Studies}
\end{center}

Bunching method applied to many settings with nonlinear budgets with convex kink points or notches (Kleven '16 survey):

\small

$\bullet$ Individual tax (Bastani-Selin '14 Sweden, Mortenson-Whitten '16 US) 

$\bullet$ Payroll tax (Tazhidinova '15 on UK)

$\bullet$ Corporate tax (Devereux-Liu-Loretz '14, Bachas-Soto '17)

$\bullet$ Wealth tax (Seim '17, Jakobsen et al. '17, Londono-Velez and Avila '18)

$\bullet$ Health spending (Einav-Finkelstein-Schrimpf '13 on Medicare Part D)

$\bullet$ Retirement savings (401(k) matches)

$\bullet$ Retirement age (Brown '13 on California Teachers)

$\bullet$ Housing transactions (Best and Kleven, 2017)

General findings: 

(1) clear bunching when information is salient and outcome easily manipulable.
Bunching comes most often from avoidance/evasion rather than real behavior.

(2) bunching is almost always small relative to conventional
elasticity estimates

%\small
%Extension: notches in budget sets have also been used to estimate behavioral responses
%(see Kleven-Waseem QJE'13). 

\end{slide}


\begin{slide}
\begin{center}
{\bf Responses to Corporate Tax Notches: Bachas-Soto '18}
\end{center}
Bachas and Soto '18 exploit the notched Costa Rica corporate tax system
to estimate compellingly the effects of the corporate tax rate on reported profits

Corporate tax applies to profits = revenue minus costs

But tax rate depends on size of revenue with 3 rates: 10\%, 20\%, 30\%

1) Firms bunch at the notches to benefit from the lower rates

2) Most importantly: clear evidence that profit rates (profits/revenue) is strongly affected
by the corporate tax rate



\end{slide}




\begin{slide}
\includepdf[pages={119, 120, 121}]{laborsupply_attach.pdf}
\end{slide}

%\begin{slide}%
%\begin{center}
%{\bf Intertemporal Labor Supply}
%\end{center}
%
%
%1) What parameter do reduced-form regressions of labor supply on
%wages or taxes identify?
%
%2) MaCurdy critique: reduced-form studies did not identify any
%parameter of interest in a dynamic model
%
%3) Instead, estimate a mix of income effects, intertemporal
%substitution effects, and compensated wage elasticities
%
%4) MaCurdy (1981) develops a structural estimation method (two
%stage budgeting) to identify preference parameters in a life-cycle
%model of labor supply (see Chetty '06 for simple exposition)
%\end{slide}


%\begin{slide}%
%\begin{center}
%{\bf Life Cycle Model of Labor Supply}
%\end{center}
%General model is of the form:
%\begin{eqnarray*}
%&&U(c_{0},..,c_{T},l_{0},..,l_{T}) \\
%&& s.t. \:\:\: A_{0}+\sum w_{t}l_{t}/(1+r)^{t}\geq \sum
%c_{t}/(1+r)^{t} \:\:\: (\lambda )
%\end{eqnarray*}
%Key Assumption for inter-temporal budget is {\bf no credit
%constraints}
%
%$\Rightarrow$ First order conditions:%
%\begin{eqnarray*}
%U_{l_{t}}(c_{0},..,c_{T},l_{0},..,l_{T})+\lambda w_{t}/(1+r)^{t} &=&0 \\
%U_{c_{t}}(c_{0},..,c_{T},l_{0},..,l_{T})-\lambda /(1+r)^{t} &=&0
%\end{eqnarray*}
%In the general case, $l_{t}(A_{0},w_{0},..,w_{T})$ same as the
%multi-good choice -- no generic results
%\end{slide}

%\begin{slide}%
%\begin{center}
%{\bf Life Cycle Model with Time Separability}
%\end{center}
%Assuming time separability, the individual maximizes:
%\[
%U=\sum_{t=0}^{T}\beta ^{t}u(c_{t},l_{t})
%\]
%\[ s.t. \:\:\: \sum_{t=0}^{T}
%c_{t}/(1+r)^{t}  \leq A_{0}+\sum_{t=0}^{T} w_{t}l_{t}/(1+r)^{t} \:\:\: (\lambda ) \]
%
%Leads to first order conditions%
%\begin{eqnarray*}
%l_{t} &:&\beta ^{t}u_{l_{t}}+\lambda w_{t}/(1+r)^{t}=0 \\
%c_{t} &:&\beta ^{t}u_{c_{t}}-\lambda /(1+r)^{t}=0
%\end{eqnarray*}
%%Combining yields: $-u_{l}=w_{t}u_{c}$
%
%Intratemporal FOC same as in static model: $-u_{l}=w_{t}u_{c}$
%
%Intertemporal FOC: $u_{c_{t}}/u_{c_{t+1}}=\beta (1+r)$ [called Euler equation]
%\end{slide}
%
%\begin{slide}
%\begin{center}
%{\bf Dynamic Life Cycle Model: Policy Rules}
%\end{center}
%1) $\lambda =u_{c_{0}}$ is the marginal utility of initial
%consumption
%
%2) The two first order conditions imply that%
%\begin{eqnarray*}
%l_{t} &=&l(w_{t},\lambda /(\beta (1+r))^{t}) \\
%c_{t} &=&c(w_{t},\lambda /(\beta (1+r))^{t})
%\end{eqnarray*}
%3) Current labor and consumption choice depends on current $w_{t}$
%
%4) All other wage rates and initial wealth enter only through the
%budget constraint multiplier $\lambda $ (MaCurdy 1981)
%
%%5) Easy to see for separable utility [$u(c)$ concave, $v(l)$
%%convex]:
%%\begin{eqnarray*}
%%u(c,l) &=&u(c)-v(l) \\
%%\Rightarrow v^{\prime }(l_{t}) &=&\lambda w_{t}/[\beta (1+r)]^{t} \\
%%\Rightarrow l_{t} &=&v^{\prime -1}(\lambda w_{t}/[\beta (1+r)]^{t})
%%\end{eqnarray*}
%5) Sufficiency of $\lambda $ greatly simplifies solution to ITLS\
%model
%\end{slide}

%\begin{slide}%
%\begin{center}
%{\bf Dynamic Life Cycle Model: Frisch Elasticity}
%\end{center}
%Frisch intertemporal labor supply elasticity defined as:
%\[
%\delta =\left (\frac{w_{t}}{l_{t}} \right )\frac{\partial l}{\partial w_{t}}|_{\lambda }
%\]
%
%Experiment: change wage rate in one period only, holding all other
%wages constant [so that $\lambda$ stays constant]
%
%Can show that $\delta >0$: work more today to take advantage of
%temporarily higher wage
%
%%In separable case:
%%\begin{eqnarray*}
%%v'(l_{t}) &=&\lambda w_{t}/[\beta (1+r)]^{t} \\
%%&\Rightarrow &\frac{\partial l}{\partial w_{t}}|_{\lambda }=\frac{\lambda }{%
%%[\beta (1+r)]^{t}v^{\prime \prime }(l_{t})}>0
%%\end{eqnarray*}
%\end{slide}

%\begin{slide}%
%\begin{center}
%{\bf Frisch vs Hicksian Elasticity:\ Illustrative Example}
%\end{center}
% Suppose that you are paid a piece rate
%
% It takes 1 hour of work to make a piece
%
% You usually work from 8am-12pm and 1pm-5pm.
%
% Suppose your employer tells you that the piece rate will be twice
%as high only during the 12pm-1pm time slot
%
% What do you do?
%
%$\rightarrow $Have lunch earlier at 11am-12pm and work from
%12pm-1pm
%
%\end{slide}

%\begin{slide}
%\begin{center}
%{\bf ITLS and Income Effects}
%\end{center}
%Single inter-temporal budget constraint: $$ \sum c_{t}/(1+r)^{t} \leq A_{0}+\sum
%w_{t}l_{t}/(1+r)^{t}$$ $\Rightarrow$
%Receiving \$ $M$ in year $0$ vs. \$ $(1+r)^t \cdot M$ in year $t$
%has the same impact on labor supply
%
%Temporary transfer has a small effect on labor in {\bf all}
%periods
%
%In reality, temporary transfers seem to have large effects on
%labor supply [e.g., severance payments, Card-Chetty-Weber QJE'08]
%$\Rightarrow$
%
%(1) Many people are credit constrained: static labor supply model
%might be a better depiction of reality
%
%(2) People might not make intertemporal choices as in ITLS model
%[behavioral economics]
%\end{slide}


%\begin{slide}
%\begin{center}
%{\bf Dynamic Life Cycle Model: Three Types of Wage Changes}
%\end{center}
%1) Evolutionary change:\ movements along profile (life-cycle)
%
%2) Parametric change: temporary tax cut
%
%3) Profile shift: changing the wage rate in all periods
%
%a) Equivalent to a permanent parametric change
%
%b) Implicitly the elasticity that static studies estimate with
%unanticipated tax changes
%\end{slide}
%
%\begin{slide}
%\includepdf[pages={70}]{laborsupply_attach.pdf}
%\end{slide}

\begin{slide}
\begin{center}
{\bf Intertemporal Labor Supply: High Frequency}
\end{center}
Frisch elasticity $e^F$: changing wages in a single period and keeping
marginal utility of income $\lambda$ constant

Compensated static elasticity $e^C$: changing wages in all periods but
keeping utility constant

Uncompensated static elasticity $e^U$: changing wages in all periods
with no compensation
\[ \text{Theoretically: } e^F > e^C > e^U \]

Frisch elasticity is of central interest for calibration of macro
business cycle models: 

Real business cycle model requires huge
elasticity to generate realistic employment fluctuations

\end{slide}





%\begin{slide}
%\begin{center}
%{\bf Frisch vs. Compensated vs. Uncompensated Elasticities}
%\end{center}
%Intertemporal substitution: Frisch elasticity $\geq$ Compensated
%static elasticity
%
%Income effects: Compensated static elasticity $\geq$ Uncompensated
%static elasticity
%
%Difference in Frisch and Compensated elasticities also loosely
%related to anticipated vs. unanticipated changes
%
%Looney and Singhal (2007) exploit this reasoning to identify
%Frisch elasticity [MTR changes predictably when filers loose a
%child exemption]
%
%Frisch elasticity is of central interest for calibration of macro
%business cycle models
%\end{slide}

\begin{slide}
\begin{center}
{\bf Intertemporal substitution: Tax Holiday in Iceland}
\end{center}

In 1987, Iceland transitioned from paying taxes on previous year's income to current income

To avoid double taxation during transition, no tax charged over 1987 incomes

Average tax rate of 14.5\% in 1986, 0\% in 1987, 8\% in 1988

Reform announced in late 1986 $\Rightarrow$ unanticipated temporary tax change

Temporary change in incentives $\Rightarrow$ ideal quasi-experiment to intertemporal substitution elasticity
(work hard in 1987, take a break in 1986 or 1988)

\small
Bianchi et al. AER'01 look at employment effects [hard to know what counterfactual is]

Siggurdsson (2019) compares high (big tax cut) vs. low earners (small tax cut) and finds larger response among high earners [but possible
that high earners are more elastic to start with]

\end{slide}

\begin{slide}
\includepdf[pages={83}]{laborsupply_attach.pdf}
\end{slide}


\begin{slide}
\begin{center}
{\bf Tax Holiday in Swiss Cantons}
\end{center}
Martinez, Saez, Siegenthaler '21 study tax holidays in Swiss cantons also created by
a transition to pay-as-you earn

Key advantage: different cantons transitioned at different times (creating staggered tax holidays across cantons)

Key findings: 

(a) precise zero effect on extensive margin

(b) some effects on intensive margin for high wage earners and self-employed (possibly avoidance rather than real)

Why smaller effects in Switzerland than Iceland? Iceland sold tax holiday as opportunity to work more (Switzerland did not)

\end{slide}

\begin{slide}
\includepdf[pages={135-142}]{laborsupply_attach.pdf}
\end{slide}

%\begin{slide}
%\begin{center}
%{\bf Income Tax Holiday for High Earners in Argentina}
%\end{center}
%
%In late August 2013, Argentina implemented an income tax holiday for wage earners with
%monthly earnings below \$15K in Jan-Aug 2013 (predetermined), lasted until Feb 2016
%
%Tortarolo (2019) proposes an \textbf{Regression Discontinuity (RD)} design using monthly
%social security admin earnings data
%
%Compares workers with earnings just below \$15K (treatment) vs. workers with earnings
%just above \$15K
%
%Evidence shows a pretty precisely estimated zero but some small significant effect on overtime hours
%
%$\Rightarrow$ High wage earners cannot respond by themselves without help from employers
%
%\end{slide}
%
%\begin{slide}
%\includepdf[pages={143-149}, scale=1.03]{laborsupply_attach.pdf}
%\end{slide}
%






%\begin{slide}%
%\begin{center}
%{\bf MaCurdy 1983}
%\end{center}
%1) Structural estimate using panel data for men and within-person
%wage variation
%
%2) Find both Frisch and compensated wage elasticity of around 0.15
%
%3) But his wage variation is not exogenous
%
%\end{slide}


%\begin{slide}%
%\begin{center}
%{\bf Pencavel 2002}
%\end{center}
%1) Instruments with trade balance interacted with schooling and
%age
%
%2) Frisch elasticity: 0.2
%
%3) Uncompensated wage elasticity: 0-0.2
%
%Instruments not credibly exogenous but results closer to
%structurally interpretable parameters
%\end{slide}
%
%\begin{slide}%
%\begin{center}
%{\bf Critique of ITLS\ models}
%\end{center}
%$\bullet$ Card critique of value of ITLS\ model
%
%a) Fails to explain most variation in hours over lifecycle
%
%b) Sheds little light on profile-shift elasticities that we care
%about for policy
%
%$\bullet$ Core \textquotedblleft structural vs.
%reduced-form\textquotedblright\ divide in applied microeconomics:
%Trade off between credible identification and well defined
%theoretical framework
%\end{slide}



%\begin{slide}%
%\begin{center}
%{\bf Blundell, Duncan, and Meghir 1998}
%\end{center}
%1) Good combination of structural and reduced form methods on
%labor supply
%
%2) Argue against standard DD approach, where treatment/control
%groups are endogenously defined
%
%a) Reduced tax rate may pull households from low income group to
%high income group
%
%b) Need group definitions that are stable over time
%
%3) Use birth cohort (decade) interacted with education (e.g. high
%school or more)
%
%\end{slide}
%
%
%\begin{slide}
%\begin{center}
%{\bf Blundell, Duncan, and Meghir 1998}
%\end{center}
%
%1) Construct group-level labor supply measures for women in
%couples
%
%2) Measure how labor supply co-varies with wages rates net of
%taxes in the UK in 1980s
%
%3) Importantly, tax reforms during this period affected groups
%very differently
%
%4) Use consumption data as a control for permanent income
%
%5) Can therefore obtain a structurally interpretable ($\lambda $
%constant) estimate
%
%\end{slide}
%
%\begin{slide}
%\includepdf[pages={71}]{laborsupply_attach.pdf}
%\end{slide}
%
%\begin{slide}%
%\begin{center}
%{\bf Blundell, Duncan, and Meghir: Results}
%\end{center}
%1) Compensated wage elasticities: 0.15-0.3, depending on number of
%kids
%
%2) No income effects when no kids, moderate income effects when
%kids present
%
%3) Identification assumption is common trends across
%cohort/education groups
%
%4) However, reforms in 80s went in opposite directions at
%different times $\rightarrow$ Secular trends cannot explain
%everything
%
%5) See Blundell and MaCurdy (1999) for additional ITLS estimates
%\end{slide}

%\begin{slide}
%\begin{center}
%{\bf Intertemporal Substitution: High Frequency Studies}
%\end{center}
%1) Recent literature focuses on high frequency substitution
%
%2) Focus on groups with highly flexible and well measured labor
%supply such as:
%
%a) cab drivers [Camerer et al. QJE'97, Farber JPE'05, AER-PP'08,
%Crawford-Meng '09]: debate on whether cab drivers are rational
%or have a daily income target
%
%b) stadium vendors [Oettinger JPE'99]
%
%c) cycling messengers randomized experiment [Fehr-Goette AER'07]
%\end{slide}

%\begin{slide}%
%\begin{center}
%{\bf Camerer et al. QJE'97}
%\end{center}
%Examine how variation across days in wage rate for cab drivers
%(arising from variation in waiting times) correlates with hours
%worked
%
%a) Striking finding: strong negative effect
%
%b) Interpret this as \textquotedblleft target
%earning\textquotedblright\ -- strongly contradicts standard
%intertemporal labor supply model
%
%c) Would suggest very counter intuitive effects for temporary tax
%changes, etc.
%
%\end{slide}%
%%EndExpansion
%
%\begin{slide}
%\includepdf[pages={72,73}]{laborsupply_attach.pdf}
%\end{slide}

%\begin{slide}
%\begin{center}
%{\bf Farber: Division Bias}
%\end{center}
%Argues that Camerer et al. evidence of target earning behavior is
%driven by econometric problems
%
%Camerer et al. regression specification:%
%\[
%h_{it}=\alpha +\beta e_{it}/h_{it}+\varepsilon _{it}
%\]
%
%Camerer et al. recognize this and try to instrument using average
%daily wage $\bar{w}_t$ across all drivers
%
%But there may be a random component to hours at the group level
%(e.g., good weather makes job more pleasant $\Rightarrow$ more
%hours and smaller wages at the group level)
%
%$\Rightarrow$ Spuriously find a negative association between
%average daily wage and average hours
%\end{slide}


%\begin{slide}%
%\begin{center}
%{\bf Farber: Within-Day Volatility}
%\end{center}
%Farber's alternative test for target earnings: hazard model
%$Quit=f(cum\_hours,cum\_inc)$
%
%Result: main determinant of quitting is hours worked (fatigue),
%NOT cumulative income $\Rightarrow$ Rejects target earning, but
%does not yield ITLS\ estimate
%
%Two other studies find positive ITLS:
%
%a) Bicycle messengers (randomized experiment with 25\% wage
%subsidy for 4 weeks): work more days and earn more when wages
%higher but effort per day $\downarrow$ [fatigue effect]
%
%b) Baseball stadium vendors (work more in high attendance games)
%
%But such structural parameters are not of
%direct interest to public finance because they are too high
%frequency
%\end{slide}

%\begin{slide}%
%\begin{center}
%{\bf Use of Frisch elasticity in RBC macro models}
%\end{center}
%
%Real business cycle (RBC) models motivated by the fact
%that output fluctuates more than hours worked
%
%{\bf Short-Run:} Hours worked are strongly pro-cyclical
%[unemployment in recessions and overtime in booms]
%
%RBC models do not have involuntary unemployment
%[questionable assumption]
%
%Shock in technology affects wages
%$\Rightarrow$ Variation in hours due to labor supply $\Rightarrow$ Output 
%varies more than hours
%
%$\Rightarrow $ Frisch elasticity must be very large (above 1.5)
%for Real business cycle macro-models to work [Prescott Nobel lecture JPE'06]
%
%Micro labor supply evidence does not support such a high Frisch elasticity
%
%\end{slide}

\begin{slide}
\begin{center}
{\bf SOCIAL DETERMINANTS OF LABOR SUPPLY}
\end{center}
Concern that taxes funding social state could discourage work

\textbf{Standard econ view:} labor supply $l(w,R)$ coming out of \\ $\max u(\underset{+}{c},\underset{-}{l})$ st $c=wl+R$
is highly incomplete

\textbf{Social determinants of labor supply:}

a) Youth labor is regulated by labor laws/education

b) Old age labor regulated by retirement programs

c) Female market labor driven by norms + child care policy

d) Hours of work regulated by overtime + vacation mandates

Social labor supply with disutility for youth, old, overtime labor

\end{slide}

\begin{slide}
\includepdf[pages={152-153, 155-157}]{laborsupply_attach.pdf}
\end{slide}


%\begin{slide}
%\begin{center}
%{\bf Macro Long-Run Evidence}
%\end{center}
%1) Macroeconomists also estimate elasticities by examining
%long-term trends/cross-country comparisons\bigskip
%
%2) Identification more questionable but estimates perhaps more
%relevant to long-run policy questions of interest\bigskip
%
%3) Use aggregate hours data and aggregate measures of taxes
%(average tax rates)\bigskip
%
%4) Highly influential in calibration of macroeconomic
%models\bigskip
%\end{slide}
%
%\begin{slide}%
%\begin{center}
%{\bf Trend-based Estimates and Macro Evidence}
%\end{center}
%{\bf Long-Run:} US real wage rates multiplied by about 5 from 1900 to
%present due to economic growth
%
%Aged 25-54 male hours have fallen 25\% and then stabilized (Ramey and Francis AEJ-macro '09)
%
%$\Rightarrow$ Uncompensated hours of work elasticity is small ($<.1$)
%
%However, taxes are rebated as transfers so can still have labor
%supply effects if large compensated elasticity/income effects
%
%Alternative more plausible story: utility depends on relative consumption and labor supply is a social norm
%(reference point labor supply theory)
%\end{slide}

\begin{slide}
\includepdf[pages={85}]{laborsupply_attach.pdf}
\end{slide}



%\begin{slide}
%\begin{center}
%{\bf Long-run cross-country panel: Prescott 2004}
%\end{center}
%Uses data on hours worked by country in 1970 and 1995 for 7 OECD\
%countries [total hours/people age 15-64]
%
%Technique to identify elasticity:\ calibration of GE\ model
%
%Rough intuition: posit a labor supply model, e.g.%
%\[
%u(c,l)=c-\frac{l^{1+1/\varepsilon }}{1+1/\varepsilon }
%\]
%
%Finds that elasticity of $\varepsilon =1.2$ best matches time
%series and cross-sectional patterns
%
%Note that this is analogous to a regression without controls for
%other variables
%
%Results developed in subsequent calibrations by
%Ohanina-Raffo-Rogerson JME'08 and others using more data
%
%\end{slide}
%
%\begin{slide}
%\includepdf[pages={74}]{laborsupply_attach.pdf}
%\end{slide}


%\begin{slide}
%\begin{center}
%{\bf Davis and Henrekson 2005}
%\end{center}
%Run regressions of hours worked on tax variables with various
%controls
%
%Some panel evidence, but primarily cross-sectional
%
%Separate intensive and extensive margin responses
%
%\end{slide}
%
%\begin{slide}
%\includepdf[pages={75,76}]{laborsupply_attach.pdf}
%\end{slide}

%\begin{slide}
%\begin{center}
%{\bf Reconciling Micro and Macro Estimates}
%\end{center}
%
%Recent interest in reconciling micro and macro elasticity
%estimates (see Chetty-Guren-Manoli-Weber '13)
%
%Three potential explanations
%
%a)  Statistical Bias: culture differs in countries
%with higher tax rates [Alesina, Glaeser, Sacerdote 2005, Steinhauer 2018 for
%Swiss communities by language]
%
%b) Macro-elasticity captures long-term response which could be larger than
%short-term response (frictions, etc. Chetty '12).
%
%c) Other programs: retirement, education affect labor supply at
%beginning and end of working life (Blundell-Bozio-Laroque '13) and child
%care affecting mothers (Kleven JEP'14)
%
%\end{slide}

%\begin{slide}
%\begin{center}
%{\bf Blundell-Bozio-Laroque '13}
%\end{center}
%
%Strong evidence that variation in aggregate hours of work across
%countries happens among the young and the old: (a) schooling-work
%margin (b) presence of young children (for women), (c) early
%retirement
%
%Serious cross-country time series analysis would require to put
%together a better tax wedge by age groups which includes all those
%additional govt programs [welfare, retirement, child care]
%
%This has been done quite successfully in the case of retirement by
%series of books by Gruber and Wise, {\em Retirement around the
%world}
%
%$\Rightarrow$ Need to develop a more comprehensive international /
%time series database of tax wedges by age and family types
%
%\end{slide}
%
%\begin{slide}
%\includepdf[pages={77,78}]{laborsupply_attach.pdf}
%\end{slide}
%
%\begin{slide}
%\includepdf[pages={79}]{laborsupply_attach.pdf}
%\end{slide}
%
%\begin{slide}
%\includepdf[pages={80,82}]{laborsupply_attach.pdf}
%\end{slide}
%
%\begin{slide}
%\includepdf[pages={81}]{laborsupply_attach.pdf}
%\end{slide}



\begin{slide}
\begin{center}
{\bf Long-term effects: Evidence from the Israeli Kibbutz}
\end{center}
Abramitzky '18 book based on series of academic papers

Kibbutz are egalitarian and socialist communities in Israel, thrived for 
almost a century within a more capitalist society

1) Social sanctions on shirkers effective in small communities with limited privacy

2) Deal with brain drain exit using communal property as bond 

3) Deal with adverse selection in entry with screening and trial period 

4) Perfect sharing in Kibbutz has negative effects on high school students performance
but effect is small in magnitude (concentrated among kids with low education parents)

\end{slide}

\begin{slide}
\begin{center}
{\bf Long-term effects: Evidence from the Israeli Kibbutz}
\end{center}
Abramitzky-Lavy ECMA'14 show that 
high school students study harder once their 
kibbutz shifts away from equal sharing

Uses a DD strategy: pre-post reform and comparing reform Kibbutz to non-reform Kibbutz.
Finds that

1) Students are 3\% points more likely to graduate

2)  Students are 6\% points more likely to achieve a matriculation 
certificate that meets university entrance requirements

3)  Students get an average of 3.6 more points in their exams

Effect is driven by students whose parents have 
low schooling; larger for males; stronger in 
kibbutz that reformed to greater degree

\end{slide}

\begin{slide}
\begin{center}
{\bf Culture of Welfare across Generations}
\end{center}
Conservative concern that welfare promotes a culture of dependency: kids growing up
in welfare supported families are more likely to use welfare

Correlation in welfare use across generations is obviously not necessarily causal 

Dahl, Kostol, Mogstad QJE'2014 analyze causal effect of parental use of Disability Insurance (DI)
on children use (as adults) of DI in Norway

Identification uses random assignment of judges to denied DI applicants who appeal [some judges are severe, some lenient] 

Find evidence of causality: parents on DI increases odds of kids on DI over next 5 years by 6 percentage points

\small
Mechanism seems to be learning about DI availability rather than reduced stigma from using DI [because no effect on other
welfare programs use]

\end{slide}

\begin{slide}
\includepdf[pages={101}]{laborsupply_attach.pdf}
\end{slide}

%\begin{slide}
%\begin{center}
%{\bf Social Determinants of Labor Supply}
%\end{center}
%Strong evidence that labor supply is not purely an individual decision based on
%standard invariant utility $u(c,l)$
%
%Firms cooperation needed in most cases 
%
%Social norms play large role: e.g. women's market labor supply
%
%US women labor force participation during World War II: 50\% increase from '40 to '45
%(2/3 reversed afterwards)
%
%Child penalties in female earnings vary a lot across countries (Kleven et al. AEA PP'19) and are not due solely
%to monetary incentives
%
%Responses to taxes and transfers likely affected by social norms
%\end{slide}



\begin{slide}
\begin{center}
{\bf REFERENCES}
\end{center}
{\small

Abramitzky, Ran \emph{The Mystery of the Kibbutz: How Socialism Succeeded}, Princeton: Princeton University Press,
2018 \href{http://elsa.berkeley.edu/~saez/course/Abramitzky_book_presentation.pdf} {(web)}

Abramitzky, Ran and Victor Lavy, 2014 ``How Responsive is Investment in Schooling to Changes in Redistributive Policies 
and in Returns?'', Econometrica, 82(4), 1241-1272 \href{http://elsa.berkeley.edu/~saez/course/Abramitzky-Lavy14.pdf} {(web)}

Alesina, A., E. Glaeser, and B. Sacerdote ``Work and Leisure in the U.S. and Europe: Why So Different?'', NBER Macroeconomics Annual 2005. \href{http://www.nber.org/papers/w11278.pdf} {(web)}

Ashenfelter, O. and M. Plant ``Non-Parametric Estimates of the Labor Supply Effects of Negative Income Tax Programs'', Journal of Labor Economics, Vol. 8, 1990, 396-415. \href{http://links.jstor.org/stable/pdfplus/2535218.pdf} {(web)} 

Bachas, Pierre and Mauricio Soto. ``Not(ch) Your Average Tax System: Corporate Taxation Under Weak Enforcement,'' World Bank Policy Research Working Paper 8524, 2018.
\href{http://elsa.berkeley.edu/~saez/course/bachas-soto18notch.pdf} {(web)}

Bastani, Spencer and Hakan Selin, ``Bunching and non-bunching at kink points of the Swedish tax schedule,'' Journal of Public Economics, 109, 2014, 36-49. 
\href{http://elsa.berkeley.edu/~saez/course/bastani-selin14.pdf} {(web)}

Bertrand, M., E. Duflo and S. Mullainhatan ``How Much Should we Trust Differences-in-Differences Estimates?'', Quarterly Journal of Economics, Vol. 119, 2004, 249-275. \href{http://links.jstor.org/stable/pdfplus/25098683.pdf} {(web)}

Best, Michael and Henrik Kleven ``Housing Market Responses to Transaction Taxes: Evidence from Notches and Stimulus in the UK,'' Review of Economic Studies 2017 forthcoming
\href{http://elsa.berkeley.edu/~saez/course/best-kleven_landnotches_may2014.pdf} {(web)}

Bianchi, M., B. R. Gudmundsson, and G. Zoega. 2001. ``Iceland's Natural
Experiment in Supply-Side Economics,'' American Economic Review, 91(5), 1564-79.
\href{http://www.jstor.org/stable/pdfplus/2677941.pdf} {(web)}

Bitler, M. J. Gelbach and H. Hoynes ``What Mean Impacts Miss: Distributional Effects of Welfare Reform Experiments'', American Economic Review, Vol. 96, 2006, 988-1012. \href{http://links.jstor.org/stable/pdfplus/30034327.pdf} {(web)}

Bitler, M.  and H. Hoynes ``The State of the Safety Net in the Post-Welfare Reform Era''  Brookings Papers on Economic Activity Fall 2010, 71-127  \href{http://elsa.berkeley.edu/~saez/course/bitler-hoynesBPEA11.pdf} {(web)}

Blau, F. and L. Kahn ``Changes in the Labor Supply Behavior of Married Women: 1980-2000'', Journal of Labor Economics, Vol. 25, 2007, 393-438. \href{http://www.jstor.org/stable/pdfplus/10.1086/513416.pdf} {(web)}

Blomquist, S. ``Restrictions in labor supply estimation: Is the MaCurdy critique correct?'', Economics Letters, Vol. 47, 1995, 229-235 \href{http://elsa.berkeley.edu/~saez/course/Blomquist_EL(1995).pdf} {(web)}

Blomquist, Soren,Whitney Newey, Anil Kumar, Che-Yuan Liang ``On Bunching and Identification of the Taxable Income Elasticity'', Journal of Political Economy 2021 \href{http://elsa.berkeley.edu/~saez/course/blomquist-newey-etalJPE21.pdf} {(web)} 

Blundell, Richard, Antoine Bozio, and Guy Laroque. 2013. ``Extensive and Intensive Margins of Labour Supply: Work and Working Hours in the US, UK and France,'' Fiscal Studies, 34(1), 1-29 \href{http://elsa.berkeley.edu/~saez/course/blundell-bozio-laroque11labor.pdf} {(web)}

Blundell, R., A. Duncan and C. Meghir ``Estimating Labor Supply Responses Using Tax Reforms'', Econometrica, Vol. 66, 1998, 827-862. \href{http://links.jstor.org/stable/pdfplus/2999575.pdf} {(web)}

Blundell, R. and T. MaCurdy ``Labor supply: a review of alternative approaches'', in the Handbook of Labor Economics, Vol. 3A, O. Ashenfelter and D. Card, eds. Amsterdam: Elsevier Science 1999. \href{http://elsa.berkeley.edu/~saez/course/Blundell-MaCurdy_Handbook.pdf} {(web)}

Brown, K. ``The Link between Pensions and Retirement Timing: Lessons from California Teachers'', 
Journal of Public Economics, 98, 2013, 1--14.
2007 \href{http://elsa.berkeley.edu/~saez/course/brown_jpube13.pdf} {(web)}

Camerer, C., L. Babcock, G. Loewenstein and R. Thaler ``Labor Supply of New York City Cabdrivers: One Day at a Time'', Quarterly Journal of Economics, Vol. 112, 1997, 407-441. \href{http://links.jstor.org/stable/pdfplus/2951241.pdf} {(web)}

Card, David, Raj Chetty, Martin Feldstein, and Emmanuel Saez ``Expanding Access to Administrative Data for Research in the United States,'' White Paper for NSF 10-069 call for papers on "Future Research in the Social, Behavioral, and Economic Sciences" September 2010. \href{http://elsa.berkeley.edu/~saez/card-chetty-feldstein-saezNSF10dataaccess.pdf} {(web)}

Card, D.,R. Chetty, and A. Weber, ``Cash-on-Hand and Competing Models of Intertemporal Behavior: New Evidence from the Labor Market'', Quarterly Journal of Economics, Vol. 122, 2007, 1511-1560. \href{http://elsa.berkeley.edu/~saez/course/Card,Chetty,Weber_QJE(2007).pdf} {(web)}

Card, David, and Dean R. Hyslop. 2005. ``Estimating the Effects of a Time-Limited
Earnings Subsidy for Welfare-Leavers'' Econometrica, 73(6), 1723-70.
\href{http://www.jstor.org/stable/pdfplus/3598750.pdf} {(web)}

Cesarini, David, Erik Lindqvist, Matthew J. Notowidigdo, Robert Ostling. 2017 ``The Effect of Wealth on Individual and Household Labor Supply: Evidence from Swedish Lotteries'', American Economic Review
\href{http://elsa.berkeley.edu/~saez/course/cesarinietalAER17.pdf} {(web)} 

Chetty, R. ``A New Method of Estimating Risk Aversion'', The American Economic Review, Vol. 96, 2006, 1821-1834. \href{http://www.jstor.org/stable/pdfplus/30034997.pdf} {(web)}

Chetty, Raj. 2012. ``Bounds on Elasticities with Optimization Frictions: A Synthesis of Micro
and Macro Evidence on Labor Supply,'' Econometrica 80(3), 969--1018.
\href{http://elsa.berkeley.edu/~saez/course/chettyEMA12.pdf} {(web)} 

Chetty, R., Adam Guren, Day Manoli, and Andrea Weber. 2013 ``Does Indivisible Labor Explain the Difference between Micro and Macro Elasticities? A Meta-Analysis of Extensive Margin Elasticities'',
NBER Macroeconomics Annual, University of Chicago Press, 27(1), 1--56.
\href{http://elsa.berkeley.edu/~saez/course/chettyetal2013NBERmacro.pdf} {(web)}

Chetty, R., J. Friedman, T. Olsen and L. Pistaferri ``Adjustment Costs, Firms
Responses, and Micro vs. Macro Labor Supply Elasticities: Evidence from Danish Tax
Records'', Quarterly Journal of Economics,  126(2), 2011, 749-804. \href{http://elsa.berkeley.edu/~saez/course/chettyetalQJE2011denmark.pdf} {(web)} 

\textbf{Chetty, R., J. Friedman and E. Saez ``Using Differences in Knowledge Across Neighborhoods
to Uncover the Impacts of the EITC on Earnings'', 
American Economic Review, 2013, 103(7), 2683-2721 \href{http://eml.berkeley.edu/~saez/chetty-friedman-saezAER13EITC.pdf} {(web)} }

Chetty, R. and E. Saez ``Teaching the Tax Code: Earnings Responses to an Experiment with Recipients'', American Economic Journal: Applied Economics 5(1), 2013, 1-31. \href{http://elsa.berkeley.edu/~saez/chetty-saezAEJ13EITC.pdf} {(web)}

Crawford, V. and J. Meng ``New York City Cabdrivers' Labor Supply Revisited: Reference-Dependence Preferences with Rational-Expectations Targets for Hours and Income'', University of California at San Diego, Economics Working Paper Series: 2008-03, 2008. \href{http://elsa.berkeley.edu/~saez/course/Crawford,Meng(2009).pdf} {(web)}

\textbf{Dahl, Gordon B., Andreas Ravndal Kostol, Magne Mogstad ``Family Welfare Cultures''
Quarterly Journal of Economics, 129(4), 2014, 1711-52 \href{http://elsa.berkeley.edu/~saez/course/dahl-kostol-mogstadQJE14.pdf} {(web)} }

Davis, J. and M. Henrekson, ``Tax Effects on Work Activity, Industry Mix and Shadow Economy Size: Evidence from Rich Country Comparisons'', in R. Gomez-Salvador, A. Lamo, B. Petrongolo, M. Ward and E. Wasmer eds., Labour Supply and Incentives to Work in Europe, 2005, 44-104. \href{http://www.nber.org/papers/w10509.pdf} {(web)}

Devereux, Michael P, Li Liu and Simon Loretz. 2014. "The Elasticity of Corporate Taxable Income: New Evidence from UK Tax Records." American Economic Journal: Economic Policy, 6(2): 19-53.
\href{http://elsa.berkeley.edu/~saez/course/devereuxetal13.pdf} {(web)}

Einav, Liran, Amy Finkelstein, Paul Schrimpf ``The Data Revolution and Economic Analysis', NBER Working Paper 19035, 2013. \href{http://www.nber.org/papers/w19035.pdf} {(web)}

Einav, Liran and Jonathan Levin ``The Data Revolution and Economic Analysis'', NBER Working Paper No. 19035, 2013 \href{http://www.nber.org/papers/w19035.pdf} {(web)}

%Eissa, N. ``Taxation and Labor Supply of Married Women: The Tax Reform Act of 1986 as a Natural Experiment'', NBER Working Paper No. 5023, 1995. \href{http://www.nber.org/papers/w5023} {(web)}

Eissa, N. and H. Hoynes ``Taxes and the labor market participation of married couples: the earned income tax credit'', Journal of Public Economics, Vol. 88, 2004, 1931-1958. \href{http://elsa.berkeley.edu/~saez/course/Eissa and Hoynes_JPubE(2004).pdf} {(web)}

Eissa, N. and J. Liebman ``Labor Supply Response to the Earned Income Tax Credit'', Quarterly Journal of Economics, Vol. 111, 1996, 605-637. \href{http://links.jstor.org/stable/pdfplus/2946689.pdf} {(web)} 

Farber, H. ``Is Tomorrow Another Day? The Labor Supply of New York City Cab Drivers'', Journal of Political Economy, Vol. 113, 2005, 46-82. \href{http://links.jstor.org/stable/pdfplus/3555297.pdf} {(web)}

Farber, H. ``Reference-Dependent Preferences and Labor Supply: The Case of New York City Taxi Drivers'', The American Economic Review, Vol. 98, 2008, 1069-1082. \href{http://www.jstor.org/stable/pdfplus/29730106.pdf} {(web)}

Fehr, E. and L. Goette ``Do Workers Work More if Wages Are High? Evidence from a Randomized Field Experiment'', American Economic Review, Vol. 97, 2007, 298-317. \href{http://links.jstor.org/stable/pdfplus/30034396.pdf} {(web)}

Friedberg, L. ``The Labor Supply Effects of the Social Security Earnings Test'', Review of Economics and Statistics, Vo. 82, 2000, 48-63. \href{http://links.jstor.org/stable/pdfplus/2646671.pdf} {(web)}

Greenberg, D. and H. Hasley, ``Systematic Misreporting and Effects of Income Maintenance Experiments on Work Effort: Evidence from the Seattle-Denver Experiment'', Journal of Labor Economics, Vol. 1, 1983, 380-407. \href{http://www.jstor.org/stable/pdfplus/2534861.pdf} {(web)}

Hausman, J. ``Stochastic Problems in the Simulation of Labor Supply'', NBER Working Paper No. 0788, 1981. \href{http://www.nber.org/papers/w0788.pdf} {(web)}

Hausman, J. ``Taxes and Labor Supply'', in A. Auerbach and M. Feldstein, eds, Handbook of Public Finance, Vol I, North Holland 1987. \href{http://elsa.berkeley.edu/~saez/course/Hausman_Handbook.pdf} {(web)}

Heckman, J. ``What Has Been Learned About Labor Supply in the Past Twenty Years?'', American Economic Review, Vol. 83, 1993, 116-121. \href{http://links.jstor.org/stable/pdfplus/2117650.pdf} {(web)}

Heckman, J. and M. Killingsworth ``Female Labor Supply: A Survey'' Handbook of Labor Economics, Vol. I, Chapter 2, 1986. \href{http://elsa.berkeley.edu/~saez/course/Heckman and Killingsworth_Handbook.pdf} {(web)}

%Hotz, J. and K. Scholz ``The Earned Income Tax Credit'', NBER Working Paper No. 8078, 2001. \href{http://www.nber.org/papers/w8078.pdf} {(web)}

\textbf{Imbens, G.W., D.B. Rubin and B.I. Sacerdote ``Estimating the Effect of Unearned Income on Labor Earnings, Savings, and Consumption: Evidence from a Survey of Lottery'', American Economic Review, Vol. 91, 2001, 778-794. \href{http://links.jstor.org/stable/pdfplus/2677812.pdf} {(web)} }

Jakobsen, Kristian, Katrine Jakobsen, Henrik Kleven and Gabriel Zucman. 2018. �Wealth
Accumulation and Wealth Taxation: Theory and Evidence from Denmark� NBER working
paper No. 24371, forthcoming Quarterly Journal of Economics
\href{http://www.nber.org/papers/w24371.pdf} {(web)}

Jones, Damon ``Information, Inertia and Public Benefit Participation: Experimental Evidence from the Advance EITC and 401(k) Savings``, AEJ: Applied Economics, Vol. 2, 2010, 147-163. \href{http://elsa.berkeley.edu/~saez/course/Jones_AEJ.pdf} {(web)}

Keane, Michael ``Labor Supply and Taxes: A Survey?'', Journal of Economic Literature,
Vol. 49(4), 2011, 961-1075. \href{http://elsa.berkeley.edu/~saez/course/keaneJEL11laborsupply.pdf} {(web)}

Kleven, Henrik  ``How Can Scandinavians Tax So Much?'', 
Journal of Economic Perspectives 28(4), 77-98, 2014
 \href{http://pubs.aeaweb.org/doi/pdfplus/10.1257/jep.28.4.77} {(web)}
 
\textbf{Kleven, Henrik 2019. ``The EITC and the Extensive Margin: A Reappraisal'', NBER working
paper No. 26405.
\href{http://www.nber.org/papers/w26405.pdf} {(web)} }

\textbf{Kleven, Henrik ``Bunching'', Annual Review of Economics, 8, 2016, 435-464.
\href{http://elsa.berkeley.edu/~saez/course/kleven_annualreview.pdf} {(web)} }

Kleven, Henrik, Camille Landais, Johanna Posch, Andreas Steinhauer, and Josef Zweimuller. 2019 ``Child penalties across countries: Evidence and explanations.'' AEA Papers and Proceedings, 109, 122-26. 
\href{https://pubs.aeaweb.org/doi/pdfplus/10.1257/pandp.20191078} {(web)}

Kleven, Henrik and Mazhar Waseem, 2013``Using notches to uncover optimization frictions and structural elasticities: Theory and evidence from Pakistan'', Quarterly Journal of Economics 2013, 669-723.
\href{http://elsa.berkeley.edu/~saez/course/kleven-waseemQJE13.pdf} {(web)}

Kline, Patrick and Melissa Tartari, 2016. "Bounding the Labor Supply Responses to a Randomized Welfare Experiment: A Revealed Preference Approach," American Economic Review, 106(4), 972-1014. 
\href{http://elsa.berkeley.edu/~saez/course/klinetkopartari13.pdf} {(web)}

Kopczuk, Wojciech  and David J. Munroe, 2015 ``Mansion Tax: The Effect of Transfer Taxes on the Residential Real Estate Market'', American Economic Journal: Economic Policy, 7(2), 214-57
\href{http://elsa.berkeley.edu/~saez/course/kopczuk-munroe15.pdf} {(web)}

%Liebman, J. and E. Saez ``Earnings Responses to Increases in Payroll Taxes'', University of California Berkeley, mimeo, 2006. \href{http://elsa.berkeley.edu/~saez/liebman-saezSSA06.pdf} {(web)}

Liebman, J. and R. Zeckhauser ``Schmeduling'', Harvard University working paper, October 2004. \href{http://elsa.berkeley.edu/~saez/course/schmeduling-zeckhauser.pdf} {(web)}

Londono-Velez, Juliana and Javier Avila. ``Can Wealth Taxation Work in Developing
Countries? Quasi-Experimental Evidence from Colombia", UC Berkeley working paper,
2018. \href{http://elsa.berkeley.edu/~saez/course/londono-wealth2018.pdf} {(web)}

MaCurdy, T. ``An Empirical Model of Labor Supply in a Life-Cycle Setting'', Journal of Political Economy, Vol. 89, 1981, 1059-1085. \href{http://www.jstor.org/stable/pdfplus/1837184.pdf} {(web)}

MaCurdy, T. ``A Simple Scheme for Estimating an Intertemporal Model of Labor Supply and Consumption in the Presence of Taxes and Uncertainty'', International Economic Review, Vol. 24, 1983, 265-289. \href{http://links.jstor.org/stable/pdfplus/2648746.pdf} {(web)}

MaCurdy, T., D. Green and H. Paarsch ``Assessing Empirical Approaches for Analyzing Taxes and Labor Supply'' Journal of Human Resources, Vol. 25, 1990, 415-490. \href{http://links.jstor.org/stable/pdfplus/145990.pdf} {(web)}

\textbf{Martinez, Isabel, Emmanuel Saez, and Michael Siegenthaler. 2021. ``Intertemporal Labor Supply Substitution? Evidence from the Swiss Income Tax Holidays,'' American Economic Review 111(2), 506-546. \href{http://elsa.berkeley.edu/~saez/martinez-saez-siegenthalerAER21.pdf} {(web)} }

Meyer, B. and D. Rosenbaum ``Welfare, the Earned Income Tax Credit, and the Labor Supply of Single Mothers'', Quarterly Journal of Economics, Vol. 116, August 2001, 1063-1114. \href{http://links.jstor.org/stable/pdfplus/2696426.pdf} {(web)}

%Meyer, B. and X. Sullivan ``The effects of welfare and tax reform: the material well-being of single mothers in the 1980s and 1990s'', Journal of Public Economics, Vol. 88, 2004, 1387-1420. \href{http://elsa.berkeley.edu/~saez/course/Meyer and Sullivan_JPubE(2004).pdf} {(web)}

Moffitt, R. ``Welfare Programs and Labor Supply'', in A. Auerbach and M. Feldstein, Handbook of Public Economics, Volume 4, Chapter 34, Amsterdam: North Holland, 2003.  \href{http://elsa.berkeley.edu/~saez/course/Moffitt_Handbook.pdf} {(web)}

Mortenson, Jacob A. and Andrew Whitten. 2016. ``Bunching to Maximize Tax Credits: Evidence from Kinks in the U.S. Tax Schedule'', OTA-JCT working paper  \href{http://elsa.berkeley.edu/~saez/course/mortenson-whitten16bunching.pdf} {(web)}

Mroz, T. ``The Sensitivity of An Empirical Model of Married Women's Hours of 	Work to Economic and Statistical Assumptions'', Econometrica, Vol. 55, 1987, 765-799. \href{http://links.jstor.org/stable/pdfplus/1911029.pdf} {(web)}

Munnell, A., Lessons from the income maintenance experiments : proceedings of a conference held at Melvin Village, New Hampshire, September 1986. Boston: Federal Reserve Bank of Boston, 1986. \href{http://elsa.berkeley.edu/~saez/course/Munnell(1986)book.pdf} {(web)}

Nichols, Austin and Jesse	Rothstein 2015. ``The	Earned	Income	Tax	Credit'',
forthcoming in Volume on US Transfer Programs edited by R. Moffitt
\href{http://elsa.berkeley.edu/~saez/course/nichols-rothstein-draft_Nov2014.pdf} {(web)}

Oettinger, G. ``An Empirical Analysis of the Daily Labor Supply of Stadium Vendors'', Journal of Political Economy, Vol. 107, 1999, 360-392. \href{http://links.jstor.org/stable/pdfplus/2990813.pdf} {(web)}

Ohanian, L., A. Raffo, and R. Rogerson ``Long-Term Changes in Labor Supply and Taxes: Evidence from OECD Countries, 1956-2004'', Journal of Monetary Economics, Vol. 55, 2008, 1353-1362. \href{http://elsa.berkeley.edu/~saez/course/Ohanianetal(2008).pdf} {(web)}

Pencavel, J. ``Labor Supply of Men: A Survey'', Handbook of Labor Economics, vol. 1, chapter 1, 1986. \href{http://elsa.berkeley.edu/~saez/course/Pencavel_Handbook.pdf} {(web)}

Pencavel, J. ``A Cohort Analysis of the Association between Work Hours and Wages among Men'', The Journal of Human Resources, Vol. 37, 2002, 251-274. \href{http://www.jstor.org/stable/pdfplus/3069647.pdf} {(web)}

Prescott, E. ``Why Do Americans Work So Much More Than Europeans?'',  NBER Working Paper No. 10316, 2004, published in FRB Minneapolis - Quarterly Review, 2004, 28(1), 2-14. \href{http://www.nber.org/papers/w10316} {(web)}

Prescott, E. ``Nobel Lecture: The Transformation of Macroeconomic Policy and Research'', Journal of Political Economy, 114, 2006, 203-235. \href{http://links.jstor.org/stable/pdfplus/3840344.pdf}{(web)}

Ramey, Valerie A. and Neville Francis, 2009. ``A Century of Work and Leisure,'' American Economic Journal: Macroeconomics, 1(2), 189-224.
\href{http://elsa.berkeley.edu/~saez/course/ramey-francisAEJ09laborsupply.pdf} {(web)}

Rees, A. ``An Overview of the Labor-Supply Results'', The Journal of Human Resources, Vol. 9, 1974, 158-180. \href{http://www.jstor.org/stable/pdfplus/144971.pdf} {(web)}

Rees-Jones, Alex and Dmitry Taubinsky. 2016 ``Measuring ``Schmeduling'' '', NBER Working Paper
No. 22884.\href{http://www.nber.org/papers/w22884} {(web)}


Rogerson, R. and J. Wallenius ``Micro and Macro Elasticities in a Life Cycle Model with Taxes'', Journal of Economic Theory, Vol. 144, 2009, 2277-2292. \href{http://elsa.berkeley.edu/~saez/course/Rogerson_JET(2009).pdf} {(web)}

Rothstein, J. `Is the EITC as Good as an NIT? Conditional Cash Transfers and Tax Incidence.'' American Economic Journal: Economic Policy 2 (1), February 2010, 177-208. \href{http://elsa.berkeley.edu/~saez/course/rothsteinAEJ10.pdf} {(web)}

Saez, E. ``Do Taxpayers Bunch at Kink Points?'', NBER Working Paper No. 7366, 1999. \href{http://www.nber.org/papers/w7366} {(web)}

\textbf{Saez, E. ``Do Taxpayers Bunch at Kink Points?'', AEJ: Economic Policy, Vol. 2, 2010, 180-212. \href{http://elsa.berkeley.edu/~saez/course/Saez_AEJ(2010).pdf} {(web)}}

\textbf{Saez, Emmanuel  ``Public Economics and Inequality: Uncovering Our Social Nature'', AEA Papers and Proceedings, 121, 2021
\href{https://eml.berkeley.edu/~saez/saez-AEAlecture.pdf} {(web)} }


Saez, E., J. Slemrod, and S. Giertz ``The Elasticity of Taxable Income with Respect to Marginal Tax Rates: A Critical Review'', Journal of Economic Literature 50(1), 2012, 3-50. \href{http://elsa.berkeley.edu/~saez/saez-slemrod-giertzJEL12.pdf} {(web)}

Seim, David. 2017. "Behavioral Responses to an Annual Wealth Tax: Evidence from
Sweden", American Economic Journal: Economic Policy, 9(4), 395-421.
\href{http://elsa.berkeley.edu/~saez/course/seimAEJ17wealth.pdf} {(web)}

Sigurdsson, Josef. 2019. ``Labor Supply Responses and Adjustment Frictions: A Tax-Free Year in Iceland'', 
Working Paper, September 2019.
\href{http://elsa.berkeley.edu/~saez/course/sigurdsson2019taxholiday.pdf} {(web)}

Steinhauer, Andreas. 2018 ``Working Moms, Childlessness, and Female Identity'',
CEPR Discussion Paper No. 12929.\href{http://elsa.berkeley.edu/~saez/course/steinhauerDP2018.pdf} {(web)}

Tazhitdinova, Alisa. 2020 ``Increasing Hours Worked: Moonlighting Responses to a Large Tax Reform'',
NBER Working Paper No. 27726, forthcoming AEJ: Economic Policy. \href{http://www.nber.org/papers/w27726.pdf} {(web)}

Tortarolo, Dario, Guillermo Cruces, and Victoria Castillo. 2019 ``It takes two to tango: labor responses to an
income tax holiday in Argentina'', UC Berkeley working paper \href{http://elsa.berkeley.edu/~saez/course/tortarolo19argentina.pdf} {(web)}

}
\end{slide}


\end{document}
