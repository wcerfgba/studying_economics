\documentclass[landscape]{slides}

\usepackage[landscape]{geometry}

\usepackage{pdfpages}
\usepackage{amsmath}
\usepackage{{hyperref}}

\def\mathbi#1{\textbf{\em #1}}

\topmargin=-1.8cm \textheight=17cm \oddsidemargin=0cm
\evensidemargin=0cm \textwidth=22cm

\author{Emmanuel Saez}

\date{Berkeley}

\title{230B: Public Economics \\
Social Security, Retirement, and Disability} \onlyslides{1-300}

\newenvironment{outline}{\renewcommand{\itemsep}{}}

\begin{document}

\begin{slide}
\maketitle
\end{slide}

\begin{slide}
\begin{center}
{\bf RETIREMENT PROBLEM}
\end{center}

{\bf Life-Cycle:} Individuals ability to work declines with aging and
continue to live after they are unwilling/unable to
work

{\bf Standard Life-Cycle Model Prediction:} Absent any government program,
rational individual would save while working to consume savings
while retired [Modigliani life cycle graph] 
%earnings, wealth,
%consumption: $T=60$ adult life, $R=40$ working life, $T-R=20$
%retirement life]

Optimal saving problem is extremely complex: uncertainty in
returns to saving, in life-span, in future ability/opportunities
to work, in future tastes/health

{\bf In practice:} When govt was small $\Rightarrow$ Many people
worked till unable to (often till death) and then were taken care
of by family members 

{\bf Today:} Govt is taxing workers to provide for retirees through
social security retirement systems

\end{slide}

\begin{slide}
\includepdf[pages={42}]{socialsecurity_attach.pdf}
\end{slide}


\begin{slide}
\includepdf[pages={53, 52, 65, 48-49}]{socialsecurity_attach.pdf}
\end{slide}





\begin{slide}
\begin{center}
{\bf GOVT INTERVENTION IN RETIREMENT POLICY}
\end{center}

{\bf Actual Retirement Programs:} All OECD countries implement
substantial retirement programs (substantial share of GDP around
6-12\%, US smaller around 6\%) 

Started in first part of 20th
century and have been growing. Common structure:

Individual pay social security contributions (payroll taxes) while
working and receive retirement benefits when they stop working
till the end of their life (annuity)

{\bf Various types of retirement programs (private or public):}

(a) Funded vs. Unfunded, (b) Defined Benefits vs. Defined Contributions,
(c) Mandatory vs. Voluntary, (d) Universal vs. Means-tested, (e) Annuitized benefits vs. lumpsum

\end{slide}

\begin{slide}
\begin{center}
{\bf FUNDED VS. UNFUNDED PROGRAMS}
\end{center}

{\bf Unfunded (pay-as-you-go):} benefits of current retirees
paid out of contributions from current workers [generational link]

current benefits = current contributions

{\bf Funded:} workers contributions are invested in financial
assets and will pay for benefits when they retire [no generational
link]

current benefits = past contributions + market returns on past contributions

\end{slide}

\begin{slide}
\begin{center}
{\bf Defined Contributions vs. Defined Benefits}
\end{center}

{\bf Defined Contributions (DC):} System specifies the level of
contributions [e.g., 10\% of earnings]. Benefits then depend on level
of contributions and returns on contributions.

{\bf Defined Benefits (DB):} System specifies the level of
benefits [e.g., 60\% of average earnings during career].
Contributions adjusted to meet required level of benefits.

DC pro: Easier to implement and contributions are not perceived as
a ``tax'' (but harder to do redistribution in DC plan)

DC con: Benefits are risky. Risk in benefits worse than risk in
contributions [as workers can adjust and absorb shocks more easily than
retirees]. 

\end{slide}

\begin{slide}
\begin{center}
{\bf EXAMPLES}
\end{center}

1) Unfunded DB: most public retirement programs (such as Social Security in the
US)

2) Funded DB: traditional US private employer pension plans [e.g.,
annual benefits = 2.5\% $\times$\# years worked$\times$ last salary], a govt DB
retirement program could also be funded [govt invests payroll
taxes into ``sovereign fund'' as in Norway]

3) Funded DC: new US private employer pensions plans [401(k)s]:
worker contributes fraction of salary and invests contributions in financial assets
in account run by private pension fund

4) Unfunded DC: Notional accounts in some government retirement programs
(Sweden, France 2020 attempt): payroll taxes yield fictitious returns and benefits are
based on contributions plus this fictitious (notional) return.

\end{slide}

\begin{slide}
\begin{center}
{\bf WHY SHOULD GOVERNMENT INTERVENE?}
\end{center}

{\bf 1) Individual failures} to behave like in standard econ model. More
of a model failure than ``individual failure''. Most individuals would not save adequately for retirement on their
own (information or self-control problems).

Large fraction of individuals do not accumulate for their retirement absent government/institution/employer
help

%Paternalism: govt imposes its preferences against individuals
%$\Rightarrow$ Individuals should oppose govt program

%Behavioral econ view: individuals understand that they need help and
%welcome / create government intervention

{\bf 2) Market Failures:} Adverse selection in annuitization market

{\bf 3) Redistribution:} \\
(a) Within Generations: Retirement programs can redistribute based
on life-time earnings (instead of annual income)

(b) Across Generations: Retirement programs can redistribute
across cohorts 

\end{slide}

\begin{slide}
\begin{center}
{\bf SOURCES OF RETIREMENT INCOME IN THE US}
\end{center}
1) Govt provided retirement benefits (US Social Security):
For 2/3 of retirees, SS is more than 50\% of income. 1/3 of
elderly households depend almost entirely on SS.

2) Home Ownership: 75\% of US elderly are homeowners

3) Employer pensions (tax favored): 40-45\% of elderly US households have employer pensions.
Two types:

a) Traditional: DB and mandatory: {\bf employer} carries full
risk [in sharp decline, many in default]

b) New: DC and elective: 401(k)s, {\bf employee}
carries full risk

60\% of workers have access to empl. pensions, 45\% contribute

4) Supplementary individual elective pensions (tax favored): IRAs (Individual Retirement Arrangements)

5) Extra savings through non-tax favored instruments:
significant only for wealthy minority [=10\% of retirees]

\textbf{Key lesson:} Bottom 90\% wealth is (a) housing (net of mortgage debt), (b) pensions,
(c) minus other debts (consumer credit, student loans) 

All 3 components are heavily affected by government policy (education finance), institutions (such as employers pensions), financial regulations (mortgage refinance, credit card and loans)

\end{slide}

\begin{slide}
\begin{center}
{\bf MODEL: MYOPIC SAVERS}
\end{center}

1) Some individuals are rational: 
\[\max u(c_1) + \delta \cdot u(c_2) \text{ subject to} \]
$c_1+s=w$ and $c_2=s\cdot(1+r)$, $c_1+c_2/(1+r)=w$ [draw graph]

FOC: $u'(c_2)/u'(c_1)=1/[(1+r)\delta]$, let $s^*$ be optimal
saving

Example: If $\delta=1$ and $r=0$ then $s^*=w/2$ and $c_1=c_2=w/2$

2) Some individuals are myopic: 
\[ \max u(c_1)  \text{ subject to} \]  $c_1+s=w$ and
$c_2=s\cdot(1+r)$ $\Rightarrow$ $c_1=w$ and $s=c_2=0$

\end{slide}

\begin{slide}
\includepdf[pages={43}]{socialsecurity_attach.pdf}
\end{slide}


\begin{slide}
\begin{center}
{\bf MODEL: MYOPIC SAVERS}
\end{center}

Social welfare is always $u(c_1) + \delta \cdot u(c_2)$

Govt imposes forced saving tax $\tau$ such that $\tau=s^*$
and benefits $b=\tau \cdot (1+r)$. We consider a funded system.
Cannot borrow against $b$ [as in
current Social Security]

1) Rational individual unaffected: adjusts $s$ one-to-one so that
outcome unchanged [rational unaffected as long as $\tau \leq
s^*$]: 100\% crowding out of private savings by forced savings

2) Myopic individual affected (0\% crowding out): new outcome
maximizes Social Welfare

Forced savings is a good solution: (a) does not affect rational individuals,
(b) affects the myopic individuals in the socially desired way


\end{slide}

\begin{slide}
\includepdf[pages={44}]{socialsecurity_attach.pdf}
\end{slide}


\begin{slide}
\begin{center}
{\bf MODEL: COMMENTS}
\end{center}

{\bf 1) Universal vs. Means-Tested Program:} Universal forced savings is better than means-tested program financed by tax on everybody [Samaritan's dilemma]. With means-tested program, 2 drawbacks:

a) Rational individuals subsidize myopic individuals

b) Incentives to under-save to get means-tested pension

$\Rightarrow$ Less sustainable politically (conservatives like means-testing)

{\bf 2) Heterogeneity in $w$}: Forced saving should be
proportional to $w$ (as long as govt does not care about
redistribution). 

%{\bf 2) Adding labor Supply Responses:}
%
%$u(c_1)-h(l_1)+\delta u(c_2)$ with $c_1=(1-\tau)wl_1-s$ and
%$c_2=(1+r)(s+\tau w l_1)$ $\Rightarrow$ $c_1+c_2/(1+r)=wl_1$
%$\Rightarrow$
%
%a) $l_1$ of the rational individuals not affected [as benefits are
%{\bf actuarially fair}]
%
%b) $l_1$ of myopic is distorted downward: $\max_{l_1}
%u((1-\tau)wl_1)-h(l_1)$ as they perceive the tax but not the
%future benefits

\end{slide}

\begin{slide}
\begin{center}
{\bf FUNDED VS. UNFUNDED SYSTEMS (skip)}
\end{center}
OLG model with 2 periods (work and retirement). Generation $t$
lives in periods $t$ and $t+1$, cohort size $N_t$, wage $w_t$

{\bf 1) Unfunded system:} Free benefits to 1st generation of retirees.
For Generation $t$:

$tax_t=\tau w_t$, $ben_t=\tau w_{t+1} N_{t+1}/N_t= \tau w_t
(w_{t+1}/w_t)(N_{t+1}/N_t)$ $\Rightarrow$ $ben_t=tax_t \cdot
(1+g)(1+n) = tax_t \cdot (1+\gamma)$

All the other generations get return equal to $\gamma \simeq n+g$
where $n$ is population growth and $g$ real wage growth per capita

{\bf 2) Funded system:} each generation gets a market return $r$ on
contributions: $ben_t=tax_t \cdot (1+r)$

\end{slide}

\begin{slide}
\begin{center}
{\bf FUNDED VS. UNFUNDED SYSTEMS (skip)}
\end{center}
Famous theoretical results:

1) Samuelson JPE'58: In OLG economy with no capital and no way to
save (chocolate economy), unfunded system generates {\bf Pareto}
improvement because it allows trade across generations [same result
with fiat-money]

2) Diamond AER'65: In OLG economy with capital and saving, unfunded
pension generates Pareto improvement iff $n+g>r$ (economy is
dynamically inefficient and has too much capital)

If $n+g<r$, unfunded pension redistributes from all generations to 1st generation.

\end{slide}

\begin{slide}
\begin{center}
{\bf FUNDED VS. UNFUNDED SYSTEMS (skip)}
\end{center}

%Famous theoretical result of Diamond AER'65: In OLG economy with capital and saving, unfunded
%pension generates Pareto improvement iff $n+g>r$ (economy is
%dynamically inefficient and has too much capital)

In practice $r>n+g$ almost everywhere: funded system delivers higher returns because it does not deliver
a free lunch to 1st generation


US economy: Annual $n=1\%$ and $g=1\%$ [$n+g$ was higher in
1940-1970].

$r=5-6\%$ if $r$ is average return on all capital
assets held by households over the long-run

Note that $r$ is much more risky than $n+g$: risk adjusted market
rate of return should be lower than average market rate $r$ but still higher than $n+g$


\end{slide}


\begin{slide}
\begin{center}
{\bf GENERATIONAL ACCOUNTING (skip)}
\end{center}
Let $\gamma=n+g$ be the generational growth rate

1) Generation $0$ nets: $V_0=- 0 \cdot w_0 N_0 + \tau w_1
N_1/(1+r)=\tau w_0 N_0 (1+\gamma)/(1+r)$

2) Generation $t$ nets: $V_t=-\tau w_t N_t + \tau w_{t+1}
N_{t+1}/(1+r)= \tau w_0 N_0 (1+\gamma)^t [-1+(1+\gamma)/(1+r)]$

3) Accounting from period $0$: $\sum_{t=0}^{\infty} V_t/(1+r)^t=$
$$ \tau  w_0 N_0\frac{1+\gamma}{1+r}+  \tau  w_0 N_0 \sum_{t=1}^{\infty}
\frac{(1+\gamma)^t}{(1+r)^t} \left [-1+\frac{1+\gamma}{1+r} \right ]=0$$ No behavioral
responses $\Rightarrow$ No net effect

Unfunded vs. Funded is about redistribution across cohorts

Originally: priority was to alleviate old age poverty so most govt
started with unfunded system

\end{slide}

\begin{slide}
\begin{center}
{\bf FUNDED VS. UNFUNDED SYSTEMS}
\end{center}
Historical development of pension systems:

1) Before 20th century: private pension arrangements are family based (kids take care of aging parents) which is an unfunded system [funded private saving was never a major source of retirement income for the majority of the population]

2) 20th century: Governments introduce unfunded pension systems to replace the family based system [workers start paying
taxes but no longer have to care for elderly parents]

3) Today: some debate on whether government systems should be funded instead of unfunded [social security
privatization debate] but funding requires transitional generation to pay twice [for the old and themselves]


%With $r>> n+g$, unfunded system looks like bad deal for current and future generations relative to funded system
 
\end{slide}



\begin{slide}
\begin{center}
{\bf SOCIAL SECURITY IN THE US}
\end{center}

1) {\bf Financed} by payroll taxes: 6.2\% on employee and 6.2\% on
employer (up to annual cap of \$138,000 in 2020, indexed for wage
growth): funds retirement and disability benefits [1.45\%+1.45\%
with no cap funds medicare]

%Obama administration offsets employee payroll tax on first \$5,200 of
%annual earnings with a refundable tax credit for 2009-2010 (stimulus bill).
%Employee rate cut to 4.2\% in 2011-2012 (tax deal).

2) {\bf Benefits} based on AIME (average indexed monthly earnings)
over the best 35 years of (indexed) taxable earnings

Indexation based on average wage growth

PIA (primary insurance amount) is a piece-wise linear function of
AIME: 90\% of first \$1000 of AIME, 32\% of AIME over \$1000 to
\$6000, 15\% of AIME above \$6000 

$\Rightarrow$ Formula is \textbf{Redistributive} but this compensates for longevity differences
by earnings groups

Average replacement  rate around 40\% (higher for low earners)
\end{slide}

\begin{slide}
\includepdf[pages={54}]{socialsecurity_attach.pdf}
\end{slide}


\begin{slide}
\begin{center}
{\bf SOCIAL SECURITY IN THE US}
\end{center}

Married couple with $PIA_H,PIA_W$ get maximum of

$1.5 \cdot \max(PIA_H,PIA_W)$ and $PIA_H+PIA_W$.

Surviving spouse gets $\max (PIA_H,PIA_W)$

Divorced spouse is eligible for benefits based on ex-spouse
$PIA$ if marriage spell longer than 10 years (no empirical
spike in divorces after 10th anniversary though!)

Benefits are fully {\bf annuitized} indexed based on consumer
price index (debate about moving to less generous chained CPI)

\end{slide}



\begin{slide}
\begin{center}
{\bf RETIREMENT AGE IN SS}
\end{center}

1) {\bf Normal Retirement Age (NRA):} Used to be 65 increasing to 67 for 1960+ cohorts. Get PIA when retiring at NRA

2) {\bf Early Retirement Age:} is 62 [Earliest age you can get SS
benefits (unless disabled)]. Benefits reduced permanently by 8\% if
retire 1 year before NRA, 16\% if 2 years before NRA, etc.
[actuarially fair on average]

3) {\bf Late Retirement:} get permanently higher benefits.
Get 8\% more permanently if delay by 1 year, 16\% for 2 year delay, 
etc. (actuarially fair). Benefits automatic at age 70.

$\Rightarrow$ Current SS system should not distort
retirement age on average (as adjustments are fair) if people
fully rational

\textbf{Early retirement age:} Availability of benefits seems to have huge
effects (inconsistent with standard model with no credit
constraints) $\Rightarrow$ {\bf Liquidity Effects}

\end{slide}



\begin{slide}
\begin{center}
{\bf EARNINGS TEST OF SS}
\end{center}

Currently: $62 \leq Age <NRA$, benefits taxed away at 50\% above
\$15,000 of annual earnings.

$Age = NRA$, benefits taxed away at 33\% above \$40,000 of annual
earnings.

No earnings test for age above NRA

Actually, not a pure tax, as benefits taxed away will be credited
back at NRA (as if you had retired later).

However, individuals may not understand this and actually bunch at
the kink point of the Earnings Test 

See Friedberg Restat '00 with CPS data and Gelber-Jones-Sacks '19 with SSA admin data
(when NRA was 65)

\end{slide}

\begin{slide}
\includepdf[pages={66}]{socialsecurity_attach.pdf}
\end{slide}

\begin{slide}
\includepdf[pages={36}, scale=1]{socialsecurity_attach.pdf}
\end{slide}


%\begin{slide}%
%\begin{center}
%{\bf Friedberg 2000: Social Security Earnings Test}
%\end{center}
%Uses CPS data on labor supply of retirees receiving Social
%Security benefits
%
%Studies bunching based on responses to Social Security earnings
%test
%
%Friedberg exploits 1983 reform (CPS age = age + 1) which
%eliminated earnings test for those aged 70-71
%
%(a) Before: test up to age 71, no test at age 72+
%
%(b) After: test up to age 69, no test at age 70+
%
%\end{slide}
%
%\begin{slide}
%\includepdf[pages={1}]{socialsecurity_attach.pdf}
%\end{slide}
%
%\begin{slide}
%\includepdf[pages={2}]{socialsecurity_attach.pdf}
%\end{slide}

\begin{slide}
\begin{center}
{\bf KEY QUESTIONS IN THE LITERATURE ABOUT SOCIAL SECURITY}
\end{center}

1) How does Social Security affect private savings?

2) How does Social Security affect retirement? 

(surveys by Lumsdaine-Mitchell 1999 and Blundell-French-Tetlow 2017)

%3) What are the distributional implications for SS?

3) Funding problems: Social Security Reform and Privatization

\end{slide}

%\begin{slide}
%\begin{center}
%{\bf SOCIAL SECURITY AND SAVINGS: THEORY}
%\end{center}
%
%Two period model $u(c_1)+\delta u(c_2)$ st $c_1=w-\tau-s$ and
%$c_2=(1+r)s+b$ where $\tau$ is SS tax and $b$ is SS benefits.
%
%1) If $b=\tau \cdot (1+r)$ (actuarially fair program) and $b\leq c^*_2$
%(optimum with no SS) then $ds/d\tau=-1$ $\Rightarrow$  SS crowds out private
%saving {\bf one-for-one}
%
%2) If $b>c^*_2$, then $s=0$ and $d s / d\tau=0$ $\Rightarrow$ 0\% crowd-out
%
%Why does this matter?
%
%Most SS programs are unfunded so if private savings fall, then
%capital stock will fall (in closed economy)
%
%Lower capital stock  per capita $k$ increases rate of return
%$r=f'(k)$ but reduces wages $w=f(k)-rk$
%
%\end{slide}
%
%\begin{slide}
%\begin{center}
%{\bf Additional effects in more complex models:}
%\end{center}
%1) Uncertainty in retirement spell+missing annuity market
%$\Rightarrow$ Precautionary savings high $\Rightarrow$ SS provides
%annuity and reduces private savings more than one-to-one %(a good
%%thing)
%
%2) Induced retirement effect: if benefits are larger than what you
%would have saved, you may decide to retire earlier, in which case
%you want to save more ($s$ increases)
%
%3) If $b<\tau(1+r)$ (not actuarially fair program), then income
%effect reduces $c_1$ and hence increases $s=w-\tau-c_1$
%
%4) Ricardian Equivalence effect: pay-go is a transfer from all
%future generations to first generation: does not change the budget
%set of the dynasty $\sum c_t/(1+r)^t \leq \sum w_t/(1+r)^t$
%
%First generation can exactly offset pay-go pension by leaving
%larger bequests to kids, etc. $\Rightarrow$ No effect on consumption
%
%\end{slide}

\begin{slide}
\begin{center}
{\bf SOCIAL SECURITY AND SAVINGS}
\end{center}
Standard econ view: you save less if you expect SS benefits

Four approaches:

1) Aggregate Time series within a country [Feldstein JPE'74]

2) Micro-cross sectional [Feldstein and Pellochio '79]

3) Cross-country [Barro-McDonald JpubE'79]

4) Reform based within a country [Attanasio- Brugiavinni QJE '03 for Italy,
Attanasio and Rohwedder AER'03 for the UK]

First 3 approaches are weak in terms of identification 
with mixed evidence (see Page, CBO'98
extensive survey).

Last approach is much more promising and could be extended to
other countries

\end{slide}

%\begin{slide}
%\begin{center}
%{\bf TIME SERIES EVIDENCE: Feldstein JPE'74}
%\end{center}
%Uses aggregate data on
%consumption 1929-1951 and constructs a series on SSWG (=  PDV of
%future benefits) and SSWN (=SSWG - PDV SS taxes paid).
%
%Regresses consumption on wealth, disposable income, and SSWG or
%SSWN, and finds that \$1 increase in SSW was associated with 30-50
%cents more consumption $\Rightarrow$ Result very fragile for several reasons:
%
%(a)\ Depends on SSW measure (current law, perfect foresight)
%
%(b)\ Depends on time period
%
%(c)\ If effects on savings are examined directly, find no effect
%
%Result challenged by Leimer and Lesnoy (error in Feldstein '74),
%Feldstein acknowledges error but find new spec restoring the
%results
%
%\end{slide}

%\begin{slide}
%\begin{center}
%{\bf Cross-sectional studies:}
%\end{center}
%Regression specification of the type:
%
%$Wealth_i = \alpha + \beta SSW_i + X_i \gamma + \varepsilon_i$
%
%Feldstein and\ Pellechio (1979) found that
%financial assets of people aged 55-60 fell by 50 cents to \$1 for
%every extra dollar of SS wealth
%
%Large set of studies in the 1980s
%found various types of results with different definitions of SS
%wealth, dispersed in the 0 to -50 cent range
%
%Deep problem with this approach:\ use of cross-sectional variation
%is questionable:
%
%$SSW_i=f(earnings_i)$ so identification issue as $W_i$ might be
%correlated with past earnings in an unknown non-linear way.
%\end{slide}
%
%\begin{slide}
%\begin{center}
%{\bf Cross-country studies}
%\end{center}
%Try to compare benefit rates across countries with rate of
%savings. Barro-McDonald JpubE 1979 find no effect (in support of dynastic view)
%
%Subsequent studies again find very mixed evidence; Page CBO'98 suggests that this is unreliable method
%
%\end{slide}

%\begin{slide}
%\begin{center}
%{\bf Italy Reform Study}
%\end{center}
%
%Italy reform 1992: Attanasio- Brugiavinni QJE '03. Cohort based
%reform: young workers affected but not older workers
%(unfortunately phased-in slowly, no sharp discontinuity)
%
%Compare saving rates of old cohort (generous SS) to new cohort
%(less generous SS) $\Rightarrow$ Find 30-40\% of SS cuts offset by
%private savings
%
%
%
%\end{slide}
%
%\begin{slide}
%\begin{center}
%{\bf UK Reform Study}
%\end{center}
%
%Attanasio and Rohwedder AER'03
%
%1) Basic State Pension (BSP) indexation change, 1975 (from ad-hoc
%to wages) and 1981 (from wages to prices). BSP is a flat rate
%pension (today replaces 15\% of past earnings)
%
%2) Introduction of SERPS in 1978 (supplemental contributory
%pension, mandatory for those with no employer pension till 1988,
%not mandatory since 1988)
%
%Heterogeneity in responses: no response for young workers (likely
%credit constrained), no response to basic pension reform (lower
%paid workers), large response to SERPS
%\end{slide}

\begin{slide}
\begin{center}
{\bf Next Steps}
\end{center}

US: use private sector DB plans or DB reforms (like freezes) and
see whether workers adjust their own savings to changes in their
retirement plans (not easy to get the data but identification
would be better).

Outside US: cohort based reforms that are not phased-in slowly are
best (allow to do RDD), even better if reforms affect different
regimes differently (public vs private sector workers).

Main difficulty is getting good savings data (few administrative
data records all wealth sources, have to rely on smaller and
noisier survey data)

Chetty et al. QJE14 in Denmark recently makes good progress in
a very micro way

\end{slide}

\begin{slide}
\begin{center}
{\bf Chetty et al. QJE 2014: Govt mandated Saving}
\end{center}
With Danish administrative data, can observe earnings, income (linked to firms) as well
as savings (both retirement savings and other financial savings)

In Denmark, starting in 1998, firms are mandated (by govt) to make automatic retirement contributions
to workers' retirement savings accounts of 1\% of earnings when earnings crosses some threshold (34.5K DKr $\simeq$ \$5K)

$\Rightarrow$ Generates a discontinuity by earnings levels: can use a \textbf{Regression Discontinuity Design}

Main finding: \$1 contribution to mandatory savings plan 
$\rightarrow$ \$1 increase in pensions and total savings

No offset of the forced contribution with reduced savings

But discontinuity is only about \$50 (minuscule)

\end{slide}

\begin{slide}
\includepdf[pages={37-41}]{socialsecurity_attach.pdf}
\end{slide}




\begin{slide}
\begin{center}
{\bf Evidence for Myopia and adequate savings}
\end{center}

1) Diamond JpubE 1977: old age poverty has fallen as SS expanded
(Gruber book graph). Poverty for other groups has not fallen
nearly as much

%2) Fall in consumption {\bf during} retirement: Hamermesh (1984)
%shows that consumption falls by 5\% per year for the elderly
%[consumption is not smooth but not necessarily suboptimal]

2) Fall in consumption {\bf at} retirement: Bernheim, Skinner,
Weinberg (2001) show that drop in consumption is significant and
sharply correlated with wealth [consistent with
myopia]

3) Countervailing view: Scholz et al. JPE '06 develops micro-model
of rational savings with uncertainty. With reasonable parameters,
80\% of families over-save, 20\% under-save [optimal savings is
low given SS, DB, Medicaid asset tests]
\end{slide}

\begin{slide}
\includepdf[pages={4}]{socialsecurity_attach.pdf}
\end{slide}



\begin{slide}
\begin{center}
{\bf Consumption drop at retirement: Aguiar-Hurst JPE' 05}
\end{center}

Starting point: Empirically, consumption falls with
retirement...but studies use expenditures as measure of consumption

Aguiar-Hurst JPE05 shows that it is important to differentiate
between consumption and expenditures. Further, the paper provides
new information on the complementarity of consumption and leisure
after retirement.

1) Confirm that  food expenditure falls by 17\% at retirement 

2) But time spent on home production rises by 60\%

3) All measures of caloric intake, vitamin intake, meat quality, etc. do not drop at retirement
(find that caloric intake falls when getting unemployed, hard to believe but suggestive)
\end{slide}

\begin{slide}
\includepdf[pages={5}]{socialsecurity_attach.pdf}
\end{slide}

%\begin{slide}
%\begin{center}
%{\bf Redistributive effects of SS}
%\end{center}
%Various studies (Liebman and Feldstein '02, handbook chapter)
%
%1) Redistribution to older generations due to pay-go structure
%
%2) Within cohort redistribution:
%
%Annual: Redistributes from workers to elderly
%
%Life-time: Roughly neutral in terms of redistribution by life-time earnings
%because:
%
%a) Redistribution in the progressive benefits formula
%
%b) Regressivity because people with higher incomes live longer
%
%\end{slide}
%
%\begin{slide}
%\begin{center}
%{\bf Redistributive effects of SS}
%\end{center}
%
%Life-time perspective best if people are rational life-time savers
%
%Annual perspective best if people totally myopic
%
%Redistribution from males to females (as substantial longevity differences by
%gender)
%
%Redistribution from single to married and
%from two earner couples to one earner couples (as non-working spouses get 50\% spousal benefits)
%
%\end{slide}

\begin{slide}
\begin{center}
{\bf SOCIAL SECURITY AND RETIREMENT: THEORY}
\end{center}

Three key elements of a social security system may affect retirement behavior:

1) Availability of benefits at \textbf{Early Retirement Age} (ERA): (62 in US)

Those effects arise because of (a) myopia,  (b) liquidity constraints, (c) focal point norm 

2) Earnings-test after claiming benefits

3) Non-actuarially fair adjustments of benefits for those retiring after the ERA:

If benefits are not adjusted in a fair way, they can create a huge implicit tax on work (US used
to have very little adjustment)

%Empirical literature not very good at distinguish those two effects

\end{slide}


\begin{slide}
\begin{center}
{\bf Social Security and Retirement: Early retirement age}
\end{center}
Conceptually early retirement age can be seen as a device to force myopic people to keep working

\textbf{(a) Rational individual:} Wants to retire at age 60 but benefits not
available till age 62=ERA. Rational individual saves ex-ante to fund retirement at age 60-61 out
of savings before getting benefits at age 62. 

$\Rightarrow$ ERA does not affect the rational person [if she can perfectly forecast retirement age]

\textbf{(b) Myopic person:} Person cannot resist retiring once benefits are available. 
Myopic person will typically have no savings so cannot retire before ERA.

$\Rightarrow$ ERA affects positively the myopic person to prevent her from retiring
too early (optimal ERA analysis yet to be done)

%Trade-off about ERA because some people should retire early even if they have no savings

\end{slide}





\begin{slide}
\begin{center}
{\bf Social Security and Retirement: Implicit tax}
\end{center}

Theory [next graph is from Germany]: life-time budget constraint: Live $T$ years, work $R$
years and retire $T-R$ years.

$C$ life-time consumption and $R$ retirement age. With constant
wage $w$ and interest rate $r=0$: $C=w \cdot R$

With a fair retirement program: $w-\tau$ when working and
$b(R)=\tau R/(T-R)$, then $C=(w-\tau)R+b(T-R)=wR$

$\Rightarrow$  No effect on lifetime
budget constraint

$\Rightarrow$ Actuarially fair system does not
affect retirement age [with no uncertainty, no myopia, and no
credit constraints]

\end{slide}

\begin{slide}
\includepdf[pages={60}]{socialsecurity_attach.pdf}
\end{slide}

\begin{slide}
\begin{center}
{\bf Social Security and Retirement}
\end{center}

Retirement systems are not actuarially fair in general:

Benefits $b(R)$

Life-time consumption: $C=(w-\tau)R+(T-R)b(R)$

$dC/dR=w-\tau-b + (T-R)b'(R)$

Distort both slope and levels: substitution and wealth effects ($dC/dR=w$ $\Rightarrow$
system is actuarially fair)

Implicit tax rate of retirement program: $t=[w-dC/dR]/w$:

If you delay
retirement by 1 year, your PDV of consumption increases by $dC/dR=w \cdot (1-t)$
[fair has $t=0$]

\end{slide}

\begin{slide}
\begin{center}
{\bf Social Security and Retirement}
\end{center}

Some European systems had $b'(R)=0$ (no adjustment of benefits)

$\Rightarrow$ $dC/dR=w-\tau-b$ If $b=0.6 \cdot w$ and $\tau=.15 \cdot w$, then $dC/dR=w \cdot (1-0.75)$ $\Rightarrow$ enormous implicit tax $t=75\%$

United States now has $b'(R)=.08$ (8\% adjustment per year) which is about actuarially fair
\end{slide}

%\begin{slide}
%\begin{center}
%{\bf Empirical Evidence}
%\end{center}
%1) US time series evidence $LFP_t = \alpha + \beta \cdot SS_t/w_t + \varepsilon_t$
%(replacement rate): growth of SS and reduced LFP of elderly are
%correlated but not clear effect is causal
%
%2) US cross-sectional evidence  $LFP_i = \alpha + \beta SS_i/w_i + \varepsilon_i$
%
%Poorly identified as replacement rate $SS_i/w_i$ is function of
%$w_i$ and regression should control for $w_i$ non-parametrically
%
%Krueger and Pischke JOLE'92 use notch generation (larger benefits for a
%couple generations due to ad-hoc adjustments, affects level of benefit
%but not the slope of lifetime budget constraint) $\Rightarrow$
%Find small wealth effects of SS on retirement
%
%All this literature does not distinguish clearly wealth and substitution
%effects
%\end{slide}
%
%\begin{slide}
%\includepdf[pages={6}]{socialsecurity_attach.pdf}
%\end{slide}


\begin{slide}
\begin{center}
{\bf Retirement Hazard Spikes}
\end{center}
Retirement hazard at age $t$ is the fraction of people who retire at age $t$ among
those still working at age $t-1$

Retirement spike at Early Retirement Age of 62 very clear and convincing: spike moves
from 65 to 62 when the ERA was reduced from 65 to 62

$\Rightarrow$ Suggests strong
liquidity effects / non-rational behavior [outside the lifetime constraint model]

Evidence from other countries also shows strong spike effects

Note: those macro-level studies do not always define carefully retirement:
claiming benefits vs. stopping to work. Stopping to work is fuzzy.

\end{slide}

\begin{slide}
\includepdf[pages={7,8}]{socialsecurity_attach.pdf}
\end{slide}

\begin{slide}
\includepdf[pages={65}]{socialsecurity_attach.pdf}
\end{slide}

\begin{slide}
\begin{center}
{\bf Early Retirement Age effect on Retirement}
\end{center}

Best evidence from Manoli-Weber (2016b). Austria changed the ERA by cohorts
for those with less than 45+ contribution years (40+ for women)

Men goes from 60 to 62, Women goes from 55 to 57 (based on birth quarter)

Use population admin data on benefits claims and work. Sample is everybody working
at age 53.

1) Very strong effect on claiming age (benefits claiming) 
%[could be due to unfair actuarial adjustment in Austria]

2) Strong effect on retirement decision (work behavior)

3) Evidence of spillover effects on groups not affected [men (women) with 45+ (40+) contribution
years]


\end{slide}

\begin{slide}
\includepdf[pages={25-32}]{socialsecurity_attach.pdf}
\end{slide}


\begin{slide}
\begin{center}
{\bf Substitution Effects on Retirement Age}
\end{center}

Best evidence from Manoli-Weber AEJ'16. Austria has a system of discontinuous severance payments
for retirees based on tenure at job [that's separate from retirement benefits]

$\Rightarrow$ Creates notches [draw graph] in the lifetime budget constraint that can be exploited to estimate substitution
effects. Information on those notches likely to be widespread.

Use complete admin earnings data linking workers/firms/benefits claims. Key results:

1) Very clear evidence of substitution effects

2) Clear evidence that some people are constrained and cannot respond
[unhealthy sample] 

3) Overall implied elasticity is fairly modest [possibly due partly to frictional constraints, 
lack of information]

\end{slide}

\begin{slide}
\includepdf[pages={33-35}]{socialsecurity_attach.pdf}
\end{slide}

\begin{slide}
\begin{center}
{\bf Focal points and social norms}
\end{center}
SS programs have a normal retirement age (NRA) but such NRAs are not always associated with real economic incentives (e.g. US SS is actuarially fair so NRA should be irrelevant)

Seibold '21 shows that in Germany, 30\% of workers retire at statutory retirement ages even with no specific underlying economic incentives and not option by default

Seibold '21 shows that statutory age bunching much larger than bunching around kinks of lifetime budget constraints created by retirement system

$\Rightarrow$ Cannot be explained within standard model

$\Rightarrow$ NRA perceived as a social injunction obeyed by workers

$\Rightarrow$ Nominal NRAs can potentially be a powerful govt tool to change the retirement
age


\end{slide}

\begin{slide}
\includepdf[pages={60, 59}]{socialsecurity_attach.pdf}
\end{slide}

\begin{slide}
\includepdf[pages={61, 62}]{socialsecurity_attach.pdf}
\end{slide}

\begin{slide}
\includepdf[pages={63, 64}]{socialsecurity_attach.pdf}
\end{slide}



%\begin{slide}
%\begin{center}
%{\bf International Empirical Evidence}
%\end{center}
%
%Gruber and Wise books: extensive analysis within each country: 2 strong
%and consistent findings:
%
%1) Large effect of tax rate $t$ on LFP of elderly $\Rightarrow$
%Key to give good incentives to elderly to keep working if you want
%to increase retirement age
%
%2) Large effect of Early entitlement age on retirement decisions:
%many individuals show liquidity/myopic/focal point effects: they retire as
%soon as they can get some benefits (even under fair system like
%US)
%
%Gruber and Wise studies do not separate cleanly early retirement
%age effects from substitution effects due to implicit tax
%\end{slide}
%
%\begin{slide}
%\includepdf[pages={11}]{socialsecurity_attach.pdf}
%\end{slide}
%
%
%\begin{slide}
%\begin{center}
%{\bf Social Security and Retirement: Other Questions}
%\end{center}
%0) Few studies on how private DB pensions affect retirement
%in US (good case study is Brown JpubE'13 bunching study for Cal teachers)
%
%1) Does increasing retirement age increase unemployment? [in
%principle: not in a US type flexible market but maybe in rigid EU
%market]. Gruber and Wise (2010) book suggests no strong link in
%most countries.
%
%2) Interactions between pay seniority (diverging from marginal
%productivity) and retirement rules is very important (Japan system
%of forced retirement from career job at 60)
%
%3) Effect of retirement on longevity
%
%\end{slide}

%\begin{slide}
%\begin{center}
%{\bf US Private Pension Evidence}
%\end{center}
%
%Many private DB programs and reforms, relatively under-studied
%
%Brown UC Berkeley PhD 07 studies Cal Public School Teacher program: pension
%system created large kink in life-time budget constraint at age 60 which was moved to age 63
%
%Used to
%be strong bunching at age 60 which moved in part to age 63
%following the reform
%
%Substitution elasticity estimates are relatively small because
%kinks are huge
%
%More studies would be valuable
%
%\end{slide}

\begin{slide}
\begin{center}
{\bf SOCIAL SECURITY REFORM: PROBLEMS WITH CURRENT SYSTEM}
\end{center}

Rate of return $n+g$ has declined from over 3\% to about 2\% due
to:

\textbf{1) Demographics:} $n$: Retirement of baby boom large cohorts born 1945-1965:
1995: 3.3 workers per beneficiary, 2030: 2 workers per beneficiaries

Due to (a) fall in fertility, (b) increased longevity at retirement age (note bottom half earners have made
no life expectancy gains over last 2 decades while top half have gained).

\textbf{2) Growth:} $g$: Slower productivity growth since 1975 ($g$ has fallen from 2\% to 1\%)

System requires adjusting taxes or benefits to remain in balance.

\end{slide}

\begin{slide}
\begin{center}
{\bf 1983 GREENSPAN COMMISSION}
\end{center}

Demographic changes are predictable, so 1st reform was implemented
in 1983 (designed to solve budget problems over next 75 years)

1) Increased payroll taxes to build a trust-fund

2) Increased retirement age in the future (from age 65 to 67)

Trust fund invested in Treasury Bills (Fed gov debt):
$TF_{t+1}=TF_{t} \cdot (1+i)+SSTax_t-SSBen_t$

Trust fund peaked at \$2.8T in 2013 and now declines and will be exhausted by
2034, taxes will then cover about 75\%
of promised benefits

Requires additional adjustment: can fix it for next 75 years by
increasing payroll tax rate now by 1.7 percentage points or wait
till 2035 and then increase tax by 3.5 pp (not huge)



\end{slide}

\begin{slide}
\begin{center}
{\bf Political Economy of the Trust Fund (skip)}
\end{center}

In principle, $TF$ should have been net additional saving by the
govt to prepare for baby boom retirement costs

$S_t= [T_t^{NSS} - G_t^{NSS} - r_r TF_t] + [T_t^{SS} - G_t^{SS} +
r_t TF_t ]$

First term $S_t^{ON}$ is on-budget, second term (SS account) is
$S_t^{OFF}$ off-budget, $S_t$ is unified budget. If govt and media
concentrated on on-budget, TF could increase total US govt saving.

In practice, govt budget deficit presented to public/media is the
unified budget $S_t$ inclusive of SS surplus. Absent SS Trust fund
build up, govt deficit would have been worse by (1.5 GDP points in recent years).

When $TF$ stops growing and starts decreasing in coming years, US govt deficit
will look worse and will require adjustments in the non-SS sector

\end{slide}

\begin{slide}
\begin{center}
{\bf Is the Trust Fund a Store of Value? (skip)}
\end{center}

If Trust Fund works as intended $S_t^{OFF}$ should have no effect
on $S_t^{ON}$.

If govt focuses on unified budget $S_t$ taking $S_t^{OFF}$ as
exogenous, then $S_t^{OFF}$ will have a negative effect on
$S_t^{ON}$

Smetters AEA-PP'04 runs regressions:

$S_t^{ON} = \alpha + \beta S_t^{OFF} + X_t \delta
+ \varepsilon_t$

Finds $\beta<0$ (even $\beta<-1$) especially in period 1970-2002
(relative to 1949-1969)

Not very well identified but suggestive evidence that Trust Fund
has not disciplined the government
\end{slide}

\begin{slide}
\includepdf[pages={12}]{socialsecurity_attach.pdf}
\end{slide}

\begin{slide}
\begin{center}
{\bf SOCIAL SECURITY REFORM OPTIONS}
\end{center}

1) Increased contributions: increase tax rate or earnings cap [eliminating cap
entirely would likely produce income shifting so often paired with similar tax increase
on capital income of top earners]

2) Reduce benefits: straight cut not politically feasible: a) Index NRA
on life expectancy, b) Index benefits using chained CPI instead of
regular CPI, c) Make benefits fully taxable

3) Means-tested benefits: bad for savings incentives and could
make program politically unstable [a program for the poor is a
poor program]. Explains conservatives support.

4) Invest Trust Fund in higher yield assets (such as stock-market,
as proposed by Clinton in 1990s). Advantage: higher return on average and govt can
be a long-term investor. Issue: Socialism (or lobbying and
corruption in investment choices), investment
choices could be left to independent board

%5) Major reform: privatization

\end{slide}

\begin{slide}
\begin{center}
{\bf SOCIAL SECURITY PRIVATIZATION}
\end{center}

Two components:

1) Funding the system

2) Replace DB by DC:

benefits = past contributions + market return

Main proponent: Feldstein, main critic: Diamond

Pros: get higher return on contributions $r>n+g$, increase $K$ stock and future wages

Some countries such as Chile, Mexico, UK have privatized (partly) their systems

\end{slide}

\begin{slide}
\begin{center}
{\bf SOCIAL SECURITY PRIVATIZATION ACCOUNTING}
\end{center}
Exactly the reverse of pay-as-you-go calculations:

1) First generation loses as they need to fund current retirees
and own contributions. All future generations gain
[generational redistribution]

2) If govt increases debt to pay for current retirees: future
generations get higher return on contributions but need to re-pay
higher govt debt $\Rightarrow$ Complete wash for all generations

tax to pay debt interest = returns on funded contributions -
returns on paygo contributions

$\Rightarrow$ Only way funding generates real changes is by
hurting some transitional generations which have to double pay

Feldstein calculations look better bc $r_{\text{contributions}} >> r_{\text{govt debt}}$

\small Should govt exploit this equity-premium opportunity?

\end{slide}

\begin{slide}
\begin{center}
{\bf ADDITIONAL PRIVATIZATION ISSUES}
\end{center}

\textbf{1) Risk:} individuals bear investment risk (stock market fluctuates
too much relative to economy) and cannot count on defined level of benefits
[$\Rightarrow$ Privatization needs to include minimum pension provision]

\textbf{2) Annuitization:} hard to impose in privatized system bc of
political constraints [sick person forced to annuitize her wealth]
$\Rightarrow$ Some people will exhaust benefits before death and
be poor in very old age [looming problem with 401(k) system]

\textbf{3) Lack of financial literacy:} Individuals do not know how to
invest [1/N rules in 401k, Sweden case]. Complicated choice, govt
can do it for people more efficiently

\textbf{4) Administrative costs:} privatized systems (Chile, UK) admin
costs very high (1\% of assets) due to wasteful advertisement
by private mutual funds [SS has very low admin costs]


\end{slide}

\begin{slide}
\begin{center}
{\bf Notional Accounts System: Sweden and Italy}
\end{center}
1) Benefits = Contribution + fictitious return set by govt

2) Return in Sweden depends on life expectancy, population growth, wage
growth to insure financial stability: return rates are low ($n+g$)
but stable [in Italy, return=GDP growth]

3) System unfunded so no transitional sacrifice

4) Individuals understand link bt contributions and benefits

5) Mandatory annuitization based on cohort-life expectancy

6) Individuals can choose retirement age freely (system is almost
actuarially fair)

7) Could add minimum pension and incentives to contribute more through savings
(e.g., matching incentives)
\end{slide}

\begin{slide}
\begin{center}
{\bf DISABILITY INSURANCE}
\end{center}
Disability is conceptually close to retirement: some people
become unable to work before old age (due to accidents, medical
conditions, etc.)

All advanced countries offer public Disability Insurance (DI) almost
always linked to the public retirement system

DI allows people to get retirement benefits before the ``Early
Retirement Age'' if they are unable to work due to disability

$\Rightarrow$ DI is a way to
screen those who really need to retire early

\textbf{Empirics:} Bound and Burkhauser Handbook Labor Economics '99 provide survey of empirical evidence

\textbf{Theory:} Diamond-Sheshinski JpubE'05 analyze optimal DI

\end{slide}

\begin{slide}
\begin{center}
{\bf US DISABILITY INSURANCE}
\end{center}
1) Federal program funded by OASDI payroll tax, pays
SS benefits to disabled workers under retirement age (similar computation
of benefits based on past earnings)

2) Program started in 1956 and became more generous overtime
(age 50+ condition removed, definition of disability liberalized,
replacement rate has grown)

3) Eligibility: Medical proof of being unable to work for at least a year,
Need some prior work experience, 5 months waiting period with no earnings
required (screening device)

4) Social security examiners rule on applications. Appeal possible
for rejected applicants. Imperfect process with big type I and II errors
(Parsons AER'91) $\Rightarrow$ Scope for Moral Hazard

5) DI tends to be an absorbing state (most beneficiaries won't ever work again). Can earn up of \$1200/month while on DI.
\end{slide}



\begin{slide}
\begin{center}
{\bf US DISABILITY INSURANCE}
\end{center}
1) In 2018, about 10.2m DI beneficiaries (not counting
widows+children), about 5-6\% of working age (20-64) population

2) Very rapid growth: In 1960, less than 1\% of working age population was on DI

3) Growth particularly strong during recessions: early 1990s, late 00s.
Slight decline from 11m in 2013 to 10.2m in 2018


\textbf{Key empirical question:} Are DI beneficiaries unable to work? or are DI beneficiaries
not working because of DI.

\end{slide}

\begin{slide}
\includepdf[pages={14-15, 17}]{socialsecurity_attach.pdf}
\end{slide}

\begin{slide}
\includepdf[pages={19}]{socialsecurity_attach.pdf}
\end{slide}


%\begin{slide}
%\includepdf[pages={18}]{socialsecurity_attach.pdf}
%\end{slide}
%
%
%\begin{slide}
%\begin{center}
%{\bf DI Empirical Effects: Observational Studies}
%\end{center}
%Parallel growth of DI recipients and non-participation rates among men aged 45-54
%but causality link not clear
%
%\textbf{Cross-Sectional Evidence (Parsons JPE'80):} Does potential DI replacement rate
%have an impact on LFP decision?
%
%Uses cross-sectional variation in potential replacement rates
%
%NLSY data on men aged 45-59 from 1966-69
%
%OLS regression
%\[ NLFP_i=\alpha+ \beta DIreprate_i + \varepsilon_i \]
%Large $\beta>0$ effect that can fully explain decline in LFP among men 45+
%
%\end{slide}
%
%\begin{slide}
%\begin{center}
%{\bf DI Empirical Effects: Observational Studies}
%\end{center}
%\textbf{Issues with Cross-Sectional Evidence:}
%
%1) $DIreprate_i$ depends on wages (higher for low wage earners)
%and likely to be correlated with $\varepsilon_i$ (likelihood to become truly disabled)
%
%2) Impossible to control non-parametrically for wages in regression because all variation
%in $DIreprate_i$ is due to wages (destroys identification)
%
%3) Bound AER'89 replicates Parson's regression on sample that never applied to
%DI and obtains similar effects implying that the OLS correlation not driven by UI
%
%\end{slide}

\begin{slide}
\begin{center}
{\bf DI EMPIRICAL EFFECTS: REJECTED APPLICANTS}
\end{center}
Bound AER'89 bounds effect of DI on LFP rate
using data on LFP on (small sample of) rejected applicants as a counterfactual

\textbf{Idea:} If rejected applicants do not work, then surely DI recipients
would not have worked absent DI
$\Rightarrow$
Rejected applicants' LFP rate is an upper bound for LFP rate of DI
recipients absent DI

\textbf{Results:} Only 1/3 of rejected applicants return to work
and they earn less than half of the mean non-DI wage

$\Rightarrow$ at most 1/3 of the trend in male LFP decline can be
explained by shift to DI

Von Waechter-Manchester-Song AER'11 replicate Bound using full pop SSA admin data
and find similar results
\end{slide}

\begin{slide}
\includepdf[pages={20}]{socialsecurity_attach.pdf}
\end{slide}


\begin{slide}
\begin{center}
{\bf DI EMPIRICAL EFFECTS: REJECTED APPLICANTS}
\end{center}
Maestas-Mullen-Strand AER'13 obtain causal effect of DI
on LFP using natural variation in DI examiners' stringency and
large SSA admin data linking DI applicants and examiners

\textbf{Idea:} (a) Random assignment of DI appplicants to examiners and
(b) examiners vary in the fraction of cases they reject $\Rightarrow$ Valid
instrument of DI receipt

\textbf{Result 1:} DI benefits reduce LFP of applicants by 28 points $\Rightarrow$ DI has an impact
but fairly small (consistent with Bound AER'89)

\textbf{Result 2:} DI has heterogeneous impact: small effect on
those severely impaired but big effect on less severly impaired

\small Tough judges marginal cases unlikely to work without DI, lenient
judges marginal case somewhat likely to work without DI

% XX DI allowance rate means "fraction of applicants admitted in DI, high means DI is lenient"
\end{slide}

\begin{slide}
\includepdf[pages={21-23}]{socialsecurity_attach.pdf}
\end{slide}

\begin{slide}
\begin{center}
{\bf  Effect of DI Processing Time: Autor et al. 2015}
\end{center}
DI requires a lengthy application process and 5 months out of the labor force
$\Rightarrow$ Process takes 10 months on average

Being out of the labor force for 10 months could hurt future job prospects
$\Rightarrow$ Could partly explain why DI rejected applicants work so little $\Rightarrow$
DI could have higher negative effects %[Parsons 1991 reply to Bound]

Autor et al. 2015 test this using (quasi-random) variation in DI applications processing 
time due to backlog

Find that 1 sd processing time delay (2.1 month) reduces  employment rate by .36 points (3.5\%)
for denied applicants

\small
Rejected applicants often appeal which reduces long-term labor supply (even if appeal not successful)
$\Rightarrow$ Rejected applicants strategy underestimates DI effects

Accounting for the delay channel boosts negative DI effects by 50\% (from 17 points to 25 points in year 6)

% XX DI allowance rate means "fraction of applicants admitted in DI, high means DI is lenient"
\end{slide}

\begin{slide}
\begin{center}
{\bf DI Generosity Effects: Regression Kink Design (RKD)}
\end{center}
DI benefits calculated like SS benefits: AIME formula 
based on average life-time earnings creates a ``kinked'' relationship 

Ideal setting for an RKD (Card et al. 2015): test whether outcome such as earnings or mortality is also ``kinky''

1) Test first for no sorting of DI recipients around kink to validate RKD design [similar to RDD validation]

2) RKD estimate: Change in slope of outcome at kink / Change in slope of benefits at kink

a) Gelber et al. '17 analyze effects on earnings of DI generosity and find
an income effect of -\$0.2 per dollar of benefits

b) Gelber et al. 18 analyze effects on mortality: at lower bend point, \$1K extra DI/year reduces annual mortality by .25 points (1 out of 400 lives saved)
\end{slide}

\begin{slide}
\includepdf[pages={54-58}]{socialsecurity_attach.pdf}
\end{slide}


\begin{slide}
\begin{center}
{\bf DI and Unemployment: Autor and Duggan QJE'03}
\end{center}
DI claims raise in recessions (as partly disabled workers have less working options)
$\Rightarrow$ Reduces unemployment rate (DI recipients outside labor force)
and labor force participation

Test this hypothesis using cross-state variation in employment
shocks (using industry mix Bartik's instrument) [e.g., car
industry shock creates employment shock in Michigan]

Negative employment shocks do increase DI applications and reduce the size
of labor force (workers+job seekers)

DI keeps beneficiaries outside labor force permanently and is an inefficient
substitute to temporary unemployment insurance benefits

\end{slide}

\begin{slide}
\includepdf[pages={24}]{socialsecurity_attach.pdf}
\end{slide}

\begin{slide}
\begin{center}
{\bf  Fall in male LFP and health}
\end{center}
LFP (=workers+job seekers/population) of prime-age males (25-54) has fallen 10 pts from 98\% in 1950s to 88\% in 2010s (2 pts drop in 2007-10). Drop particularly large among least educated.

What can explain it? [Black et al. 2016 review potential explanations]

Generosity of govt programs (e.g. DI) or incarceration cannot explain it

Consistent with reduced work and pay opportunities (due to surge in inequality)

Possible that this is related to deteriorating health [see Case and Deaton 2015] in which case
DI increase recipiency is a symptom of the problem (not the cause)

\end{slide}

\begin{slide}
\includepdf[pages={46}]{socialsecurity_attach.pdf}
\end{slide}

\begin{slide}
\includepdf[pages={45}]{socialsecurity_attach.pdf}
\end{slide}

\begin{slide}
\includepdf[pages={47}]{socialsecurity_attach.pdf}
\end{slide}


\begin{slide}
\begin{center}
{\bf REFERENCES}
\end{center}
{\small

Aguiar, M. and E. Hurst ``Consumption vs. Expenditure'', Journal of Political Economy, Vol. 113, 2005, 919-948. \href{http://www.jstor.org/stable/pdfplus/10.1086/491590.pdf} {(web)}

Attanasio, O. and A. Brugiavinni ``Social Security and Household's Saving'', Quarterly Journal of Economics, Vol 118, 2003, 1499-1521. \href{http://links.jstor.org/stable/pdfplus/25053931.pdf} {(web)}

Attanasio, O. and S. Rohwedder ``Pension Wealth and Household Saving: Evidence from Pension Reforms in the United Kingdom'', American Economic   Review, Vol 93,  2003, 1121-1157. \href{http://links.jstor.org/stable/pdfplus/3132139.pdf} {(web)}

Autor, David H., and Mark G. Duggan. 2003. ``The Rise in the Disability Rolls and the Decline in
Unemployment.'' Quarterly Journal of Economics, 118(1): 157�205.
\href{http://www.jstor.org/stable/pdfplus/25053901.pdf} {(web)}

Autor, David, Nicole Maestas, Kathleen Mullen, Alexander Strand ``Does Delay Cause Decay? The Effect of Administrative Decision Time on the Labor Force Participation and Earnings of Disability Applicants'', NBER Working Paper No, 20840, 2015. \href{http://www.nber.org/papers/w20840.pdf} {(web)}

Barro, R. and G. MacDonald ``Social Security and Consumer Spending in an International Cross Section'', Journal of Public Economics, Vol. 11, 1979, 275-289. \href{http://elsa.berkeley.edu/~saez/course/Barro and MacDonald_JPubE(1979).pdf} {(web)}

Bernheim, D., J. Skinner, and S. Weinberg, ``What Accounts for the Variation in Retirement Wealth Among U.S. Households?'', American Economic Review,  Vol. 91, 2001, 832-857. \href{http://www.jstor.org/stable/pdfplus/2677815.pdf} {(web)} 

Black, Sandra, Jason Furman, Emma Rackstraw, Nirupama Rao. ``The long-term decline in US prime-age male labour force participation'', Voxeu.org column , July 2016. \href{http://voxeu.org/article/long-term-decline-us-prime-age-male-labour-force-participation-and-policies-address-it} {(web)} 

Blundell, Richard, Eric French, and Gemma Tetlow. 2017 ``Retirement Incentives and Labor Supply'',
\emph{Handbook of Population Aging} edited by John Piggott and Alan Woodland, Elsevier: North Holland.
\href{http://elsa.berkeley.edu/~saez/course/blundelletal17handbook.pdf} {(web)}


\textbf{Bound, John ``The Health and Earnings of Rejected Disability Insurance Applicants,'' \emph{ American Economic Review} 79 (1989), 482-503.
\href{http://www.jstor.org/stable/pdfplus/1806858.pdf} {(web)} }

Bound, John ``The Health and Earnings of Rejected Disability Insurance Applicants: Reply,''
\emph{American Economic Review} 81 (December 1991), 1427-1434.
\href{http://www.jstor.org/stable/pdfplus/2006931.pdf} {(web)}

Bound, John and Richard V. Burkhauser ``Economic analysis of transfer programs targeted on people with disabilities,'' In: Orley C. Ashenfelter and David Card, Editor(s), \emph{Handbook of Labor Economics,} Elsevier, 1999, Volume 3, Part C, 3417-3528.
\href{http://elsa.berkeley.edu/~saez/course/bound-burkhauserHLE99.pdf} {(web)}

Brown, K. ``The Link between Pensions and Retirement Timing: Lessons from California Teachers'', 
Journal of Public Economics, 98, 2013, 1--14.
2007 \href{http://elsa.berkeley.edu/~saez/course/brown_jpube13.pdf} {(web)}

Card, David, David S. Lee, Zhuan Pei, and Andrea Weber. 2015. ``Inference on Causal Effects in a
Generalized Regression Kink Design.'' Econometrica 83 (6): 2453-83.
\href{https://www.jstor.org/stable/pdf/43866417.pdf} {(web)}

Case, Anne and Angus Deaton.  ``Rising morbidity and mortality in midlife among white non-Hispanic Americans in the 21st century'',  PNAS 112(49), 2015. \href{http://www.pnas.org/content/112/49/15078.full.pdf} {(web)}

Chetty, Raj, John Friedman, Soren Leth-Petersen, Torben Nielsen, and Tore Olsen ``Active vs. Passive Decisions and Crowd-out in Retirement Savings Accounts: Evidence from Denmark.'' Quarterly Journal of Economics 129(3): 1141-1219, 2014  \href{http://elsa.berkeley.edu/~saez/course/chettyatQJE14savings.pdf} {(web)} 

Diamond, P. ``National Debt in a Neoclassical Growth Model'', American Economic Review, Vol. 55, 1965, 1126-1150. \href{http://links.jstor.org/stable/pdfplus/1809231.pdf} {(web)}

Diamond, P. ``A Framework for Social Security Analysis'', Journal of Public Economics, Vol. 8, 1977, 275-298. \href{http://elsa.berkeley.edu/~saez/course/Diamond_JPubE(1977).pdf} {(web)}

Diamond, P. and J. Mirrlees, ``A Model of Social Insurance with Variable Retirement,'' Journal of Public Economics, (1978)  295-336
\href{http://elsa.berkeley.edu/~saez/course/diamond-mirrleesJPubE78retirement.pdf} {(web)}

Diamond, P. and E. Sheshinski, ``Economic Aspects of Optimal Disability Benefits,'' Journal of Public Economics 57 (1995), 1-24.
\href{http://elsa.berkeley.edu/~saez/course/diamond-sheshinskiJpubE95DI-benefits.pdf} {(web)}

Feldstein, M. ``Social Security, Induced Retirement and Aggregate Capital Formation'', Journal of Political Economy, Vol. 82, 1974, 905-926. \href{http://links.jstor.org/stable/pdfplus/1829174.pdf} {(web)}

Feldstein, M. and J. Liebman ``Social Security'' in A. Auerbach and M. Feldstein, Handbook of Public Economics, Sections 1-5. \href{http://elsa.berkeley.edu/~saez/course/Feldstein and Liebman_Handbook.pdf} {(web)}

Feldstein, M. and A. Pellechio ``Social Security and Household Wealth Accumulation: New Microeconometric Evidence'', The Review of Economics and Statistics, Vol. 61, 1979, 361-368 \href{http://www.jstor.org/stable/pdfplus/1926065.pdf} {(web)}

Friedberg, L. ``The Labor Supply Effects of the Social Security Earnings Test'', Review of Economics and Statistics, Vol. 82, 2000, 48-63. \href{http://links.jstor.org/stable/pdfplus/2646671.pdf} {(web)}

Gelber, Alex, Damon Jones, and Dan Sacks. 2020
``Estimating Earnings Adjustment Frictions: Method and Evidence from the Earnings Test'', 
American Economic Journal: Applied Economics 12(1), 1-31
\href{http://elsa.berkeley.edu/~saez/course/gelber-jones-sacks19.pdf} {(web)} 

\textbf{Gelber, Alex, Timothy Moore, and Alexander Strand. 2018. ``The Effect of Disability Insurance Payments on Beneficiaries' Earnings,'' American Economic Journal: Economic Policy, 9(3),  229-261
\href{http://elsa.berkeley.edu/~saez/course/gelbermoorestrandDI17earnings.pdf} {(web)} }

Gelber, Alex, Timothy Moore, and Alexander Strand. 2018
``Disability Insurance Income Saves Lives'' UC Berkeley Working Paper
\href{http://elsa.berkeley.edu/~saez/course/gelbermoorestrandDI18mortality.pdf} {(web)} 

Gruber, Jon ``Disability Insurance Benefits and the Labor Supply of Older Persons,''
Journal of Political Economy, 108(6), 2000.
\href{http://www.jstor.org/stable/pdfplus/10.1086/317682.pdf} {(web)}

Gruber, J. and D. Wise (editors) \emph{Social Security and Retirement Around the World}. Chicago: University of Chicago Press, 1999 \href{http://www.nber.org/books/grub99-1} {(book online)}
  (Introduction Chapter is required reading)
\href{http://elsa.berkeley.edu/~saez/course/gruber-wiseNBER99ch1.pdf} {(web)} 

Gruber, J. and D. Wise (editors) \emph{Social Security Programs and Retirement Around the World: Micro Estimation}. Chicago, University of Chicago Press: 2004.
\href{http://www.nber.org/books/grub04-1} {(book online)}

Gruber, J. and D. Wise (editors) \emph{Social Security Programs and Retirement Around the World: The Relationship to Youth Employment}. Chicago: University of Chicago Press, 2010.

Hamermesh, D. ``Consumption During Retirement: The Missing Link in the Life-Cycle Hypothesis'', Review of Economics and Statistics, Vol. 66, 1984, 1-7. \href{http://links.jstor.org/stable/pdfplus/1924689.pdf} {(web)}

Krueger, Alan B.  and Jorn-Steffen Pischke ``The Effect of Social Security on Labor Supply: A Cohort Analysis of the Notch Generation'',
Journal of Labor Economics ,10(4), 1992, 412-437.
\href{http://www.jstor.org/stable/pdfplus/2535254.pdf} {(web)}

Leimer, R. and D. Lesnoy ``Social Security and Private Saving: New Time-Series Evidence'', The Journal of Political Economy, Vol. 90, 1982, 606-629. \href{http://www.jstor.org/stable/pdfplus/1831373.pdf} {(web)}

Lumsdaine, R., Mitchell, O. S., 1999. ``New developments in the economic analysis of retirement.''
In: Ashenfelter, O. C., Card, D. (Eds.), Handbook of Labor Economics, Volume 3C.
North Holland. \href{http://elsa.berkeley.edu/~saez/course/lumsdaine-mitchell1999handbook.pdf} {(web)} 

\textbf{Maestas, Nicole, Kathleen Mullen and Alexander Strand
``Does Disability Insurance Receipt Discourage Work?
Using Examiner Assignment to Estimate Causal Effects of SSDI Receipt'', American Economic Review, 103(5), 2013,
1797-1829.
\href{http://elsa.berkeley.edu/~saez/course/maestas-mullen-strandAER13.pdf} {(web)} }

Manoli, Day and Andrea Weber, ``Nonparametric Evidence on the Effects of Financial Incentives on Retirement Decisions,'' American Economic Journal: Economic Policy 8(4), 2016, 160--182.
\href{http://elsa.berkeley.edu/~saez/course/manoli-weberAEJ16.pdf} {(web)} 

\textbf{Manoli, Day and Andrea Weber, ``The Effects of the Early Retirement Age on Retirement Decisions'' NBER Working Paper 22561, 2016b \href{http://www.nber.org/papers/w22561.pdf} {(web)} }

Page, B. ``Social Security and Private Saving: A Review of the Empirical Evidence'',
Congressional Budget Office, 1998
\href{http://elsa.berkeley.edu/~saez/course/pageCBO98.pdf} {(web)}

Parsons, Donald ``The Decline of Male Labor Force Participation,'' Journal of Political Economy 88 (February 1980), 117-134.
\href{http://www.jstor.org/stable/pdfplus/1830962.pdf} {(web)}

Parsons, Donald, ``The Health and Earnings of Rejected Disability Insurance Applicants: Comment,''
\emph{American Economic Review} 81 (December 1991), 1419-1426.
\href{http://www.jstor.org/stable/pdfplus/2006930.pdf} {(web)}

Saez, Emmanuel  ``Public Economics and Inequality: Uncovering Our Social Nature'', AEA Papers and Proceedings, 121, 2021, 1-27
\href{https://eml.berkeley.edu/~saez/saez-AEAlecture.pdf} {(web)} 


Samuelson, P. ``An Exact Consumption Loan Model of Interest With or Without the Social Contrivance of Money'', Journal of Political Economy, Vol. 66, 1958. \href{http://links.jstor.org/stable/pdfplus/1826989.pdf} {(web)}

Scholz, J., A. Seshadri and S. Khitatrakun ``Are Americans Saving ``Optimally'' for Retirement?'', Journal of Political Economy, Vol. 114, 2006, 607-643. \href{http://links.jstor.org/stable/pdfplus/3840335.pdf} {(web)}

\textbf{Seibold, Arthur. 2021 ``Reference Points for Retirement Behavior: Evidence from German
Pension Discontinuities'', American Economic Review forthcoming  \href{http://elsa.berkeley.edu/~saez/course/seiboldAER21.pdf} {(web)} }

Social Security Administration \emph{Annual Statistical Report on the Social Security Disability Insurance Program.} \href{http://www.ssa.gov/policy/docs/statcomps/di_asr/} {(web)}

Smetters, K. ``Is the Social Security Trust Fund a Store of Value?'',  American Economic Review, Vol. 94, 2004, 176-181. \href{http://links.jstor.org/stable/pdfplus/3592878.pdf} {(web)} 

Von Wachter, Till, Jae Song. and Joyce Manchester, ``Trends in Employment and Earnings of Allowed and Rejected Social Security Disability Insurance Applicants'' \emph{American Economic Review}, 2011, Vol. 101 No. 7, 3308-29.
\href{http://elsa.berkeley.edu/~saez/course/vonwachterAER-DI.pdf} {(web)}

}
\end{slide}















\end{document}
