\documentclass[landscape]{slides}

\usepackage[landscape]{geometry}

\usepackage{pdfpages}

\usepackage{{hyperref}}

\def\mathbi#1{\textbf{\em #1}}

\topmargin=-1.8cm \textheight=17cm \oddsidemargin=0cm
\evensidemargin=0cm \textwidth=22cm

\author{Emmanuel Saez}

\date{Berkeley}

\title{230B: Public Economics \\
Tax Favored Retirement Accounts: \\ IRAs and 401(k)s} \onlyslides{1-300}

\newenvironment{outline}{\renewcommand{\itemsep}{}}

\begin{document}

\begin{slide}
\maketitle
\end{slide}


\begin{slide}
\begin{center}
{\bf RETIREMENT PROBLEM}
\end{center}

Individuals ability to work declines with aging $\Rightarrow$
Individuals continue to live after they are unwilling/unable to
work

{\bf Standard Model Prediction:} Absent any government program,
rational individual would save while working to consume savings
while retired

Optimal saving problem is extremely complex: uncertainty in
returns to saving, in life-span, in future ability/opportunities
to work, in future tastes/health

{\bf In practice:} When govt was small $\Rightarrow$ Many people
worked till unable to (often till death) and then were taken care
of by family members (paygo system not funded) [US elderly poverty
rate very high before SS]
\end{slide}



\begin{slide}
\begin{center}
{\bf SOURCES OF RETIREMENT INCOME}
\end{center}
1) Social Security retirement benefits: more than 50\%
of retirement income for 2/3 of US elderly families

2) Home Ownership: 75\% of US elderly are homeowners

3) Employer pensions (tax favored): 40-45\% of elderly US households have employer pensions.
Two types:

a) Traditional: DB and mandatory: {\bf employer} carries full
risk [in sharp decline, many in default]

b) New: DC and elective: 401(k)s, {\bf employee}
carries full risk

4) Supplementary individual elective pensions (tax favored): IRAs and Keoghs
(self-employed)

5) Extra savings through non-tax favored instruments:
significant only for wealthy minority [=10\% of retirees]
\end{slide}



\begin{slide}
\begin{center}
{\bf PRIVATE RETIREMENT PROGRAMS}
\end{center}

Used to be traditional DB plans: mandatory, employer manages
contributions and investment, benefits are annuitized and depend
on retirement age, tenure, and past salaries: highly regulated,
lots of risk for employers, risk for employees if employer goes
bankrupt [govt provides minimal insurance]

Shift to DC plans called 401(k)s: individual chooses level of
contributions (as \% of salary), investment choices
(through a mutual fund) $\Rightarrow$ All the decisions and risk
is on the employee

DB coverage used to be 50\% of workforce, 401(k) coverage is
around 60\% of workforce but only 40-45\% of employees participate
[CPS Contingent Work Supplements, CWS, 2005).

IRAs: Individual Retirement Arrangements, start in the 1970s,
additional private contributions for workers with no employer
pension or low incomes.
\end{slide}



\begin{slide}
\begin{center}
{\bf TAX ADVANTAGE}
\end{center}

All private pensions (DB+DC) and IRAs have always been tax favored:
contributions are not considered income, contributions grow tax
free (no tax on annual return), benefits or withdrawals are taxed
as ordinary income when received

Constant annual return $r$ and flat tax on capital and labor
income at rate $\tau$: \$1 earned and invested has value $V$ after $T$ years

1) NO TAXES: $V_{NT}=(1+r)^T$

2) TAXABLE ACCOUNT: $V_T=(1-\tau) (1+r(1-\tau))^T$

3) 401(k) or deductible IRA (back-end, postpaid tax): 
contributions deducted from taxable income: $V_{D}=(1+r)^T (1-\tau)$

10\% tax penalty on early withdrawals (before age 59.5)
\end{slide}



\begin{slide}
\begin{center}
{\bf BACK-END VS FRONT-END TAXES}
\end{center}
4)  Roth IRA (front-end, prepaid tax, introduced in 1998):
$V_R=(1-\tau) (1+r)^T$

$V_T<V_D=V_R<V_{NT}$

Note: $V_D \neq V_R$ if tax rates are not constant over time (bracket
change or tax reform)

Tax on dividends and capital gains is also less than
labor income tax

Note that investments in tax favored accounts still pay the corporate income
tax (but incidence is not clear)

Switch from Traditional to Roth IRA makes current federal budget look better
at the expense of future budgets. Switch is a net looser for govt revenue
if average return in IRAs is bigger than return of government debt
\end{slide}



\begin{slide}
\begin{center}
{\bf KEY QUESTIONS ABOUT IRAs and 401(k)s}
\end{center}
{\bf 1) Effects on Savings:}

a) Do they increase household savings and financial security in
retirement?

b) Do they increase national savings? National Savings = Household
+ Corporate + Govt savings

c) Identify the elasticity of savings or wealth with respect to
rate of return.

{\bf 2) Understanding Savings (behavioral economics):}

a) Do households respond solely to financial incentives? (net-rate
of return, match, etc.)

b) Do households respond to institutional features? (defaults, menu of investment choices, framing, etc.)
\end{slide}



\begin{slide}
\begin{center}
{\bf IRAs: Individual Retirement Arrangements}
\end{center}
Started in 1974 for workers with no employer pension

Eligibility extended to all workers in 1981

TRA'86 restricted eligibility only for those with no pension or
AGI below \$50K [non-deductible contributions possible, but not as
advantageous and not used much]

In 1998, Roth IRA introduced for AGI below \$100K (front-end tax
instead of back-end). 

%Govt budget looks better in short-term but
%negative long-term effects if IRA returns $>$ Govt debt interest.

2001-08 contribution limits increase from \$2K/year to \$5K/year
[low so bind in most cases], now indexed to inflation (limit is \$5.5K/year in 2014)

Individuals choose contributions and investment through mutual
funds [little regulation]
\end{slide}



\begin{slide}
\begin{center}
{\bf EMPLOYER BASED 401(k) PLANS}
\end{center}
Start in 1978, Key differences with IRAs:

1) worker can contribute to 401(k) only if his employer sponsors
such a plan. 60\% of workers eligible, 40-45\% participate

2) higher contribution limit: in 2014 \$17.5K/year (indexed)

3) contributions deducted from paycheck automatically once
enrolled in the plan

4) employers often offer matches to induce higher participation:
typical match 50\% up to contributions of 6\% of salary.

5) 401(k)s organized around the workplace: spillovers across
employees, financial education at the workplace

6) Opt-in vs. opt-out [employers can set default option]
\end{slide}



\begin{slide}
\begin{center}
{\bf IRAs and 401(k)s: Theoretical Effects on Savings}
\end{center}
Key questions: Absent IRAs or 401(k)s, how much less would
households save? How much less wealth would they have? Do
contributions represent new savings or simply shifting of other
saving?

Show graph: $(c_1,c_2)$: Savings $s=w-c_1$ and retirement wealth
$c_2=s (1+r)$ or $c_2=s(1+r(1-\tau))$: Tax subsidy increases $c_2$
(income+substitution effects), ambiguous effect on $c_1$ (and
hence savings $s$) [show also graph with IRA limit]

Controversial empirical question because no perfect identification
source. Survey JEP 1996: Engen-Gale-Scholz argue no effects on
savings, Poterba-Venti-Wise, argue strong effects on savings.

Bernheim Handbook chapter 2002 provides detailed survey of this older
literature
\end{slide}



\begin{slide}
\begin{center}
{\bf EMPIRICAL FINDINGS: Big Picture}
\end{center}

1) Engen-Gale-Scholz: Aggregate personal savings rate in the US has
decreased from 10\% in late 1970s to about 0\% in the 2000s in
spite of increase in 401(k)+IRA contributions

$\Rightarrow$ Suggests no effect
on savings but not conclusive as savings could have fallen more
absent 401k+IRAs.

2) Sum of total retirement savings to payroll have been stable

$\Rightarrow$ Suggests that increase in 401(k)s has just replaced
disappearing DB plans with no overall increase in retirement
savings.
\end{slide}

\begin{slide}
\includepdf[pages={1}]{ira401k_attach.pdf}
\end{slide}

\begin{slide}
\includepdf[pages={2,3}]{ira401k_attach.pdf}
\end{slide}

\begin{slide}
\begin{center}
{\bf IRA Effects}
\end{center}

Ideal experiment: randomized variation in IRA eligibility
and compare subsequent saving and wealth accumulation behavior.

Pb: IRA eligibility is not randomized (depends on employer pension
and AGI) and contribution decision is endogenous

1) Early literature: Compares wealth $W_i$ of contributors vs. non
contributors: OLS regression: $W_i = \alpha + \beta IRA_i + \varepsilon_i$.
Pb: contributors may have higher taste for savings

Solution: Control for observables (income, initial wealth, fixed
effects). Pb: omitted variable bias, results
sensitive to controls (Gale-Scholz AER 94 vs. Venti-Wise papers).

2) Exogenous changes in eligibility: 1982 expansion for workers
with employer pension (treatment) compared to workers with no
employer pension (control). DD estimator, pb is that no good data
are available
\end{slide}


%\begin{slide}
%\begin{center}
%{\bf IRA Non-Standard Effects}
%\end{center}
%
%1) TRA'86: AGI limit to IRA eligibility: contributions jumped in
%'82-'86 but collapsed after TRA 86, but even for those still
%eligible (Poterba-Venti-Wise, JpubE '95)
%
%Explanation: mutual funds stopped advertising IRAs (as most
%clients dropped out) and explains lower take-up among low income
%still eligible group
%
%2) Individuals delay IRA contributions for year $t$ till they file
%a return in year $t+1$ because (a) they need tax refunds to
%finance contributions (b) they think about it then
%\end{slide}


\begin{slide}
\begin{center}
{\bf 401(k) Effects on Savings}
\end{center}
Identification strategy: compare workers eligible to workers non
eligible [i.e. whether employer offers plan]

Data quality on saving+employer eligibility is poor: only SCF, SIPP, and
Health Retirement Survey (HRS) have decent information on this

Poterba-Venti-Wise: financial wealth $W_i= \alpha + \beta Elig_i +
X_i \gamma + \varepsilon_i$

They find large effects of eligibility on financial wealth even
controlling for observables $X$

\end{slide}


\begin{slide}
\begin{center}
{\bf 401(k) Effects on Savings}
\end{center}

\textbf{Issues:}

1) 401(k) eligibility is not randomized. Better employers more
likely to offer 401(k)s [even controlling for $X$] or employees
self-select into employers offering 401(k)s

2) Gap in assets between eligible and not-eligible is larger than
401(k) balances [Bernheim and Garrett]

\textbf{More recent study:} Gelber AEJ:EP '11

Uses the fact that many firms have a 1 or 2 year waiting
period for 401(k)s

Uses SIPP longitudinal data and finds both IRA and real savings 
crowd-out but results imprecise
\end{slide}


\begin{slide}
\begin{center}
{\bf 401(k) margins of substitution}
\end{center}
1) Housing wealth: Engen-Gale 1995 say that 401(k)s might crowd out
housing equity wealth (bigger or longer mortgage).

2) DB substitution: Engelhardt 2000 vs Poterba-Venti-Wise '01
debate.

Engelhardt includes DB imputed wealth in $W_i$ and 401(k) effects
go away but data very noisy so result is sensitive

In principle, want to use total net wealth including DB pensions
(not only financial wealth) in Poterba-Venti-Wise regression but
no good data.

Question is still controversial and does not even tell apart
wealth vs. saving outcomes, tax favored effects, match effects,
etc. 

$\Rightarrow$ Chetty et al. QJE14 study for Denmark makes huge progress

\end{slide}



\begin{slide}
\begin{center}
{\bf Behavioral Effects: Match rates}
\end{center}
Match produces huge incentives to contribute [dwarfs the tax
advantage]. In principle, great source of variation in incentives
to measure savings effects

Problem is that no good data on match and
savings $\Rightarrow$ Studies focus on contributions [using
company administrative data], two results :

1) Matches increase participation substantially

2) Substantial bunching at the kink point where match stops
[consistent with theory, could back-out an elasticity]

Such responses do not imply 401(k)s raise savings, could be all
reshuffling, although match likely increases retirement wealth

Engelhardt-Kumar, JpubE'07 using HRS data find no effect of
match on total savings (but results imprecise) 
\end{slide}


\begin{slide}
\begin{center}
{\bf Why do Employers offer 401(k) matching?}
\end{center}
Match creates a distortion in savings behavior $\Rightarrow$ inefficient
in a rational model

Two hypotheses have been put forward to explain company matches:

1) {\bf Equity regulations} put tight limits on how enrollment rates
can differ between highly compensated vs. other employees. Match
is a way to increase enrollment among non-highly compensated
employees

2) \textbf{Sophisticated hyperbolic employees:} if employees know they will save
too little because of self-control problems, they value ex-ante the match that
gives them incentives to save and overcome self-control problems

\end{slide}




\begin{slide}
\begin{center}
{\bf Match rate Effects: \\ IRA Randomized H\&R Block experiment}
\end{center}
Duflo at al. QJE'06: provide matches for IRA at the time of tax preparation (funded out of tax refund).
3 key findings:

1) significant effect of matches on probability of contributing
and contribution levels

2) people do not game the system (by contributing and withdrawing contributions with
10\% penalty afterwards)

3) effects of randomized simple and salient matches much larger than the effect
of the Saver's Credit which provides tax credit for IRA-401(k) contributions of low income earners

Saez AEJ-EP'09: compares the effects of matches to equivalent rebate: 50\% match is equivalent
to a 33\% rebate but match generates much larger effect (behavioral effect)
\end{slide}

\begin{slide}
\includepdf[pages={30,4,31,33,5-7}]{ira401k_attach.pdf}
\end{slide}

\begin{slide}
\begin{center}
{\bf Behavioral Effects: Default Effects in 401(k) decisions}
\end{center}
Madrian-Shea QJE'01: tremendous impact in economics:
effect of switching to automatic participation for new hires:

Before= [opt-in] new employees needed to voluntarily enroll

After = [opt-out] new employees are automatically enrolled by
default at a given contribution/investment [3\% salary, money
market fund]

Strategy: compare 401(k) outcomes for hires before and after
reform:

\end{slide}

\begin{slide}
\begin{center}
{\bf Behavioral Effects: Default Effects}
\end{center}

Two key findings of Madrian and Shea (2001)

1) Auto-enrollment has enormous impact on enrollment in short-term
(60 points) and substantial effect remains in long-run (30 points)

2) Most employees stick to default choice which could be bad for
long-term investment [2\% contribution default even though 50\% match
offered up to 6\% of contributions]

$\Rightarrow$ Individuals do not behave as in standard model where
defaults are irrelevant
\end{slide}

\begin{slide}
\includepdf[pages={8,9}]{ira401k_attach.pdf}
\end{slide}



\begin{slide}
\begin{center}
{\bf Default Effects, Extensions}
\end{center}
Series of papers by Choi-Laibson-Madrian-Metrick have confirmed
and replicated those results.

Quick enrollment (active choice required, need to choose) has also
a positive impact but not as large

Effect on savings and retirement wealth unknown [very hard to get
data on both 401(k) features and actual total savings and wealth]
(see Chetty et al. QJE'14 study below)

Default effects also found in match allocation, cash distributions, and annuitization decisions

\end{slide}

\begin{slide}
\includepdf[pages={10,11}]{ira401k_attach.pdf}
\end{slide}



\begin{slide}
\begin{center}
{\bf Framing Effects in Retirement Savings Decisions}
\end{center}
Many employers also provide mandatory employee or employer DC benefits:
e.g., employer provides 5\% of salary in DC pension, employer forces employees
to contribute 3\% of salary in DC pension.

Card and Ransom Restat'11 analyze whether changes in employer or employee mandatory
contributions have an impact on voluntary supplemental contributions (401k type)

In rational model, \$1 extra of employer and employee contribution should lead to
\$1 less of voluntary 401k contribution (as they are perfect substitutes)
\end{slide}

\begin{slide}
\begin{center}
{\bf Framing Effects in Retirement Savings Decisions}
\end{center}

Card and Ransom Restat'11 findings:

1) \$1 extra of employee mandatory contribution reduces voluntary contribution
by 70 cents

2) \$1 extra of employer mandatory contribution reduces voluntary contribution
by 30 cents

$\Rightarrow$ Two departures from standard model:

1) No one-to-one crowd out

2) Crowd-out rate is not the same for employer vs. employee mandatory contribution

Likely explanation: Employees do not pay attention. Employee mandatory contribution
reduce wages and hence are more visible

\end{slide}


\begin{slide}
\begin{center}
{\bf Active vs. Passive Savings Decisions: Chetty et al. '14}
\end{center}
They use admin data in Denmark on contributions and wealth
to analyze savings responses to retirement contributions. Two policies are analyzed: 

(a) Automatic contributions by firms (either voluntary or govt mandated) to workers� retirement savings accounts 

(b) Tax subsidies for retirement savings [similar to 401(k)]

Key results:

(a) Automatic contributions raise total savings much more than price subsidies 
because 85\% of people are passive

(b) Only 15\% exploit tax incentives and they do so with crowding out
(not real savings) %[Regression kink design]

Paper deals a devastating blow to 401(k) US policy agenda 

\end{slide}

\begin{slide}
\includepdf[pages={17-19,21}]{ira401k_attach.pdf}
\end{slide}

\begin{slide}
\begin{center}
{\bf Active vs. Passive Savings Decisions: Chetty et al. '14}
\end{center}
They exploit reduction in subsidy for capital pensions in 1999 for
upper income earners (above 250K DKr)

First stage: negative effect on capital pensions very clear: 
Does this come from reduced savings or by shifting into other forms of savings?

Second stage: Denmark has another form of tax favored pension savings called annuity pensions:
positive effect on annuity pensions very clear (crowd-out is 56\%)

Third stage: Effect on taxable savings: positive effect on taxable savings so that
in net, there is no reduction at all in total pension+regular savings: complete crowd-out

%Heterogeneity: the rich and the highly educated are much more likely to shift savings
%following reform
\end{slide}

\begin{slide}
\includepdf[pages={24, 34,25-26,43-49,42}]{ira401k_attach.pdf}
\end{slide}


\begin{slide}
\begin{center}
{\bf Default Effects in Asset Allocation}
\end{center}
Choi, Laibson, Madrian '07 study a firm that used two match systems
in their 401(k) plan

\textbf{1) Default Case:} Match allocated to employer stock and workers can reallocate 
(default is employer stock)

\textbf{1) No Default Case:} Match allocated to an asset actively chosen by workers; workers required to make an active designation.

Economically, these two systems are identical.
They both allow workers to do whatever the worker wants.

\end{slide}



\begin{slide}
\includepdf[pages={13}]{ira401k_attach.pdf}
\end{slide}

\begin{slide}
\begin{center}
{\bf Cash Distributions for Employees who Move}
\end{center}
What happens to savings plan balances when employees leave
their jobs?

1) Employees can request a cash distribution or roll balances over into another account

a) Balances $>\$5000$: default leaves balances with former employer

b) Balances $<\$5000$: default distributes balances as cash transfer

2) Vast majority of employees accept default (Choi et al. 2002, 2004a and 2004b)

3) When employees receive small cash distributions, balances typically consumed (Poterba, Venti and Wise 1998)
\end{slide}

\begin{slide}
\begin{center}
{\bf Post-Retirement Distributions}
\end{center}
\textbf{1) Social Security:}

a) Joint and survivor annuity (reduced benefits)

\textbf{2) Defined benefit pension:}

a) Annuity

b) Lump sum payout if offered

\textbf{3) Defined contribution savings plan:}

a) Lump sum payout

b) Annuity if offered
\end{slide}

\begin{slide}
\begin{center}
{\bf  Defined Benefit Pension Annuitization}
\end{center}
1) Annuity income and economic welfare of the elderly

a) Social Security replacement rate relatively low on average

b) 17\% of women fall into poverty after the death of their spouse (Holden and Zick 2000)

2) For married individuals, three distinct annuitization regimes

a) Pre-1974: no regulation

b) ERISA I (1974): default joint-and-survivor annuity with
option to opt-out: joint-and-survivor annuitization increases 25 percentage points (Holden and Nicholson 1998)

c) ERISA II (1984 amendment): opting out required notarized permission of spouse:
joint-and-survivor annuitization increases 5 to 10 percentage points (Aura 2005)


\end{slide}

\begin{slide}
\begin{center}
{\bf Saving More Tomorrow}
\end{center}
Thaler and Benartzi JPE '04: experiment in a medium sized firm with 300
employees:

Program has a consultant talk to employees and run them through an
savings software to determine required 401(k) saving rate.

Individuals can decide to commit to invest a fixed percentage of
their future pay raises to 401(k) (like 50\% of all future pay
rises).

Results: individuals who commit obtain much higher contribution
rates than those who did not.

Looks like a non-binding commitment can have a huge effect on
savings.
\end{slide}


\begin{slide}
\begin{center}
{\bf Financial Education and Peer Effects}
\end{center}
Various studies on the effects of financial educations: pamphlets,
seminars, etc.

Two studies have shown that there are strong peer effects at
the workplace about 401(k) decisions: 

1) observational study (Duflo
and Saez, JpubE '03) 

2) randomized experiment (Duflo and Saez, QJE '03)

Effects of financial education and peer effects are very small relative to default
effects
\end{slide}


\begin{slide}
\begin{center}
{\bf Financial Education: Duflo and Saez QJE'03}
\end{center}
Randomized experiment within one university to induce individuals to
attend the benefits fair (providing information on benefits including
401k).

Offer a \$20 reward for attending fair for a random group of employees
within a random sample of departments

{\bf 1st stage:} Attendance rate: 28\% for treated individuals
in treated depts, 15\% for untreated individuals in treated depts, 5\% in untreated
depts $\Rightarrow$ Strong peer effects in decision to attend benefits fair

{\bf 2nd stage:} Use 401k enrollment:
Enrollment rates in treated departments significantly higher (2 percentage points) than
in control departments with same positive effect on treated and untreated
individuals within treated departments

\end{slide}



\begin{slide}
\begin{center}
{\bf Bottom line on Behavioral Effects}
\end{center}
Financial education, peer effects, framing effects, and \textbf{especially
enrollment procedures} can have a large effect on participation.

Based on Chetty et al. QJE'14 (for Denmark), they likely have large effects on total personal savings 
[hard to believe people are swayed by small
things in 401(k) decisions but then offset it all rationally along
other dimensions].

This psychological or behavioral evidence suggests that 401(k)
have strong effects and that it is much cheaper to affect
savings through other channels than pure economic incentives

Libertarian paternalism (Thaler and Sunstein 2005, 2008): changing the default
imposes minimal costs on rational individuals and can \textbf{nudge} non-rational agents
in a desirable direction.

\end{slide}

\begin{slide}
\begin{center}
{\bf REFERENCES}
\end{center}
{\small

Aura, S. ``Does the Balance of Power Within a Family Matter? The Case of the Retirement Equity Act'', Journal of Public Economics, Vol. 89, 2005, 1699-1717. \href{http://elsa.berkeley.edu/~saez/course/Aura_JPubE(2005).pdf} {(web)}

\textbf{Bernheim, D. ``Taxation and Saving'', in A. Auerbach and M. Feldstein, Handbook of Public Economics, Volume 3, Chapter 18, Amsterdam: North Holland, 2002, Section 4. \href{http://elsa.berkeley.edu/~saez/course/Bernheim_Handbook.pdf} {(web)} }

Bernheim, B.D. and D.M. Garrett ``The Determinants and Consequences of Financial Education in the Workplace: Evidence from a Survey of Households'', NBER Working Paper No. 5667, 1996. \href{http://www.nber.org/papers/w5667} {(web)}

Card, David  and Michael Ransom (2011) ``Pension Plan Characteristics and Framing Effects in Employee Savings Behavior'', 
Review of Economics and Statistics, 93(1), 228-243,
NBER Working Paper No. 13275, July 2007
\href{http://www.nber.org/papers/w13275.pdf} {(web)}

\textbf{Chetty, Raj, John Friedman, Soren Leth-Petersen, Torben Nielsen, and Tore Olsen ``Active vs. Passive Decisions and Crowd-out in Retirement Savings Accounts: Evidence from Denmark.'' 
NBER Working Paper No. 18565, 2012, Quarterly Journal of Economics, 2014\href{http://www.nber.org/papers/w18565} {(web)} }

Choi, J., D. Laibson and B. Madrian ``Reducing the Complexity Costs of 401(k) Participation Through Quick Enrollment'', NBER Working Paper No. 11979, 2006. \href{http://www.nber.org/papers/w11979.pdf} {(web)}

Choi, J., D. Laibson and B. Madrian ``\$100 Bills on the Sidewalk: Suboptimal Saving in 401(k) Plans'', NBER Working Paper No. 11554, 2005. \href{http://www.nber.org/papers/w11554} {(web)}

Choi, J., D. Laibson, B. Madrian. and A. Metrick  ``For Better or For Worse: Default Effects and 401(k) Savings Behavior''. In David Wise, editor, Perspectives on the Economics of Aging, pp. 81-121. Chicago: University of Chicago Press, 2004, also NBER Working Paper No. 8651, 2001. \href{http://www.nber.org/papers/w8651} {(web)}

Choi, J., D. Laibson, B. Madrian. and A. Metrick ``Optimal Defaults'', American Economic Review, Vol. 93, 2003, 180-185. \href{http://links.jstor.org/stable/pdfplus/3132221.pdf} {(web)}

Choi, J., D. Laibson, B. Madrian. and A. Metrick ``Defined Contribution Pensions: Plan Rules, Participant Decisions, and the Path of Least Resistance'' In James Poterba, editor, Tax Policy and the Economy 16, 2002, pp. 67-114, also NBER Working Paper No. 8655, 2001. \href{http://www.nber.org/papers/w8655} {(web)}

Choi, J., D. Laibson, B. Madrian. and A. Metrick ``Passive Decisions and Potent Defaults'', NBER Working Paper No. 9917, 2003. \href{http://www.nber.org/papers/w9917} {(web)}

Duflo, E. and E. Saez ``Participation and Investment Decisions in a Retirement Plan: The Influence of Colleagues' Choices'', Journal of Public Economics, Vol. 85, 2002, 121-148. \href{http://elsa.berkeley.edu/~saez/course/Duflo and Saez_JPubE(2002).pdf} {(web)}

Duflo, E. and E.Saez ``The Role of Information and Social Interactions in Retirement Plan Decisions: Evidence from a Randomized Experiment'', Quarterly Journal of Economics, Vol. 118, 2003, 815-842. \href{http://www.jstor.org/stable/pdfplus/25053924.pdf} {(web)}

Duflo, E., W. Gale, J. Liebman, P. Orszag and E. Saez ``Saving Incentives for Low- and Middle-Income Families: Evidence from a Field Experiment with H\&R Block'', Quarterly Journal of Economics, Vol. 121, 2006, 1311-1346. \href{http://elsa.berkeley.edu/~saez/course/Duflo et al,QJE,2006.pdf} {(web)}

Engen, E., and W. Gale, "Debt,Taxes and the Effects of 401 (k) Plans on Household Wealth Accumulation," mimeo, October 1995. \href{http://elsa.berkeley.edu/~saez/course/Engen,Gale(1997).pdf} {(web)}

Engen, E., W. Gale and J. Scholz ``The Illusory Effects of Saving Incentives'', Journal of Economic Perspectives, Vol. 10, 1996, 113-138. \href{http://links.jstor.org/stable/pdfplus/2138557.pdf} {(web)} 

Engelhardt, G. ``Have 401(k)s Raised Household Saving? Evidence from the Health and Retirement Study'', Syracuse Center for Policy Research Working Paper, 2000. \href{http://elsa.berkeley.edu/~saez/course/Engelhardt(2000).pdf} {(web)}

Engelhardt, G. and A. Kumar (2007) ``Employer Matching and 401(k) Saving: Evidence from the Health and Retirement Study'', 
Journal of Public Economics, 91(10), 1920-43. \href{http://elsa.berkeley.edu/~saez/course/engelhardt-kumarJpubE07.pdf} {(web)}
%\href{http://www.nber.org/papers/w12447} {(web)}

Gale, W. and J. Scholz ``IRAs and Household Saving'', American Economic Review, Vol. 84, 1994, 1233-1260. \href{http://links.jstor.org/stable/pdfplus/2117770.pdf} {(web)}

Gelber, Alexander  (2011) ``How do 401(k)s Affect Saving? Evidence from Changes in 401(k) Eligibility,'' American Economic Journal: Economic Policy, 3:4, 103-122
\href{http://elsa.berkeley.edu/~saez/course/gelberAEJ11.pdf} {(web)}

Holden, K., Nicholson, S., 1998. ``Selection of a joint-and-survivor pension.''
Institute for Research on Poverty Discussion Paper No. 1175�98.
 \href{http://elsa.berkeley.edu/~saez/course/HoldenNicholson98.pdf} {(web)}

Holden, K., and K. Zick ``Distributional changes in income and wealth upon widowhood: Implications for private insurance and public policy'' In: Retirement needs framework, 69�C79. SOA Monograph M-RS00-1. Schaumburg, IL: Society of Actuaries, 2000. \href{http://elsa.berkeley.edu/~saez/course/Holden,Zick(2000).pdf} {(web)}

\textbf{Madrian, B. and D. Shea ``The Power of Suggestion: Inertia in 401(k) Participation and Savings Behavior'', Quarterly Journal of Economics, Vol.116, 2001, 1149-1188. \href{http://links.jstor.org/stable/pdfplus/2696456.pdf} {(web)} }

Poterba, J., S. Venti and D. Wise ``Do 401(k) Contributions Crowd Out Other Personal Saving?'', Journal of Public Economics, Vol. 58, 1995, 1-32. \href{http://elsa.berkeley.edu/~saez/course/Poterbaetal_JPubE(1995).pdf} {(web)}

Poterba, J., S. Venti and D. Wise ``How Retirement Saving Programs Increase Saving'', Journal of Economic Perspectives, Vol. 10, 1996, 91-112. \href{http://links.jstor.org/stable/pdfplus/2138556.pdf} {(web)} 

Poterba, J., S. Venti and D. Wise ``401 (k) Plans and Future Patterns of Retirement Saving'', The American Economic Review, Vol. 88, No. 2, Papers and Proceedings of the Hundred and Tenth Annual Meeting of the American Economic Association (May, 1998), 179-184. \href{http://www.jstor.org/stable/pdfplus/116915.pdf} {(web)}

Poterba, J., S. Venti and D. Wise ``The Transition to Personal Accounts and Increasing Retirement Wealth: Macro and Micro Evidence'', NBER Working Paper No. 8610, 2001. \href{http://www.nber.org/papers/w8610} {(web)}

Saez, E. ``Details Matter: The impact of Presentation and Information in the Take-up of Financial Incentives for Retirement Savings'', American Economic Journal: Economic Policy, Vol. 1, 2009, 204-228. \href{http://elsa.berkeley.edu/~saez/course/Saez_AEJ(2009).pdf} {(web)}

Thaler, R. and S. Benartzi ``Save More Tomorrow: Using Behavioral Economics to Increase Employee Saving'', Journal of Political Economy, Vol. 112, 2004, 164-187. \href{http://links.jstor.org/stable/pdfplus/3555217.pdf} {(web)}

\textbf{Thaler, Richard H. and Cass R. Sunstein ``Libertarian Paternalism''
American Economic Review, Vol. 93, No. 2, 2003, 175-179.
\href{http://www.jstor.org/stable/pdfplus/3132220.pdf} {(web)} }

Thaler, Richard H. and Cass R. Sunstein  \emph{Nudge: Improving Decisions About Health, Wealth, and Happiness},
2008.

Venti, S. and D. Wise ``Have IRAs Increased U.S. Saving? Evidence from Consumer Expenditure Surveys'', Quarterly Journal of Economics, Vol. 105, 1990, 661-698. \href{http://links.jstor.org/stable/pdfplus/2937894.pdf} {(web)}


}
\end{slide}



\end{document}
