\documentclass[landscape]{slides}

\usepackage[landscape]{geometry}

\usepackage{pdfpages}

\usepackage{hyperref}

\def\mathbi#1{\textbf{\em #1}}

\topmargin=-1.8cm \textheight=17cm \oddsidemargin=0cm
\evensidemargin=0cm \textwidth=22cm

\author{Emmanuel Saez}
\date{}

\title{Graduate Public Economics \\
Introduction and Road Map } \onlyslides{1-300}

\newenvironment{outline}{\renewcommand{\itemsep}{}}

\begin{document}

\begin{slide}
\maketitle
\end{slide}

\begin{slide}
\begin{center}
{\bf PUBLIC ECONOMICS DEFINITION}
\end{center}
Public economics = Study
of the role of the government in the economy

Government is instrumental in most aspects of economic life:

1) Government in charge of huge regulatory structure

2) Taxes: governments in advanced economies collect 30-50\% of National Income in taxes

3) Expenditures: tax revenue funds traditional \textbf{public goods} (infrastructure, public order and safety, defense),
and \textbf{social state} (education, retirement benefits, health care, income support)

4) Macro-economic stabilization through central bank (interest rate, inflation control), fiscal stimulus, bailout policies

\end{slide}

\begin{slide}
\includepdf[pages={37, 36}, scale=1.05]{introPE_attach.pdf}
\end{slide}

\begin{slide}
\begin{center}
{\bf Bigger view on government (Saez 2021)}
\end{center}
Economists have a narrow minded view of individual behavior: selfish and rational individuals interacting through markets

But social interactions critical for humans:  we cooperate at many levels: families, workplaces, communities, nation states; 
Beyond subsistence, value of income is largely relative

Governments are a formal way to organize cooperation 

Archaic human societies depended on social cooperation for protection and taking care of the young, sick, and old

$\Rightarrow$ Explains best why our modern nation states provide defense and education, health care, and retirement benefits

Replacing social institutions by markets does not always work  

\small E.g., Retirement benefits: Saving for your own retirement is economically rational but in practice most people unable to do so unless institutions (employers/government) help them 


\end{slide}






\begin{slide}
\begin{center}
{\bf For Economists: \\ Two General Rules for Government Intervention}
\end{center}

1) Failure of 1st Welfare Theorem: Government intervention can help
if there are market or individual failures

2) Fallacy of the 2nd Welfare Theorem: Distortionary Government
intervention is required to reduce economic inequality
\end{slide}

\begin{slide}
\begin{center}
{\bf Role 1: 1st Welfare Theorem Failure}
\end{center}

{\bf 1st Welfare Theorem:} If (1) no externalities, (2) perfect
competition, (3) perfect information, (4) agents are rational,
then private market equilibrium is Pareto efficient

Government intervention may be desirable if:

1) Externalities require government interventions (Pigouvian
taxes/subsidies, public good provision)

2) Imperfect competition requires regulation (typically studied in
Industrial Organization)

3) Imperfect or Asymmetric Information (e.g., adverse
selection may call for mandatory insurance)

4) Agents are not rational ({\bf = individual failures}
analyzed in behavioral economics, field in huge expansion): e.g.,
myopic or hyperbolic agents may not save enough for retirement

\end{slide}

\begin{slide}
\begin{center}
{\bf Role 2: 2nd Welfare Theorem Fallacy}
\end{center}

Even with no market failures, free market might generate
substantial inequality. 
Inequality is an issue because human are social beings: people care
about their relative situation.

{\bf 2nd Welfare Theorem:} Any Pareto Efficient outcome can be
reached by (1) Suitable redistribution of initial endowments
[individualized {\bf lump-sum} taxes based on indiv.
characteristics and not behavior], (2) Then letting markets work
freely

$\Rightarrow$ No conflict between efficiency and equity [1st best taxation]

Redistribution of initial endowments is not feasible
(information pb) $\Rightarrow$ govt needs to use {\bf distortionary} taxes
and transfers  $\Rightarrow$ Trade-off between
efficiency and equity [2nd best taxation]

This class will focus primarily but not exclusively on role 2

\end{slide}


\begin{slide}
\begin{center}
{\bf Illustration of 2nd Welfare Theorem Fallacy}
\end{center}
Suppose economy is populated 50\% with disabled people unable to work (hence they earn \$0) and 50\% with able people
who can work and earn \$100

\textbf{Free market outcome:} disabled have \$0, able have \$100

\textbf{2nd welfare theorem:} govt is able to tell apart the disabled from the able [even if the able do not work] 

\small
$\Rightarrow$
can tax the able by \$50 [regardless of whether they work or not] to give \$50 to each disabled person $\Rightarrow$ the able keep working [otherwise they'd have zero income and still have to pay \$50]

\normalsize 
 
\textbf{Real world:} govt can't tell apart disabled from non working able

\small
$\Rightarrow$ \$50 tax on workers + \$50 transfer on non workers destroys all incentives to work $\Rightarrow$ govt can no longer do full redistribution $\Rightarrow$ Trade-off between equity and size of the pie

\small



\end{slide}


\begin{slide}
\begin{center}
{\bf Normative vs. Positive Public Economics}
\end{center}

{\bf Normative Public Economics:} Analysis of How Things Should be
(e.g., should the government intervene in health insurance market?
how high should taxes be?, etc.)



{\bf Positive Public Economics:} Analysis of How Things Really Are
(e.g., Does govt provided health care crowd out private health
care insurance? Do higher taxes reduce labor supply?)

Positive Public Economics is a required 1st step before we can
complete Normative Public Economics

Positive analysis is primarily empirical and Normative analysis is
primarily theoretical

Positive Public Economics overlaps with Labor Economics

{\bf Political Economy} is a positive analysis of govt outcomes
[public choice is political economy from a libertarian view]
%focusing on \textbf{government failures}]


\end{slide}


\begin{slide}
\begin{center}
{\bf Individual Failures vs. Paternalism}
\end{center}

In many situations, individuals may not or do not seem to act in
their best interests [e.g., many individuals are not able to save
for retirement]

Two Polar Views on such situations:

{\bf 1)  Individual Failures [Behavioral Economics View]}
Individual do not behave as in standard model: Self-control problems, Cognitive
limitations, Social behavior

{\bf 2) Paternalism [Libertarian Chicago View]} Individual
failures do not exist and govt wants to impose on individuals its
own preferences against individuals' will


Key way to distinguish those 2 views: Under Paternalism,
individuals should be opposed to govt programs such as Social
Security. If individuals understand they have failures, they will
tend to support govt programs such as Social Security.

\end{slide}




\begin{slide}
\begin{center}
{\bf Plan for 230B Lectures}
\end{center}
1) {\bf Labor Income Taxation and Redistribution (SAEZ):} (a) Normative Aspects: Optimal Income Taxes and Transfers,
(b) Empirical Aspects: Labor Supply and Taxes and Transfers, (c) Social security retirement and disability benefits

2)  {\bf Wealth inequality and taxing capital income (ZUCMAN):} (a) Wealth inequality, (b) Taxation of capital income, (c) International tax and tax enforcement issues

%3)  {\bf Environmental Economics (JOE SHAPIRO):} Guest lecture, last day of class

%3)  {\bf Local public finance (YAGAN):} (a) Property taxation, the Tiebout model, and education finance, (b) Place-based policies and capitalization

%2) {\bf Capital Income Taxation and Redistribution} (a) Empirical Aspects: Wealth Accumulation, Savings, and Taxation, (b) Normative Aspects: Optimal Capital Income Taxation

%3) {\bf Social Insurance:} (a) Social Security and Retirement and Savings Decisions, (b) Disability Insurance

\end{slide}

\begin{slide}
\begin{center}
{\bf Income Inequality: Labor vs. Capital Income}
\end{center}
Individuals derive market income (before tax) from {\bf labor} and
{\bf capital}: $z=wl+rk$ where $w$ is wage, $l$ is labor supply,
$k$ is wealth, $r$ is rate of return on wealth

1) {\bf Labor income inequality} is due to differences in working abilities
(education, talent, physical ability, etc.), work effort (hours of
work, effort on the job, etc.), and luck (labor effort might
succeed or not)

2) {\bf Capital income inequality} is due to differences in
wealth $k$ (due to past saving behavior and inheritances
received), and in rates of return $r$ (varies dramatically
overtime and across assets)

Entrepreneurs start with labor which then transmutes into
wealth (e.g., Zuckerberg with Facebook)

\end{slide}


\begin{slide}
\begin{center}
{\bf Macro-aggregates: Labor vs. Capital Income}
\end{center}
National Income=GDP - depreciation of K+net foreign income

Labor income $wl \simeq$ 70-75\% of national income $z$

Capital income $ rk \simeq$ 25-30\% of national income $z$ (has increased
in recent decades)

Wealth stock  $k \simeq 500\%$ of national income $z$ (and increasing).
Wealth comes from past savings and price effects.

Rate of return on capital $ r \simeq 6\%$

$\alpha = \beta \cdot r$ where $\alpha= r k/z$ share of capital income and
$\beta=k/z$ wealth to income ratio

In GDP, gross capital share is higher (35-40\%) because it includes
depreciation of capital ($\simeq 10\%$ of GDP)


\end{slide}

%\begin{slide}
%\includepdf[pages={19}, scale=.9]{introPE_attach.pdf}
%\end{slide}
%
%\begin{slide}
%\includepdf[pages={9,10}]{introPE_attach.pdf}
%\end{slide}


\begin{slide}
\begin{center}
{\bf Income Inequality: Labor vs. Capital Income}
\end{center}

Capital Income (or wealth) is more concentrated than Labor
Income. In the US:

Top 1\% wealth holders have 40\% of total private wealth (Saez-Zucman 2016). Bottom 50\% wealth holders
hold almost no wealth.

Top 1\% incomes earn about 20\% of total national income on a pre-tax basis (Piketty-Saez-Zucman, 2018)

Top 1\% labor income earners have about 15\% of total labor income

\end{slide}





\begin{slide}
\begin{center}
{\bf Income Inequality Measurement}
\end{center}
Inequality can be measured by indexes such as Gini, log-variance,
quantile income shares which are functions of the income
distribution $F(z)$

Gini = 2 * area between 45 degree line and Lorenz curve

Lorenz curve $L(p)$ at percentile $p$ is fraction of total income
earned by individuals below percentile $p$

$0 \leq L(p) \leq p$

Gini=0 means perfect equality

Gini=1 means complete inequality (top person has all the income)


\end{slide}

\begin{slide}
\includepdf[pages={29}, scale=1]{introPE_attach.pdf}
\end{slide}


\begin{slide}
\begin{center}
{\bf Key Empirical Facts on Income/Wealth Inequality}
\end{center}
1) In the US, labor income inequality has increased substantially
since 1970: due to skilled biased technological progress
vs. institutions (min wage and Unions) [Autor-Katz'99]

2) US top income shares dropped dramatically from 1929 to
1950 and increased dramatically since 1980. Bottom 50\% incomes
have stagnated in real terms since 1980 [Piketty-Saez-Zucman '18 distribute
full National Income]

%3) Top US incomes used to be primarily capital income. Now, top
%incomes are divided 50/50 between labor and capital income (explosion of top labor incomes with stock-options, etc.)

3) Fall in top income shares from 1900-1950 happened in most
OECD countries. Surge in top income shares has happened primarily
in English speaking countries, and not as much in Continental Europe
and Japan [Atkinson, Piketty, Saez JEL'11]

\end{slide}

\begin{slide}
\includepdf[pages={1}, scale=.9]{introPE_attach.pdf}
\end{slide}

\begin{slide}
\includepdf[pages={30}, scale=1]{introPE_attach.pdf}
\end{slide}



\begin{slide}
\includepdf[pages={23}, scale=.9]{introPE_attach.pdf}
\end{slide}

%\begin{slide}
%\includepdf[pages={24}, scale=.9]{introPE_attach.pdf}
%\end{slide}

\begin{slide}
\includepdf[pages={25}, scale=1.05]{introPE_attach.pdf}
\end{slide}


%\begin{slide}
%\includepdf[pages={17-18}, scale=.9]{introPE_attach.pdf}
%\end{slide}


\begin{slide}
\begin{center}
{\bf Measuring Intergenerational Income Mobility}
\end{center}
Strong consensus that children's success should not depend too much on parental income [Equality of Opportunity]

Studies linking adult children to their parents can measure link between children and parents income

Simple measure: average income rank of children by income rank of parents [Chetty et al. 2014]

1) US has less mobility than European countries (especially Scandinavian countries such as Denmark)

2) Substantial heterogeneity in mobility across cities in the US

3) Places with low race/income segregation, low income inequality, good K-12 schools, high social capital, high family stability tend to have high mobility [these are correlations and do not imply causality]

\end{slide}

\begin{slide}
\includepdf[pages={11-16}]{introPE_attach.pdf}
\end{slide}


\begin{slide}
\begin{center}
{\bf Govt Redistribution with Taxes and Transfers}
\end{center}

Government taxes individuals based on income and consumption and
provides transfers: $z$ is pre-tax income, $y=z-T(z)+B(z)$ is
post-tax income

1) If inequality in $y$ is less than inequality in $z$
$\Leftrightarrow$ tax and transfer system is redistributive (or
progressive)

2) If inequality in $y$ is more than inequality in $z$
$\Leftrightarrow$ tax and transfer system is regressive

\small
a) If $y=z \cdot (1-t)$ with constant $t$, tax/transfer system is
neutral

b) If $y=z \cdot (1-t)+G$ where $G$ is a universal (lumpsum)
allowance, then tax/transfer system is progressive 

c) If $y=z-T$ where $T$ is a uniform tax (poll tax), then
tax/transfer system is regressive

Current tax/transfer systems in rich countries look roughly like b) 
\end{slide}



\begin{slide}
\begin{center}
{\bf US Distributional National Accounts}
\end{center}
Piketty-Saez-Zucman (2018) distribute both pre-tax and post-tax US national income across adult individuals 

Pre-tax income is income before taxes and transfers

Post-tax income is income net of all taxes and adding all transfers and public good spending

Both concepts add up to national income, consistent with national accounts aggregates, and provide a comprehensive view of the mechanical impact
of government redistribution

\end{slide}

%\begin{slide}
%\includepdf[pages={28}, scale=.9]{introPE_attach.pdf}
%\end{slide}

\begin{slide}
\includepdf[pages={26}, scale=.9]{introPE_attach.pdf}
\end{slide}

\begin{slide}
\includepdf[pages={38}, scale=.9]{introPE_attach.pdf}
\end{slide}

\begin{slide}
\begin{center}
{\bf US tax/transfer System: Progressivity and Evolution}
\end{center}
{\bf 0) US Tax/Transfer system is progressive overall:} pre-tax national income is less
equally distributed than post-tax/post-transfer national income 

{\bf 1) Medium Term Changes:} Federal Tax Progressivity has declined
since 1950 (Saez and Zucman 2019) but govt redistribution through transfers has increased (Medicaid, Social Security retirement, DI, UI
various income support programs)

{\bf 2) Long Term Changes:} Before 1913, US taxes were primarily
tariffs, excises, and real estate property taxes [slightly
regressive], minimal social state (and hence small govt)

http://www.treasury.gov/education/fact-sheets/taxes/ustax.shtml
\end{slide}



\begin{slide}
\includepdf[pages={33-35}]{introPE_attach.pdf}
\end{slide}




\begin{slide}
\begin{center}
{\bf Federal US Tax System (2/3 of total taxes)}
\end{center}
1) Individual income tax (on both labor+capital income)
[progressive](40\% of fed tax revenue)

2) Payroll taxes (on labor income) financing social security programs [about
neutral] (40\% of revenue)

3) Corporate income tax (on capital income) [progressive if incidence on
capital income] (15\% of revenue)

4) Estate taxes (on capital income) [very progressive] (1\% of
revenue)

5) Minor excise taxes (on consumption) [regressive] (3\%
of revenue)

Fed agencies (CBO, Treasury, Joint Committee on Taxation) and think-tanks (Tax Policy Center)
provide distributional Fed tax tables

\end{slide}


\begin{slide}
\begin{center}
{\bf State+Local Tax System (1/3 of total taxes)}
\end{center}
Decentralized governments can experiment, be tailored to local views, create tax competition and make redistribution harder (famous Tiebout 1956 model)
hence favored by conservatives

1) Individual + Corporate income taxes
[progressive] (1/3 of state+local tax revenue)

2) Sales taxes + Excise taxes (tax on consumption) [regressive] (1/3 of
revenue)

3) Real estate property taxes (on capital income)
[slightly progressive] (1/3 of revenue)

See ITEP (2018) ``Who Pays'' for systematic state level distributional tax tables

US Census provides Census of Government data

%http://www.census.gov/govs/www/qtax.html

\end{slide}


\begin{slide}
\begin{center}
{\bf Government Redistribution in Practice}
\end{center}
\textbf{1) Tax system:} Taxes can be more or less progressive (right vs. left debate).
Most OECD countries today have fairly
flat tax systems. Taxes used to be very progressive in US and UK.

\textbf{2) Social state:} (size of social state also right vs. left debate)

a) Publicly funded education: everybody gets access
to quality education $\Rightarrow$ Redistributive and gives opportunity

b) Universal health care (outside US): everybody gets access
to quality health care $\Rightarrow$ Redistributive by income and health 

c) Retirement benefits: old get support $\Rightarrow$ redistributive in cross-section but not necessarily
on life-time basis

d) Income support: direct redistribution but tends to be targeted to specific groups (children, unemployed, disabled, elderly)
or in-kind (housing, nutrition, training) 


\end{slide}


%\begin{slide}
%\begin{center}
%{\bf Why has govt grown so much over 20th century?}
%\end{center}
%
%1) {\bf Demand Side Argument: Wagner law} Govt provided goods
%(education, health, social insurance) are luxury goods
%
%2) {\bf Supply Side Argument: Tax Enforcement Ability} Ability of govt to tax
%increases dramatically over the course of economic development
%[easy to tax large companies which need careful records for their
%operations]
%
%
%\end{slide}

\begin{slide}
\begin{center}
{\bf REFERENCES CITED}
\end{center}
{\small

Alvaredo, F., Atkinson, A., T. Piketty and E. Saez ``The Top 1 Percent in International and Historical Perspective.''  \emph{Journal of Economic Perspectives} 27(3), 2013, 3-20. \href{http://eml.berkeley.edu/~saez/alvaredo-atkinson-piketty-saezJEP13top1percent.pdf} {(web)}

Alvaredo, F., Atkinson, A., T. Piketty, E. Saez, and G. Zucman \emph{World Inequality Database},
\href{http://www.wid.world/} {(web)}

Alvaredo, F., Atkinson, A., T. Piketty, E. Saez, and G. Zucman. 2018 \emph{World Inequality Report},
\href{https://wir2018.wid.world/} {(web)}

Atkinson, A., T. Piketty and E. Saez ``Top Incomes in the Long Run of History'', Journal of Economic Literature
49(1), 2011, 30--71. \href{http://elsa.berkeley.edu/~saez/atkinson-piketty-saezJEL10.pdf} {(web)}

Chetty, Raj, Nathan Hendren, Patrick Kline, and Emmanuel Saez, ``Where is the Land of Opportunity? The Geography of Intergenerational Mobility in the United States,'' \emph{Quarterly Journal of Economics}, 129(4), 2014, 1553-1623.
\href{http://eml.berkeley.edu/~saez/chetty-friedman-kline-saezQJE14mobility.pdf}{(web)}

ITEP (Institute on Taxation and Economic Policy). 2018. ``Who Pays: A Distributional Analysis of the Tax Systems in All 50 States'', 6th edition. 
\href{https://itep.org/whopays/}{(web)}

%Kleven, Henrik, Claus Kreiner, and Emmanuel Saez ``Why Can Modern Governments Tax So Much? An Agency Model of Firms as Fiscal Intermediaries,'' NBER Working Paper No. 15218, August 2009.
%\href{http://www.nber.org/papers/w15218} {(web)}

Kopczuk, Wojciech, Emmanuel Saez, and Jae Song ``Earnings Inequality and Mobility in the United States: Evidence from Social Security Data since 1937,'' Quarterly Journal of Economics 125(1), 2010, 91-128. \href{http://www.econ.berkeley.edu/~saez/kopczuk-saez-songQJE10mobility.pdf} {(web)}

Piketty, Thomas, \emph{Capital in the 21st Century},  Cambridge: Harvard University Press, 2014,   
\href{http://piketty.pse.ens.fr/en/capital21c2}{(web)}

Piketty, Thomas, \emph{Capital and Ideology},  Cambridge: Harvard University Press, 2020,  Chapter 10,
\href{http://piketty.pse.ens.fr/en/ideology}{(web)}

Piketty, Thomas and Emmanuel Saez ``Income Inequality in the United States, 1913-1998'', Quarterly Journal of Economics, 118(1), 2003, 1-39. \href{http://links.jstor.org/stable/pdfplus/25053897.pdf} {(web)} 

Piketty, Thomas and Emmanuel Saez ``How Progressive is the U.S. Federal Tax System? A Historical and International Perspective,'' Journal of Economic Perspectives, 21(1), Winter 2007, 3-24.
 \href{http://www.econ.berkeley.edu/~saez/piketty-saezJEP07taxprog.pdf} {(web)} 
 
\textbf{Piketty, Thomas, Emmanuel Saez, and Gabriel Zucman,  ``Distributional National Accounts:
Methods and Estimates for the United States'', Quarterly Journal of Economics, 133(2), 553-609, 2018
\href{https://eml.berkeley.edu/~saez/PSZ2018QJE.pdf} {(web)} }

Piketty, Thomas and Gabriel Zucman,  ``Capital is Back: Wealth-Income Ratios in Rich Countries, 1700-2010'',  \emph{Quarterly Journal of Economics} 129(3), 2014, 1255-1310
\href{http://www.econ.berkeley.edu/~saez/PikettyZucman2014QJE.pdf} {(web)}

\textbf{Saez, Emmanuel  ``Public Economics and Inequality: Uncovering Our Social Nature'', AEA Papers and Proceedings, 121, 2021
\href{https://eml.berkeley.edu/~saez/saez-AEAlecture.pdf} {(web)} }

Saez, Emmanuel and Gabriel Zucman, ``Wealth Inequality in the United States since 1913: Evidence from Capitalized Income Tax Data'', \emph{Quarterly Journal of Economics}  131(2), 2016, 519-578
\href{http://eml.berkeley.edu/~saez/SaezZucman2016QJE.pdf}{(web)}

\textbf{Saez, Emmanuel and Gabriel Zucman. ``The Rise of Income and Wealth Inequality in America: Evidence from Distributional Macroeconomic Accounts,'' Journal of Economic Perspectives 34(4), Fall 2020, 3-26.
\href{https://eml.berkeley.edu/~saez/SaezZucman2020JEP.pdf}{(web)} }

\textbf{Saez, Emmanuel and Gabriel Zucman. The Triumph of Injustice: How the Rich Dodge Taxes and How to Make them Pay, New York: W.W. Norton, 2019. 
\href{https://eml.berkeley.edu/~saez/SZ2019Slides.pdf} {(web)} }

Tiebout,  Charles M.  ``A Pure Theory of Local Expenditures''
Journal of Political Economy, 64(5), 1956, 416-424
\href{https://www.jstor.org/stable/pdf/1826343.pdf}{(web)}

}


\end{slide}

\begin{slide}
\begin{center}
{\bf GENERAL BOOK REFERENCES}
\end{center}
{\small

\textbf{Graduate Level}

Atkinson, A.B. and J. Stiglitz, Lectures on Public Economics, New York: McGraw Hill, 1980.

Auerbach, A. and M. Feldstein, eds., Handbook of Public Economics, 4 Volumes, Amsterdam: North Holland, 1985, 1987, 2002, and 2002.
\href{http://www.sciencedirect.com/science/handbooks/15734420/} {(web)}

Auerbach, A., Chetty, R., M. Feldstein, and E. Saez, eds., Handbook of Public Economics, Volume 5,
Amsterdam: North Holland, 2013
\href{http://www.sciencedirect.com/science/handbooks/15734420/} {(web)}

Kaplow, L. The Theory of Taxation and Public Economics.  Princeton University Press, 2008.

Mirrlees, J. Reforming the Tax System for the 21st Century The Mirrlees Review, Oxford University Press, (2 volumes) 2009 and 2010.
\href{https://www.ifs.org.uk/publications/mirrleesreview} {(web)}

Piketty, Thomas, \emph{Capital in the 21st Century},  Cambridge: Harvard University Press, 2014,   
\href{http://piketty.pse.ens.fr/en/capital21c2}{(web)}

Piketty, Thomas, \emph{Capital and Ideology},  Cambridge: Harvard University Press, 2020,   
\href{http://piketty.pse.ens.fr/en/ideology}{(web)}

Saez, Emmanuel and Gabriel Zucman. The Triumph of Injustice: How the Rich Dodge Taxes and How to Make them Pay, New York: W.W. Norton, 2019. 
\href{http://www.taxjusticenow.org} {(web)}

Salani\'e, B. The Economics of Taxation, Cambridge: MIT Press, 2nd Edition 2010.

Slemrod, Joel and Christian Gillitzer. Tax Systems, Cambridge: MIT Press, 2014.



\pagebreak

\textbf{Under-Graduate Level}

Gruber, J. Public Finance and Public Policies, 6th edition, Worth Publishers, 2019.

Rosen, H. and T. Gayer Public Finance, 10th edition, McGraw Hill, 2014.

Stiglitz, J. and J. Rosengard. Economics of the Public Sector,  4th edition, Norton, 2015.

Slemrod, J. and J. Bakija. Taxing Ourselves: A Citizen's Guide to the Debate over Taxes. 5th edition, MIT Press, 2017.

}

\end{slide}

\begin{slide}
\begin{center}
{\bf REFERENCES ON EMPIRICAL METHODS:}
\end{center}
{\small


Angrist, J. and A. Krueger, ``Instrumental Variables and the Search for Identification: From Supply and Demand to Natural Experiments,'' Journal of Economic Perspectives, 15 (4), 2001, 69-87 \href{http://www.jstor.org/stable/pdfplus/2696517.pdf} {(web)}

Angrist, J. and Steve Pischke. Mostly Harmless Econometrics: An Empiricist's Companion, Princeton University Press, 2009. \href{http://www.mostlyharmlesseconometrics.com/} {(web)}

Bertrand, M. E. Duflo et S. Mullainhatan, ``How Much Should we Trust Differences-in-Differences Estimates?,'' Quarterly Journal of Economics, Vol. 119, No. 1, 2004, pp. 249-275. \href{http://www.jstor.org/stable/pdfplus/25098683.pdf} {(web)}

Imbens, Guido and Jeffrey Wooldridge (2007) What's New in Econometrics? NBER SUMMER INSTITUTE MINI COURSE 2007. \href{http://www.nber.org/minicourse3.html} {(web)}

Meyer, B. ``Natural and Quasi-Experiments in Economics,'' Journal of Business and Economic Statistics, 13(2), April 1995, 151-161. \href{http://www.jstor.org/stable/pdfplus/1392369.pdf} {(web)}

}
\end{slide}



\end{document}
