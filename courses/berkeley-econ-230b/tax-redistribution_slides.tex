\documentclass[landscape]{slides}

\usepackage[landscape]{geometry}

\usepackage{pdfpages}
\usepackage{amsmath}
\usepackage{hyperref}

\def\mathbi#1{\textbf{\em #1}}

\topmargin=-1.8cm \textheight=17cm \oddsidemargin=0cm
\evensidemargin=0cm \textwidth=22cm

\author{Emmanuel Saez}
\date{}

\title{Graduate Public Economics \\
Optimal Labor Income Taxes/Transfers} \onlyslides{1-300}

\newenvironment{outline}{\renewcommand{\itemsep}{}}

\begin{document}

\begin{slide}
\maketitle
\end{slide}

\begin{slide}
\begin{center}
{\bf TAXATION AND REDISTRIBUTION}
\end{center}

{\bf Key question:} By how much should government reduce inequality?

1) Governments use {\bf taxes} to raise revenue

2) This revenue funds \textbf{public goods} and \textbf{social state}. Social state has 2 components:

\small

a) Universal programs: Education, Health Care
(only 65+ in the US), Retirement and Disability

b) Means-tested programs: In-kind (e.g., public housing, nutrition,
Medicaid in the US) and cash (direct welfare and refundable tax credits)
\normalsize

Means-tested transfers relatively small relative to universal transfers
and generally in-kind and targeted to children

%Modern governments raise large fraction of  in taxes (30-45\%)
%and spend significant fraction of GDP on transfers 

This lecture follows Piketty and Saez '13 \textbf{handbook chapter}

\end{slide}

\begin{slide}
\includepdf[pages={75}, scale=.95]{tax-redistribution_attach.pdf}
\end{slide}


\begin{slide}
\begin{center}
{\bf FACTS ON US TAXES AND TRANSFERS}
\end{center}
{\bf References:} Comprehensive description in Gruber undergrad
textbook (taxes/transfers) and Slemrod-Bakija (taxes)

http://www.taxpolicycenter.org/taxfacts/

{\bf A) Taxes:} (1) individual income tax (fed+state), (2) payroll
taxes on earnings (fed, funds Social Security+Medicare), (3)
corporate income tax (fed+state), (4) sales taxes (state)+excise
taxes (state+fed), (5) property taxes (state)

{\bf B) Means-tested Transfers:} (1) refundable tax credits (fed),
(2) in-kind transfers (fed+state): Medicaid, public housing, nutrition
(SNAP), education (3) cash welfare: TANF for single parents
(fed+state), SSI for old/disabled (fed)

\end{slide}


\begin{slide}
\begin{center}
{\bf FEDERAL US INCOME TAX}
\end{center}

US income tax assessed on {\bf annual} {\bf family} income (not
individual) [most other OECD countries have shifted to individual
assessment]

Sum all cash income sources from family members (both from labor
and capital income sources) = called {\bf Adjusted Gross Income
(AGI)}

Main exclusions: fringe benefits (health insurance, pension
contributions and returns), imputed rent of homeowners, undistributed corporate profits, unrealized capital gains, interest from state+local bonds 

$\Rightarrow$ AGI base is only 70\% of national income

\end{slide}


\begin{slide}
\begin{center}
{\bf FEDERAL US INCOME TAX}
\end{center}

Taxable income = AGI - deduction

%personal exemptions = \$4K * \# family members (in 2016)

deduction is max of standard deduction or itemized deductions

Standard deduction is a fixed amount 
(\$12K for singles, \$24K for married couple)

Itemized deductions: (a) state and local taxes paid (only up to \$10K since 2018), (b) mortgage interest payments (capped), (c) charitable giving,  various small other items

[about 10\% of AGI lost through itemized deductions, called tax
expenditures]

In 2018+, only 10\% of tax filers itemize (30\% before 2018)

\end{slide}


\begin{slide}
\begin{center}
{\bf FEDERAL US INCOME TAX: TAX BRACKETS}
\end{center}

Tax $T(z)$ is piecewise linear and continuous function of taxable
income $z$ with constant marginal tax rates (MTR) $T'(z)$ by
brackets

In 2018+, 6 brackets with MTR 10\%,12\%,22\%,24\%,32\%,35\%, 37\% (top
bracket for $z$ above \$600K), indexed on price inflation

Lower preferential rates (up to a max of 20\%) apply to dividends
(since 2003), realized capital gains [in part to offset double
taxation of corporate profits].

20\% of business profits are exempt since 2018

Tax rates change frequently over time. Top MTRs have declined
drastically since 1960s (as in many OECD countries)
\end{slide}

\begin{slide}
\includepdf[pages={44, 45, 70}]{tax-redistribution_attach.pdf}
\end{slide}


\begin{slide}
\begin{center}
{\bf FEDERAL US INCOME TAX: AMT AND CREDITS}
\end{center}

{\bf Alternative minimum tax (AMT)} is a parallel tax system
(quasi flat tax at 28\%) with fewer deductions: actual tax =$\max
(T(z),AMT)$ (hits $<1\%$ of taxpayers in 2018+)

{\bf Tax credits:} Additional reduction in taxes

(1) {\bf Non refundable} (cannot reduce taxes below zero): foreign
tax credit, child care expenses, education credits, energy credits

(2) {\bf Refundable} (can reduce taxes below zero, i.e., be net
transfers): EITC (earned income tax credit, up to \$3.5K, \$5.7K, \$6.5K for working
families with 1, 2, 3+ kids), Child Tax Credit (\$2K per kid, partly
refundable)

Refundable credits have become the largest means-tested cash transfer
in the US
\end{slide}

\begin{slide}
\begin{center}
{\bf FEDERAL US INCOME TAX: TAX FILING}
\end{center}
Taxes on year $t$ earnings are withheld on paychecks during year
$t$ (pay-as-you-earn)

Income tax return filed in late January-April 15th, year $t+1$ [filers use
either software or tax preparers, big private industry]

Most tax filers get a tax refund as withholdings $>$ taxes
owed

Payers (employers, banks, etc.) send income information to IRS (US tax administration)
(3rd party reporting)

Third party reporting + withholding at source is key for successful
enforcement

\end{slide}

\begin{slide}
\begin{center}
{\bf MAIN MEANS-TESTED TRANSFER PROGRAMS}
\end{center}
1) {\bf Traditional transfers:} managed by welfare agencies, paid
on monthly basis, high stigma and take-up costs $\Rightarrow$ low
take-up rates

Main programs: Medicaid (health insurance for low incomes), SNAP
(former food stamps), public housing, TANF (traditional welfare), SSI
(aged+disabled)

2) {\bf Refundable income tax credits:} managed by tax
administration, paid as an annual lumpsum in year $t+1$, low
stigma and take-up cost $\Rightarrow$  high take-up rates

Main programs: EITC and Child Tax Credit [large expansion since
the 1990s] for low income working families with children
\end{slide}

\begin{slide}
\includepdf[pages={42}, scale=1.1]{tax-redistribution_attach.pdf}
\end{slide}

\begin{slide}
\begin{center}
{\bf BOTTOM LINE ON ACTUAL TAXES/TRANSFERS}
\end{center}
1) Based on current income, family situation, and disability
(retirement) status $\Rightarrow$ Strong link with {\bf current
ability to pay}

2) Some allowances made to reward / encourage certain behaviors:
charitable giving, home ownership, savings, energy conservation,
and more recently work (refundable tax credits such as EITC)

3) Provisions pile up overtime making tax/transfer system more and
more complex until significant simplifying reform happens (such as
US Tax Reform Act of 1986, or TCJA 2018)

\end{slide}



\begin{slide}
\begin{center}
{\bf KEY CONCEPTS FOR TAXES/TRANSFERS}
\end{center}
1) Transfer benefit with zero earnings $-T(0)$ [sometimes called
demogrant or lumpsum grant]

2) Marginal tax rate (or phasing-out rate) $T'(z)$: individual
keeps $1-T'(z)$ for an additional \$1 of earnings (intensive labor
supply response)

3) Participation tax rate $\tau_p=[T(z)-T(0)]/z$: individual keeps
fraction $1-\tau_p$ of earnings when moving from zero earnings to
earnings $z$ (extensive labor supply response): \[z-T(z)=-T(0)+z - [T(z)-T(0)] = -T(0) + z \cdot
(1-\tau_p)\] 

4) Break-even earnings point $z^*$: point at which $T(z^*)=0$
\end{slide}



\begin{slide}
\includepdf[pages={46,47}]{tax-redistribution_attach.pdf}
\end{slide}

\begin{slide}
\includepdf[pages={3,40}]{tax-redistribution_attach.pdf}
\end{slide}



\begin{slide}
\begin{center}
{\bf OPTIMAL TAXATION: SIMPLE MODEL WITH NO BEHAVIORAL RESPONSES}
\end{center}

Utility $u(c)$ strictly increasing and concave

Same for everybody where $c$ is after tax income.

Income is $z$ and is fixed for each individual, $c=z-T(z)$ where
$T(z)$ is tax on $z$. $z$ has density distribution $h(z)$

Government maximizes {\bf Utilitarian} objective: 
\[ \int_0^{\infty}
u(z-T(z))h(z)dz \]
subject to {\bf budget constraint} $\int T(z)h(z)dz \geq E$
(multiplier $\lambda$)

\end{slide}

\begin{slide}
\begin{center}
{\bf SIMPLE MODEL WITH NO BEHAVIORAL RESPONSES}
\end{center}

Form lagrangian: $L=[u(z-T(z))+\lambda \cdot T(z)] \cdot h(z)$

First order condition (FOC) in $T(z)$: 
\[ 0= \frac{\partial L}{\partial T(z) } =
[-u'(z-T(z))+\lambda] \cdot h(z) \Rightarrow u'(z-T(z))=\lambda \]
$\Rightarrow$ $z-T(z)=$ constant for all $z$.

$\Rightarrow$ $c=\bar{z}-E$ where $\bar{z}=\int z h(z)dz$ average
income.

100\% marginal tax rate. Perfect equalization of after-tax income.

Utilitarianism with decreasing marginal utility leads to perfect
egalitarianism [Edgeworth, 1897]
\end{slide}

\begin{slide}
\includepdf[pages={55}]{tax-redistribution_attach.pdf}
\end{slide}

\begin{slide}
\includepdf[pages={76}]{tax-redistribution_attach.pdf}
\end{slide}

\begin{slide}
\begin{center}
{\bf ISSUES WITH SIMPLE MODEL}
\end{center}

1) {\bf No behavioral responses:} Obvious missing piece: 100\%
redistribution would destroy incentives to work and thus the
assumption that $z$ is exogenous is unrealistic

$\Rightarrow$ Optimal income tax theory incorporates behavioral
responses $u(\underset{+}{c},\underset{-}{z})$ (Mirrlees REStud '71):  \textbf{equity-efficiency trade-off}

2) {\bf Issue with Utilitarianism:} Even absent behavioral
responses, many people would object to 100\% redistribution
[perceived as confiscatory]

$\Rightarrow$ Citizens' views on fairness impose {\bf bounds} on
redistribution 

We will discuss at the end of the lecture alternatives to utilitarianism
(and be agnostic for now on how society's preferences for redistribution are shaped)
\end{slide}

\begin{slide}
\includepdf[pages={77}]{tax-redistribution_attach.pdf}
\end{slide}


%\begin{slide}
%\begin{center}
%{\bf 2ND WELFARE THEOREM FALLACY}
%\end{center}
%Suppose individuals differ in their ability to earn
%
%{\bf 2nd Welfare Theorem:} Any Pareto Efficient outcome can be
%reached by (1) Suitable redistribution of initial endowments
%[individualized {\bf lump-sum} taxes based on ability and not
%behavior], (2) Then letting markets work freely
%
%$\Rightarrow$ No conflict between efficiency and equity
%
%In reality, redistribution of initial endowments is not feasible
%(information pb) and govt needs to use {\bf distortionary} taxes
%and transfers based on income and consumption to redistribute
%
%$\Rightarrow$ Real conflict between efficiency and equity
%
%\end{slide}


%\begin{slide}
%\begin{center}
%{\bf EQUITY-EFFICIENCY TRADE-OFF}
%\end{center}
%
%Taxes can be used to raise revenue for transfer programs which can
%reduce inequality in disposable income $\Rightarrow$ Desirable if
%society feels that inequality is too large
%
%Taxes (and transfers) reduce incentives to work $\Rightarrow$ High
%tax rates create economic inefficiency if individual respond to
%taxes
%
%Size of behavioral response limits the ability of govt to
%redistribute with taxes/transfers
%
%$\Rightarrow$ Generates an \textbf{equity-efficiency trade-off}
%
%Empirical tax literature estimates the size of behavioral
%responses to taxation
%\end{slide}


\begin{slide}
\begin{center}
{\bf MIRRLEES OPTIMAL INCOME TAX MODEL (skip)}
\end{center}

{\bf 1) Standard labor supply model:} Individual maximizes
$u(c,l)$ subject to $c=wl-T(wl)$ where $c$ consumption, $l$ labor
supply, $w$ wage rate, $T(.)$ nonlinear income tax $\Rightarrow$
taxes affect labor supply

{\bf 2) Individuals differ in ability $w$:} $w$ distributed with
density $f(w)$.

{\bf 3) Govt social welfare maximization:} Govt maximizes
\[ SWF=\int G(u(c,l))f(w)dw\] ($G(.)$ $\uparrow$ concave) subject to

(a) budget constraint $\int T(wl) f(w)dw \geq E$ (multiplier
$\lambda$)

(b) individuals' labor supply $l$ depends on $T(.)$

\end{slide}

\begin{slide}
\begin{center}
{\bf MIRRLEES MODEL RESULTS (skip)}
\end{center}
Optimal income tax trades-off redistribution and efficiency (as
tax based on $w$ only not feasible) 

$\Rightarrow$ $T(.)<0$ at
bottom (transfer) and $T(.)>0$ further up (tax) [full integration
of taxes/transfers]

Mirrlees formulas complex, only a couple fairly general results:

1) $0 \leq T'(.) \leq 1$, $T'(.)\geq 0$ is non-trivial (rules out
EITC) [Seade '77]

2) Marginal tax rate $T'(.)$ should be zero at the top (if skill
distribution bounded) [Sadka '76-Seade '77]

3) If everybody works and lowest $wl>0$, $T'(.)=0$ at bottom


\end{slide}

\begin{slide}
\begin{center}
{\bf BEYOND MIRRLEES (skip)}
\end{center}
Mirrlees '71 had a huge impact on information economics:
models with asymmetric information in contract theory

Discrete 2-type version of Mirrlees model developed by Stiglitz
JpubE '82 with individual FOC replaced by Incentive Compatibility
constraint [high type should not mimick low type]

Till late 1990s, Mirrlees results not closely connected to
empirical tax studies and  little impact on tax policy
recommendations

Since late 1990s, Diamond AER'98, Piketty '97, Saez ReStud '01
have connected Mirrlees model to practical tax policy / empirical
tax studies 

[new approach summarized in Diamond-Saez JEP'11 and
Piketty-Saez Handbook'13]
\end{slide}

\begin{slide}
\begin{center}
{\bf INTENSIVE LABOR SUPPLY CONCEPTS}
\end{center}
 \[ \max_{c,z} u(\underset{+}{c},\underset{-}{z}) \text{ subject to  } c=z \cdot (1-\tau)+R\] 
$R$ is virtual income and
$\tau$ marginal tax rate.   FOC in $c,z$ $\Rightarrow$
$(1-\tau) u_c+u_z=0$ $\Rightarrow$ Marshallian labor supply
$z=z(1-\tau,R)$
\[ \mathrm{Uncompensated\:\:elasticity} \quad \varepsilon^u=\frac{(1-\tau)}{z}\frac{\partial z}
{\partial (1-\tau)} \]
\[ \mathrm{Income\:\:effects} \quad \eta=(1-\tau)\frac{ \partial z}{\partial R} \leq 0 \]
Substitution effects: Hicksian labor supply: $z^c(1-\tau,u)$
minimizes cost needed to reach $u$ given slope $1-\tau$ $\Rightarrow$
\[ \mathrm{Compensated\:\:elasticity} \quad \varepsilon^c=\frac{(1-\tau)}{z}\frac{\partial z^c}
{\partial (1-\tau)} > 0 \]
\[\mathrm{Slutsky\:\:equation} \quad \frac{\partial z}{\partial (1-\tau)} = \frac{\partial z^c}{\partial (1-\tau)}
+ z \frac{ \partial z}{\partial R} \Rightarrow \varepsilon^u = \varepsilon^c + \eta\]
\end{slide}

\begin{slide}
\includepdf[pages={56-60}]{tax-redistribution_attach.pdf}
\end{slide}

\begin{slide}
\begin{center}
{\bf Labor Supply Effects of Taxes and Transfers}
\end{center}
Taxes and transfers change the slope $1-T'(z)$ of the budget constraint and net disposable income $z-T(z)$
(relative to the no tax situation where $c=z$)

Positive MTR $T'(z)>0$ reduces labor supply through substitution effects

Net transfer ($T(z)<0$) reduces labor supply through income effects

Net tax ($T(z)>0$) increases labor supply through income effects

\end{slide}

\begin{slide}
\includepdf[pages={61}]{tax-redistribution_attach.pdf}
\end{slide}



\begin{slide}
\begin{center}
{\bf WELFARE EFFECT OF SMALL TAX REFORM}
\end{center}
Indirect utility: $V(1-\tau,R)=\max_z u((1-\tau)z+R,z)$ where $R$
is virtual income intercept

Small tax reform: $d\tau$ and $dR$:

$dV=u_c \cdot [-z d\tau + dR ] + dz \cdot [(1-\tau)u_c+u_z] = u_c \cdot [-z d\tau + dR ]$

\textbf{Envelope theorem:} no effect of $dz$ on $V$ because $z$ is already chosen
to maximize utility ($(1-\tau)u_c+u_z=0$)

$[-z d\tau + dR]$ is the \textbf{mechanical} change in disposable income due to tax reform


Welfare impact of a small tax reform is given by $u_c$ times the money metric
mechanical change in tax

Remains true for any nonlinear tax system $T(z)$ (just need to look at $dT(z)$, mechanical change
in taxes)
\end{slide}

\begin{slide}
\begin{center}
{\bf SOCIAL WELFARE FUNCTIONS (SWF)}
\end{center}
Welfarism = social welfare based solely on individual utilities

\small
Any other social objective will lead to Pareto dominated outcomes in some
circumstances (Kaplow and Shavell JPE'01)
\normalsize

Most widely used welfarist SWF:

1) Utilitarian: $SWF = \int_i u^i$

2) Rawlsian (also called Maxi-Min): $SWF=\min_i u^i$

3) $SWF= \int_i G(u^i)$ with $G(.) \uparrow$  and concave, e.g., $G(u)=u^{1-\gamma}/(1-\gamma)$ 
(Utilitarian is $\gamma=0$, Rawlsian is $\gamma=\infty$)

4) General Pareto weights: $SWF=\int_i \mu_i \cdot u^i$ with $\mu_i \geq 0$ exogenously given


\end{slide}

\begin{slide}
\begin{center}
{\bf SOCIAL MARGINAL WELFARE WEIGHTS}
\end{center}

Key sufficient statistics in optimal tax formulas are \textbf{Social Marginal Welfare Weights} for each
individual: 

Social Marginal Welfare Weight on individual $i$ is $g_i = \mu_i u^i_c / \lambda$ ($\lambda$ multiplier of govt budget constraint) measures \$ value for
govt of giving \$1 extra to person $i$ 

No income effects $\Rightarrow \int_i g_i =1$: giving \$1 to all costs \$1 (pop. has measure 1)
and increases SWF (in \$ terms) by $\int_i g_i$

$g_i$ typically depend on tax system (endogenous variable)

Utilitarian case: $g_i$ decreases with $z_i$ due to decreasing marginal utility of consumption 

Rawlsian case: $g_i$ concentrated on most disadvantaged (typically those with $z_i=0$)

\end{slide}

\begin{slide}
\includepdf[pages={78-79}]{tax-redistribution_attach.pdf}
\end{slide}


\begin{slide}
\begin{center}
{\bf OPTIMAL LINEAR TAX RATE: LAFFER CURVE (skip)}
\end{center}
$c=(1-\tau)\cdot z + R$ with $\tau$ linear tax rate and $R$ demogrant funded
by taxes $\tau Z$ with $Z$ aggregate earnings

Population of size one (continuum) with heterogeneous preferences $u^i(c,z)$
[differences in earnings ability are built in utility function]

Individual $i$ chooses $z$ to maximize $u^i((1-\tau)\cdot z + R,z)$
labor supply choices $z^i(1-\tau,R)$ aggregate to economy wide earnings
$Z(1-\tau) = \int_i z^i $

Tax Revenue $R(\tau)=\tau \cdot Z(1-\tau)$ is inversely U-shaped
with $\tau$: $R(\tau=0)=0$ (no taxes) and $R(\tau=1)=0$ (nobody
works): called the Laffer Curve

\end{slide}

\begin{slide}
\begin{center}
{\bf OPTIMAL LINEAR TAX RATE: LAFFER RATE (skip)}
\end{center}
Top of the Laffer Curve corresponds to tax rate $\tau^*$
maximizing tax revenue: inefficient to have $\tau>\tau^*$
\[R'(\tau^*)=Z - \tau^* d Z/ d(1-\tau)=0 \Rightarrow \]
Revenue maximizing tax rate (Laffer rate): 
\[ \tau^*=\frac{1}{1+e} \quad \mathrm{ with } \quad e=\frac{1-\tau}{Z} \frac{d Z}{ d(1-\tau)} \] 
elasticity of earnings with respect to the net-of-tax rate

$\tau>\tau^*$ is (second-best) Pareto inefficient: cutting $\tau$ increases individuals welfare and
government revenue (and hence $R$)

$\tau=\tau^*$ is the optimum in Rawlsian case (if some people have zero earnings)
\end{slide}

\begin{slide}
\includepdf[pages={54}]{tax-redistribution_attach.pdf}
\end{slide}


\begin{slide}
\begin{center}
{\bf OPTIMAL LINEAR TAX RATE: FORMULA (skip)}
\end{center}
Government chooses $\tau$ to maximize
\[ \int_i G[u^i((1-\tau)z^i+\tau Z(1-\tau),z^i)] \]
Govt FOC (using the envelope theorem as $z^i$ maximizes $u^i$):
\[ 0 = \int_i G'(u^i) u^i_c \cdot
\left [-z^i + Z - \tau \frac{dZ}{d(1-\tau)} \right ] ,\]
\[ 0 = \int_i G'(u^i) u^i_c \cdot
\left [(Z-z^i) - \frac{\tau}{1-\tau} e Z \right ],\]
First term $(Z-z^i)$ is mechanical redistributive effect of $d\tau$, second term is efficiency cost due to behavioral 
response of $Z$

$\Rightarrow$ we obtain the following optimal linear income tax formula
\[
\tau= \frac{1-\bar{g}}{1-\bar{g} + e} \quad \mathrm{with} \quad
\bar{g}= \frac{ \int g_i \cdot z_i  }{Z \cdot \int g_i}, \quad g_i =  G'(u^i) u^i_c
\]

\end{slide}

\begin{slide}
\begin{center}
{\bf OPTIMAL LINEAR TAX RATE: FORMULA (skip)}
\end{center}
\[
\tau= \frac{1-\bar{g}}{1-\bar{g} + e} \quad \mathrm{with} \quad
\bar{g}= \frac{ \int g_i \cdot z_i  }{Z \cdot \int g_i}, \quad g_i =  G'(u^i) u^i_c
\]
$0\leq \bar{g} <1$ if $g_i$ is decreasing with $z_i$ (social marginal welfare weights
fall with $z_i$). 

$\bar{g}$ low when (a) inequality is high, (b) $g^i$ $\downarrow$ sharply with $z^i$

Formula captures the equity-efficiency trade-off \textbf{robustly} ($\tau \downarrow \bar{g}$, $\tau \downarrow e$)

Rawlsian case: $g_i \equiv 0$ for all $z_i>0$ so $\bar{g}=0$ and $\tau=1/(1+e)$

\end{slide}




\begin{slide}
\begin{center}
{\bf OPTIMAL TOP INCOME TAX RATE (SAEZ '01)}
\end{center}
Consider constant MTR $\tau$ above fixed $z^*$. Goal is to
derive optimal $\tau$

%We assume that social marginal value of consumption of the rich
%is very small	(relative to average) so goal is to maximize tax revenue
%from the rich

Assume w.l.o.g there is a continuum of measure one of individuals above $z^*$

Let
$z(1-\tau)$ be their average income [depends on net-of-tax rate
$1-\tau$], with elasticity $e=[(1-\tau)/z]\cdot
d z/d(1-\tau)$

Note that $e$ is a mix of income and substitution effects (see Saez '01)

%[$a$ is $z^m/(z^m-z^*)>1$ as we will see]

\end{slide}


\begin{slide}
\includepdf[pages={24-25}]{tax-redistribution_attach.pdf}
\end{slide}

\begin{slide}
\begin{center}
{\bf OPTIMAL TOP INCOME TAX RATE}
\end{center}
Consider small $d\tau>0$ reform above $z^*$.

1) {\bf Mechanical increase} in tax revenue: $$dM=
[z-z^*] d\tau$$

2) {\bf Welfare effect:} $$dW=-\bar{g} \cdot dM = -\bar{g} \cdot [z-z^*] d\tau$$ where $\bar{g}$ is the social marginal welfare weight for top earners

%Money-metric utility loss is $dM$ by envelope theorem: govt values
%marginal consumption of rich at $0 \leq \bar{g} <1$: $dW=-\bar{g}
%dM$ [formally $\bar{g}=\int_{z^*}^\infty G'(u) \cdot u_c \cdot
%h(z)dz/ ((1-H(z))\lambda)$]

3) {\bf Behavioral response} reduces tax revenue:
$$dB=  \tau \cdot dz= - \tau \frac{dz}{d(1-\tau)}d\tau = -
\frac{\tau}{1-\tau} \cdot \frac{1-\tau}{z}
\frac{dz}{d(1-\tau)} \cdot z d\tau$$ $$\Rightarrow dB=  - 
 \frac{\tau}{1-\tau} \cdot  e \cdot z d\tau$$ 
 

\end{slide}



\begin{slide}
\begin{center}
{\bf OPTIMAL TOP INCOME TAX RATE}
\end{center}
$$dM+dW+dB=d\tau \left [ (1-\bar{g}) [z-z^*]   - e
\frac{\tau}{1-\tau} z \right ] $$ Optimal $\tau$ such that
$dM+dW+dB=0$ $\Rightarrow$
$$\frac{\tau}{1-\tau}=\frac{(1-\bar{g})[z-z^*]}
{e \cdot z}$$
$$\tau=\frac{1-\bar{g}}{1-\bar{g}+a \cdot e} \quad \mathrm{with} \quad
a=\frac{z}{z-z^*}$$

Optimal $\tau$
$\downarrow$ $\bar{g}$ [redistributive tastes]

Optimal $\tau$ $\downarrow$ with $e$ [efficiency]

Optimal $\tau$ $\downarrow$ $a$ [thinness of top tail]

\end{slide}


\begin{slide}
\begin{center}
{\bf ZERO TOP RATE RESULT (skip)}
\end{center}
Suppose top earner earns $z^T$ 
 
When $z^* \rightarrow z^T$ $\Rightarrow$ $z  \rightarrow  z^T $

\[ dM=d\tau [z-z^*]<< dB= d\tau \cdot  e \cdot
\frac{\tau}{1-\tau} z \quad \mathrm{when} \quad z^* \rightarrow z^T \]

Intuition: extra tax applies only to earnings above $z^*$ but
behavioral response applies to full $z$ $\Rightarrow$

Optimal $\tau$ should be zero when $z^*$ close to $z^T$
(Sadka-Seade zero top rate result) but result applies only to top earner

Top is uncertain: If actual distribution is finite draw from an underlying Pareto
distribution then expected revenue maximizing rate is $1/(1+a\cdot e)$
(Diamond and Saez JEP'11)
\end{slide}


%\begin{slide}
%\begin{center}
%{\bf ZERO TOP RATE RESULT}
%\end{center}
%Suppose top earner earns $z^T$, 
%and second top earner earns $z^S$,
%then $z = z^T$ when $z^*>z^S$ $\Rightarrow$ $z/z^*
%\rightarrow 1$ when $z^* \rightarrow z^T$ $\Rightarrow$
%
%\[ dM=d\tau [z-z^*]<< dB= d\tau \cdot  e \cdot
%\frac{\tau}{1-\tau} z \quad \mathrm{when} \quad z^* \rightarrow z^T \]
%
%Intuition: extra tax applies only to earnings above $z^*$ but
%behavioral response applies to full $z$ $\Rightarrow$
%
%Optimal $\tau$ should be zero when $z^*$ close to $z^T$
%(Sadka-Seade zero top rate result) but result applies only to top earner
%
%Top is uncertain: If actual distribution is finite draw from an underlying Pareto
%distribution then expected revenue maximizing rate is $1/(1+a\cdot e)$
%(Diamond and Saez JEP'11)
%\end{slide}

\begin{slide}
\includepdf[pages={26}]{tax-redistribution_attach.pdf}
\end{slide}


\begin{slide}
\begin{center}
{\bf OPTIMAL TOP INCOME TAX RATE}
\end{center}
Empirically: $a=z/(z-z^*)$ very stable above $z^*=\$400K$

Pareto distribution $1-F(z)=(k/z)^{\alpha}$, $f(z)=\alpha \cdot k^{\alpha}/z^{1+{\alpha}}$,
with ${\alpha}$ Pareto parameter
$$z(z^*) = \frac{ \int_{z^*}^{\infty} s f(s)ds }{\int_{z^*}^{\infty} f(s)ds }
= \frac{ \int_{z^*}^{\infty} s^{-{\alpha}} ds
}{\int_{z^*}^{\infty} s^{-{\alpha}-1} ds } = \frac{{\alpha}}{{\alpha}-1} \cdot
z^*$$  ${\alpha}=z/(z-z^*)=a$ measures {\em thinness} of top tail of the
distribution

Empirically $a \in (1.5,3)$, US has $a=1.5$, Denmark has $a=3$
$$\tau=\frac{1-\bar{g}}{1-\bar{g}+a \cdot e}$$
Only difficult parameter to estimate is $e$
\end{slide}

\begin{slide}
\begin{center}
{\bf TAX REVENUE MAXIMIZING TAX RATE (skip)}
\end{center}
Utilitarian criterion with $u_c  \rightarrow 0$ when $c
\rightarrow \infty$ $\Rightarrow$ $\bar{g} \rightarrow 0$ when
$z^* \rightarrow \infty$

Rawlsian criterion (maximize utility of worst off person) $\Rightarrow$ $\bar{g}=0$ for any
$z^*>\min(z)$

In the end, $\bar{g}$ reflects the value that society puts on
marginal consumption of the rich

$\bar{g}=0$ $\Rightarrow$ Tax Revenue Maximizing Rate $\tau=1/(1+a \cdot
e)$ (upper bound on top tax rate)

Example: $a=2$ and $e=0.25$ $\Rightarrow$
$\tau=2/3=66.7\%$

Laffer linear rate is a special case with $z^*=0$,
$z^m/z^*=\infty=a/(a-1)$ and hence $a=1$,
$\tau=1/(1+e)$
\end{slide}


%\begin{slide}
%\begin{center}
%{\bf EXTENSIONS AND LIMITATIONS}
%\end{center}
%1) Model includes only intensive earnings response. Extensive
%earnings responses [entrepreneurship decisions, migration
%decisions] $\Rightarrow$ Formulas can be modified
%
%2) Model does not include {\bf fiscal externalities}: part of the
%response to $d\tau$ comes from {\bf income shifting} which affects
%other taxes $\Rightarrow$ Formulas can be modified
%
%3) Model does not include {\bf classical externalities}: (a)
%charitable contributions, (b) positive spillovers (trickle down)
%[top earners underpaid], (c) negative spillovers [top earners
%overpaid]
%
%Classical general equilibrium effects on prices are NOT
%externalities and do not affect formulas [Diamond-Mirrlees AER
%'71, Saez JpubE '04]
%\end{slide}

\begin{slide}
\begin{center}
{\bf EXTENSION: MIGRATION EFFECTS (skip)}
\end{center}
Migration issues may be particularly important at the top end (brain
drain). Some theory papers (Mirrlees '82, Lehmann-Simula QJE'14). 
%Little empirical work (on individual side).

Migration depends on average tax rate. Define $P(z-T(z)|z)$
fraction of $z$ earners in the country: Elasticity
$$\eta^m=\frac{z-T(z)}{P} \cdot \frac{\partial P}{\partial (z-T(z))}
$$
Tax revenue maximizing formula (Brewer-Saez-Shepard '10):
$$\tau=\frac{1}{1+a \cdot e+ \bar{\eta}^m}$$
Note: $\bar{\eta}^m$ depends on size of jurisdiction: large for
cities, zero worldwide $\Rightarrow$ (1) Redistribution easier in
large jurisdictions, (2) Tax coordination across countries
increases ability to redistribute (big issue currently in EU)
\end{slide}

\begin{slide}
\begin{center}
{\bf REAL VS. TAX AVOIDANCE RESPONSES (skip)}
\end{center}
Behavioral response to income tax comes not only from reduced
labor supply but also shifts to other forms of income or
activities: (untaxed fringe benefits, deferred compensation, shift to corporate
income tax base, shift toward tax favored capital gains, etc.)

Real responses vs. tax avoidance responses is critical for 2 reasons:

1) Govt can control tax avoidance through other tools: closing loopholes,
broadening the tax base $\Rightarrow$ Elasticity $e$ is endogenous to
tax system design (Slemrod) 

2) Most tax avoidance responses create ``fiscal externalities'' in the
sense that tax revenue increases at other time periods or in other tax bases
(Saez-Slemrod-Giertz JEL' 12)
\end{slide}


%\begin{slide}
%\begin{center}
%{\bf REAL VS. AVOIDANCE RESPONSES THEORY}
%\end{center}
%Fraction $s$ of response $dz$ to $d\tau$ due to avoidance (fraction $1-s$ is real)
%and ``shifted income'' $s \cdot dz$ is taxed at rate $t \leq \tau$
%
%$\Rightarrow$ Tax revenue maximizing rate is (Saez, Slemrod, Giertz '12)
%\[ \tau=\frac{1+a \cdot t \cdot s \cdot e }{1+a \cdot e} \]
%1) $t=0 \Rightarrow \tau=1/(1+a \cdot e)$ (tax avoidance response vs. real
%response response is irrelevant, Feldstein '99)
%
%2) $t>0 \Rightarrow \tau>1/(1+a \cdot e)$ because of ``fiscal externality''
%
%3) \textbf{Fully optimal policy:} $t=\tau$ and $\tau=1/[1+a \cdot (1-s)e]$ with $(1-s)e$ real elasticity (avoidance response $s \cdot e$ irrelevant) $\Rightarrow$ (a) broaden the base/close loopholes, (b) then $\uparrow$ top rates
%
%\end{slide}


\begin{slide}
\begin{center}
{\bf REAL VS. AVOIDANCE RESPONSES (skip)}
\end{center}
{\bf Key policy question:} Is it possible to eliminate avoidance elasticity using base broadening, etc.? or would new avoidance schemes keep popping up?

a) Some forms of tax avoidance are due to \textbf{poorly designed tax codes} (preferential treatment for some income forms, deductions)

b) Some forms of tax avoidance/evasion can only be addressed with \textbf{international cooperation} (off-shore
tax evasion in tax heavens)

c) Some forms of tax avoidance/evasion are due to technological limitations of tax collection
(impossible to tax informal cash businesses)

\end{slide}


%\begin{slide}
%\begin{center}
%{\bf CLASSICAL EXTERNALITIES}
%\end{center}
%1) Classical externalities require additional Pigouvian correction
%on top of the regular optimal income tax (Sandmo '75,
%Cremer-Gahvari-Ladoux JpubE '98). 
%
%If externality cannot be directly targeted, tweak the income tax as follows
%(Piketty-Saez-Stantcheva AEJ'14):
%
%a) If top pay = marginal productivity, then no externalities,
%standard theory.
%
%b) If top pay $<$ marginal productivity (e.g., unions divert
%surplus from top to bottom workers or firm insurance)$\Rightarrow$
%labor supply of top earners has positive externality and optimal
%tax rate should be lower
%
%c) If top pay $>$ marginal productivity (e.g., executives skim
%their companies)$\Rightarrow$ skimming is a negative externality
%for shareholders, tax on top pay may mitigate the externality
%\end{slide}
%
%
%\begin{slide}
%\begin{center}
%{\bf RENT-SEEKING RESPONSES THEORY}
%\end{center}
%In models with frictions or imperfect information, pay $z$ does not always equal
%marginal product $y$ $\Rightarrow$ scope for rent-seeking bargaining $\Rightarrow$ Classical Externality
%
%Suppose fraction $s$ of the response $dz$ to $d\tau$ is due to bargaining (and fraction $1-s$ is real so that $dy=(1-s)dz$)
%
%Tax revenue maximizing rate (Piketty, Saez, Stantcheva '14):
%$$\tau=\frac{1+ a \cdot s \cdot e }{1+a \cdot e}$$
%%$s$ depends both on bargaining responses and whether top earners
%%are overpaid
%{\bf 1) Trickle-up:} If top earners overpaid $y<z$, then $s>0$ and
%$\tau>1/(1+a \cdot e)$
%
%{\bf 2) Trickle-down:} If top earners underpaid, then $s<0$ is
%possible and $\tau<1/(1+a \cdot e)$
%
%\end{slide}


\begin{slide}
\begin{center}
{\bf GENERAL NON-LINEAR INCOME TAX $T(z)$}
\end{center}
(1) Lumpsum grant given to everybody equal to $-T(0)$

(2) Marginal tax rate schedule $T'(z)$ describing how (a) lump-sum
grant is taxed away, (b) how tax liability increases with income

Let $H(z)$ be the income CDF [population normalized to 1] and
$h(z)$ its density [endogenous to $T(.)$]

Let $g(z)$ be the social marginal value of consumption for
taxpayers with income $z$ in terms of public funds.
With no income effects $\Rightarrow$
$\int g(z)h(z)dz=1$

Redistribution valued $\Rightarrow$ $g(z)$ decreases with $z$
%[government
%indifferent between giving $1/g(z_1)$ to person with $z_1$ and
%giving $1/g(z_2)$  to person with  $z_2$]

Let $G(z)$ the {\em average} social marginal value of $c$ for
taxpayers with income above $z$ [$G(z)=\int_z^{\infty}
g(s)h(s)ds/(1-H(z))$]
\end{slide}


\begin{slide}
\includepdf[pages={34}]{tax-redistribution_attach.pdf}
\end{slide}

\begin{slide}
\begin{center}
{\bf GENERAL NON-LINEAR INCOME TAX (skip)}
\end{center}
Assume away income effects $\varepsilon^c = \varepsilon^u
=e$ [Diamond AER'98 shows this is the key theoretical
simplification]

Consider small reform: increase $T'$ by $d\tau$ in small band $z$
and $z+dz$

Mechanical effect $dM=dz d\tau [1-H(z)]$

Welfare effect $dW=-dz d\tau [1-H(z)]G(z)$

Behavioral effect: substitution effect $\delta z$ inside small
band $[z,z+dz]$: $dB=h(z)dz \cdot T' \cdot \delta z = -h(z)dz
\cdot T' \cdot d\tau \cdot z \cdot e_{(z)}/(1-T')$

Optimum $dM+dW+dB=0$

\end{slide}

\begin{slide}
\begin{center}
{\bf GENERAL NON-LINEAR INCOME TAX}
\end{center}
\[ T'(z) =\frac{1-G(z)}{1-G(z) + \alpha(z) \cdot e_{(z)} } \]

1) $T'(z)$ decreases with $e_{(z)}$ (elasticity efficiency
effects)

2) $T'(z)$ decreases with $ \alpha(z)= (z h(z))/(1-H(z))$ (local Pareto
parameter)

3) $T'(z) $ decreases with  $G(z)$ (social weight above $z$)

Asymptotics: $G(z) \rightarrow \bar{g}$, $\alpha(z)
\rightarrow a$, $e_{(z)} \rightarrow
e$ $\Rightarrow$ Recover top rate formula
$\tau=(1-\bar{g})/(1-\bar{g}+a \cdot e)$
\end{slide}

\begin{slide}
\includepdf[pages={26}]{tax-redistribution_attach.pdf}
\end{slide}

\begin{slide}
\begin{center}
{\bf Negative Marginal Tax Rates Never Optimal (skip)}
\end{center}
Suppose $T'<0$ in band $[z,z+dz]$

Increase $T'$ by $d\tau>0$ in band $[z,z+dz]$: $dM+dW>0$ and
$dB>0$ because $T'(z)<0$

$\Rightarrow$ Desirable reform

$\Rightarrow$  $T'(z)<0$ cannot be optimal

EITC schemes are not desirable in Mirrlees '71 model 
\end{slide}

%pset idea: demonstration of top rate>0 Pareto inefficient and bottom rate<0 Pareto inefficient
% if bottom guy works

\begin{slide}
\begin{center}
{\bf NUMERICAL SIMULATIONS (skip)}
\end{center}
$H(z)$ [and also $G(z)$] endogenous to $T(.)$. Calibration method
(Saez Restud '01):

Specify utility function (e.g. constant elasticity):
$$u(c,z)=c- \frac{1}{1+\frac{1}{e}} \cdot  \left ( \frac{z}{n} \right )
^{1+\frac{1}{e}}$$ Individual FOC $\Rightarrow$
$z=n^{1+e} (1-T')^{e}$

Calibrate the exogenous skill distribution $F(n)$ so that, using
{\bf actual} $T'(.)$, you recover {\bf empirical} $H(z)$

Use Mirrlees '71 tax formula (expressed in terms of $F(n)$) to
obtain the optimal tax rate schedule $T'$.

\end{slide}

\begin{slide}
\begin{center}
{\bf NUMERICAL SIMULATIONS (skip)}
\end{center}
$$\frac{T'(z(n))}{1-T'(z(n))} =\left ( 1+\frac{1}{e} \right ) \left
(\frac{1}{n f(n)} \right ) \int_n^{\infty} \left
[1-\frac{G'(u(m))}{\lambda} \right ] f(m)dm,$$

Iterative Fixed Point method: start with $T'_0$, compute $z^0(n)$
using individual FOC, get $T^0(0)$ using govt budget, compute
$u^0(n)$, get $\lambda$ using $\lambda=\int G'(u)f$, use formula
to estimate $T'_1$, iterate till convergence

Fast and effective method (Brewer-Saez-Shepard '10)
\end{slide}


\begin{slide}
\begin{center}
{\bf NUMERICAL SIMULATION RESULTS (skip)}
\end{center}
\[ T'(z) =\frac{1-G(z)}{1-G(z) + \alpha(z) \cdot e_{(z)} } \] Take utility
function with $e$ constant

2) $\alpha(z)=(z h(z))/(1-H(z))$ is inversely U-shaped empirically

3) $1-G(z)$ increases with $z$ from $0$ to $1$ ($\bar{g}=0$)

$\Rightarrow$ Numerical optimal $T'(z)$ is U-shaped with $z$:
reverse of the general results $T'=0$ at top and bottom [Diamond
AER'98 gives theoretical conditions to get U-shape]
\end{slide}

\begin{slide}
\includepdf[pages={8}]{tax-redistribution_attach.pdf}
\end{slide}


\begin{slide}
\begin{center}
{\bf OPTIMAL TRANSFERS: MIRRLEES MODEL}
\end{center}
Mirrlees model predicts that optimal transfer at bottom takes the
form of a ``Negative Income Tax'':

1) Lumpsum grant $-T(0)$ for those with no earnings

2) High MTRs $T'(z)$ at the bottom to phase-out the lumpsum grant
quickly: intuition

Intuition: high MTRs at bottom are efficient because:

(a) they target transfers to the most needy

(b) earnings at the bottom are low to start with so intensive
response does not generate large output losses

\end{slide}


\begin{slide}
\begin{center}
{\bf OPTIMAL TRANSFERS: MIRRLEES MODEL}
\end{center}

Diamond-Saez JEP'11: 
\[ T'(0)=(g_0-1)/(g_0-1+e_0) \]

$e_0$ elasticity of the fraction non-working wrt to $1-T'(0)$ 

$g_0$
social marginal welfare weight on non workers

$ \Rightarrow T'(0)$ large: e.g. $g_0=3$ and $e_0=.5$ $\Rightarrow$ $T'(0)=80\%$

However if $g_0<1$ then $T'(0)<0$ $\Rightarrow$ EITC is optimal

\end{slide}

%\begin{slide}
%\includepdf[pages={9}]{tax-redistribution_attach.pdf}
%\end{slide}



\begin{slide}
\includepdf[pages={67-69}]{tax-redistribution_attach.pdf}
\end{slide}


%\begin{slide}
%\begin{center}
%{\bf EXTENSIONS (skip)}
%\end{center}
%
%1) Income effects can be introduced (Saez Restud '01). Keeping
%$\varepsilon_c(z)$ and $g(z)$ constant: Higher income effects
%$\Rightarrow$ Higher $T'(z)$ for high incomes
%
%2) Inverted problem: use current $T(z)$ and $H(z)$  to back out
%welfare weights $g(z)$ [very sensitive to assumptions on
%$e_{(z)}$]
%
%3) Pareto Efficient taxation (Werning '07): any tax schedule such
%that $g(z) \geq 0$ for all $z$ is Pareto Efficient (and
%conversely)
%
%If $g(z)<0$ in some range, can design a tax reform that keeps
%utilities constant and raises tax revenue [tax system is locally
%on the wrong side of the Laffer curve]
%
%\end{slide}




\begin{slide}
\begin{center}
{\bf Optimal Transfers: Participation Responses}
\end{center}
Empirical literature shows that participation labor supply
responses [due to fixed costs of working] are large at the bottom
[much larger and clearer than intensive responses]

Diamond JpubE'80, Saez QJE'02, Laroque EMA'05 incorporate such
extensive labor supply responses in the optimal income tax model

Participation depends on participation tax rate:
$\tau_p=[T(z)-T(0)]/z$: individual keeps fraction $1-\tau_p$ of
earnings when moving from zero earnings to earnings $z$:
\[z-T(z)=-T(0)+z - [T(z)-T(0)] = -T(0) + z \cdot (1-\tau_p) \]

{\bf Key result:} in-work subsidies with $T'(z)<0$ (such as EITC)
become optimal when labor supply responses are concentrated along
extensive margin and social marginal welfare weight on low skilled
workers $>1$.

\end{slide}



%\begin{slide}
%\begin{center}
%{\bf SAEZ QJE'02 PARTICIPATION MODEL (skip)}
%\end{center}
%Model with discrete earnings outcomes: $w_0=0<w_1<...<w_I$
%
%Tax/transfer $T_i$ when earning $w_i$, $c_i=w_i-T_i$
%
%Participation labor supply: Skill $i$ individual compares $c_i$
%and $c_0$ when deciding to work $\Rightarrow$ Participation tax
%rate $\tau_i$ such that $c_i-c_0=w_i \cdot (1-\tau_i)$
%
%Person works iff $c_i - \theta \geq c_0$ where $\theta$ is fixed cost of working
%
%$\Rightarrow$ In aggregate, fraction $h_i(c_i-c_0)$ of population
%earns $w_i$
%
%Participation elasticity $e_i=(c_i-c_0)/h_i \cdot \partial h_i /
%\partial (c_i-c_0)$
%
%Social Welfare function is summarized by social marginal welfare
%weights at each earnings level $g_i \downarrow i$, and average to
%one $\sum_i g_i h_i =1$ (if no income effects)
%
%\end{slide}

\begin{slide}
\includepdf[pages={63-66}]{tax-redistribution_attach.pdf}
\end{slide}


%\begin{slide}
%\includepdf[pages={13-16}]{tax-redistribution_attach.pdf}
%\end{slide}
%
%\begin{slide}
%\begin{center}
%{\bf SAEZ QJE'02: OPTIMAL TAX DERIVATION}
%\end{center}
%Small reform $dc_i=-dT_i>0$. Three effects:
%
%1) Mechanical Change in tax revenue $dM=h_i dT_i$
%
%2) Behavioral Effect: $dh_i = -e_i h_i dT_i / (c_i-c_0)$
%$\Rightarrow$ Tax loss: $dB=-(T_i-T_0) dh_i=-e_i h_i dT_i
%(T_i-T_0)/ (c_i-c_0)$
%
%3)  Welfare Effect: each worker in job $i$ looses $dT_i$ so
%welfare loss $dW=-g_i h_i dT_i$ [No first order welfare loss for
%switchers]
%
%FOC: $dM+dB+dW=0$ $\Rightarrow$
%$$\frac{\tau_i}{1-\tau_i} = \frac{T_i-T_0}{c_i-c_0} = \frac{1}{e_i} (1-g_i)$$
%$g_1>1 \Rightarrow T_1-T_0<0 \Rightarrow$ in-work subsidy
%\end{slide}

\begin{slide}
\begin{center}
{\bf ACTUAL TAX/TRANSFER SYSTEMS}
\end{center}

1) Transfer programs used to be of the traditional form with high
phasing-out rates (sometimes above 100\%) $\Rightarrow$ No
incentives to work (even with modest elasticities)

Initially designed for
groups not expected to work [widows in the US] but later attracting
groups who could potentially work [single mothers]

2) In-work benefits have been introduced and expanded in OECD
countries since 1980s (US EITC, UK Family Credit, etc.) and have
been politically successful 

$\Rightarrow$ (a) Redistribute to low
income workers, (b) improve incentives to work

\end{slide}


%\begin{slide}
%\begin{center}
%{\bf OPTIMAL TRANSFERS IN RECESSIONS (GUESS)}
%\end{center}
%1) The models we have covered consider only voluntary unemployment
%[people compare costs of work vs. benefits of work and can find a
%job if they want to]. Reasonable approximation during good times
%with low involuntary unemployment
%
%2) During recessions (such as US in 2008-2010), many unemployed
%would like to work but cannot find a job
%
%$\Rightarrow$ Labor supply participation responses shut down
%during recession [unemployed cannot find jobs, workers do not want
%to abandon jobs]
%
%$\Rightarrow$ Redistribution becomes close to lumpsum [no
%efficiency costs while labor supply is frozen]
%
%$\Rightarrow$ Redistributing more to non-workers during recessions
%is efficient [justification for increasing unemployment benefits
%during recessions, Landais-Michaillat-Saez '10]
%\end{slide}


%\begin{slide}
%\begin{center}
%{\bf TAGGING}
%\end{center}
%We have assumed that $T(z)$ depends only on earnings $z$.
%
%In reality, govt can observe many other characteristics $X$ also
%correlated with ability [gender, race, age, disability, family
%structure, height,...] and set $T(z,X)$. Two theory results:
%
%1) If characteristic $X$ is {\bf immutable} then redistribution
%across the $X$ groups will be complete [until average social
%marginal welfare weights are equated across $X$ groups]
%
%2) If characteristic $X$ can be \textbf{manipulated} [behavioral response
%or cheating] but $X$ correlated with ability then taxes will still
%depend on both $X$ and $z$.
%
%References: Akerlof AER'78 (welfare), Nichols-Zeckhauser AER'82
%(welfare), Weinzierl '11 (age), Mankiw-Weinzierl '10 (height),
%Kaplow '08 (chapter 7)
%\end{slide}

%notes: for result 2), sign of T(z,1) - T(z,0) does not depend
%only on average social marginal weights but also on whether X is
%positively or negatively related to leisure: do switchers work more or less (which has 1st order
%fiscal consequences)?

%\begin{slide}
%\begin{center}
%{\bf TAGGING WITH IMMUTABLE CHARACTERISTICS}
%\end{center}
%Consider $X$ binary immutable ({\bf T}alls vs. {\bf S}horts)
%
%With $T(z)$ independent of $X$, Talls have higher ability on
%average $\Rightarrow$ Average social marginal welfare weights
%$\bar{g}^T<\bar{g}^S$ $\Rightarrow$ Transfer from Talls to Shorts
%is desirable (surtax on Talls which finances an allowance on
%Shorts)
%
%Optimal height transfers should be up to point where
%$\bar{g}^T=\bar{g}^S$
%
%Mankiw-Weinzierl '09 compute the optimal $T^{Tall}(z)$ and
%$T^{Short}(z)$ based on calibrated mode: optimal transfer
%$T^{Tall}(z)-T^{Short}(z)$ not trivial ($\simeq$ 10\% of income)
%
%They also show that you can get a (very modest) {\bf
%Pareto} improvement using taxes on height and income instead of
%only income
%
%\end{slide}


%\begin{slide}
%\begin{center}
%{\bf PROBLEM WITH TAGGING}
%\end{center}
%
%In practice public would oppose height based redistribution
%because height does not cause high earnings $\Rightarrow$
%
%1) {\bf Horizontal Equity} concerns [people with same
%``ability-to-pay'' should pay the same tax] impose constraints on
%feasible policies [not captured by utilitarian framework]
%
%2) Constrained optimization analysis [$T(z)$ instead of $T(z,X)$]
%remains valid even with heterogeneity in preferences
%
%3) In practice $T(z,X)$ depends on $X$ only when $X$ is {\bf
%directly} related to welfare [family structure, \# kids, medical
%expenses] or ability to earn [disability status]
%(``ability-to-pay'' intuition)
%\end{slide}
%

%\begin{slide}
%\begin{center}
%{\bf IN-KIND REDISTRIBUTION (skip)}
%\end{center}
%Majority of actual transfers are in-kind (health care, child care, education, public housing, nutrition
%subsidies) %[care not cash San Francisco reform]
%
%1) {\bf Rational Individual perspective:}
%
%(a) In-kind transfer is {\bf tradeable} at market price
%$\Rightarrow$ in-kind equivalent to cash
%
%(b) In-kind transfer {\bf non-tradeable} $\Rightarrow$ in-kind
%inferior to cash.
%
%\end{slide}
%
%
%\begin{slide}
%\begin{center}
%{\bf IN-KIND REDISTRIBUTION  (skip)}
%\end{center}
%2) {\bf Social perspective:} 4 justifications:
%
%a) Commodity Egalitarianism: some goods (education, health,
%shelter, food) seen as {\bf rights} and ought to be provided to
%all
%
%b) Paternalism: society imposes its preferences on recipients
%[recipients prefer cash]
%
%c) Behavioral: Recipients do not make choices in their best
%interests (self-control, myopia) [recipients understand that
%in-kind is better for them]
%
%d) Under standard welfarist objective: Efficiency considerations
%in a 2nd best context
%
%\end{slide}


%\begin{slide}
%\begin{center}
%{\bf EFFICIENCY OF IN-KIND REDISTRIBUTION}
%\end{center}
%Depends on what income tax tools are available:
%
%1) No income tax: Income $z$ not observable (devo countries)
%$\Rightarrow$ In-kind provision or subsidies for necessities
%desirable
%
%2) Linear tax model (Ramsey): Guesnerie-Roberts EMA'84
%$\Rightarrow$ rationing goods encouraged by the tax system is
%desirable [and forcing consumption of goods discouraged by tax]
%
%3) Nonlinear income tax: Under Atkinson-Stiglitz assumption
%[weak-separability and homogeneity $U^h(v(c_1,..,c_K),z)$]
%$\Rightarrow$ Any distortion (quota, rationing, subsidy) involving
%$c$ choices not desirable provided $T(z)$ optimal
%
%If good $c_k$ related to leisure/ability [soup kitchen with
%queuing requirement] then A-S fails and in-kind redistribution
%possibly desirable even with optimal $T(z)$
%
%\end{slide}

%\begin{slide}
%\begin{center}
%{\bf IMPOSING ORDEALS ON TRANSFER RECIPIENTS  (skip)}
%\end{center}
%Many actual transfer programs impose requirements on beneficiaries
%(complex application, job search, training, or work requirements)
%and hence have low take-up (often $<50\%$)
%
%\small
%
%1) If social objective is welfarist and income $z$ observable:
%ordeals unlikely to be desirable:
%
%Compare ordeal to benefit cut: (a) only benefit cut saves money
%mechanically, (b) both reduce welfare of recipients, (c) both
%reduce take-up [good fiscally]
%
%Need implausible sorting effects for ordeal to be desirable [e.g.,
%ordeal does not hurt much deserving beneficiaries and discourages
%undeserving take-up, conditional on $z$]
%
%2) If $z$ is not observable then ordeal could be desirable (kitchen soup line)
%
%3) With non-welfarist objective [such as poverty alleviation], ordeal can be desirable [Besley-Coate
%AER'92]
%
%\end{slide}
%
%\begin{slide}
%\begin{center}
%{\bf WORK RESTRICTIONS AND MINIMUM WAGE  (skip)}
%\end{center}
%Minimum wage creates rationing of low skilled work. Could minimum
%wage be desirable on top of nonlinear tax/transfer?
%
%Lee and Saez JpubE'12 use a job choice model [Saez QJE '02 with
%endogenous wages]. Two results:
%
%1) Minimum wage desirable if (a) govt wants to redistribute to low
%skilled workers ($g_1>1$) and (b) rationing created by min wage is
%{\bf efficient} 
%
%2) If labor supply responses along extensive margin only then
%minimum wage with positive tax rate on low skilled work $\tau_1>0$
%is 2nd best Pareto inefficient 
%
%$\Rightarrow$ EITC and min wage are
%complementary 
%%[delivers strong policy reform
%%prescription]
%\end{slide}

%\begin{slide}
%\includepdf[pages={17-23}]{tax-redistribution_attach.pdf}
%\end{slide}


%\begin{slide}
%\begin{center}
%{\bf FAMILY TAXATION: MARRIAGE AND CHILDREN}
%\end{center}
%Two important issues in policy debate:
%
%1) Marriage: What is the optimal taxation of couples vs. singles?
%Should secondary earnings be treated differently?
%
%2) Children: What should be the net transfer (transfer or tax
%reduction) for family with children (as a function of family
%income and structure)?
%
%Theoretical literature is not great in part because utilitarian
%framework is not satisfactory
%\end{slide}
%
%\begin{slide}
%\begin{center}
%{\bf TAXATION OF COUPLES}
%\end{center}
%1) Economies of scale and sharing in consumption within families
%$\Rightarrow$ Welfare best measured by family income relative to
%size [$\equiv$ {\bf normalized income}]
%
%$\Rightarrow$ Taxes/Transfers should be based on normalized family income
%which can create a marriage penalty / subsidy
%
%Note: Impossible to have a tax/transfer system that 
%
%(1) is  family
%income based $T(z^h+z^w)$
%
%(2) has marriage neutrality $T(z^h,z^w)=T(z^h)+T(z^w)$
%
%(3) is progressive (i.e., not strictly linear)
%
%Proof: (1)+(2) $\Rightarrow T(z^h+z^w)=T(z^h)+T(z^w) \Rightarrow T(z)=\tau \cdot z$
%
%\normalsize
%
%
%\end{slide}
%
%\begin{slide}
%\begin{center}
%{\bf TAXATION OF COUPLES}
%\end{center}
%
%2) If marriage responds to tax/transfer differential $\Rightarrow$
%better to reduce marriage penalty and move toward
%individualized system
%
%Particularly important cohabitation is close
%substitute to marriage (Scandinavian countries)
%
%3) Labor supply of secondary earners more elastic than labor
%supply of primary earner $\Rightarrow$ Secondary earnings should
%be taxed less (standard Ramsey intuition, Boskin-Sheshinski
%JpubE'83)
%
%But labor supply elasticity differential is decreasing as earnings
%gender gap decreases [Blau and Kahn JOLE'07]
%
%%4) Welfare effect of spousal earnings $\downarrow$ primary
%%earnings $\Rightarrow$ Transfer for having a non-working spouse
%%[=tax on secondary earnings] $\downarrow$ primary earnings
%%[Kleven-Kreiner-Saez EMA'09]
%
%In OECD countries: income tax systems have become {\bf individual
%based} but means tested transfers have remained {\bf family based}
%
%\end{slide}
%
%\begin{slide}
%\begin{center}
%{\bf TRANSFERS OR TAX CREDITS FOR CHILDREN}
%\end{center}
%1) Children reduce {\bf normalized income} $\Rightarrow$ Transfer
%for children $T_{kid}$ should be positive
%
%In practice, transfers for children are always positive
%
%2) Should $T_{kid}(z)$ increase with income $z$?
%
%Pro: they reduce normalized income most for upper earners [e.g.,
%France computes taxes as $N \cdot T(z/N)$ where $N$ is \# family
%members, kids count as .5 $\Rightarrow$ $T_{kid}(z) $ increases with $ z$].
%
%Cons: lower earners need child transfers most [most OECD countries
%have means-tested transfers conditional on number of kids
%$\Rightarrow$ $T_{kid}(z)$ decreases with $ z$, US has  $T_{kid}(z)$
%inverted U-shape due to EITC and Child Tax Credit]
%
%\end{slide}
%
%\begin{slide}
%\begin{center}
%{\bf TRANSFERS OR TAX CREDITS FOR CHILDREN}
%\end{center}
%
%3) Family does not make decisions as a single unit (Chiappori JPE'92):
%transfers to mothers has bigger effects on children's consumption
%than transfers to fathers [Lundberg et al. '97, Duflo '03]
%
%4) Children create externalities [positive: pay-as-you-go retirement programs,
%negative: global warming]. If fertility responds to transfers,
%case for subsidizing/taxing children
%
%5) Child care costs are positively related to work
%$\Rightarrow$ Such costs should be subsidized by Atkinson-Stiglitz
%[often they are in practice]:
%
%Public pre-kindergarten in Europe is a huge in-work subsidy for
%mothers $\Rightarrow$ Large effect on mothers' labor force participation
%(bigger effect than US EITC)
%\end{slide}

%\begin{slide}
%\begin{center}
%{\bf CHILDREN AND LIMITS OF UTILITARIAN MODEL}
%\end{center}
%If fertility decisions unrelated to children tax/transfers
%$\Rightarrow$ Social marginal utility should be equated across
%families with 0 children, families with 1 child, etc.
%
%If ability uncorrelated with children $\Rightarrow$ Families with
%kids will get fully compensating transfers
%
%If ability positively correlated with children $\Rightarrow$
%Families with kids might be taxed more heavily [as in the height
%tax case]
%
%Seems an absurd model to think about transfers for children
%$\Rightarrow$ Need to come up with more realistic alternative
%
%\end{slide}

\begin{slide}
\begin{center}
{\bf LIMITS OF WELFARIST APPROACH}
\end{center}
Welfarism is the dominant approach in optimal taxation 

Welfarism: social objective is a sole function of individual utilities

Tractable and coherent framework that captures the equity-efficiency trade-off but generates puzzles:

\small

(a) 100\% taxation absent behavioral responses

(b) Whether income is deserved or due to luck is irrelevant

(c) What transfer recipients would have done absent transfers is irrelevant

%(d) Tags correlated with ability should be heavily used

\normalsize
A number of alternatives to welfarism have been proposed 

Saez-Stantcheva '16 propose ``generalized social welfare weights'' to resolve
those puzzles. Stantcheva '20: surveys on how people think about taxes.

\end{slide}


\begin{slide}
\begin{center}
{\bf TESTING PEOPLE SOCIAL PREFERENCES}
\end{center}
Saez-Stantcheva '16 survey people online (using Amazon MTurk) by asking hypothetical questions
to elicit social preferences:

1) People typically do not have ``utilitarian'' social justice principles (consumption lover not seen as more
deserving than frugal person)

2) People put weight on whether income has been earned through effort vs. not (hard working
vs. leisure lover)

3) People put weight of what people would have done absent the government intervention
(deserving poor vs. free loaders)

4) People put weight on taxes paid conditional on $c$: person z=\$50K, T=\$15K, c=\$35K more deserving than z=\$40K, T=\$5K, c=\$35K


\small
$\Rightarrow$ Aversion for free loaders and liking tax contributors  is like an intuitive or evolutionary optimal tax tool
(needed to sustain cooperative societies)

%Lakoff (1996) shows that framing (conservative/liberal) is critical to shape views
\end{slide}


\begin{slide}
\includepdf[pages={74, 73, 72}]{tax-redistribution_attach.pdf}
\end{slide}


\begin{slide}
\includepdf[pages={71}]{tax-redistribution_attach.pdf}
\end{slide}



%\begin{slide}
%\begin{center}
%{\bf STANDARD TAX REFORM APPROACH}
%\end{center}
%Consider classic optimal income tax problem. Individual $i$ has utilities $u^i(c,z)$ increasing
%in consumption $c$, decreasing in earnings $z$.
%
%Planner sets $T(z)$ to maximize $\int_i G(u_i)$ 
%
%subject to (a) $\int_i T(z_i) \geq E$, (b) $z_i$ maximizes $u^i(z_i-T(z_i),z_i)$
%
%Equivalent to \textbf{tax reform approach:} $T(z)$ is optimal if
%
%For any budget  \textbf{neutral} small tax reform $dT(z)$,
%$\int_i g_i dT(z_i)=0$ with $g_i=G_{u^i} u^i_c \geq 0$ social marginal welfare weight on indiv. $i$
%
%Sum mechanical gains and losses across individuals, behavioral responses matter only for govt budget constraint
%
%$\Rightarrow$ Generates optimal tax formulas function of $g_i$'s and behavioral response elasticities
%
%\end{slide}

\begin{slide}
\begin{center}
{\bf Generalized Social Marginal Welfare Weights (skip)}
\end{center}
Social planner uses \textbf{generalized social marginal welfare weights} $g_i \geq 0$ to value marginal consumption
of individual $i$

Standard utilitarist case $SWF=\int_i u^i$ has $g_i=u^i_c$

But we can define generalized $g_i$ that might depend on fairness judgements as well

\textbf{Optimal tax criterion:} $T(z)$ is optimal if for any budget \textbf{neutral} small tax reform $dT(z)$,
$\int_i g_i \cdot dT(z_i)=0$ with $g_i \geq 0$ generalized social marg. welfare weight on indiv. $i$

%\small

1) Generates same optimal tax formulas as welfarist approach 

2) Respects (local) constrained Pareto efficiency ($g_i \geq 0$)

3) No social objective is maximized
\end{slide}



\begin{slide}
\begin{center}
{\bf Application 1: Optimal Tax with Fixed Incomes (skip)}
\end{center}
%$u(c)$ with $c=z-T(z)$

Utilitarian approach has degenerate solution with 100\% taxation when $u'(c)$ decreases with $c$

Public may not support confiscatory taxation even absent behavioral responses

Generalized social marginal welfare weights: $g_i=g(c_i,T_i)$

$g(c,T) $ decreases with $c$ (ability to pay)

$g(c,T)$ increases with $T$ (contribution to society)

Optimum: $g(z-T(z),T(z))$ equalized across $z$ 

$\Rightarrow T'(z)=1/(1-g_T/g_c)$ and $0 \leq T'(z) \leq 1$

\end{slide}


\begin{slide}
\begin{center}
{\bf Application 1: Optimal Tax with Fixed Incomes (skip)}
\end{center}
Preferences for redistributions embodied in $g(c,T)$

Polar cases:

1) Utilitarian case: $g(c,T)=u'(c)  \Rightarrow T'(z)\equiv 1$

2) Libertarian case: $g(c,T)=g(T)  \Rightarrow T'(z) \equiv 0$

We use Amazon mTurk online survey to estimate $g(c,T)$

We find that revealed preferences depend on \textbf{both} $c$ and $T$

\small
z=\$40K, T=\$10K, c=\$30K more deserving than z=\$50K, T=\$10K, c=\$40K

z=\$50K, T=\$15K, c=\$35K more deserving than z=\$40K, T=\$5K, c=\$35K

\end{slide}

%\begin{slide}
%\begin{center}
%{\bf Application 2: DESERVED VS. LUCK INCOME}
%\end{center}
%Taxing luck income is fair while taxing deserved income is not
%
%Suppose $z=w+y$ with $w$ deserved income and $y$ luck income ($w,y$ mix not observable)
%
%Person is deserving if $c=z-T \leq w+ Ey$ with $Ey$ average luck income 
%$\Rightarrow g_i=1$ if $c_i \leq w_i+Ey$, $g_i=0$ if not.
%
%$\mathrm{Prob}[g_i=1 | w+y=z]$ provides micro-foundation for $g(c,T)$ increasing in $T$
%
%Beliefs in share of income due to luck at each income level is key
%
%\end{slide}




\begin{slide}
\begin{center}
{\bf Application 2: FREE LOADERS (skip)}
\end{center}
Saez-Stantcheva '16 online survey shows strong public preference for redistributing toward ``deserving poor''
(unable to work or trying hard to work) rather than ``undeserving poor'' (who would work absent
transfers)

Generalized social welfare weights can capture this by setting $g_i=0$ on free loaders (=transfer recipients who would have worked absent the transfer) $\Rightarrow$

1) Behavioral responses reduce desirability of transfers (over and above standard budgetary effect)

2) In-work benefit ($T'(0)=(g_0-1)/(g_0-1+e_0)<0$ at bottom becomes optimal in Mirrlees (1971) optimal tax model
if $g_0<1$

%3) Extending UI benefits in recessions (when jobs are hard to find)
\end{slide}


%\begin{slide}
%\begin{center}
%{\bf Application 4: TAGS AND HORIZONTAL EQUITY}
%\end{center}
%Tagging (e.g. taxing height) desirable in welfarist framework
%
%In practice horizontal equity concerns
%
%Horizontal equity concerns can be captured with generalized social marginal welfare weights
%as follows:
%
%If individual suffers from horizontal inequity (e.g., she is taxed more because of her height) then
%her generalized social welfare weight increases sharply (Endogenous Rawlsian criterion)
%
%$\Rightarrow$ Tagging is desirable only if it benefits those discriminated against by the tax system
%
%
%$\Rightarrow$ Dramatically reduces the scope for tagging
%
%\end{slide}

\begin{slide}
\begin{center}
{\bf LINK WITH OTHER JUSTICE PRINCIPLES (skip)}
\end{center}
Various alternatives to welfarism have been proposed (survey Fleurbaey-Maniquet '11)

Each alternative can be recast in terms of implied \textbf{generalized social marginal welfare weights} (as long
as it generates constrained Pareto efficient optima)

In all cases, we can use simple and tractable optimal income tax formula for heterogeneous population from Saez Restud'01 (case with no income effects):
\[ T'(z)=\frac{1-G(z)}{1-G(z)+\alpha(z) \cdot e(z)} \] 
with $G(z)$ average of $g_i$ above $z$

$g_i$ average to one in the full population and hence $G(0)=1$
\end{slide}

\begin{slide}
\begin{center}
{\bf LINK WITH OTHER JUSTICE PRINCIPLES (skip)}
\end{center}

\textbf{1) Rawlsian:}  $g_i$ concentrated on worst-off individual $\Rightarrow$ $G(z)=0$ for $z>0$ and
$T'(z)=1/(1+ \alpha(z) \cdot e)$ revenue maximizing

\textbf{2) Libertarian:} $g_i \equiv 1$ $\Rightarrow$ $G(z)\equiv 1$ and $T'(z) \equiv 0$

%\textbf{3) Fair income tax:} Fleurbaey-Maniquet '06, '11 capture the compensation and responsibility principles: 
%individuals should be compensated for differences in ability but not for differences in taste for work
%
%$g_i$ concentrated
%on minimum wage workers working full-time (earnings $z_0>0$) $\Rightarrow$ $G(z)=0$ if $z>z_0$,
%$G(z)=1/(1-H(z))>1$ if $z<z_0$. Therefore, $T'(z)<0$ for $z<z_0$ and $T'(z)=1/(1+ \alpha(z) \cdot e)$ for $z>z_0$

\textbf{3) Equality of Opportunity:} (Roemer '98) $g_i$ concentrated on those coming from disadvantaged background.
$G(z)$= relative fraction of individuals above $z$ coming from disadvantaged background

$\Rightarrow G(z)$ decreases with $z$ for reasons unrelated to decreasing marginal utility


\end{slide}


\begin{slide}
\includepdf[pages={41}]{tax-redistribution_attach.pdf}
\end{slide}

%\begin{slide}
%\begin{center}
%{\bf CONCLUSION ON LIMITS OF WELFARISM}
%\end{center}
%\textbf{Tax reform approach + generalized social marginal welfare weights} create a tractable and powerful 
%framework for optimal tax analysis
%
%1) Nests welfarist case, retains the same optimal tax formulas, can address most puzzles
%
%2) Puts social preferences at the core of the optimal tax problem: 2 approaches
%
%a) Positive approach: analyze actual social preferences for redistribution using online surveys
%
%b) Normative approach: set social marginal welfare weights based on justice principles
%
%\end{slide}


\begin{slide}
\begin{center}
{\bf COMMODITY VS. INCOME TAXATION}
\end{center}
Suppose we have $K$ consumption goods $c=(c_1,..,c_K)$ with
pre-tax price $p=(p_1,..,p_K)$. Individual $h$ has utility
$u^h(c_1,..,c_K,z)$

Key question: Can government increase $SWF$ using differentiated
commodity taxation $t=(t_1,..,t_K)$ (after tax price $q=p+t$) in
addition to nonlinear Mirrlees income tax on earnings $z$?

In practice, govt (a) exempts some goods (food, education, health)
from sales tax or value-added-tax, (b) imposes additional excise
taxes on some goods (cars, gasoline, luxury goods)

$\max_{t,T(.)} SWF \geq \max_{t=0,T(.)} SWF$ because more
instruments cannot hurt

\end{slide}

\begin{slide}
\begin{center}
{\bf ATKINSON-STIGLITZ THEOREM }
\end{center}

Famous Atkinson-Stiglitz JpubE' 76 shows that $$\max_{t,T(.)} SWF
= \max_{t=0,T(.)} SWF$$ (i.e, commodity taxes not useful over and above $T(z)$) under
\textbf{two assumptions} on utility functions $u^h(c_1,..,c_K,z)$

1) Weak separability between $(c_1,..,c_K)$ and $z$ in utility

2) Homogeneity across individuals in the sub-utility of
consumption $v(c_1,..,c_K)$ [does not vary with $h$]
\[ \text{(1) and (2):} \quad u^h(c_1,..,c_K,z)=U^h(v(c_1,..,c_K),z) \]

%$\Rightarrow$ Given after tax income $y=z-T(z)$, everybody
%consumes the same bundle $c$ [$\max_{c} v(c_1,..,c_K)$ st $q \cdot
%c \leq y$]

Original proof was based on optimum conditions, new
straightforward proof by Laroque EL '05, and Kaplow JpubE '06.

\end{slide}

\begin{slide}
\begin{center}
{\bf ATKINSON-STIGLITZ THEOREM PROOF}
\end{center}

Let $V(y,p+t)=\max_{c} v(c_1,..,c_K)$ st $(p+t) \cdot c \leq y$ be
the indirect utility of consumption $c$ [common to all
individuals]

Start with $(T(.),t)$. Let $c(t)$ be consumer choice.

Replace $(T(.),t)$ with $(\bar{T}(.),t=0)$ where $\bar{T}(z)$ such
that $V(z-T(z),p+t)=V(z-\bar{T}(z),p)$ $\Rightarrow$ Utility
$U^h(V,z)$ and labor supply choices $z$ unchanged for all
individuals.

Attaining $V(z-\bar{T}(z),p)$ at price $p$ costs at least
$z-\bar{T}(z)$

Consumer also attains $V(z-\bar{T}(z),p)=V(z-T(z),p+t)$ when
choosing $c(t)$ $\Rightarrow$ $ z - \bar{T}(z) \leq p \cdot c(t)
=z -T(z) - t \cdot c(t)$

$\Rightarrow$ $\bar{T}(z) \geq T(z) + t \cdot c(t)$: the
government collects more taxes with $(\bar{T}(.),t=0)$

\end{slide}

\begin{slide}
\begin{center}
{\bf ATKINSON-STIGLITZ INTUITION}
\end{center}

With separability and homogeneity, conditional on earnings $z$,
consumption choices $c=(c_1,..,c_K)$ do not provide any
information on ability

$\Rightarrow$ Differentiated commodity taxes $t_1,..,t_K$ create a
tax distortion with no benefit $\Rightarrow$ Better to do all the
redistribution with the individual income tax

Note: With weaker linear income taxation tool (Diamond-Mirrlees
AER '71, Diamond JpubE '75), need $v(c_1,..,c_K)$ homothetic
(linear Engel curves, Deaton EMA '81) to obtain no
commodity tax result

\small
[Unless Engel curves are linear, commodity taxation can be useful
to ``non-linearize'' the tax system]
\end{slide}


\begin{slide}
\begin{center}
{\bf WHEN ATKINSON-STIGLITZ ASSUMPTIONS FAIL}
\end{center}

Thought experiment: force high ability people to work less and
earn only as much as low ability people: if higher ability consume
more of good $k$ than lower ability people, then taxing good $k$
is desirable. Happens when:

1) High ability people have a relatively higher taste for good $k$
(independently of income) [indirect tagging]

2) Good $k$ is positively related to leisure (consumption of $k$
increases when leisure increases keeping after-tax income
constant) [tax on holiday trips, subsidy on work
related expenses such as child care]

In general Atkison-Stiglitz assumption is a good starting place
for most goods $\Rightarrow$ Zero-rating on some goods under VAT
for redistribution is inefficient and administratively burdensome
[Mirrlees 2010 review]

\end{slide}

\begin{slide}
\begin{center}
{\bf ATKINSON-STIGLITZ AND TAX ON SAVINGS (skip)}
\end{center}
Standard two period model ($w$=wage rate in period 1, retired in
period 2)
$$u^h(c_1,c_2,z)=u(c_1)+\frac{u(c_2)}{1+\delta}-b(z/w)$$
$\delta$ is the discount rate, $b(.)$ is the disutility of effort,
budget $c_1+c_2/(1+r(1-t_K)) \leq z-T(z)$

Aktinson-Stiglitz implies that savings taxation $t_K$ (equivalent
to tax on $c_2$) is useless in the presence of an optimal income
tax if $\delta$ is the same for everybody

If low ability people have higher $\delta$ [empirically plausible]
then savings tax $t_K>0$ is desirable (Saez JpubE '02)

%Diamond-Spinnewijn '09 consider nonlinear savings tax
\end{slide}

%\begin{slide}
%\begin{center}
%{\bf ATKINSON-STIGLITZ AND TAX ON SAVINGS II}
%\end{center}
%{\bf Conjecture to verify:}
%
%Suppose now that labor supply decision is about retirement age
%[length of work life vs. retirement life]
%
%Savings are used for retirement consumption
%
%$\Rightarrow$ Retirement consumption is positively related to
%leisure [high skill person retiring earlier and earning life-time
%like a low skilled person needs to save more to finance smooth
%consumption profile]
%
%$\Rightarrow$ Retirement savings should be taxed
%
%\end{slide}





\begin{slide}
\begin{center}
{\bf REFERENCES}
\end{center}
{\small

Akerlof, G. ``The Economics of Tagging as Applied to the Optimal Income Tax, Welfare Programs, and Manpower Planning'', American Economic Review, Vol. 68, 1978, 8-19. \href{http://links.jstor.org/stable/pdfplus/1809683.pdf} {(web)} 

Atkinson, A.B. and J. Stiglitz ``The design of tax structure: Direct versus indirect taxation'', Journal of Public Economics, Vol. 6, 1976, 55-75. \href{http://elsa.berkeley.edu/~saez/course/AtkinsonStiglitz_JPubE(1976).pdf} {(web)}

Besley, T. and S. Coate ``Workfare versus Welfare: Incentives Arguments for Work Requirements in Poverty-Alleviation Programs'', American Economic Review, Vol. 82, 1992, 249-261. \href{http://links.jstor.org/stable/pdfplus/2117613.pdf} {(web)}

Blau, F. and L. Kahn ``Changes in the Labor Supply Behavior of Married Women: 1980-2000'', Journal of Labor Economics, Vol. 25, 2007, 393-438. \href{http://www.jstor.org/stable/pdfplus/10.1086/513416.pdf} {(web)}

Boskin, M. and E. Sheshinski ``Optimal tax treatment of the family: Married couples'', Journal of Public Economics, Vol. 20, 1983, 281-297 \href{http://elsa.berkeley.edu/~saez/course/Boskin,Sheshinski_JpubE(1983).pdf} {(web)}

Brewer, M., E. Saez, and A. Shephard ``Means Testing and Tax Rates on Earnings'', in The Mirrlees Review: Reforming the Tax System for the 21st Century, Oxford University Press, 2010. \href{http://eml.berkeley.edu/~saez/brewer-saez-shephardMR10book.pdff} {(web)}

Chiappori, P-A ``Collective Labor Supply and Welfare'', Journal of Political Economy, 100(3), 1992, 437--467. \href{http://www.jstor.org/stable/pdfplus/2138727.pdf} {(web)}

Cremer, H., F. Gahvari, and N. Ladoux ``Externalities and optimal taxation'', Journal of Public Economics, Vol. 70, 1998, 343-364. \href{http://elsa.berkeley.edu/~saez/course/Cremer et.al._JPubE(1998).pdf} {(web)}

Deaton, A. ``Optimal Taxes and the Structure of Preferences'', Econometrica, Vol. 49, 1981, 1245-1260 \href{http://www.jstor.org/stable/pdfplus/1912753.pdf} {(web)}

Diamond, P. ``A many-person Ramsey tax rule'', Journal of Public Economics, Vol.4, 1975, 335-342. \href{http://elsa.berkeley.edu/~saez/course/Diamond_JPubE(1975).pdf} {(web)}

Diamond, P. ``Income Taxation with Fixed Hours of Work''Journal of Public Economics, Vol. 13, 1980, 101-110. \href{http://elsa.berkeley.edu/~saez/course/Diamond_JPubE(1980).pdf} {(web)}

Diamond, P. ``Optimal Income Taxation: An Example with a U-Shaped Pattern of Optimal Marginal Tax Rates'', American Economic Review, Vol. 88, 1998, 83-95. \href{http://links.jstor.org/stable/pdfplus/116819.pdf} {(web)}

Diamond, P. and J. Mirrlees ``Optimal Taxation and Public Production I: Production Efficiency'', American Economic Review, Vol. 61, 1971, 8-27. \href{http://links.jstor.org/stable/pdfplus/1910538.pdf} {(web)}

Diamond, P. and J. Mirrlees ``Optimal Taxation and Public Production II: Tax Rules'', American Economic Review, Vol. 61, 1971, 261-278. \href{http://links.jstor.org/stable/pdfplus/1813425.pdf} {(web)}

\textbf{Diamond, P. and E. Saez ``From Basic Research to Policy Recommendations:
The Case for a Progressive Tax'', Journal of Economic Perspectives, 25(4), 2011, 165-190.
\href{http://elsa.berkeley.edu/~saez/diamond-saezJEP11full.pdf} {(web)} }

%Diamond, P. and J. Spinnewijn ``Capital Income Taxes with Heterogeneous Discount Rates'', NBER Working Paper No. 15115, 2009. \href{http://www.nber.org/papers/w15115.pdf} {(web)}

Duflo, E. ``Grandmothers and Granddaughters: Old-Age Pensions and Intrahousehold Allocation in South Africa'', The World Bank Economic Review
Vol. 17, 2003, 1-25 \href{http://www.jstor.org/stable/pdfplus/3990043.pdf} {(web)}

Edgeworth, F. ``The Pure Theory of Taxation'', The Economic Journal, Vol. 7, 1897, 550-571. \href{http://www.jstor.org/stable/pdfplus/2956603.pdf} {(web)}

Feldstein, M. ``Tax Avoidance and the Deadweight Loss of the Income Tax'', Review of Economics and Statistics, Vol. 81, 1999, 674-680. \href{http://links.jstor.org/stable/pdfplus/2646716.pdf} {(web)}

%Fleurbaey, Marc and Fran\c{c}ois Maniquet. 2006.
%\textquotedblleft Fair Income Tax,\textquotedblright\ \textit{Review of
%Economic Studies} 73, 55-83.  \href{http://www.jstor.org/stable/3700617} {(web)}

%Fleurbaey, Marc and Fran\c{c}ois Maniquet. 2011. \textit{%
%A Theory of Fairness and Social Welfare}, Cambridge: Cambridge University
%Press.

Guesnerie, R. and K. Roberts, ``Effective Policy Tools and Quantity Controls'', Econometrica, Vol. 52, 1984, 59-86. \href{http://links.jstor.org/stable/pdfplus/1911461.pdf} {(web)}

Kaplow, L. ``On the undesirability of commodity taxation even when income taxation is not optimal'', Journal of Public Economics, Vol. 90, 2006, 1235-1250. \href{http://elsa.berkeley.edu/~saez/course/Kaplow_JPubE(2006).pdf} {(web)}

Kaplow, L. \emph{The Theory of Taxation and Public Economics.}  Princeton University Press, 2008.

Kaplow, L. and S. Shavell ``Any Non-welfarist Method of Policy Assessment Violates the Pareto Principle,''
Journal of Political Economy, 109(2), (April 2001), 281-286
\href{http://www.jstor.org/stable/pdfplus/10.1086/319553.pdf} {(web)}

Kleven, H., C. Kreiner and E. Saez ``The Optimal Income Taxation of Couples'', Econometrica, Vol. 77, 2009, 537-560. \href{http://elsa.berkeley.edu/~saez/course/Kleven et al_Econometrica.pdf} {(web)}

Lakoff, George, 1996. \emph{Moral Politics: How Liberals and Conservatives Think}, 2nd edition 2010.
\href{https://georgelakoff.com/} {(web)}

Landais, Camille, Pascal Michaillat, and Emmanuel Saez ``Optimal Unemployment Insurance over the Business Cycle,'' NBER Working Paper No. 16526, November 2010.
\href{http://www.nber.org/papers/w16526.pdf} {(web)}

Laroque, G. ``Income Maintenance and Labor Force Participation'', Econometrica, Vol. 73, 2005, 341-376. \href{http://links.jstor.org/stable/pdfplus/3598791.pdf} {(web)}

\textbf{Laroque, G. ``Indirect Taxation is Superfluous under Separability and Taste Homogeneity: A Simple Proof'', Economic Letters, Vol. 87, 2005, 141-144. \href{http://elsa.berkeley.edu/~saez/course/Laroque(2005).pdf} {(web)} }

Lee, D. and E. Saez ``Optimal Minimum Wage in Competitive Labor Markets'', Journal of Public Economics 96(9-10), 2012, 739-749. \href{http://elsa.berkeley.edu/~saez/lee-saezJpubE12minwage.pdf} {(web)}

Lehmann, E., L. Simula, A. Trannoy ``Tax Me if You Can! Optimal Nonlinear Income Tax between Competing Governments,'' Quarterly Journal of Economics 129(4), 2014, 1995-2030.
\href{http://elsa.berkeley.edu/~saez/course/lehmannetalQJE14.pdf} {(web)}

Lundberg, S. R. Pollak and T. Wales ``Do Husbands and Wives Pool Their Resources? Evidence from the United Kingdom Child Benefit'', The Journal of Human Resources, Vol. 32, 1997, 463-480 \href{http://www.jstor.org/stable/pdfplus/146179.pdf} {(web)}

Mankiw, G. and M. Weinzierl ``The Optimal Taxation of Height: A Case Study of Utilitarian Income Redistribution'', AEJ: Economic Policy, Vol. 2, 2010, 155-176. \href{http://elsa.berkeley.edu/~saez/course/Mankiw and Weinzierl_AEJ(2010).pdf} {(web)}

Mirrlees, J. ``An Exploration in the Theory of Optimal Income Taxation'', Review of Economic Studies, Vol. 38, 1971, 175-208. \href{http://links.jstor.org/stable/pdfplus/2296779.pdf} {(web)}

Mirrlees, J. ``Migration and Optimal Income Taxes'', Journal of Public Economics, Vol. 18, 1982, 319-341. \href{http://elsa.berkeley.edu/~saez/course/Mirrlees_JPubE(1982).pdf} {(web)}

Mirrlees, J. Reforming the Tax System for the 21st Century The Mirrlees Review, Oxford University Press, (2 volumes) 2009 and 2010.
\href{http://www.ifs.org.uk/mirrleesReview} {(web)}

Nichols, A. and R. Zeckhauser``Targeting Transfers Through Restrictions on Recipients'', American Economic Review, Vol. 72, 1982, 372-377.  \href{http://links.jstor.org/stable/pdfplus/1802361.pdf} {(web)}

Piketty, T. ``La redistribution fiscale face au ch\^{o}mage'', Revue fran\c{c}aise d'\'{e}conomie, Vol.12, 1997, 157-201. \href{http://elsa.berkeley.edu/~saez/course/Piketty(1997).pdf} {(web)}

\textbf{Piketty, Thomas and Emmanuel Saez ``Optimal Labor Income Taxation,'' Handbook of Public Economics, Volume 5, Amsterdam: Elsevier-North Holland, 2013.
\href{http://www.nber.org/papers/w18521.pdf} {(web)} }

Piketty, Thomas, Emmanuel Saez, and Stefanie Stantcheva "Optimal Taxation of Top Labor Incomes: A Tale of Three Elasticities", American Economic Journal: Economic Policy, 6(1), 2014, 230-271
\href{http://eml.berkeley.edu/~saez/piketty-saez-stantchevaAEJ14.pdf} {(web)} 

Sadka, E. ``On Income Distribution, Incentives Effects and Optimal Income Taxation'', Review of Economic Studies, Vol. 43, 1976, 261-268. \href{http://www.jstor.org/stable/pdfplus/2297322.pdf} {(web)}

Saez, E. ``Using Elasticities to Derive Optimal Income Tax Rates'', Review of Economics Studies, Vol. 68, 2001, 205-229. \href{http://links.jstor.org/stable/pdfplus/2695925.pdf} {(web)} 

Saez, E. ``Optimal Income Transfer Programs: Intensive Versus Extensive Labor Supply Responses'', Quarterly Journal of Economics, Vol. 117, 2002, 1039-1073.  \href{http://links.jstor.org/stable/pdfplus/4132495.pdf} {(web)} 

Saez, E. ``The Desirability of Commodity Taxation under Non-linear Income Taxation and Heterogeneous Tastes'', Journal of Public Economics, Vol. 83, 2002, 217-230. \href{http://elsa.berkeley.edu/~saez/course/Saez_JPubE(2002).pdf} {(web)}

Saez, E. ``The Optimal Treatment of Tax Expenditures'', Journal of Public Economics, Vol. 88, 2004, 2657-2684. \href{http://elsa.berkeley.edu/~saez/course/Saez_JPubE(2004).pdf} {(web)}

Saez, E., J. Slemrod, and S. Giertz (2012) ``The Elasticity of Taxable Income with Respect to Marginal Tax Rates: A Critical Review'', Journal of Economic Literature. \href{http://elsa.berkeley.edu/~saez/course/Saez et al(2010).pdf} {(web)} 

\textbf{Saez, Emmanuel and Stefanie Stantcheva ``Generalized Social Marginal Welfare Weights for Optimal Tax Theory,'' American Economic Review 2016. \href{http://eml.berkeley.edu/~saez/saez-stantchevaAER16.pdf} {(web)} }

Sandmo, A. ``Optimal Taxation in the Presence of Externalities'', The Swedish Journal of Economics, Vol. 77, 1975, 86-98. \href{http://www.jstor.org/stable/pdfplus/3439329.pdf} {(web)}

Seade, Jesus K. ``On the shape of optimal tax schedules.'' Journal of public Economics 7.2 (1977): 203-235.
\href{http://elsa.berkeley.edu/~saez/course/seadeJpubE77zerotopresults.pdf} {(web)} 

Stantcheva, Stefanie. ``Understanding Tax Policy: How Do People Reason?'', NBER Working Paper No. 27699, 2020
\href{http://www.nber.org/papers/w27699.pdf} {(web)} 

Stiglitz, J. ``Self-selection and Pareto Efficient Taxation'', Journal of Public Economics, Vol. 17, 1982, 213-240. \href{http://elsa.berkeley.edu/~saez/course/Stiglitz_JPubE(1982).pdf} {(web)}

%Werning, I. ``Pareto Efficient Income Taxation'', Mimeo MIT, 2007. \href{http://elsa.berkeley.edu/~saez/course/Werning(2007).pdf} {(web)}

%Weinzierl, M. ``The Surprising Power of Age-Dependent Taxes'', Harvard Business School Working Paper, No. 11-114, 2011 \href{http://elsa.berkeley.edu/~saez/course/Weinzierl(2011).pdf} {(web)}

}
\end{slide}

\end{document}
