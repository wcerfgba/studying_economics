\documentclass[landscape]{slides}

\usepackage[landscape]{geometry}

\usepackage{pdfpages}

\usepackage{pdfpages}

\usepackage{{hyperref}}

\def\mathbi#1{\textbf{\em #1}}

\topmargin=-1.8cm \textheight=17cm \oddsidemargin=0cm
\evensidemargin=0cm \textwidth=22cm

\author{Emmanuel Saez}

\date{Berkeley}

\title{230B: Public Economics \\
Tax Enforcement} \onlyslides{1-300}

\newenvironment{outline}{\renewcommand{\itemsep}{}}

\begin{document}

\begin{slide}
\maketitle
\end{slide}

\begin{slide}
\begin{center} {\bf Tax Enforcement Problem} \end{center}

Most models of optimal taxation (income or commodity) assume away
enforcement issues. In practice:

\vskip2ex

1) Enforcement is costly (eats up around 10\% of taxes collected
in the US) when combining costs for government (tax administration) and private
agents (tax compliance costs)

\vskip2ex

2) Substantial tax evasion (15\% of under-reported income in the
US federal taxes). Tax evasion much worse in developing countries

\vskip2ex

Two widely used surveys:

Andreoni, Erard, Feinstein JEL 1998

Slemrod and Yitzhaki Handbook of PE, 2002

\end{slide}

\begin{slide}
\begin{center} {\bf ALLINGHAM-SANDMO JPUBE'72 MODEL} \end{center}

Seminal in the theoretical tax evasion literature. Uses the Becker
crime model

Individual taxpayer problem:
$$\max_{\bar{w}} \:\: (1-p) \cdot u(w-\tau \cdot \bar{w}) + p \cdot u(w -\tau \cdot \bar{w} -
\tau (w-\bar{w})(1+\theta)),$$ \noindent where $w$ is true income,
$\bar{w}$ reported income, $\tau$ tax rate, $p$ audit probability,
$\theta$ fine factor, $u(.)$ concave.

Let $c^{No \:Audit}=w-\tau \cdot \bar{w}$ and $c^{Audit}=w -\tau
\cdot \bar{w} - \tau (w-\bar{w})(1+\theta)$

FOC in $\bar{w}$: $-\tau(1-p)u'(c^{No \:Audit})+p \theta \tau
u'(c^{Audit})=0 \Rightarrow$
$$\frac{u'(c^{Audit})}{u'(c^{No
\:Audit})} = \frac{1-p}{p \theta}$$

SOC $\Rightarrow$ $\tau^2 (1-p)u''(c^{No
\:Audit})+p\tau^2\theta^2u''(c^{Audit})<0$

\end{slide}

\begin{slide}
\begin{center} {\bf ALLINGHAM-SANDMO JPUBE'72 MODEL} \end{center}

{\bf Result:} Evasion $w-\bar{w}$ $\downarrow$ with $p$ and
$\theta$

Proof of $d\bar{w}/dp>0$: Differentiate FOC with respect to $p$
and $\bar{w}$:

$-dp \cdot \tau u'(c^{No \:Audit}) - d\bar{w} \cdot  \tau^2 (1-p)
u''(c^{No \:Audit})  = dp \cdot \theta \tau u'(c^{Audit}) +
d\bar{w} \cdot p \theta^2 \tau^2 u''(c^{Audit})$

$\Rightarrow$ $ d\bar{w} \cdot [- \tau^2 (1-p) u''(c^{No \:Audit})
- p \theta^2 \tau^2 u''(c^{Audit}) ] = dp \cdot [ \theta \tau
u'(c^{Audit}) + \tau u'(c^{No \:Audit}) ]$

Similar proof for $d\bar{w}/d\theta>0$

%In this model $d\bar{w}/d\tau>0$: $d\tau$ creates income effect
%$\bar{w} \uparrow$ [risk aversion] and substitution effect
%$\bar{w} \downarrow$ [pays to evade more]

Huge literature built from the {\bf A-S} model [including optimal
auditing rules]
\end{slide}



\begin{slide}
\begin{center} {\bf Why is tax evasion so low in OECD countries?} \end{center}

{\bf Key puzzle:} US has low audit rates ($p=.01$) and low fines
($\theta \simeq .2$). With reasonable risk aversion (say CRRA
$\gamma=1$), tax evasion should be much higher than observed
empirically

Two types of explanations for puzzle

1) {\bf Unwilling to Cheat:} Social norms and morality [people
dislike being dishonest and hence voluntarily pay taxes]

2) {\bf Unable to Cheat:} Probability of being caught is much
higher than observed audit rate because of {\bf 3rd party
reporting}:

Employers double report wages to govt (W2 forms), companies and
financial institutions double report capital income paid out to
govt (US 1099 forms)

\end{slide}

\begin{slide}
\begin{center} {\bf DETERMINANTS OF TAX EVASION} \end{center}

Large empirical literature studies tax evasion levels and the link
between tax evasion and (a) tax rates, (b) penalties, (c) audit
probabilities, (d) prior audit experiences, (e) socio-economic
characteristics

Early literature relies on observational [non-experimental]
data which creates serious identification and measurement issues:

(1) Evasion is difficult to measure

(2) Most independent variables [audits, penalties, etc.] are
endogenous responses to evasion and also difficult to measure

$\Rightarrow$ Requires to use experimental data or to find good
instruments: (a) IRS Tax Compliance Measurement Studies (TCMP),
(b) lab experiments, (c) field experiments
\end{slide}

\begin{slide}
\begin{center} {\bf TCMP: IMPACT OF THIRD PARTY REPORTING} \end{center}

IRS Tax Compliance Measurement Study (TCMP) is a thorough audit of
stratified sample of tax returns done periodically. TCMP shows that:

1) Tax Gap is about 15\%

2) Tax Gap concentrated among income items with no 3rd party
reporting (such as self-employment income)

$\bullet$ tax gap over 50\% when little 3rd party reporting
[consistent with Allingham-Sandmo]

$\bullet$ Tax Gap very small ($<5\%$) with 3rd party reporting

\end{slide}

\begin{slide}
\includepdf[pages={1,2}]{taxenforcement_attach.pdf}
\end{slide}

\begin{slide}
\begin{center} {\bf TCMP: IMPACT OF TAX WITHHOLDING} \end{center}

3) Tax Withholding further reduces tax gap: liquidity constraint
effect is most likely explanation: some taxpayers can never
pay the tax due unless it is withheld at source

$\Rightarrow$ wage income withholding is critical for enforcement
of broad based income tax and payroll taxes

Numbers from TCMP are rough estimates because audits cannot
uncover all evasion [IRS blows up uncovered evasion by factor 3-4]
$\Rightarrow$ Thorough audits detect evasion of only about 4\% of income

%TCMP cannot be used to study convincingly causal impact of audits
%or fines on evasion

\end{slide}


\begin{slide}
\begin{center} {\bf LAB EXPERIMENTS} \end{center}

Multi-period reporting games involving participants (mostly
students) who receive and report income, pay taxes, and face risks
of being audited and penalized

1) Lab experiments have consistently shown that penalties, audit
probabilities, and prior audits increase compliance (e.g., Alm,
Jackson, and McKee, 1992)

2) But when penalties and audit probabilities are set at realistic
levels, their deterrent effect is quite small [Alm, Jackson, and
McKee 1992] $\Rightarrow$  Laboratory experiments tends to predict
more evasion than we observe in practice

Issues: Lab environment is artificial, and therefore likely to
miss important aspects of the real-world reporting environment
[3rd party information and social norms]
\end{slide}


\begin{slide}
\begin{center} {\bf FIELD EXPERIMENTS} \end{center}

1) Blumenthal, Christian, Slemrod NTJ'01 study the effects of
normative appeals to comply: treatment group receives letter
encouraging compliance on normative grounds ``support valuable
services'' or ``join the compliant majority'', control group [no
letter]

$\Rightarrow$ No (statistically significant) effect of normative
appeals on compliance overall

2) Slemrod, Blumenthal, Christian JPubE'01 study the effects of
``threat-of-audit'' letters

$\Rightarrow$ Statistically significant effect on reported income
increase, especially among the self-employed [``high opportunity
group''] but very small sample size

\small
Recently: (a) Hallsworth et al. (2014) show that normative appeals help in collecting
overdue taxes [but small quantitatively], (b) Bott et al. 2014 for a randomized experiment in Norway on
foreign income [threat of audit more effective than normative appeal], (c) see survey Luttmer-Singhal '14
 

\end{slide}

\begin{slide}
\includepdf[pages={3}]{taxenforcement_attach.pdf}
\end{slide}

\begin{slide}
\includepdf[pages={4}]{taxenforcement_attach.pdf}
\end{slide}

\begin{slide}
\begin{center} {\bf TAX AUDIT EXPERIMENT FROM DENMARK} \end{center}

Kleven-Knudsen-Kreiner-Pedersen-Saez '11 analyze bigger Danish
income tax auditing experiment [stratified sample 40,000]

Overall detected evasion [no adjustment] is around 2.5\% but:

1) Evasion rate for self-reported items is almost 40\%

2) Evasion rate for third party reported items is only 0.3\%

3) Overall evasion rate is so low because 95\% of income is third
party reported in Denmark

Role of 3rd party reports [information structure] seem to trump
social factors and economic factors:

$Evade_i = \alpha + \beta Self\: Reported \: Income_i + \gamma
Social\: Factors_i + \varepsilon_i$
\end{slide}



\begin{slide}
\includepdf[pages={5,6}]{taxenforcement_attach.pdf}
\end{slide}

\begin{slide}
\includepdf[pages={15}]{taxenforcement_attach.pdf}
\end{slide}


\begin{slide}
\includepdf[pages={16}]{taxenforcement_attach.pdf}
\end{slide}

%
%\begin{slide}
%\includepdf[pages={14,15}]{pdf/TaxComplianceExperimentOct2009K.pdf}
%\end{slide}

\begin{slide}
\begin{center} {\bf TAX AUDIT EXPERIMENT FROM DENMARK} \end{center}
Kleven et al. '11 also provide experimental causal effects of:

1) Marginal tax rates: use bunching evidence before and after audit:
Most bunching not due to evasion but avoidance $\Rightarrow$ Effect of MTR on evasion is
modest

2) Prior-audit effects: compare next year outcomes of 100\% audit
group and a 0\% audit group [as audited tax filers may update
upward beliefs on $p$]

$\Rightarrow$ Find significant effects on reported income
increases, concentrated among self-reported items [nothing on 3rd
party income]: Extra tax collected through this indirect effect is about 50\% of
extra taxes collected due to base year audits

3) Threat-of-audit letters: Find significant effects on
self-reported income increases [as in Slemrod et al.] and letter prob matters

\end{slide}

\begin{slide}
\includepdf[pages={7,8,9,10,11}]{taxenforcement_attach.pdf}
\end{slide}


%\begin{slide}
%\includepdf[pages={19,20}]{pdf/TaxComplianceExperimentOct2009K.pdf}
%\end{slide}


%\begin{slide}
%\begin{center} {\bf ADDING THIRD PARTY REPORTING IN A-S MODEL: KLEVEN-KREINER-SAEZ '09} \end{center}
%
%Income $w=w_t+w_s$ where $w_t$ is third party reported (observed
%by govt at no cost) and $w_s$ is self-reported (as in standard
%Allingham-Sandmo model). Individual reports $\bar{w}_t$ and
%$\bar{w}_s$
%
%1) $\bar{w}_t=w_t$ because audit rate is 100\% for this income
%category
%
%2) Government audits $\bar{w}_s$ with probability $p<1$ (costly):
%$$\max_{\bar{w}_s} \:\: (1-p) u(w- \tau w_t - \tau \bar{w}_s) + p u(w-\tau w_t - \tau \bar{w}_s -
%\tau (w_s-\bar{w}_s)(1+\theta))$$ $$\Leftrightarrow
%\max_{\bar{w}=w_t+\bar{w}_s} \:\: (1-p) u(w - \tau \bar{w}) + p
%u(w - \tau \bar{w} - \tau (w-\bar{w})(1+\theta))$$
%
%$\Rightarrow$ {\bf 3rd Party Irrelevance:} If no constraints on
%$\bar{w}_s$, 3rd party reporting does not help enforcement
%
%Note: irrelevance result remains true if $p(\bar{w})$
%\end{slide}
%
%
%\begin{slide}
%\begin{center} {\bf BREAKING THE IRRELEVANCE RESULT} \end{center}
%
%Irrelevance result depends on 2 strong assumptions:
%
%(1) Self-reported losses are allowed
%
%(2) Audit rate does not depend on (sign of) $\bar{w}_s$
%
%More realistic models where irrelevance breaks down:
%
%(1) Disallow self-reported losses 
%
%(2) Audit rate $p$ depends (negatively) on $\bar{w}_s$
%
%$\Rightarrow$ 3rd party reporting helps government enforce taxes
%
%\end{slide}

\begin{slide}
\begin{center} {\bf EXPLAINING ACTUAL TAX POLICIES} \end{center}
Income $w=w_t+w_s$ where $w_t$ is third party reported (observed
by govt at no cost) and $w_s$ is self-reported (as in standard
Allingham-Sandmo model).

Incorporating 3rd party reporting solves
puzzles of the {\bf Allingham-Sandmo} model:

1) Evasion rates are high in $s$ sector (consistent with {\bf Allingham-Sandmo})
and low in $t$ sector

2) IRS sets audit rate $p$ higher when $\bar{w}_s<0$ (small business losses,
undocumented deductions, etc.) to protect $w_t$ base

3) $\bar{w}_s$ losses not allowed against $w_t$ (example: US
limits capital gain losses and passive business losses)

4) Use of schedular income taxes (tax separately various bases): Earliest
income taxes (1800-1900) are {\bf schedular}
\end{slide}


\begin{slide}
\begin{center} {\bf SIMPLER MODEL OF TAX EVASION} \end{center}
$$u=(1-p(\bar{w}))[w-\tau \bar{w}]+ p(\bar{w})[ w(1-\tau) - \theta
\tau (w-\bar{w})]$$

FOC $du/d\bar{w}=0$ $\Rightarrow$ $[p ( \bar{w}) - p' ( \bar{w})
(w-\bar{w})]( 1+\theta) =1$

Introduce the elasticity of the detection probability with respect
to undeclared income: $\varepsilon = -(w-\bar{w})
p'(\bar{w})/p(\bar{w})>0$
$$1=p ( \bar{w}) \cdot ( 1+\theta) \cdot (1+\varepsilon )$$ 
%Marginal benefit of evading \$1 extra = Marginal
%cost of evading \$1 extra

If $\varepsilon=0$, then always evade if $1>p \cdot (1+\theta)$

If $\varepsilon>0$, then evading more increases risk of being
caught on all infra-marginal evaded taxes $\Rightarrow$ Even with
$\theta=0$, full evasion is not always optimal

Shape of $p(\bar{w})$ depends crucially on 3rd party income
\end{slide}

\begin{slide}
\includepdf[pages={12}]{taxenforcement_attach.pdf}
\end{slide}


\begin{slide}
\begin{center} {\bf WHY DOES THIRD PARTY REPORTING WORK?} \end{center}
In theory, employer and employee could collude to evade taxes
$\Rightarrow$ third-party does not help (Yaniv 1992)

In practice, such collusion is fragile in modern companies because
of combination of:

1) Accounting and payroll records that are widely used within the
firm [records need to report true wages in order to be useful to run
a complex business]

2) A single employee can denounce collusion between employer and
employees. Likely to happen
in a large business [disgruntled employee, honest newly hired
employee, whistle blower seeking govt reward]

$\Rightarrow$ Taxes can be enforced even with low penalties and
low audit rates [Kleven-Kreiner-Saez, 2016]
\end{slide}


%\begin{slide}
%\begin{center} {\bf FORMAL MODEL OF 3RD PARTY} \end{center}
%
%1) Firm has $N$ employees where wages=marginal productivity
%$w=(w_1,..,w_N)$ (assume away profits).
%
%2) Firm and employees cooperatively report
%$\bar{w}=(\bar{w}_1,..,\bar{w}_N)$ to govt which applies constant
%tax rate $\tau$
%
%3) If firm uses {\bf accounting records} then $w, \bar{w}$ known
%within firm by all employees
%
%4) If $w \neq \bar{w}$, any employee can show {\bf accounting
%records} to govt and denounce cheating
%
%5) Govt cannot observe $w$ if all employees collude
%
%6) Govt applies fine at rate $\theta>0$ for evaded taxes
%
%\end{slide}
%
%
%\begin{slide}
%\begin{center} {\bf FORMAL MODEL OF 3RD PARTY} \end{center}
%
%Firm and all employees can {\bf collude} to report
%$\bar{w}=(0,..,0)$ and evade taxes entirely
%
%But collusive equilibrium is fragile as a single employee can
%reveal cheating. Can happen because of:
%
%1) Random Shocks: Work conflict, Moral Concerns, Mistake
%
%2) Whistle blowing reward: Govt offers fraction $\delta$ of unpaid
%taxes to whistle blowers
%
%$\Rightarrow$ Collusive equilibrium harder to sustain in large
%firms
%
%\end{slide}
%
%
%\begin{slide}
%\begin{center} {\bf FORMAL MODEL OF 3RD PARTY: RANDOM SHOCKS} \end{center}
%
%If $w \neq \bar{w}$, each employee denounces firm with probability
%$\varepsilon$ (iid) $\Rightarrow$ Firm successfully evades with
%prob. $(1-\varepsilon)^N$
%
%Firm/workers set $\bar{w}$ to maximize ex-ante expected surplus
%(assuming risk neutrality):
%$$S=\sum_n [w_n - \tau \cdot \bar{w}_n - (1-(1-\varepsilon)^N) \cdot
%\tau \cdot (1+\theta) \cdot (w_n-\bar{w}_n)]$$ $$\partial S /
%\partial \bar{w}_n = \tau \cdot
%[-1+(1+\theta)(1-(1-\varepsilon)^N)]$$ Firm/workers evade (fully)
%iff $(1-\varepsilon)^N > \theta/(1+\theta)$
%
%Large firms do not evade even for small $\varepsilon$ and $\theta$
%
%\end{slide}
%
%\begin{slide}
%\begin{center} {\bf FORMAL MODEL OF 3RD PARTY: WHISTLE BLOWER
%MODEL}\end{center}
%
%Govt offers reward fraction $\delta$ of uncovered taxes to whistle
%blowers ($\delta<\theta$).
%
%Audit probability is 0 if nobody whistle blows and 1 if anybody
%whistle blows. Whistle blowers share fraction $\delta$ of unpaid
%taxes $\tau \sum_{n'} (w_{n'} - \bar{w}_{n'})$
%
%If $w \neq \bar{w}$, nobody whistle blows iff
%
%$w_n - \tau \bar{w}_n \geq w_n - \tau \bar{w}_n - (1+\theta)\tau
%(w_n-\bar{w}_n) + \delta \tau \sum_{n'} (w_{n'} - \bar{w}_{n'})$
%
%iff $(1+\theta)(w_n-\bar{w}_n) \geq \delta \sum_{n'} (w_{n'} -
%\bar{w}_{n'})$ for all $n$ $\Rightarrow$ $1+\theta \geq N \delta$
%
%$\Rightarrow$ No collusive tax cheating is sustainable iff $\delta
%> (1+\theta)/N$ $\Rightarrow$ Large firms do not evade even with
%small $\delta$ and $\theta$
%\end{slide}

%\begin{slide}
%\begin{center}
%{\bf INCOME TAXATION IN DEVO COUNTRIES}
%\end{center}
%Progressive individual income taxes in devo countries are small and limited to a small
%fraction of upper income taxpayers (vast majority of the population are informal self-employed workers)
%
%Kleven and Waseem QJE'13 study income tax in Pakistan
%
%Tax creates notches because \textbf{average} tax rate jumps
%$\Rightarrow$ Bunching below the notch and gap in density just above the notch
%
%\textbf{Empirically:} Evidence of bunching (primarily among self-employed)
%but size of the response is quantitatively small
%
%Large fraction of taxpayers are unresponsive to notch likely  due to lack of information
%
%\end{slide}
%
%\begin{slide}
%\includepdf[pages={20,18-19,21}]{taxenforcement_attach.pdf}
%\end{slide}
%
%
%\begin{slide}
%\begin{center}
%{\bf Kleven and Waseem QJE'13 notch analysis}
%\end{center}
%With optimization frictions (lack of information, costs of adjustment),
%a fraction of individuals fail to respond to notch
%
%Kleven-Waseem use density above notch to measure the fraction
%of unresponsive individuals
%
%This allows them to back up the frictionless elasticity (i.e. the elasticity
%among responsive individuals)
%
%The frictionless elasticity is much higher than the reduced form elasticity
%but remains still relatively modest
%
%
%\end{slide}



\begin{slide}
\begin{center} {\bf HISTORY OF TAX COLLECTION} \end{center}
Interesting to understand why taxes develop the way they do
[Webber-Wildavsky '86 book, Ardant '71 book in French]

During most of history, governments were under the tax enforcement
constraint: they were collecting as much taxes as possible given
the economic / informational conditions

Many developing countries today still face such tax enforcement
constraints

Earliest taxes are tributes: conquerors / rulers realize that it
is more lucrative to raise periodic tributes than outright
stealing

\end{slide}

\begin{slide}
\begin{center}
{\bf Taxation as the Origin of States}
\end{center}
States first arise through warfare and conquest in productive areas (e.g. Nile Valley) to extract taxes (see Carneiro, 1970)

Modern test of this theory: Sanchez (2015) surveys Eastern Congo villages in war areas 

Bandits establish ``local states'' (=order and taxes) when village tax potential is high

(a) villages with coltan mineral have tax potential particularly when coltan price is high

(b) villages with gold mineral do not have tax potential (bc gold can be easily hidden)

Likelihood of taxation of coltan mining sites follows coltan price
\end{slide}

\begin{slide}
\includepdf[pages={27}]{taxenforcement_attach.pdf}
\end{slide}

\begin{slide}
\includepdf[pages={28}]{taxenforcement_attach.pdf}
\end{slide}


\begin{slide}
\begin{center} {\bf ARCHAIC TAXES} \end{center}

Governments try to extract revenue through rules without
destroying economic activity and without generating tax revolts

Colbert (17th century France) famous expression: ``plucking the
goose while minimizing hissing''

{\bf Direct taxes:} taxes on property, businesses, or people

{\bf Indirect taxes:} taxes on transactions and exchanges

Classification is no longer very meaningful: [estate tax is
direct, inheritance tax is indirect but economically equivalent]
\end{slide}





\begin{slide}
\begin{center} {\bf ARCHAIC DIRECT TAXES}\end{center}

{\bf Poll tax} (fixed amount per person). Cannot raise much
revenue as poor cannot pay much [people flee or rebel, serfdom is
a way to prevent fleeing behavioral response]. Later
differentiated by class (nobility, peasants, professions).

{\bf Land tax} (amount per lot), later differentiated by quality.
Cannot raise much unless carefully differentiated with expensive
land registry [otherwise marginal lands abandoned]

{\bf Product taxes} (such as tithe = fraction of gross
agricultural product): Tax requires monitoring production. Tax on
gross product can be overwhelming for marginal lands

$\Rightarrow$ Archaic direct taxes can hardly raise more than 5\%
of total product in primitive economies. Hard to collect in barter
economies. Only minimal govt can be supported.
\end{slide}

\begin{slide}
\begin{center} {\bf ARCHAIC INDIRECT TAXES} \end{center}

Indirect taxes require exchange economies %because based on
%movements of people and goods

{\bf Tolls} for use of roads, rivers, entering towns, crossing
borders, harbors, mountain pass. Initially based on people, later
based on goods transported [overused when no coordination across
jurisdictions]

{\bf Excise and Sales Taxes} on exchanged goods. In early
economies, only few goods are traded: salt, metal, alcohol
beverages. Fairs where exchanges are concentrated also allow
governments to impose sales taxes

{\bf Govt Monopoly} Some economic activities require use of heavy
equipment (grinding wheat, pressing grapes) $\Rightarrow$ Can be
controlled/monitored by govt

$\Rightarrow$ Archaic indirect taxes can raise substantial
additional revenue in jurisdictions with substantial trading
activity %[Italian towns, Holland towns]. Easier to collect as
%money is used for exchanges.
\end{slide}

\begin{slide}
\begin{center} {\bf MODERN TAXES} \end{center}
Modern taxes exploit {\bf accounting information} that is required
in large/complex business activities and {\bf withholding at
source}

Shift from differentiated capitation and presumptive taxes (on
businesses and individuals) toward modern income taxation

Shift from excise taxes toward general sales taxes and VAT

Modern taxes can collect 50\% of GDP without harming
growth

Modern taxes in rich countries today are threatened primarily by
(a) tax havens [enforcement difficult], (b) international tax
competition [requires international coordination], (c) marginally
the informal sector

\small IMF recommendations for poor countries to switch from archaic
tariffs to modern VAT reduced tax revenue bc VAT enforcement failed [Cage-Gadenne 13]
\end{slide}



\begin{slide}
\begin{center}
{\bf EMPIRICAL PATH OF GOVERNMENT GROWTH}
\end{center}
1) Govt size is small (typically $<$ 10\% of GDP) in Western
countries before industrialization (Flora '83). Use {\bf archaic}
taxes: [poll taxes, land-property taxes, product taxes, excise
taxes, tolls, tariffs]

2) Govt size increases sharply in all advanced economies during
20th century. Increase corresponds to the development of {\bf
modern} taxes enforced using business records [income taxes,
payroll taxes, value added taxes]

3) Govt growth has slowed or stopped in most advanced economies
over last 3 decades

This general historical pattern applies to almost all rich
countries although timing and final govt size varies across
countries
\end{slide}

\begin{slide}
\includepdf[pages={13,14}]{taxenforcement_attach.pdf}
\end{slide}





%\begin{slide}
%\begin{center}
%{\bf Explaining Pattern of Govt Growth with Tax Enforcement (Kleven-Kreiner-Saez '09)}
%\end{center}
%
%
%1) Early development: economy is rudimentary and business are
%small / do not need business records: government cannot use modern
%taxes and is constrained by limited fiscal capacity 
%$\Rightarrow$
%Government size is too small relative to citizens' preferences
%
%2) Middle development: businesses grow in size and start using
%records $\Rightarrow$ Government can start using modern taxes and
%grows (still constrained by fiscal capacity as too high taxes
%would make businesses go back to informality)
%
%3) Late development: economy is largely formal $\Rightarrow$
%Government no longer constrained, govt size optimal given citizens'
%preferences $\Rightarrow$ Government size (relative to
%GDP) is stable
%\end{slide}
%
\begin{slide}
\begin{center}
{\bf ALTERNATIVE THEORIES OF GOVT GROWTH}
\end{center}
1) Demand elasticity for public goods has income elasticity above
one [{\bf Wagner's law} $\sim 1900$] (can't explain stability since 1980)

2) Supply side: Stagnating productivity in the government sector
[Baumol's '67 {\bf Cost Disease Theory}] (can't explain stability since 1980)

3) {\bf Ratchet effect theory}: temporary shocks (e.g., wars)
raise government expenditures, which do not fall back after the
shock because of changed social norms [Peacock-Wiseman '61,
Besley-Persson '08] (can't explain Sweden and pre-20th century
wars)

4) {\bf Political economy} theories based on voting and
democratization, etc.

%5) {\bf Leviathan theory} Bureaucrats maximize govt size subject
%to fiscal capacity and political constitution constraints
\end{slide}


\begin{slide}
\begin{center}
{\bf VARIOUS SALES TAXES}
\end{center}
{\bf Turnover taxes} used to tax all sales: business to consumer
(B-C) and business to business (B-B):

Creates multiple layers of taxes along a production chain
$\Rightarrow$ Higher total tax when B-B-C than B-C

{\bf Retail Sales Tax} is imposed on B-C sales only [B-B exempt]:
difficult to distinguish B-B and B-C (shifting), strong evasion
incentive for B-C [sales tax does not work well with small retailers]

{\bf Value-Added-Tax (VAT)} taxes only value added [sales minus
purchases] in all transactions (B-B and B-C): equivalent to retail
sales economically but easier to enforce [automatic upstream enforcement]

VAT first introduced in France in 1950s, has spread to most countries
[US only rich country without VAT] yet little research

%VAT is huge quantitatively yet little VAT research (lack of data)

\end{slide}

%\begin{slide}
%\begin{center}
%{\bf NO EVASION: VAT $\Leftrightarrow$ RETAIL SALES TAX }
%\end{center}
%(1) Supplier $S$ produces material using only labor inputs and
%sells it for $s$, pays VAT $\tau \cdot s$
%
%(2) Manufacturer $M$ buys material for $s$ and sells product for
%$m$, pays VAT $\tau \cdot (m-s)$
%
%(3) Retailer $R$ buys product for $m$ and sells good to consumers
%for $r$, pays VAT $\tau \cdot (r-m)$
%
%Total VAT is $\tau \cdot r$
%
%Retail sales tax paid only by $R$: $\tau \cdot r$
%
%VAT $\Leftrightarrow$ Retail sales tax
%\end{slide}
%
%\begin{slide}
%\begin{center}
%{\bf INTRODUCING EVASION}
%\end{center}
%Government matches the purchases and sales VAT reports: need to be
%consistent: $\bar{s}$, $\bar{m}$, $\bar{r}$
%
%If $M$ and $R$ truthfully report $\bar{m}=m$, $\bar{r}=r$: if $S$
%decides to evade $\bar{s}<s$, $M$ has to pay $\tau \cdot
%(m-\bar{s})$, $M$ will only purchase at lower price $\Rightarrow$
%No gain for $S$ to evade
%
%Similarly, if $R$ truthfully reports $\bar{r}=r$, then $M$ (and
%hence $S$) cannot evade
%
%VAT compliance down the chain forces compliance upstream [even if
%upstream businesses are informal]
%
%If $R$ is big and uses business records (Walmart) then $R$ cannot
%misreport $\bar{r}$ $\Rightarrow$ VAT will work well [but retail
%sales tax would also work]
%
%\end{slide}
%
%\begin{slide}
%\begin{center}
%{\bf WHY VAT WORKS BETTER?}
%\end{center}
%
%If $R$ is small / informal, it can evade but needs to report at
%least $\bar{r}=\bar{m}$ [otherwise VAT credit would attract tax
%audit]
%
%If $M$ is small / informal and if $R$ evades and sets
%$\bar{r}=\bar{m}$, then $M$ can evade VAT by colluding with $R$:
%both $R$ and $M$ can decide to lower both $\bar{r}$ and $\bar{m}$
%equally
%
%... $S$ can also evade if $M$ and $R$ evade
%
%If all firms are small / informal, VAT enforcement is impossible
%
%If bottom firm $R$ is small / informal $\Rightarrow$ Retail sales
%tax breaks down entirely but VAT does not:
%
%If bottom firm $R$ is small / informal but $M$ is large / formal,
%VAT enforcement will work from $M$ and upstream
%
%\end{slide}

%\begin{slide}
%\begin{center}
%{\bf ENFORCEMENT OF VAT VS RETAIL SALES TAX}
%\end{center}
%
%1) Sales tax enforcement depends critically on retailers. Sales tax
%can be enforced with large retailers but much harder with small retailers
%
%2) VAT: when there is a large/formal firm in the production chain, then
%enforcement upstream takes place automatically. Imports often play this role
%as they are easy to observe and tax [Keen '07]
%
%3) VAT Issues: (a) VAT evasion easier with international
%transactions [carousel fraud], (b) VAT cannot tax easily financial
%services.
%\end{slide}

\begin{slide}
\begin{center}
{\bf POMERANZ AER'15 VAT EXPERIMENT}
\end{center}
Randomized experiment with 445,000 firms in Chile: sent threat of VAT audit letters to sub-sample of businesses

\textbf{Key Results:}

1) Significant effect of letters on VAT collection (+10\% over 12 months)

2) Smaller impact on
reported transactions that already have a paper trail
(intermediate sales) than on those which don't (final sales)

3) Effect of random audit announcement is transmitted up the VAT chain, increasing compliance
by firms' suppliers

\end{slide}

\begin{slide}
\includepdf[pages={22-26}]{taxenforcement_attach.pdf}
\end{slide}

\begin{slide}
\begin{center}
{\bf WEALTH IN TAX HEAVENS ZUCMAN QJE'13}
\end{center}
Official statistics substantially underestimate the net foreign asset positions of rich countries 
bc they do not capture most of the assets held by households in off-shore tax havens

Example: Wealthy US individual opens a Cayman Islands account and buys
mutual fund shares (composed of US corporate stock): Cayman Islands record a liability
but US do not record an asset (because this is not reported in the US) 

$\Rightarrow$ Total world liabilities are larger than world total assets

Zucman compiles all financial stats and estimates that around 8\% of the global financial wealth of households is held in tax
havens (three-quarters of which goes unrecorded = 6\%)

If top 1\% hold about 50\% of total financial wealth, then about 12\% of financial wealth of the rich is
hidden in tax heavens

\end{slide}

\begin{slide}
\begin{center}
{\bf CURBING OFF-SHORE TAX EVASION}
\end{center}
Off-shore tax evasion possible because of bank secrecy: US cannot get a list
of US individuals owning Swiss bank accounts from Switzerland

$\Rightarrow$ No 3rd party reporting makes tax enforcement very difficult

In principle, problem could be solved with exchange of information across
countries BUT need all countries to cooperate

Johannesen-Zucman AEJ-EP'14 analyze tax haven crackdown: 
G20 countries forced number of tax havens to sign bilateral treaties
on bank information sharing

Key result: Instead of repatriating funds, tax evaders
shifted deposits to havens not covered by treaty with home country.

\end{slide}


\begin{slide}
\begin{center}
{\bf CURBING OFF-SHORE TAX EVASION}
\end{center}

\textbf{FATCA'13 US regulations} try to impose info exchange for all entities dealing with US: 

If foreign bank B
does not provide list of all its US account holders, any financial transaction between B and US will carry
30\% tax withholding $\Rightarrow$ Interesting to see what it will do


\textbf{Long-term solution} will require:

a) Systematic registration of assets to ultimate owners [already exists within countries for domestic
tax enforcement]

b) Systematic information exchange between tax countries with no exceptions for tax heavens 

$\Rightarrow$ Could be enforced with tariffs threats on tax heavens [Zucman JEP'14 and book '15]

\end{slide}


%\begin{slide}
%\includepdf[pages={17}]{materials\zucman2013.pdf}
%\end{slide}


\begin{slide}
\begin{center}
{\bf REFERENCES}
\end{center}
{\small

\textbf{Allingham, M. and A. Sandmo ``Income tax evasion: a theoretical analysis'', Journal of Public Economics, Vol. 1, 1972, 323-338. \href{http://elsa.berkeley.edu/~saez/course/Allingham\&SandmoJPubE(1972).pdf} {(web)} }

Alm, J., B. Jackson and M. McKee ``Institutional Uncertainty and Taxpayer Compliance'', American Economic Review, Vol. 82, 1992, 1018-1026. \href{http://links.jstor.org/stable/pdfplus/2117358.pdf} {(web)}

\textbf{Andreoni, J., B. Erard and J. Feinstein ``Tax Compliance'', Journal of Economic Literature, Vol. 36, 1998, 818-60.  \href{http://links.jstor.org/stable/pdfplus/2565123.pdf} {(web)} }

Ardant, G.: Histoire de l'imp\^{o}t (Volumes 1 and 2), Paris: Fayard, 1971.

Besley, T., T. Persson ``The Incidence of Civil War: Theory and Evidence'', NBER Working Paper, 14585, 2008. \href{http://www.nber.org/papers/w14585.pdf} {(web)}

Blumenthal, M., C. Christian and J. Slemrod ``Do Normative Appeals
Affect Tax Compliance? Evidence from a Controlled Experiment in
Minnesota'', National Tax Journal, Vol. 54, 2001, 125-238.
\href{http://elsa.berkeley.edu/~saez/course/BlumenthaletalNTJ(2001).pdf}
{(web)}

Bott, Kristina, Alexander W. Cappelen Erik Sorensen, Bertil Tungodden (2014) ``You�ve got mail: A randomised field
experiment on tax evasion'', Norwegian School of Economics Working Paper
\href{http://elsa.berkeley.edu/~saez/course/bottetal14.pdf}
{(web)}

Cage, Julia and Lucie Gadenne ``The Fiscal Cost of Trade Liberalization'', Harvard Working Paper, 2012
\href{http://elsa.berkeley.edu/~saez/course/CageGadenne.pdf}
{(web)}

Carneiro, Robert (1970) ``A Theory of the Origin of the State,'' Science, 169(3947), 733--738. 
\href{http://elsa.berkeley.edu/~saez/course/carneiro70.pdf}
{(web)}

Cowell, F. Cheating the Government: The Economics of Evasion (MIT Press, Cambridge, 1990).

Dwenger, Nadja , Henrik Kleven, Imran Rasul, Johannes Rincke 2016. ``Extrinsic and Intrinsic Motivations for Tax Compliance: Evidence from a Field Experiment in Germany'', American Economic Journal: Economic Policy?
\href{http://elsa.berkeley.edu/~saez/course/dwenger-kleven-rasul-rincke_aejpol2015.pdf} {(web)}

Hallsworth, Michael, John A. List, Robert D. Metcalfe, and Ivo Vlaev (2014) , ``The Behavioralist As Tax Collector: Using Natural Field Experiments to Enhance Tax Compliance'', NBER Working Paper No. 20007
 \href{http://www.nber.org/papers/w20007.pdf} {(web)}

IRS, 2012 ``Tax Gap for Tax Year 2006: Overview''
\href{http://elsa.berkeley.edu/~saez/course/overview_tax_gap_2006.pdf} {(web)}

Johannesen, Niels  and Gabriel Zucman ``The End of Bank Secrecy? An Evaluation of the G20 Tax Haven Crackdown,''
American Economic Journal: Economic Policy, 6(1), 2014.
\href{http://elsa.berkeley.edu/~saez/course/johannesen-zucmanAEJ14.pdf} {(web)}

Keen, M. ``VAT, Tariffs, and Withholding: Border Taxes and Informality in Developing Countries'', IMF Working Paper WP/07/174. \href{http://elsa.berkeley.edu/~saez/course/Keen_IMF.pdf} {(web)}

\textbf{Kleven, H., M. Knudsen, C. Kreiner, S. Pedersen, and E. Saez ``Unwilling or Unable to Cheat? Evidence from a Randomized Tax Audit Experiment in Denmark'', Econometrica 79(3), 2011, 651-692. \href{http://elsa.berkeley.edu/~saez/kleven-knudsen-kreiner-pedersen-saezEMA11taxaudit.pdf} {(web)} }

Kleven, H. C. Kreiner, and E. Saez ``Why Can Modern Governments Tax So Much? An Agency Model of Firms as Fiscal Intermediaries'', Economica,  2016. \href{http://eml.berkeley.edu/~saez/kleven-kreiner-saezEMA15.pdf} {(web)}

Kopczuk, W. and C. Pop-Eleches ``Electronic filing, tax preparers, and participation in the earned income tax credit'', Journal of Public Economics, Vol. 91, 2007, 1351-1367. \href{http://elsa.berkeley.edu/~saez/course/Kopczuk\&Pop-Eleches_JPubE(2007).pdf} {(web)}

Luttmer, Erzo F. P.  and Monica Singhal (2014) ``Tax Morale'', Journal of Economic Perspectives 28(4), 149--168.
\href{http://elsa.berkeley.edu/~saez/course/taxmorale.pdf} {(web)}

Peacock, A. T. and Wiseman, J. (1961) The Growth of Government Expenditure in the United Kingdom, Princeton, NJ: Princeton University Press.

\textbf{Pomeranz, Dina. 2015. "No Taxation without Information: Deterrence and Self-Enforcement in the Value Added Tax." American Economic Review, 105(8): 2539-69.
\href{http://elsa.berkeley.edu/~saez/course/pomeranzAER15.pdf} {(web)}}

Sanchez de la Sierra, Raul (2015) ``On the Origin of States: Stationary Bandits and Taxation in Eastern Congo''
Working Paper. \href{http://elsa.berkeley.edu/~saez/course/sanchez15.pdf} {(web)}

Shaw, J., J. Slemrod, and J. Whiting ``Administration \& Compliance'', in IFS, The Mirrlees Review: Reforming the Tax System for the 21st Century, Oxford University Press, 2009. \href{http://elsa.berkeley.edu/~saez/course/Shawetal_IFS(2009).pdf} {(web)}

Slemrod, J. and S. Yitzhaki ``Tax Avoidance, Evasion and Administration'', in A. Auerbach and M. Feldstein (eds.), Handbook of Public Economics, Vol. 3 (Amsterdam: North-Holland, 2002), 1423-1470. \href{http://elsa.berkeley.edu/~saez/course/Slemrod,Yitzhaki PE Handbook chapter.pdf} {(web)}

Slemrod, J., M. Blumenthal and C. Christian ``Taxpayer response to an increased probability of audit: evidence from a controlled experiment in Minnesota'', Journal of Public Economics, Vol. 79, 2001, 455-483. \href{http://elsa.berkeley.edu/~saez/course/Slemrodetal_JPubE(2001).pdf} {(web)}

Webber, C., A. Wildavsky, A history of taxation and expenditure in the Western world. New York : Simon and Schuster, 1986.

Yaniv, G. ``Collaborated Employee-Employer Tax Evasion'', Public Finance/ Finances Publiques, Vol. 47, 1992, 312-321. \href{http://elsa.berkeley.edu/~saez/course/Yaniv(1992).pdf} {(web)}

Zucman, G. ``The Missing Wealth of Nations: Are Europe and the US Net
Debtors or Net Creditors'', Quarterly Journal of Economics, 2013, 1321-1364. \href{http://elsa.berkeley.edu/~saez/course/zucmanQJE13.pdf} {(web)}

Zucman, G. \emph{The Hidden Wealth of Nations}, September 2015, University of Chicago Press.
\href{http://gabriel-zucman.eu/hidden-wealth/} {(web)}

Zucman, G.  ``Taxing across Borders: Tracking
Personal Wealth and Corporate Profits'' Journal of Economic Perspectives, 28(4), 2014, 121-148.
\href{http://elsa.berkeley.edu/~saez/course/Zucman2014JEP.pdf} {(web)}


}

\end{slide}



\end{document}
