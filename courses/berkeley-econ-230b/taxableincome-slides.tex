\documentclass[landscape]{slides}

\usepackage[landscape]{geometry}

\usepackage{pdfpages}
\usepackage{hyperref}
\usepackage{amsmath}
\usepackage{epstopdf}

\def\mathbi#1{\textbf{\em #1}}

\topmargin=-1.8cm \textheight=17cm \oddsidemargin=0cm
\evensidemargin=0cm \textwidth=22cm

\author{Emmanuel Saez}

\date{UC Berkeley}

\title{230B: Public Economics \\
Taxable Income Elasticities} \onlyslides{1-300}

\newenvironment{outline}{\renewcommand{\itemsep}{}}

\begin{document}

\begin{slide}

\maketitle

\end{slide}

\begin{slide}
\begin{center}
{\bf TAXABLE INCOME ELASTICITIES}
\end{center}

Modern public finance literature focuses on taxable income
elasticities instead of hours/participation elasticities

Two main reasons:

1) What matters for policy is the total behavioral response
to tax rates (not only hours of work but also occupational choices,
avoidance, etc.)

2) Data availability: taxable income is precisely measured in tax
return data

Overview of this literature: Saez-Slemrod-Giertz JEL'12

\end{slide}

%\begin{slide}
%\begin{center}
%{\bf FELDSTEIN RESTAT'99}
%\end{center}
%Consider two sources of responses to tax rates:
%
%1) Labor supply: $u(c,z)$ model where $z$ is earnings and is equal
%to reported income $y$ with $c=y \cdot (1-\tau)+R$
%
%Individual chooses $y$ to maximize $u(y(1-\tau)+R,y)$
%
%2) Avoidance: $z$ earnings is fixed but reported income $y=z-d$
%where $d$ is non-taxable compensation (health benefits or perks):
%$u(c,d)$ with $c=(1-\tau) y+R$
%
%Individual chooses $y$ to maximize $u(y(1-\tau)+R,z-y)$ [$z$
%fixed]
%
%Models are formally identical and generate the same efficiency and
%optimal tax analysis
%\end{slide}

%\begin{slide}
%\begin{center}
%{\bf TAXABLE INCOME ELASTICITY}
%\end{center}
%Taxable income elasticity due to both real responses (labor supply) and
%avoidance responses (e.g., non-taxable fringe benefits compensation)
%
%Feldstein '99 shows that real vs. avoidance responses is irrelevant for
%deadweight burden and revenue raising considerations but this view is too narrow
%
%Real response vs. avoidance response is critical because:
%
%a) avoidance elasticity depends on tax law (loopholes, etc.) and hence can be
%reduced by base broadening
%
%b) avoidance responses often generate fiscal externalities
%
%{\bf Key policy question:} Is it possible to eliminate avoidance elasticity using base broadening, etc.? or would new avoidance schemes keep popping up?
%\end{slide}

\begin{slide}
\begin{center}
{\bf FEDERAL US INCOME TAX CHANGES}
\end{center}
Tax rates change frequently over time

Biggest tax rate changes have happened at the top:

Reagan I: ERTA'81: top rate $\downarrow$ 70\% to 50\% (1981-1982)

Reagan II: TRA'86: top rate $\downarrow$ 50\% to 28\% (1986-1988)

Clinton: OBRA'93: top rate $\uparrow$ 31\% to 39.6\% (1992-1993)

Bush: EGTRRA '01: top rate $\downarrow$ 39.6\% to 35\% (2001-2003)

Obama '13:  top rate $\uparrow$ 35\% to 39.6\%+3.8\% (2012-2013)

Trump '17: top rate $\downarrow$ 37\%+3.8\% (2017-2018)

Taxable Income = Ordinary Income + Realized Capital Gains -
Deductions $\Rightarrow$ Each component can respond to $MTR$s
\end{slide}

\begin{slide}
\includepdf[pages={1}]{taxableincome_attach.pdf}
\end{slide}

%\begin{slide}
%\includepdf[pages={3}]{taxableincome_attach.pdf}
%\end{slide}

\begin{slide}
\includepdf[pages={4}]{taxableincome_attach.pdf}
\end{slide}




\begin{slide}
\begin{center}
{\bf LONG-RUN EVIDENCE IN THE US}
\end{center}
Goal: evaluate whether top \textbf{pre-tax} incomes respond to changes in one minus
the marginal tax rate (=net-of-tax rate)

Focus is on pre-tax income before deductions and excluding realized
capital gains

Pioneered by Feenberg-Poterba TPE'93 for period 1951-1990

Piketty-Saez QJE'03 estimate top income shares since 1913 [IRS
tabulations for 1913-1959, IRS micro-files since 1960]

Saez TPE'04 proposes detailed analysis for 1960-2000 period using
TAXSIM calculator at NBER linked to IRS micro-files

Piketty-Saez-Stantcheva AEJ'14 look at 1913-2010 period for the US
\end{slide}

%\begin{slide}
%\includepdf[pages={5}]{taxableincome_attach.pdf}
%\end{slide}
%
%\begin{slide}
%\includepdf[pages={6}]{taxableincome_attach.pdf}
%\end{slide}

%\begin{slide}
%\includepdf[pages={7}]{taxableincome_attach.pdf}
%\end{slide}
%
%\begin{slide}
%\includepdf[pages={8}]{taxableincome_attach.pdf}
%\end{slide}

\begin{slide}
\includepdf[pages={53}]{taxableincome_attach.pdf}
\end{slide}




\begin{slide}
\begin{center}
{\bf INCOME SHARE BASED ELASTICITY ESTIMATION}
\end{center}
1) {\bf Tax Reform Episode:} Compare top \textbf{pre-tax} income shares at $t_0$
(before reform) and $t_1$ (after reform)
$$e=\frac{\log sh_{t_{1}}-\log sh_{t_{0}}}{\log (1-\tau
_{t_{1}})-\log (1-\tau _{t_{0}})}$$ where $sh_t$ is top income
share and $\tau_t$ is the average MTR for top group

Identification assumption: absent tax change, $sh_{t_0}=sh_{t_1}$

2) {\bf Full Time Series:} Run regression:
$$\log sh_{t}=\alpha+e\cdot \log (1-\tau_{t})+\varepsilon _{t}$$
and adding time controls to capture non-tax related top income
share trends

ID assumption: non-tax related changes in $sh_t$ $\perp \tau_t$
\end{slide}

\begin{slide}
\includepdf[pages={9}]{taxableincome_attach.pdf}
\end{slide}

%\begin{slide}
%\includepdf[pages={10}]{taxableincome_attach.pdf}
%\end{slide}

\begin{slide}
\begin{center}
{\bf LONG-RUN EVIDENCE IN THE US}
\end{center}
1) Clear correlation between top incomes and top income rates both
in several short-run tax reform episodes and in the long-run [but
hard to assess long-run tax causality]

2) Correlation largely absent below the top 1\% (such as the next
9\%)

3) Top income shares sometimes do not respond to large tax rate
cuts [e.g., Kennedy Tax Cuts of early 1960s]

2) and 3) suggest that context matters (such as opportunities to respond / avoid taxes matter),
response not due to a universal labor supply elasticity

%4) Top income shares can change for non-tax related
%reasons: (a) Great Depression 1928-1931 (MTR stable and top income
%shares $\downarrow$, (b) 1990s: MTR $\uparrow$ and top income
%shares $\uparrow$

\end{slide}

\begin{slide}
\begin{center}
{\bf SPECIFIC TAX REFORM STUDIES}
\end{center}
Literature initially developed by analyzing specific tax reforms (instead of
full time series)

Lindsey JpubE'87 analyzes ERTA'81 using {\bf repeated
cross-section} tax data and finds large elasticities

Feldstein JPE'95 uses {\bf panel} tax data to study TRA'86

Goolsbee JPE'00 uses {\bf executive compensation} data to study
OBRA'93

Gruber-Saez JpubE'02 uses 1979-1990 {\bf panel} tax data 

Saez TPE'17 uses income share to study 2013 top tax rate increase

%Romer-Romer '11 uses 1919-1939 tabulated data

Many other studies in the US and abroad (survey by Saez-Slemrod-Giertz JEL'12)
\end{slide}

%\begin{slide}
%\begin{center}
%{\bf FELDSTEIN JPE'95: METHODOLOGY}
%\end{center}
%Feldstein (1995)\ estimates the effect of TRA86 on taxable income
%for top earners using {\bf panel} tax data
%
%1) Constructs three income groups $M$ (Medium), $H$ (High), $HH$ (Highest) based on before
%reform income in 1985
%
%2) Looks at how incomes and MTRs evolve from 1985 to 1988 for
%individuals in each group using panel: forms DD estimates
%$$\hat{e} = \frac{ \Delta \log (z^H) - \Delta \log (z^M) }
%{\Delta \log (1-\tau^H) - \Delta \log (1-\tau^M)}$$ where $z^H$,
%$z^M$ and $\tau^H$, $\tau^M$ are income and MTRs of the $H$ and
%$M$ groups
%\end{slide}
%
%\begin{slide}
%\includepdf[pages={11,12}]{taxableincome_attach.pdf}
%\end{slide}
%
%\begin{slide}
%\begin{center}
%{\bf FELDSTEIN JPE'95: RESULTS}
%\end{center}
%
%{\bf Results:} Feldstein obtains very high elasticities (above 1)
%for top earners
%
%$\Rightarrow$ US was on the wrong side of the Laffer curve for the
%rich
%
%$\Rightarrow$ Laffer rate $\tau=1/(1+a \cdot e)=1/(1+2 \cdot 1)
%=33\%$ Cutting top tax rate from 50\% to 28\% raised revenue
%\end{slide}
%
%\begin{slide}
%\begin{center}
%{\bf FELDSTEIN JPE'95: ISSUES}
%\end{center}
%1) Non-tax related {\bf changes in inequality} [same criticism as
%top share analysis]: panel helps only if inequality changes due to
%arrival of new people
%
%2) Short-term vs. Long-term response [same criticism as top share
%analysis]
%
%3) {\bf Mean reversion:} rich people in year $t$ tend to revert to
%the mean in year $t+1$ $\Rightarrow$ Panel analysis introduces
%downward bias in $e$ [when $\tau \downarrow$ for rich]
%
%4) {\bf Very small sample} in panel data [57 tax filers in $HH$
%group] [Auten-Carroll RESTAT'99 use larger Treasury panel data and
%find smaller elasticity 0.65]
%
%In net, not clear panel data adds value relative to
%repeated-cross-section
%\end{slide}
%
%\begin{slide}
%\begin{center}
%{\bf FELDSTEIN JPE'95: ISSUES}
%\end{center}
%5) DD can give very biased results when elasticity differs across
%groups:
%
%Example: (a) $M$ group has $e^M=0$ so that $\Delta \log (z^M)=0$
%and that $H$ group has $e^H=e>0$ so that $\Delta \log (z^H) = e
%\Delta \log (1-\tau^H)$.
%
%Suppose that $\Delta \log (1-\tau^M)=0.5 \cdot \Delta \log
%(1-\tau^H)$.
%
%Then, the estimated elasticity
%$\hat{e}^{DD}= e \Delta \log
%(1-\tau^H) /[\Delta \log (1-\tau^H)- \Delta \log (1-\tau^M)]= 2 e$
%
%In Feldstein JPE'95: Simple Difference $\Delta \log (z)/\Delta
%\log (1-\tau)$ uniformly smaller than DD
%
%$\Rightarrow$ Better to focus on a single group as in top share
%analysis than on the comparison with lower income group control
%\end{slide}

\begin{slide}
\begin{center}
{\bf GRUBER AND SAEZ JPUBE'02 (skip)}
\end{center}
%Generalization of Feldstein JPE'95 using IV regression analysis

Use panel data from 1979-1990 on all tax changes available rather
than a single reform

{\bf Model:} $z_{it}=z_{it}^0 \cdot (1-\tau_{it})^e$ where
$z^0_{it}$ is potential income (if MTR=0), $e$ is elasticity
$$\log \left ( \frac{z_{i t+3}}{z_{it}} \right ) = \alpha + e \cdot \log \left ( \frac{1-\tau_{i t+3}}{1-\tau_{i
t}} \right ) + \varepsilon_{it}$$ $\tau_{i t+3}$ and $\varepsilon_{it}$ are
correlated [because $\tau_{i t+3} = T_{t+3}'(z_{i t+3})]$

{\bf Instrument:} predicted change in MTR assuming income stays
constant: $\log [(1-\tau^p_{i t+3})/(1-\tau_{i t})]$ where $\tau^p_{i
t+3}=T'_{t+3}(z_{it})$

Isolates changes in tax law ($T_{t}(.)$) as the only source of
variation in tax rates
\end{slide}

\begin{slide}
\includepdf[pages={13}]{taxableincome_attach.pdf}
\end{slide}

\begin{slide}
\begin{center}
{\bf GRUBER AND SAEZ JPUBE'02 (skip)}
\end{center}
Find an elasticity of roughly 0.3-0.4 BUT results are very fragile
[Saez-Slemrod-Giertz JEL'12]

1) Sensitive to exclusion of low incomes 

2) Sensitive to controls for mean reversion

3) Subsequent studies find smaller elasticities using data from
other countries [Kleven-Schultz AEJ-EP'14 for Denmark]

4) Bundles together small tax changes and large tax changes: if
individuals respond only to large changes in short-medium run,
then estimated elasticity is too low [Chetty et al. QJE'11]
\end{slide}

\begin{slide}
\begin{center}
{\bf KLEVEN AND SCHULTZ AEJ-EP'14}
\end{center}
\textbf{Key Advantages:} 

a) Use full population of tax returns in Denmark since 1980 
(large sample size, panel structure, many demographic variables, stable inequality)

b) A number of reforms changing tax rates differentially across three income
brackets and across tax bases (capital income taxed separately from labor income)

c) Show compelling visual DD-evidence of tax responses around the 1986 large reform:

Define treatment and control group in year 1986 (pre-reform), follow the same group in years before and years after
the reform (panel analysis)

\end{slide}

\begin{slide}
\includepdf[pages={31-32}]{taxableincome_attach.pdf}
\end{slide}

\begin{slide}
\begin{center}
{\bf KLEVEN AND SCHULTZ AEJ-EP'14}
\end{center}

\textbf{Key Findings:} 

a) Small labor income elasticity (.1)

b) bigger capital income elasticities (.2-.3)

c) bigger elasticities for large reforms

d) modest income shifting between labor and capital in Denmark
(likely because top rates on labor and capital are carefully aligned)

$\Rightarrow$ Danish tax system
optimized to have broad base and few avoidance opportunities

\end{slide}


%\begin{slide}
%\begin{center}
%{\bf RELATED: NON-LINEAR ELECTRICITY PRICING}
%\end{center}
%Nonlinear pricing used in other settings (electricity, utility, cell phones consumption, etc.)
%
%Ito AER'14 is a great example using household electricity consumption to estimate consumer
%responses:
%
%1) Electricity is priced nonlinearly based on monthly consumption
%
%2) Ito compares people at the border of two electricity areas in Southern California:
%comparable groups facing different schedules and changes over time
%
%3) Finds very compelling evidence that consumers respond to average price and not marginal price
%
%To date, no such compelling design has been found for income taxation
%
%\end{slide}
%
%
%\begin{slide}
%\includepdf[pages={24-30}]{taxableincome_attach.pdf}
%\end{slide}


%\begin{slide}
%\begin{center}
%{\bf FISCAL EXTERNALITIES}
%\end{center}
%Feldstein RESTAT'99: nature of behavioral response (labor supply,
%avoidance, etc.) does not matter AS LONG AS the behavioral
%response does not generate a {\bf fiscal externality}
%
%A {\bf Fiscal externality} is a change in tax revenue that occurs
%in any tax base $z^B$ other than $z$ due to the behavioral
%response to the tax change in the initial base $z$
%
%(1) $z^B$ can be a different tax base in the same time period
%(such as corporate income tax base) $\Rightarrow$ {\bf Income
%shifting}
%
%(2) $z^B$ can be the same tax base in a different time period
%(such as future income) $\Rightarrow$ {\bf Inter-temporal
%Substitution}
%
%Efficiency and optimal tax analysis depend on effect on {\bf
%total} tax revenue
%\end{slide}

\begin{slide}
\begin{center}
{\bf FISCAL EXTERNALITIES}
\end{center}

Tax changes due to tax avoidance often generate \textbf{fiscal externalities}

A {\bf Fiscal externality} is a change in tax revenue that occurs
in any tax base $z^B$ other than $z$ due to the behavioral
response to the tax change in the initial base $z$

(1) $z^B$ can be a different tax base in the same time period
(such as corporate income tax base) $\Rightarrow$ {\bf Income
shifting}

(2) $z^B$ can be the same tax base in a different time period
(such as future income) $\Rightarrow$ {\bf Inter-temporal
Substitution}

Efficiency and optimal tax analysis depend on effect on {\bf
total} tax revenue so critical to identify fiscal externalities
\end{slide}




\begin{slide}
\begin{center}
{\bf Inter-Temporal Substitution: Realized Capital Gains}
\end{center}
Realized capital gains occur when individual sells asset at a
higher price than buying price

Individuals have flexibility in the timing of asset sales and
capital gains realizations

TRA'86 lowered the top tax rate on ordinary income from 50\% to
28\% but increased the top tax rate on realized capital gains from
20\% to 28\%

2013: tax rate on KG increased from 15\% to 20\%+3.8\% (Saez TPE'17 proposes
simple analysis)

$\Rightarrow$ Surge in capital gains realizations in 1986 and 2012 [and
depressed capital gains in 1987 and 2013] 

\normalsize
$\Rightarrow$ Short-term elasticity is very large but long-term
elasticity is certainly much smaller


\end{slide}

\begin{slide}
\includepdf[pages={48}, scale=.95]{taxableincome_attach.pdf}
\end{slide}

\begin{slide}
\includepdf[pages={49}]{taxableincome_attach.pdf}
\end{slide}

%\begin{slide}
%\begin{center}
%{\bf INTER-TEMPORAL SUBSTITUTION: STOCK-OPTIONS}
%\end{center}
%Goolsbee JPE'00 hypothesizes that top earners' ability to retime
%income drives much of observed responses 
%[Frisch elasticity instead of compensated elasticity]
%
%Fixed effects regression specification:
%$$TLI_{it}= e_{1}\log (1-MTR_{it})+ e_{2}\log
%(1-MTR_{i t+1})+ \alpha_i + \beta_t$$ Short-run elasticity is
%$e_1$
%
%$e_2<0$ if future MTR increase shifts income to present
%
%Long run elasticity is $e_{1}+e_{2}$
%
%Uses ExecuComp panel data to study effects of the 1993 Clinton top
%tax rate $\uparrow$ [from 31\% in 1992 to 39.6\% in 1993 announced
%in late 1992] on executive pay
%\end{slide}

\begin{slide}
\begin{center}
{\bf INTER-TEMPORAL SUBSTITUTION: STOCK-OPTIONS}
\end{center}
Goolsbee JPE'00 analyzes CEO pay around the 1993 Clinton top tax rate increase
$\uparrow$ [from 31\% in 1992 to 39.6\% in 1993 announced
in late 1992] on executive pay

Finds a strong re-timing response through stock-option exercise (executive can choose
the timing of their stock-option exercises)

$\Rightarrow$ Large short-term response due to re-timing, small long-term response

Some response but smaller around the 2013 tax increase

\end{slide}


\begin{slide}
\begin{center}
{\bf STOCK OPTIONS}
\end{center}
Major form of compensation of US top executives. Theoretical goal
is to motivate executives to increase the value of the company
(stock price $P(t)$)

Stock-options granted at date $t_0$ allow executives to buy $N$
company shares at price $P(t_0)$ on or after $t_1$ (in general
$t_1-t_0 \simeq 3-5$ years = vesting period)

Executive exercises option at (chosen) time $t_2 \geq t_1$: pays $N
\cdot P(t_0)$ to get shares valued $N \cdot P(t_2)$. Exercise
profit $N \cdot [P(t_2)-P(t_0)]$ (taxed as wage income in
the US)

After $t_2$, executive owns $N$ shares, eventually sold at time
$t_3 \geq t_2$: realized capital gain $N \cdot [P(t_3)-P(t_2)]$ (taxed
as KG)
\end{slide}

\begin{slide}
\includepdf[pages={16}]{taxableincome_attach.pdf}
\end{slide}

%\begin{slide}
%\includepdf[pages={17}]{taxableincome_attach.pdf}
%\end{slide}
%
%\begin{slide}
%\begin{center}
%{\bf GOOLSBEE JPE'00: INTER-TEMPORAL SUBSTITUTION}
%\end{center}
%Executives had a surge in income in 1992 (when reform was
%announced) relative to 1991 followed by a sharp drop in 1993
%
%1) Simple DD estimate for '92 vs '93 would find a large effect
%here, but it would be picking up pure re-timing
%
%2) Concludes that long run effect $e_1+e_2$ is much smaller than
%substitution effect $e_1$ [long-run elasticity is the relevant
%parameter for policy]
%
%3) Effects driven almost entirely by re-timing exercise of
%stock-options [executives knew tax rate would $\uparrow$ in '93
%when Clinton elected in Nov. '92 $\Rightarrow$ Exercise stock
%options]
%\end{slide}

%\begin{slide}
%\begin{center}
%{\bf INCOME SHIFTING: CORPORATE AND INDIVIDUAL TAX BASE}
%\end{center}
%Businesses can be organized as {\bf corporations} or {\bf
%unincorporated businesses} [also called {\bf pass-through}
%entities]
%
%Corporate profits are first taxed by corporate tax [tax rate
%$\tau_c$ ]
%
%Net-of-tax profits are taxed again when finally distributed to
%shareholders. 2 distribution options:
%
%a) dividends [tax rate $\tau_d$ ]
%
%b) retained profits increase stock price: shareholders realize
%capital gains when finally selling the stock [tax rate
%$\tau_{cg}$]
%
%For {\bf unincorporated businesses}: sole proprietorships,
%partnerships (including LLCs), S-corporations profits are taxed directly and
%solely as individual income (rate $\tau_i$)
%\end{slide}


\begin{slide}
\begin{center}
{\bf Income Shifting: Corporate and Individual Tax Base}
\end{center}
Businesses can be organized as {\bf corporations} or {\bf
unincorporated businesses} [also called {\bf pass-through}
entities]

Corporate profits first taxed by corporate tax [rate
$\tau_c = 21\%$]

Net-of-tax profits are taxed again at rate $\tau_{\text{distrib}}$  when finally distributed to
shareholders. Two distribution options:

a) dividends [tax rate $\tau_d = 20\%$ today]

b) retained profits increase stock price: shareholders realize
capital gains when finally selling the stock [tax rate
$\tau_{cg}=20\%$]

\vspace{-0.5cm}

But distributions can be deferred so that $\tau_{\text{distrib}}<<\tau_d, \tau_{cg}$

For {\bf unincorporated businesses} (sole proprietorships,
partnerships, S-corporations) profits are taxed directly and
solely as individual income (tax rate $\tau_i = 37\%$ top MTR or even $30\%$ with 20\% business profit deduction since 2018)
\end{slide}


\begin{slide}
\begin{center}
{\bf CORPORATE AND INDIVIDUAL TAX BASE}
\end{center}
Corporate form best if $(1-\tau_c) \cdot (1-\tau_{\text{distrib}})>1-\tau_i$

US fed taxes in 2018: $\tau_c=21\%$, $\tau_{cg}=\tau_d=20\%$,
(but $\tau_{\text{distrib}}<<20\%$ if distribution deferred), 
$\tau_i=37\%$ or $30\%$ 

After 2018 Trump change: corporate form is best, especially if 
wealthy business owner can defer distribution

Pre 2018, $\tau_c=35\%$ and $\tau_i=39.6\% \Rightarrow$ individual form better

$\Rightarrow$ wealthy people likely to incorporate their businesses in 2018+

\small
Before TRA'86 (and especially before ERTA'81), top individual rate
$\tau_i$ was much higher so corporate form was best

Shifts from corporate to individual base increases business
profits at the expense of dividends and realized capital gains

Large part of TRA'86 response is due to such shifting
\end{slide}

\begin{slide}
\includepdf[pages={18, 50}]{taxableincome_attach.pdf}
\end{slide}

%\begin{slide}
%\begin{center}
%{\bf INCOME SHIFTING: US EXECUTIVE COMPENSATION AND TAX RATES}
%\end{center}
%1) Executive compensation (relative to average pay) stagnated from
%1936-1970 and surged in the US since 1970 [Frydman-Saks '08]
%
%2) Top income rates on earnings high till 1970 and lower
%afterwards: tempting to attribute surge in executive compensation
%to lower tax rates
%
%3) BUT stock-options from 1950 to 1975 used to be ``Qualified
%stock options'' taxed as realized capital gains only when stock
%sold [no tax when exercised] but not deductible for corporate tax
%
%$\Rightarrow$ Tax rate on stock-option compensation was therefore
%very low (25\%) as realized capital gains
%
%Yet stock-option compensation was only used in moderation
%
%$\Rightarrow$ Frydman-Molloy '09 find no overall elasticity of
%executive compensation with respect to MTRs when taking this into
%account
%\end{slide}
%
%\begin{slide}
%\includepdf[pages={38}]{pdf/frydman-saks08executivepay.pdf}
%\end{slide}
%
%\begin{slide}
%\includepdf[pages={40}]{pdf/frydman-molloy09tax-executives.pdf}
%\end{slide}
%
%\begin{slide}
%\includepdf[pages={40}]{pdf/frydman-molloy09tax-executivesb.pdf}
%\end{slide}

%\begin{slide}
%\begin{center}
%{\bf Bottom Line On Behavioral Responses To Taxes}
%\end{center}
%1) Clear evidence of strong responses to tax changes due to
%re-timing or income shifting
%
%2) Heterogeneity in tax responses due to heterogeneity in shifting
%opportunities [e.g., Kennedy tax cuts of '61 vs. TRA'86]
%
%3) Top income shares can change drastically without changes in tax
%rates [Great Depression, 1993-2000]
%
%4) Difficult to know from single country time series the role
%played by top tax rate cuts in the surge of top incomes
%$\Rightarrow$ International evidence can cast further useful
%evidence
%\end{slide}

%\begin{slide}
%\begin{center}
%{\bf INTERNATIONAL EVIDENCE ON TOP INCOMES AND TAX RATES}
%\end{center}
%Atkinson-Piketty-Saez JEL'10 summarize recent effort to construct
%top income share series [and MTR series in some cases]
%
%Two empirical regularities from those top income studies:
%
%(1) 1900-1950: Most Western countries experience substantial drop
%in top income shares due to fall in top capital incomes
%
%possibly long run consequence of high tax rates which reduce
%ability to accumulate wealth $W_t$
%
%$W_{t+1}=W_t \cdot (1+r_t(1-\tau_t)) + Savings_t$ $\Rightarrow$
%High $\tau$ makes accumulating wealth $W_t$ harder
%\end{slide}
%
%\begin{slide}
%\includepdf[pages={19}]{taxableincome_attach.pdf}
%\end{slide}
%
%\begin{slide}
%\includepdf[pages={20}]{taxableincome_attach.pdf}
%\end{slide}

\begin{slide}
\begin{center}
{\bf TOP RATES AND TOP INCOMES INTERNATIONAL EVIDENCE}
\end{center}

1) Use pre-tax top 1\% income share data from 18 OECD countries since 1960 using
the {\bf World Inequality Database}

2) Compute top (statutory) individual income tax rates using OECD data [including both
central and local income taxes].

%Those tax rates do not include payroll taxes,
%corporate taxes, or VAT and Sales taxes

Plot top 1\% pre-tax income share against top MTR in 1960-4,  in 2005-9, and 1960-4 vs. 2005-9

%OLS basic regressions:
%$$\log (Top\:\: 1\% \:\: Share) = \alpha + e \cdot \log(1-MTR) + \varepsilon$$
%$$\Delta \log (Top\:\: 1\% \:\: Share) = \alpha + e \cdot  \Delta \log(1-MTR) + \varepsilon$$
\end{slide}

\begin{slide}
\includepdf[pages={34-36}]{taxableincome_attach.pdf}
\end{slide}

\begin{slide}
\includepdf[pages={37}]{taxableincome_attach.pdf}
\end{slide}


\begin{slide}
\begin{center}
{\bf ECONOMIC EFFECTS OF TAXING THE TOP 1\%}
\end{center}
Strong empirical evidence that \textbf{pre-tax} top incomes are affected by top tax rates

\textbf{3} potential scenarios with very different policy consequences

{\bf 1) Supply-Side:} Top earners work less and earn less when top tax rate increases
$\Rightarrow$ Top tax rates should not be too high

{\bf 2) Tax Avoidance/Evasion:} Top earners avoid/evade more when top tax rate increases

$\Rightarrow$ a) Eliminate loopholes, b) Then increase top tax rates

{\bf 3) Rent-seeking:} Top earners extract more pay (at the expense of the 99\%)
when top tax rates are low 
$\Rightarrow$ High top tax rates are desirable

\end{slide}


%\begin{slide}
%\begin{center}
%{\bf ANATOMY OF BEHAVIORAL RESPONSES}
%\end{center}
%{\bf 1) Avoidance:} Is the surge in US top income shares explained by reduced tax avoidance/evasion
%since 1970s instead of change in real income?
%
%\textbf{Test:} Under avoidance scenario, narrower measures of taxable income should
%be much more responsive to marginal tax rates than broader measures including tax preferred
%income items
%
%First pass is to compare income excluding and including tax preferred realized capital gains
%$\Rightarrow$ Does not support tax avoidance scenario
%
%{\bf 2) Rent-Seeking:} Has top 1\% income share surge come at the expense of the 99\%?
%
%\textbf{Test:} First pass is to look at correlation between economic growth and top tax rate cuts
%$\Rightarrow$ No correlation supports trickle-up (more work needed)
%
%\end{slide}

\begin{slide}
\begin{center}
{\bf Real changes vs. tax avoidance?}
\end{center}
Long-term Correlation between \textbf{pre-tax} top reported incomes and top tax rates

If due solely to tax avoidance, true top income shares were high in the 1950s-1970s but
top earners could lower their taxable income (by retaining earnings in businesses and benefit
from lower tax rate on capital gains)

But top income share including K gains follows the same U-shape (Piketty, Saez, Stantcheva '14)

%2) fully comprehensive national income definition
%from Distributional National Accounts (Piketty, Saez, Zucman '18)

Piketty, Saez, Zucman QJE'18: comprehensive national income estimates are also U-shaped over the century

$\Rightarrow$ Long-run evolution of inequality is not an artifact of tax avoidance or evasion

\end{slide}


\begin{slide}
\includepdf[pages={54}, scale=.95]{taxableincome_attach.pdf}
\end{slide}

\begin{slide}
\includepdf[pages={58}, scale=.95]{taxableincome_attach.pdf}
\end{slide}

\begin{slide}
\begin{center}
{\bf Real changes vs. tax Avoidance? Charitable giving}
\end{center}
Test using charitable giving behavior of top income earners (Saez TPE '17)

Because charitable is tax deductible, incentives to give are stronger
when tax rates are higher

Under the tax avoidance scenario, reported incomes and reported charitable
giving should move in opposite directions

Empirically, charitable giving of top income earners has grown in close tandem with
top incomes

$\Rightarrow$ Incomes at the top have grown for real 

%Empirical correlation is very similar ruling out the pure tax avoidance scenario

\end{slide}



\begin{slide}
\includepdf[pages={51,52}, scale=.95]{taxableincome_attach.pdf}
\end{slide}

\begin{slide}
\begin{center}
{\bf Supply-Side or Rent-Seeking? (Piketty-Saez-Stantcheva)}
\end{center}
Correlation between \textbf{pre-tax} top incomes and top tax rates

If rent-seeking: growth in top 1\% incomes should come at the expense of
bottom 99\% (and conversely)

Two macro-preliminary tests:

1) In the US, top 1\% incomes grow slowly from 1933 to 1975 and fast afterwards.
Bottom 99\% incomes grow fast from 1933 to 1975 and slowly afterwards
$\Rightarrow$ Consistent with rent-seeking effects

2) Look at cross-country correlation between economic growth and top tax rate cuts
$\Rightarrow$ No correlation supports trickle-up 

One micro-test using CEO pay data

\end{slide}

\begin{slide}
\includepdf[pages={55}]{taxableincome_attach.pdf}
\end{slide}


%\begin{slide}
%\includepdf[pages={38-39}]{taxableincome_attach.pdf}
%\end{slide}
%
%
%\begin{slide}
%\includepdf[pages={40}]{taxableincome_attach.pdf}
%\end{slide}

\begin{slide}
\begin{center}
{\bf INTERNATIONAL CEO PAY EVIDENCE}
\end{center}
Recent micro-data for 2006 gathered by Fernandes, Ferreira, Matos, Murphy RFS'12. 

1) CEO pay across countries strongly negatively correlated with top tax rates

2) Correlation remains as strong even when controlling for firms' characteristics
and performance 

$\Rightarrow$ Consistent with bargaining effects

\end{slide}

\begin{slide}
\includepdf[pages={41-43}]{taxableincome_attach.pdf}
\end{slide}


\begin{slide}
\begin{center}
{\bf INTERNATIONAL MIGRATION}
\end{center}
Public debate concern that top skilled individuals move to low tax countries (e.g., in EU context)
or low tax states (within US Federation)

Migration concern bigger in public debate than supply-side concern within a country 

%Relatively little work on tax induced international migration of top skilled workers

Interesting variation due to proliferation of special low tax schemes for highly paid foreigners
in Europe

\small

Kleven-Landais-Saez AER'13 look at \textbf{football players} in Europe (highly mobile group, many tax
reforms) $\Rightarrow$ Find significant migration responses to taxes after European football market
was de-regulated in '95

Akcigit-Baslandze-Stantcheva AER'16 look at \textbf{innovators} (using patent data) mobility and find significant tax
effects for top innovators

\normalsize

Various US states studies: Moretti-Wilson AER17 , 2019, Rauh-Shyu '19 (huge effects), Young et al. '16 (modest effects)



\end{slide}


\begin{slide}
\begin{center}
{\bf KLEVEN-LANDAIS-SAEZ-SCHULTZ QJE'14}
\end{center}
Exploit the 1991 Danish tax scheme:  immigrants with high earnings ($ \geq 103,000$ Euros/year)
taxed at flat 25\% rate (instead of regular progressive tax with top 59\% rate) for 3 years

Use population wide Danish tax data and DD strategy: compare immigrants above eligibility
earnings threshold (treatment) to immigrants below threshold (control) 

\textbf{Key Finding:} Scheme doubles the number of highly paid
foreigners in Denmark relative to controls

$\Rightarrow$ Elasticity of migration with respect to the net-of-tax rate above one
(much larger than the within country elasticity of earnings)

$\Rightarrow$ Tax coordination will be key to
preserve progressive taxation in the EU (but tax competition hard-coded in EU treaties)

\end{slide}

\begin{slide}
\includepdf[pages={33}]{taxableincome_attach.pdf}
\end{slide}



%\begin{slide}
%\begin{center}
%{\bf RESPONSE OF TAX EXPENDITURES TO MTRs}
%\end{center}
%Taxable Income = Ordinary Income + Realized Capital Gains -
%Deductions
%
%Deductions include mortgage interest payments, charitable
%contributions, state and local taxes
%
%Deductions could also respond to MTRs: MTRs $\uparrow$
%$\Rightarrow$ Deductions more attractive
%
%This response is captured in taxable income response but not
%ordinary income response
%
%Harder to construct long time series of taxable income because
%definition changes overtime
%
%Large literature has analyzed response of charitable contributions
%to tax rates [Andreoni Handbook Chapter, Fack-Landais '09]
%\end{slide}
%
%\begin{slide}
%\includepdf[pages={23}]{taxableincome_attach.pdf}
%\end{slide}




\begin{slide}
\begin{center}
{\bf REFERENCES}
\end{center}
{\small

Akcigit, Ufuk , Salom\'e Baslandze, and Stefanie Stantcheva. ``Taxation and the International Mobility of Inventors'',
American Economic Review 106 (10), 2016, 2930--2981 \href{http://elsa.berkeley.edu/~saez/course/Akcigit_Baslandze_Stantcheva_Taxation_Superstars.pdf} {(web)}

Akcigit, Ufuk, John Grigsby, Tom Nicholas, and Stefanie Stantcheva. 2018. ``Taxation and Innovation in the 20th Century.'' National Bureau of Economic Research No. 24982.
\href{https://www.nber.org/papers/w24982.pdf} {(web)}

Alvaredo, F., A. Atkinson, T. Piketty, E. Saez, G. Zucman \emph{The World Wealth and Income Database},
\href{http://www.wid.world/} {(web)}

Andreoni, J. ``Philanthropy'', In Serge-Christophe Kolm and Jean Mercier Ythier, Handbook on the Economics of Giving, Reciprocity and Altruism, Volume 2, Applications, 2006, 1201-1269 \href{http://elsa.berkeley.edu/~saez/course/Andereoni_Handbook chapter.pdf} {(web)}

Atkinson, A., T. Piketty and E. Saez ``Top Incomes in the Long Run of History'', Journal of Economic Literature, 49(1), 2011, 3-71. \href{http://elsa.berkeley.edu/~saez/atkinson-piketty-saezJEL10.pdf} {(web)}

Auten,G., R. Carroll ``The Effect of Income Taxes on Household Income'', Review of Economics and Statistics, Vol. 81, 1999, 681-693. \href{http://www.jstor.org/stable/pdfplus/2646717.pdf} {(web)}

Break, G. ``Income Taxes and Incentives to Work: An Empirical Study'', American Economic Review, Vol. 47, 1957, 529-549. \href{http://links.jstor.org/stable/pdfplus/1811736.pdf} {(web)}

Chetty, R., J. Friedman, T. Olsen and L. Pistaferri ``Adjustment Costs, Firms
Responses, and Micro vs. Macro Labor Supply Elasticities: Evidence from Danish Tax
Records'', Quarterly Journal of Economics,  126(2), 2011, 749-804. \href{http://elsa.berkeley.edu/~saez/course/chettyetalQJE2011denmark.pdf} {(web)} 

Fack, G and C. Landais ``Are Tax Incentives for Charitable Giving Efficient? Evidence from France'', American Economic Journal: Economic Policy, vol. 2, 2010, 117-41 \href{http://elsa.berkeley.edu/~saez/course/Fack and Landais(2010).pdf} {(web)}

Feenberg, D. and J. Poterba, ``Income Inequality and the Incomes of Very High Income Households: Evidence from Tax Returns'', in J. Poterba, ed., Tax Policy and the Economy, Volume 7, 145-177, Cambridge and London: MIT Press, 1993. \href{http://www.jstor.org/stable/pdfplus/20060632.pdf} {(web)}

Feldstein, M. ``The Effect of Marginal Tax Rates on Taxable Income: A Panel Study of the 1986 Tax Reform Act'', Journal of Political Economy, Vol. 103, 1995, 551-572. \href{http://links.jstor.org/stable/pdfplus/2138698.pdf} {(web)} 

Feldstein, M. ``Tax Avoidance and the Deadweight Loss of the Income Tax'', Review of Economics and Statistics, Vol. 81, 1999, 674-680. \href{http://links.jstor.org/stable/pdfplus/2646716.pdf} {(web)}

Goolsbee, A. ``What Happens When You Tax the Rich? Evidence from Executive Compensation'', Journal of Political Economy, Vol. 108, 2000, 352-378. \href{http://links.jstor.org/stable/pdfplus/3038281.pdf} {(web)} 

Gordon, R.H. and J. Slemrod ``Are ``Real'' Responses to Taxes Simply Income Shifting Between Corporate and Personal Tax Bases?'', NBER Working Paper No. 6576, 2000. \href{http://www.nber.org/papers/w6576} {(web)}

Gruber, J. and E. Saez ``The Elasticity of Taxable Income: Evidence and Implications'', Journal of Public Economics, Vol. 84, 2002, 1-32. \href{http://elsa.berkeley.edu/~saez/course/Gruber and Saez_JPubE(2002).pdf} {(web)}

Kleven, Henrik, Camille Landais, and Emmanuel Saez ``Taxation and International Mobility of Superstars: Evidence from the European Football Market,'' American Economic Review, 103(5), 2013, 1892-1924. \href{http://elsa.berkeley.edu/~saez/kleven-landais-saezAER13football.pdf} {(web)}

Kleven, Henrik, Camille Landais, Emmanuel Saez, and Esben Schultz ``Taxation and International Migration of Top Earners: Evidence from the Foreigner Tax Scheme in Denmark,'' Quarterly Journal of Economics, 129(1), 2014, 333-378.
\href{http://elsa.berkeley.edu/~saez/kleven-landais-saez-schultzQJE14danishscheme.pdf} {(web)}

\textbf{Kleven, Henrik and Esben Schultz  ``Estimating Taxable Income Responses
using Danish Tax Reforms'', American Economic Journal: Economic Policy, 6(4), 2014, 271-301
\href{http://elsa.berkeley.edu/~saez/course/kleven-schultz_may2012.pdf} {(web)} }

Kleven, Henrik and Mazhar Waseem ``Tax Notches in Pakistan: Tax Evasion, Real Responses, and Income Shifting'', Quarterly Journal of Economics, 128, 669-723, 2013
\href{http://elsa.berkeley.edu/~saez/course/kleven-waseem_qje2013.pdf} {(web)}

Ito, Koichiro. ``Do Consumers Respond to Marginal or Average Price?
Evidence from Nonlinear Electricity Pricing'', 
American Economic Review, 104(2), 2014, 537-567 \href{http://elsa.berkeley.edu/~saez/course/koichiroAER14.pdf} {(web)} 

Lindsey, L. ``Individual Taxpayer Response to Tax Cuts, 1982-1984: With Implications for the Revenue Maximizing Tax Rate'', Journal of Public Economics, 33, 1987, 173-206. \href{http://elsa.berkeley.edu/~saez/course/Lindsey_JPubE(1987).pdf} {(web)}

Moretti, Enrico and Daniel Wilson 2017. ``The Effect of State Taxes on the Geographical Location of Top Earners: Evidence from Star Scientists'', American Economic Review 107(7), 1858-1903 \href{http://elsa.berkeley.edu/~saez/course/moretti-wilsonAER17mobility.pdf} {(web)}

Moretti, Enrico, and Daniel J. Wilson. Taxing Billionaires: Estate Taxes and the Geographical Location of the Ultra-Wealthy. National Bureau of Economic Research Working Paper No. 26387.
\href{https://www.nber.org/papers/w26387.pdf} {(web)}

Piketty, T. and E. Saez ``Income Inequality in the United States, 1913-1998'', Quarterly Journal of Economics, Vol. 116, 2003, 1-39. \href{http://links.jstor.org/stable/pdfplus/25053897.pdf} {(web)}

\textbf{Piketty, Thomas, Emmanuel Saez, and Stefanie Stantcheva ``Optimal Taxation of Top Labor Incomes: A Tale of Three Elasticities,'' \emph{American Economic Journal: Economic Policy} 2014, 6(1), 2014, 230-271.
\href{http://elsa.berkeley.edu/~saez/piketty-saez-stantchevaAEJ14.pdf} {(web)} }

Piketty, Thomas, Emmanuel Saez, and Gabriel Zucman,  ``Distributional National Accounts:
Methods and Estimates for the United States'', Quarterly Journal of Economics, 133(2), 553-609, 2018
\href{https://eml.berkeley.edu/~saez/PSZ2018QJE.pdf} {(web)}

Rauh, Joshua, and Ryan J. Shyu. 2019. ``Behavioral Responses to State Income Taxation of High Earners: Evidence from California.'' National Bureau of Economic Research Working Paper No. 26349.
\href{https://www.nber.org/papers/w26349.pdf} {(web)} 

Saez, E. ``Reported Incomes and Marginal Tax Rates, 1960-2000: Evidence and Policy Implications'', in J. Poterba, ed., Tax Policy and the Economy, Volume 18, Cambridge: MIT Press, 2004. \href{http://www.jstor.org/stable/pdfplus/20061888.pdf} {(web)}

\textbf{Saez, E. ``Taxing the Rich More: Preliminary Evidence from the 2013 Tax Increase'', in R. Moffitt, ed., Tax Policy and the Economy, Volume 31, Cambridge: MIT Press, 2017. \href{http://eml.berkeley.edu/~saez/saezNBERWP16TPE.pdf} {(web)} }

\textbf{Saez, E., J. Slemrod, and S. Giertz ``The Elasticity of Taxable Income with Respect to Marginal Tax Rates: A Critical Review,'' Journal of Economic Literature 50(1), 2012, 3-50. \href{http://elsa.berkeley.edu/~saez/saez-slemrod-giertzJEL12.pdf} {(web)}}

Saez, Emmanuel and Gabriel Zucman. ``The Rise of Income and Wealth Inequality in America: Evidence from Distributional Macroeconomic Accounts,'' Journal of Economic Perspectives 34(4), Fall 2020, 3-26.
\href{https://eml.berkeley.edu/~saez/SaezZucman2020JEP.pdf}{(web)} 


Slemrod, J. ``Income Creation or Income Shifting? Behavioral Responses to the Tax Reform Act of 1986'', American Economic Review, Vol. 85, 1995, 175-180. \href{http://links.jstor.org/stable/pdfplus/2117914.pdf} {(web)}

Young, Cristobal, Charles Varner, Ithai Lurie, Richard Prisinzano, 2016 ``Millionaire Migration and the Taxation of the Elite: Evidence from Administrative Data'', American Sociological Review 81(3), 421--446
\href{http://elsa.berkeley.edu/~saez/course/Millionaire_Migration_and_the_Taxtion_of_the_Elite.pdf} {(web)}


}

\end{slide}

\end{document}
