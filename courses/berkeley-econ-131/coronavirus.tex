\documentclass[landscape]{slides}

\usepackage[landscape]{geometry}

\usepackage{pdfpages}

\usepackage{hyperref}
\usepackage{amsmath}
\usepackage{epstopdf}

\def\mathbi#1{\textbf{\em #1}}

\topmargin=-1.8cm \textheight=17cm \oddsidemargin=0cm
\evensidemargin=0cm \textwidth=22cm


\author{131 Undergraduate Public Economics \\ Emmanuel Saez \\ UC Berkeley}
\date{\today}



\title{Public Policy Response to Coronavirus} \onlyslides{1-300}

\newenvironment{outline}{\renewcommand{\itemsep}{}}

\begin{document}

\begin{slide}

\maketitle

\end{slide}


%\includegraphics[scale=1.5]{materials/fig1A_slide}
%
%\includegraphics[scale=1.5]{materials/fig1A_slide2}
%
%\includegraphics[scale=1.5]{materials/fig1A_slide3}
%
%\includegraphics[scale=1.5]{materials/fig1B_slide}



\begin{slide}
\begin{center}
{\bf CORONAVIRUS CRISIS OF 2020}
\end{center}
Coronavirus has created a global pandemic and economic crisis

Governments have quickly launched massive lockdowns to slow down epidemic
that in turn disrupt the economy

Governments have also created new policies to alleviate economic hardship

Clear that government policy is absolutely central for health response and
economic response

How do economists grapple with the situation?

\end{slide}


\begin{slide}
\begin{center}
{\bf EPIDEMIOLOGY}
\end{center}

\textbf{Severity.} Covid-19 infection is serious in 15\% of cases (need oxygen), very serious in 4\% of cases (need
ventilators or blood oxygenation for weeks)

Fatality rate around 1\% with best health care, possibly up to 4\% with no health care

\textbf{Spread.} Virus is highly contagious: each infected person infects in turn $R_0 \simeq 2.5$ others on average

$\Rightarrow$ unchecked epidemic grows exponentially until $1-1/R_0=60\%$ of population has been infected
(herd immunity)

Social distancing reduces $R$. $R < 1$ $\Rightarrow$ outbreak dies off. 

Coronavirus particularly tough because of asymptomatic but contagious phase/cases
and very long and intensive care needed

\end{slide}



\begin{slide}
\includepdf[pages={2}]{coronavirus_attach.pdf}
\end{slide}

\begin{slide}
\includepdf[pages={9}]{coronavirus_attach.pdf}
\end{slide}

\begin{slide}
\includepdf[pages={3}, scale=1]{coronavirus_attach.pdf}
\end{slide}

\begin{slide}
\begin{center}
{\bf HEALTH CARE CHOICES}
\end{center}
Do nothing approach not appealing 
$\Rightarrow$ 60\% of population gets infected,
health care sector  overwhelmed,
mortality rate up from 1\% to 4\%, economy
paralyzed for several months %(Correia et al. 2020 for Great Influenza)

Countries with big outbreaks have imposed drastic measures to flatten the curve:

\textbf{Mitigation:} Reduce $R$ to slow down epidemic and fraction eventually infected down to $1-1/R$
(e.g. if $R=1.5$ then only 33\% eventually infected instead of 60\%) but still catastrophic

\textbf{Suppression.} Getting $R<1$ so that outbreak dies off

China and Korea succeeded in suppressing outbreaks and can wait for 
vaccine or better treatments with less drastic measures

Size of outbreak depends on how fast social distancing measures are taken
(Wuhan, Italy slow, some US states also slow)

\end{slide}

\begin{slide}
\includepdf[pages={5, 7}, scale=1]{coronavirus_attach.pdf}
\end{slide}

\begin{slide}
\includepdf[pages={14, 13}, scale=1]{coronavirus_attach.pdf}
\end{slide}


\begin{slide}
\includepdf[pages={11}, scale=1]{coronavirus_attach.pdf}
\end{slide}

\begin{slide}
\includepdf[pages={12}, scale=1]{coronavirus_attach.pdf}
\end{slide}

\begin{slide}
\includepdf[pages={1}, scale=1]{coronavirus_attach.pdf}
\end{slide}


\begin{slide}
\begin{center}
{\bf SUPPRESSING BIG OUTBREAKS}
\end{center}
With big outbreaks (as in most EU countries and now many US states), drastic social distancing measures
are needed to bring $R<1$

Government needs to shutdown large fraction of the economy (scope of shutdown varies across countries,
Sweden, Netherlands, some US states, devo countries do less)

In many places (many EU countries, CA, NY, NJ, etc.): \textbf{drastic lockdown} where only remote and essential work
is allowed

With drastic lockdown, GDP falls by 1/3, and 1/3 of workers
are idled, 1/3 work from home, 1/3 still work (recent estimation for France)

How to cope with the economic crisis created by drastic lockdowns that are likely to last several months?

\end{slide}

\begin{slide}
\includepdf[pages={10}, scale=.9]{coronavirus_attach.pdf}
\end{slide}


\begin{slide}
\begin{center}
{\bf ECONOMIC RESPONSE: DO NOTHING}
\end{center}
Lockdown businesses lose their revenue and can no longer pay their workers and maintenance costs

$\Rightarrow$ most idled workers get laid off (unemployment rate of 30\%) 

Many businesses will go bankrupt and have to liquidate (especially small ones which cannot borrow)

Self-employed lose their earnings (e.g. UBER drivers)

$\Rightarrow$ Economic hardship for tens of millions families (as many families don't have savings
to dip in)

$\Rightarrow$ Slow recovery as it takes time for businesses and jobs to be re-created

Do nothing creates economic catastrophe $\Rightarrow$ Govt shuts down the economy, govt should also mitigate economic hardship

\end{slide}


\begin{slide}
\begin{center}
{\bf MACRO PERSPECTIVE}
\end{center}
Lockdowns shut down part of the economy and this economic output is lost
[supply side shock]

But government can change distribution of losses [=who absorbs the losses]
to alleviate hardship and keep businesses alive

Govt can issue public debt to fund transfers to individuals or businesses hit by lockdown

Implicitly, new public debt is bought by individuals who are saving more [maybe haven't lost income but
can't consume as much because of shutdown]

%Explicitly, central bank (fed in US) buys government bonds and gives cash to govt 

Extra public debt will be repaid with higher taxes in future decades

%Business sector is getting emergency loans from central banks
\end{slide}


\begin{slide}
\begin{center}
{\bf BUSINESS PERSPECTIVE}
\end{center}
Lockdown forces businesses to stop or reduce operations (e.g., restaurants, airlines)
temporarily

Businesses can layoff workers but still have maintenance costs to pay (such as rent, interest on debt,
maintaining equipment, essential workers, etc.)

In principle, businesses could borrow to cover these costs until they can reopen

\textbf{Liquidity issue:} Businesses may not be able to borrow. Government can provide loans
(done through central bank). Business absorbs the loss but can survive.

\textbf{Solvency issue:} Businesses may not be able to repay the loan (if shutdown is long).
Government can provide grants (=forgiving loans). Govt absorbs the loss.

\end{slide}



\begin{slide}
\begin{center}
{\bf EU COUNTRIES RESPONSE: HIBERNATE}
\end{center}
Most EU countries have adopted plans to ``hibernate'' the economy and avoid
mass layoffs and business destruction

Govt pays for the wages of idled workers and maintenance costs of idled businesses

Example: UK pays 80\% of wages of idled workers (up to $\pounds$2,500/month) and idled business maintenance costs.

Businesses and workers can resume work once lockdown ends

$\Rightarrow$ Alleviates hardship and allows for fast recovery

$\Rightarrow$ Can work if shutdown is not too long ($<$ 6 months)

Challenge is how to generate government funding (e.g. Spain, Italy pay higher interest
on their public debt than Germany), on-going attempts to find EU level solution (Eurobonds)

\end{slide}


\begin{slide}
\begin{center}
{\bf US RESPONSE}
\end{center}
Enormous wave of layoffs in the US: 3.3m in week 3/15-3/21

US passed historically large \$2.2T stimulus on 3/27/2020

Expands unemployment insurance (more generous, covers more people
including self-employed)

Direct one time checks to families (\$1200/adult+\$500/child), not well targeted 
but can be administered fast

Emergency loans for businesses. Loan allows to avoid bankruptcy but
has to be repaid (some businesses might become insolvent)

Loans can convert into grants for small businesses if they don't layoff workers
but not systematic as in EU

$\Rightarrow$ Alleviates hardship but won't be enough prevent mass layoffs
and rise in fraction of people with no health insurance 
\end{slide}


\begin{slide}
\begin{center}
{\bf DEVELOPING COUNTRIES}
\end{center}
Harder to track epidemic in developing countries (Iran outbreak possibly
worst to date)

Huge variation in policy responses from abrupt lockdown in India
to ``nothing to worry about for now'' in Brazil or Mexico 

Devo countries have weaker health care capacity ($\Rightarrow$ higher mortality)
and weaker state capacity to impose lockdown ($\Rightarrow$ bigger outbreaks)

$\Rightarrow$ Barring climate help, health crisis in devo will be bigger

Devo countries have less ability to issue public debt to respond (face high interest rate)
and suffer from huge capital outflows 

$\Rightarrow$ Economic crisis will also be more severe in devo countries

Help to devo countries key to help eradicate pandemic
\end{slide}


\begin{slide}
\begin{center}
{\bf RECOVERING FROM ECONOMIC CRISIS}
\end{center}
Economy will restart once the outbreak is controlled and has shrunk in size

With systematic testing, tracing, quarantining, possible
to sustain $R<1$ with less business lockdown (as in South Korea)

But still risky situation as outbreaks can explode again, various restrictions
will still need to be in place (banning large gatherings, etc.)

Vaccine could become available in 12-18 months

Economy recovers faster if fewer businesses, work-employer, business-customers
relationships have been destroyed

\end{slide}

\begin{slide}
\begin{center}
{\bf LESSONS FROM THE CRISIS}
\end{center}
Crisis is global and huge and evolving incredibly fast $\Rightarrow$ making predictions is hard

Health and economic systems much more fragile that we thought 

Markets fail in emergency  and collective action through government 
becomes crucial (highlights social aspect of humans)

Response through government moves faster and bolder than we could have imagined
(left-right divide muted) 

$\Rightarrow$ Human societies put in a big fight to save their
vulnerable people (care for the sick and old)

How do we build a more resilient health and economic system?

%If collective response fails, the crisis will likely be worse 

\end{slide}





\begin{slide}
\begin{center}
{\bf REFERENCES}
\end{center}
{\small

Baldwin, Richard, Beatrice Weder di Mauro. 
\emph{Mitigating the COVID Economic Crisis: Act Fast and Do Whatever It Takes},
March 18, 2020.
 \href{https://voxeu.org/content/mitigating-covid-economic-crisis-act-fast-and-do-whatever-it-takes} {(web)} 


Correia, Sergio, Stephan Luck, and Emil Verner. ``Pandemics Depress the Economy, Public Health Interventions Do Not: Evidence from the 1918 Flu'',
March 27, 2020,  \href{https://papers.ssrn.com/sol3/papers.cfm?abstract_id=3561560} {(web)} 

Econfip, Economists for Inclusive Prosperity, Covid-19 briefs, March 2020
 \href{https://econfip.org/} {(web)} 
 
IGM Forum, \emph{COVID-19}, March 2020
 \href{ http://www.igmchicago.org/covid-19/s} {(web)} 

Pueyo, Tomas.  ``Coronavirus: The Hammer and the Dance
What the Next 18 Months Can Look Like, if Leaders Buy Us Time'', March 19, 2020
 \href{https://medium.com/@tomaspueyo/coronavirus-the-hammer-and-the-dance-be9337092b56} {(web)} 
 
Saez, Emmanuel and Gabriel Zucman, ``Keeping Businesses Alive: The Government Will Pay'',
Econfip, Covid-19 brief, March 2020
 \href{https://econfip.org/policy-brief/keeping-businesses-alive-the-government-will-pay/} {(web)} 
 
Saez, Emmanuel and Gabriel Zucman, ``Jobs Aren�t Being Destroyed This Fast Elsewhere. Why Is That?'',
New York Times, March 30, 2020
 \href{https://www.nytimes.com/2020/03/30/opinion/coronavirus-economy-saez-zucman.html} {(web)} 
 
%Saez, Emmanuel and Gabriel Zucman, ``Keeping Businesses Alive: The Government Will Pay'',
%Econfip, Covid-19 brief, March 2020
% \href{https://econfip.org/policy-brief/keeping-businesses-alive-the-government-will-pay/} {(web)} 

Washington Center for Equitable Growth, ``U.S. economic policy principles for confronting the coronavirus recession''
\href{https://equitablegrowth.org/u-s-economic-policy-principles-for-confronting-the-coronavirus-recession/} {(web)}

}

\end{slide}

\end{document}
