\documentclass[landscape]{slides}

\usepackage[landscape]{geometry}

\usepackage{pdfpages}

\usepackage{hyperref}
\usepackage{amsmath}
\usepackage{epstopdf}

\def\mathbi#1{\textbf{\em #1}}

\topmargin=-1.8cm \textheight=17cm \oddsidemargin=0cm
\evensidemargin=0cm \textwidth=22cm


\author{131 Undergraduate Public Economics \\ Emmanuel Saez \\ UC Berkeley}
\date{}



\title{Responses of Reported Taxable Income to Taxes} \onlyslides{1-300}

\newenvironment{outline}{\renewcommand{\itemsep}{}}

\begin{document}

\begin{slide}

\maketitle

\end{slide}


%\includegraphics[scale=1.5]{materials/fig1A_slide}
%
%\includegraphics[scale=1.5]{materials/fig1A_slide2}
%
%\includegraphics[scale=1.5]{materials/fig1A_slide3}
%
%\includegraphics[scale=1.5]{materials/fig1B_slide}



\begin{slide}
\begin{center}
{\bf TAXABLE INCOME ELASTICITIES}
\end{center}

Modern public finance literature focuses on taxable income
elasticities instead of hours/participation elasticities

Two main reasons:

1) What matters for policy is the total behavioral response
to tax rates (not only hours of work but also occupational choices,
avoidance, etc.)

2) Data availability: taxable income is precisely measured in tax
return data

Recent overview of this literature: Saez-Slemrod-Giertz JEL'12

\end{slide}

%\begin{slide}
%\begin{center}
%{\bf FELDSTEIN RESTAT'99}
%\end{center}
%Consider two sources of responses to tax rates:
%
%1) Labor supply: $u(c,z)$ model where $z$ is earnings and is equal
%to reported income $y$ with $c=y \cdot (1-\tau)+R$
%
%Individual chooses $y$ to maximize $u(y(1-\tau)+R,y)$
%
%2) Avoidance: $z$ earnings is fixed but reported income $y=z-d$
%where $d$ is non-taxable compensation (health benefits or perks):
%$u(c,d)$ with $c=(1-\tau) y+R$
%
%Individual chooses $y$ to maximize $u(y(1-\tau)+R,z-y)$ [$z$
%fixed]
%
%Models are formally identical and generate the same efficiency and
%optimal tax analysis
%\end{slide}


\begin{slide}
\begin{center}
{\bf FEDERAL US INCOME TAX CHANGES}
\end{center}
%Tax rates change frequently over time

Biggest tax rate changes have happened at the top:

Reagan I: ERTA'81: top rate $\downarrow$ 70\% to 50\% (1981-1982)

Reagan II: TRA'86: top rate $\downarrow$ 50\% to 28\% (1986-1988)

Clinton: OBRA'93: top rate $\uparrow$ 31\% to 39.6\% (1992-1993)

Bush: EGTRRA '01: top rate $\downarrow$ 39.6\% to 35\% (2001-2003)

Obama '13:  top rate $\uparrow$ 35\% to 39.6\%+3.8\% (2012-2013)

Trump '17: top rate cut down to 37\%+3.8\% (2017-2018)

Taxable Income = Ordinary Income + Realized Capital Gains -
Deductions $\Rightarrow$ Each component can respond to MTRs
\end{slide}

%\begin{slide}
%\includepdf[pages={1}]{taxableincome_attach.pdf}
%\end{slide}

\begin{slide}
\includepdf[pages={1}]{taxableincome_new_attach.pdf}
\end{slide}



\begin{slide}
\begin{center}
{\bf LONG-RUN EVIDENCE IN THE US}
\end{center}
Goal: evaluate whether top \textbf{pre-tax} incomes respond to changes in one minus
the marginal tax rate (=net-of-tax rate)

Focus is on \textbf{pre-tax} income before deductions and excluding realized
capital gains (because they are taxed at lower separate rate)

%Pioneered by Feenberg-Poterba TPE'93 for period 1951-1990

Piketty-Saez QJE'03 estimate top income shares since 1913 [IRS
tabulations for 1913-1959, IRS micro-files since 1960]

Piketty-Saez-Stantcheva AEJ-EP'14 estimate the effect of top MTR
on top income shares in the US since 1913

%Landais '09 estimates MTRs by income groups since 1913
%
%Saez TPE'04 proposes detailed analysis for 1960-2000 period using
%TAXSIM calculator at NBER linked to IRS micro-files
\end{slide}

\begin{slide}
\includepdf[pages={2}]{taxableincome_new_attach.pdf}
\end{slide}

%\begin{slide}
%\includepdf[pages={5}]{taxableincome_attach.pdf}
%\end{slide}
%
%\begin{slide}
%\includepdf[pages={6}]{taxableincome_attach.pdf}
%\end{slide}

%\begin{slide}
%\includepdf[pages={7}]{taxableincome_attach.pdf}
%\end{slide}
%
%\begin{slide}
%\includepdf[pages={8}]{taxableincome_attach.pdf}
%\end{slide}

\begin{slide}
\begin{center}
{\bf INCOME SHARE BASED ELASTICITY ESTIMATION}
\end{center}
1) {\bf Tax Reform Episode:} Compare top \textbf{pre-tax} income shares at $t_0$
(before reform) and $t_1$ (after reform)
$$e=\frac{\log sh_{t_{1}}-\log sh_{t_{0}}}{\log (1-\tau
_{t_{1}})-\log (1-\tau _{t_{0}})}$$ where $sh_t$ is pre-tax top income
share and $\tau_t$ is the average MTR for top group in year $t$

Identification assumption: absent tax change, $sh_{t_0}=sh_{t_1}$

2) {\bf Full Time Series:} Run regression:
$$\log sh_{t}=\alpha+e\cdot \log (1-\tau_{t})+\varepsilon _{t}$$
and adding time controls to capture non-tax related top income
share trends

Identification assumption: non-tax related changes in $sh_t$ $\perp \tau_t$
\end{slide}

%\begin{slide}
%\includepdf[pages={9}]{taxableincome_attach.pdf}
%\end{slide}

%\begin{slide}
%\includepdf[pages={10}]{taxableincome_attach.pdf}
%\end{slide}

\begin{slide}
\begin{center}
{\bf LONG-RUN EVIDENCE IN THE US}
\end{center}
1) Clear correlation between top incomes and top income rates both
in several short-run tax reform episodes and in the long-run: estimated elasticities
are large: around 0.7 for long-run, and sometimes over 1 for short-run episodes (such as '86-'88).

2) Correlation between tax rates and income shares 
largely absent below the top 1\% (such as the next
9\%)

3) Top income shares sometimes do not respond to large tax rate
cuts [e.g., Kennedy Tax Cuts of early 1960s]

2) and 3) suggest that context matters (such as opportunities to respond / avoid taxes matter),
response unlikely to be due to a universal labor supply elasticity


\end{slide}


%\begin{slide}
%\begin{center}
%{\bf KLEVEN AND SCHULTZ '12}
%\end{center}
%\textbf{Key Advantages:}
%
%a) Use full population of tax returns in Denmark since 1980
%(large sample size, panel structure, many demographic variables, stable inequality)
%
%b) A number of reforms changing tax rates differentially across three income
%brackets and across tax bases (capital income taxed separately from labor income)
%
%c) Show compelling visual DD-evidence of tax responses around the 1986 large reform
%
%\end{slide}
%
%\begin{slide}
%\includepdf[pages={31-32}]{taxableincome_attach.pdf}
%\end{slide}
%
%\begin{slide}
%\begin{center}
%{\bf KLEVEN AND SCHULTZ '12}
%\end{center}
%
%\textbf{Key Findings:}
%
%a) Small labor income elasticity (.05-.1)
%
%b) bigger capital income elasticities (.2)
%
%c) bigger elasticities for large reforms
%
%d) modest income shifting between labor and capital in Denmark
%(top rates on labor and capital are carefully aligned)
%
%$\Rightarrow$ Danish tax system
%optimized to have broad base and few avoidance opportunities
%
%\end{slide}


%\begin{slide}
%\begin{center}
%{\bf FISCAL EXTERNALITIES}
%\end{center}
%%Feldstein RESTAT'99: nature of behavioral response (labor supply,
%%avoidance, etc.) does not matter AS LONG AS the behavioral
%%response does not generate a {\bf fiscal externality}
%
%A {\bf Fiscal externality} is a change in tax revenue that occurs
%in any tax base $z^B$ other than $z$ due to the behavioral
%response to the tax change in the initial base $z$
%
%(1) $z^B$ can be a different tax base in the same time period
%(such as corporate income tax base) $\Rightarrow$ {\bf Income
%shifting}
%
%(2) $z^B$ can be the same tax base in a different time period
%(such as future income) $\Rightarrow$ {\bf Inter-temporal
%Substitution}
%
%Efficiency and optimal tax analysis depend on effect on {\bf
%total} tax revenue
%\end{slide}

\begin{slide}
\begin{center}
{\bf TAX AVOIDANCE}
\end{center}
%Feldstein RESTAT'99: nature of behavioral response (labor supply,
%avoidance, etc.) does not matter AS LONG AS the behavioral
%response does not generate a {\bf fiscal externality}

Behavioral response to income tax comes not only from reduced
work effort and economic activity but also from tax avoidance. Two main forms of tax avoidance:


\textbf{1) Intertemporal substitution:} Shift income over time to take advantage of
tax changes: Example: If tax rates increase next year, shift income from next year into this year.

\textbf{2) Income shifting:} Shift income to another tax base that is taxed less. Example: shift 
business profits from corporate tax base to the individual tax base if this is advantageous taxwise

Such tax avoidance affect tax revenue through these other tax bases and such revenue effects
need to be accounted for in optimal tax analysis

\end{slide}


\begin{slide}
\begin{center}
{\bf Intertemporal Substitution: Realized Capital Gains}
\end{center}
Realized capital gains occur when individual sells asset at a
higher price than buying price

Individuals have flexibility in the timing of asset sales and
capital gains realizations

TRA'86 lowered the top tax rate on ordinary income from 50\% to
28\% but increased the top tax rate on realized capital gains from
20\% to 28\%

2013: tax rate on capital gains increased from 15\% to 20\%+3.8\% %(see Saez 2017)

$\Rightarrow$ Surge in capital gains realizations in 1986 and 2012 [and
depressed capital gains in 1987 and 2013] 

\normalsize
$\Rightarrow$ Short-term elasticity is very large but long-term
elasticity is certainly much smaller

\end{slide}

\begin{slide}
\includepdf[pages={17, 18}]{taxableincome_new_attach.pdf}
\end{slide}

%\begin{slide}
%\includepdf[pages={4}]{taxableincome_new_attach.pdf}
%\end{slide}

%\begin{slide}
%\begin{center}
%{\bf INTER-TEMPORAL SUBSTITUTION: STOCK-OPTIONS}
%\end{center}
%Goolsbee JPE'00 hypothesizes that top earners' ability to retime
%income drives much of observed responses
%[Frisch elasticity instead of compensated elasticity]
%
%Fixed effects regression specification:
%$$TLI_{it}= e_{1}\log (1-MTR_{it})+ e_{2}\log
%(1-MTR_{i t+1})+ \alpha_i + \beta_t$$ Short-run elasticity is
%$e_1$
%
%$e_2<0$ if future MTR increase shifts income to present
%
%Long run elasticity is $e_{1}+e_{2}$
%
%Uses ExecuComp panel data to study effects of the 1993 Clinton top
%tax rate $\uparrow$ [from 31\% in 1992 to 39.6\% in 1993 announced
%in late 1992] on executive pay
%\end{slide}

%\begin{slide}
%\begin{center}
%{\bf Intertemporal Substitution: Stock-Options}
%\end{center}
%Goolsbee JPE'00 analyzes CEO pay around the 1993 Clinton top tax rate increase
%[from 31\% in 1992 to 39.6\% in 1993 announced
%in late 1992]
%
%Finds a strong re-timing response through stock-option exercise (executives can choose
%the timing of their stock-option exercises)
%
%$\Rightarrow$ Large short-term response due to re-timing, small long-term response
%
%The 2013 Obama top tax rate increase has also generated income shifting from
%2013 to 2012 (Saez '17)
%
%\end{slide}
%
%
%\begin{slide}
%\includepdf[pages={5}]{taxableincome_new_attach.pdf}
%\end{slide}
%
%\begin{slide}
%\begin{center}
%{\bf STOCK OPTIONS}
%\end{center}
%Major form of compensation of US top executives. Theoretical goal
%is to motivate executives to increase the value of the company
%(stock price $P(t)$)
%
%Stock-options granted at date $t_0$ allow executives to buy $N$
%company shares at price $P(t_0)$ on or after $t_1$ (in general
%$t_1-t_0 \simeq 3-5$ years = vesting period)
%
%Executive exercise option at (chosen) time $t_2 \geq t_1$: pays $N
%\cdot P(t_0)$ to get shares valued $N \cdot P(t_2)$. Exercise
%profit $N[P(t_2)-P(t_0)]$ (considered and taxed as wage income in
%the US)
%
%After $t_2$, executive owns $N$ shares, eventually sold at time
%$t_3 \geq t_2$: realized capital gain $N [P(t_3)-P(t_2)]$ (taxed
%as capital gains)
%\end{slide}

%\begin{slide}
%\includepdf[pages={17}]{taxableincome_attach.pdf}
%\end{slide}
%
%\begin{slide}
%\begin{center}
%{\bf GOOLSBEE JPE'00: INTER-TEMPORAL SUBSTITUTION}
%\end{center}
%Executives had a surge in income in 1992 (when reform was
%announced) relative to 1991 followed by a sharp drop in 1993
%
%1) Simple DD estimate for '92 vs '93 would find a large effect
%here, but it would be picking up pure re-timing
%
%2) Concludes that long run effect $e_1+e_2$ is much smaller than
%substitution effect $e_1$ [long-run elasticity is the relevant
%parameter for policy]
%
%3) Effects driven almost entirely by re-timing exercise of
%stock-options [executives knew tax rate would $\uparrow$ in '93
%when Clinton elected in Nov. '92 $\Rightarrow$ Exercise stock
%options]
%\end{slide}

\begin{slide}
\begin{center}
{\bf Income Shifting: Corporate vs. Individual Tax Base}
\end{center}
Businesses can be organized as {\bf corporations} or {\bf
unincorporated businesses} [also called {\bf pass-through}
entities]

Corporate profits first taxed by corporate tax [rate
$\tau_c = 21\%$]

Net-of-tax profits are taxed again at rate $\tau_{\text{distrib}}$  when finally distributed to
shareholders. Two distribution options:

a) dividends [tax rate $\tau_d = 20\%$ today]

b) retained profits increase stock price: shareholders realize
capital gains when finally selling the stock [tax rate
$\tau_{cg}=20\%$]

\vspace{-0.5cm}

But distributions can be deferred so that $\tau_{\text{distrib}}<<\tau_d, \tau_{cg}$

For {\bf unincorporated businesses} (sole proprietorships,
partnerships, S-corporations) profits are taxed directly and
solely as individual income (tax rate $\tau_i = 37\%$ top MTR reduced to $\simeq 30\%$ with 20\% business profit deduction since 2018)
\end{slide}


\begin{slide}
\begin{center}
{\bf CORPORATE AND INDIVIDUAL TAX BASE}
\end{center}
Corporate form best if $(1-\tau_c) \cdot (1-\tau_{\text{distrib}})>1-\tau_i$

US fed taxes in 2018+: $\tau_c=21\%$, $\tau_{cg}=\tau_d=20\%$,
(but $\tau_{\text{distrib}}<<20\%$ if distribution deferred), 
$\tau_i=37\%$ or $30\%$ 

After 2018 Trump tax cut: corporate form is best, especially if 
wealthy business owner can defer distribution

Pre 2018, $\tau_c=35\%$ and $\tau_i=39.6\% \Rightarrow$ individual form better

$\Rightarrow$ the rich are likely to incorporate their businesses in '18+

\small
Before 1986 (and especially before 1981), top individual rate
$\tau_i$ was much higher so corporate form was best

Shifts from corporate to individual base increases business
profits at the expense of dividends and realized capital gains

Large part of 1986 response is due to such shifting
\end{slide}

\begin{slide}
\includepdf[pages={19}]{taxableincome_new_attach.pdf}
\end{slide}

%\begin{slide}
%\begin{center}
%{\bf INCOME SHIFTING: US EXECUTIVE COMPENSATION AND TAX RATES}
%\end{center}
%1) Executive compensation (relative to average pay) stagnated from
%1936-1970 and surged in the US since 1970 [Frydman-Saks '08]
%
%2) Top income rates on earnings high till 1970 and lower
%afterwards: tempting to attribute surge in executive compensation
%to lower tax rates
%
%3) BUT stock-options from 1950 to 1975 used to be ``Qualified
%stock options'' taxed as realized capital gains only when stock
%sold [no tax when exercised] but not deductible for corporate tax
%
%$\Rightarrow$ Tax rate on stock-option compensation was therefore
%very low (25\%) as realized capital gains
%
%Yet stock-option compensation was only used in moderation
%
%$\Rightarrow$ Frydman-Molloy '09 find no overall elasticity of
%executive compensation with respect to MTRs when taking this into
%account
%\end{slide}
%
%\begin{slide}
%\includepdf[pages={38}]{pdf/frydman-saks08executivepay.pdf}
%\end{slide}
%
%\begin{slide}
%\includepdf[pages={40}]{pdf/frydman-molloy09tax-executives.pdf}
%\end{slide}
%
%\begin{slide}
%\includepdf[pages={40}]{pdf/frydman-molloy09tax-executivesb.pdf}
%\end{slide}

\begin{slide}
\begin{center}
{\bf Bottom Line on Behavioral Responses to Taxes}
\end{center}
1) Clear evidence of strong responses to tax changes due to
re-timing or income shifting

2) Heterogeneity in tax responses due to heterogeneity in shifting
opportunities [e.g., Kennedy tax cuts of '61 vs. TRA'86]

3) Top income shares can change drastically without changes in tax
rates [e.g., 1993-2000]

4) Difficult to know from single country time series the role
played by top tax rate cuts in the surge of top incomes
$\Rightarrow$ International evidence can cast further useful
evidence
\end{slide}

\begin{slide}
\begin{center}
{\bf TOP RATES AND TOP INCOMES INTERNATIONAL EVIDENCE}
\end{center}

1) Use pre-tax top 1\% income share data from 18 OECD countries since 1960 using
the {\bf World Top Incomes Database}

2) Compute top (statutory) individual income tax rates using OECD data [including both
central and local income taxes].

%Those tax rates do not include payroll taxes,
%corporate taxes, or VAT and Sales taxes

Plot top 1\% pre-tax income share against top MTR in 1960-4,  in 2005-9, and 1960-4 vs. 2005-9

%OLS basic regressions:
%$$\log (Top\:\: 1\% \:\: Share) = \alpha + e \cdot \log(1-MTR) + \varepsilon$$
%$$\Delta \log (Top\:\: 1\% \:\: Share) = \alpha + e \cdot  \Delta \log(1-MTR) + \varepsilon$$
\end{slide}


\begin{slide}
\includepdf[pages={7-9}]{taxableincome_new_attach.pdf}
\end{slide}


\begin{slide}
\begin{center}
{\bf TOP RATES AND TOP INCOMES EVIDENCE}
\end{center}

1) Pre-tax Top income shares have increased significantly in some but not all countries
[Atkinson-Piketty-Saez JEL'11]

2) Top tax rates have come down significantly in a number of countries since 1960s

3) Correlation between 1) and 2) is strong but not perfect: lower top tax rates
are a necessary but not sufficient condition for surge in top incomes

%$\Rightarrow$ Total elasticity is large but could be a mix of real effects, avoidance effects,
%or bargaining effects
\end{slide}


\begin{slide}
\begin{center}
{\bf ECONOMIC EFFECTS OF TAXING THE TOP 1\%}
\end{center}
Strong empirical evidence that \textbf{pre-tax} top incomes are affected by top tax rates 

\textbf{3} potential scenarios with very different policy consequences

{\bf 1) Supply-Side:} Top earners work less and earn less when top tax rate increases
$\Rightarrow$ Top tax rates should not be too high

{\bf 2) Tax Avoidance/Evasion:} Top earners avoid/evade more when top tax rate increases

$\Rightarrow$ a) Eliminate loopholes, b) Then increase top tax rates

{\bf 3) Rent-seeking:} Top 1\% earners extract more income (at the expense of the 99\%)
when top tax rates are low
$\Rightarrow$ High top tax rates are desirable

\end{slide}


\begin{slide}
\includepdf[pages={10}]{taxableincome_new_attach.pdf}
\end{slide}

\begin{slide}
\includepdf[pages={22}]{taxableincome_new_attach.pdf}
\end{slide}



\begin{slide}
\begin{center}
{\bf Real changes vs. tax Avoidance? Charitable giving}
\end{center}
Test using charitable giving behavior of top income earners (Saez TPE '17)

Because charitable is tax deductible, incentives to give are stronger
when tax rates are higher

Under the tax avoidance scenario, reported incomes and reported charitable
giving should move in opposite directions

Empirically, charitable giving of top income earners has grown in close tandem with
top incomes

$\Rightarrow$ Incomes at the top have grown for real 

%Empirical correlation is very similar ruling out the pure tax avoidance scenario

\end{slide}




\begin{slide}
\includepdf[pages={20, 21}]{taxableincome_new_attach.pdf}
\end{slide}


\begin{slide}
\begin{center}
{\bf Supply-Side or Rent-Seeking? (Piketty-Saez-Stantcheva AEJ'13)}
\end{center}
Correlation between \textbf{pre-tax} top incomes and top tax rates

If rent-seeking: growth in top 1\% incomes should come at the expense of
bottom 99\% (and conversely)

Two macro-preliminary tests:

1) In the US, top 1\% incomes grow slowly from 1933 to 1975 and fast afterwards.
Bottom 99\% incomes grow fast from 1933 to 1975 and slowly afterwards
$\Rightarrow$ Consistent with rent-seeking effects

2) Look at cross-country correlation between economic growth and top tax rate cuts
$\Rightarrow$ No correlation supports rent-seeking

One micro-test using CEO pay data

\end{slide}

\begin{slide}
\includepdf[pages={11}]{taxableincome_new_attach.pdf}
\end{slide}


%\begin{slide}
%\includepdf[pages={12,13}]{taxableincome_new_attach.pdf}
%\end{slide}

\begin{slide}
\begin{center}
{\bf INTERNATIONAL CEO PAY EVIDENCE}
\end{center}
Recent micro-data for 2006 gathered by Fernandes, Ferreira, Matos, Murphy RFS'12.

1) CEO pay across countries strongly negatively correlated with top tax rates

2) Correlation remains as strong even when controlling for firms' characteristics
and performance

$\Rightarrow$ Consistent with rent-seeking effects

\end{slide}

\begin{slide}
\includepdf[pages={14-15}]{taxableincome_new_attach.pdf}
\end{slide}


\begin{slide}
\begin{center}
{\bf INTERNATIONAL MIGRATION}
\end{center}
Public debate concern that top skilled individuals move to low tax countries (e.g., in EU context)
or low tax states (within US Federation)

Migration concern bigger in public debate than supply-side concern within a country 

%Relatively little work on tax induced international migration of top skilled workers

Interesting variation due to proliferation of special low tax schemes for highly paid foreigners
in Europe

\small

Kleven-Landais-Saez AER'13 look at \textbf{football players} in Europe (highly mobile group, many tax
reforms) $\Rightarrow$ Find significant migration responses to taxes after European football market
was de-regulated in '95

Akcigit-Baslandze-Stantcheva AER'16 look at \textbf{innovators} (using patent data) mobility and find significant tax
effects for top innovators

\normalsize

Various US states studies: Moretti-Wilson AER17 , 2019, Rauh-Shyu '19 (big effects), Young et al. '16 (modest effects)



\end{slide}



\begin{slide}
\begin{center}
{\bf KLEVEN-LANDAIS-SAEZ-SCHULTZ QJE'14}
\end{center}
Exploit the 1991 tax scheme in Denmark: immigrants with high earnings ($ \geq 103,000$ Euros/year)
taxed at flat 25\% rate (instead of regular tax with top 59\% rate) for 3 years

Use population wide Danish tax data and DD strategy: compare immigrants above eligibility
earnings threshold (treatment) to immigrants slightly below threshold (control)

\textbf{Key Finding:} Scheme doubles the number of highly paid
foreigners in Denmark relative to controls

$\Rightarrow$ Aggressive tax competition can be desirable from a one
country perspective but undermines tax progressivity in other countries

$\Rightarrow$ Tax coordination will be key to
preserve progressive taxation in the European Union

\end{slide}

\begin{slide}
\includepdf[pages={16}]{taxableincome_new_attach.pdf}
\end{slide}


\begin{slide}
\begin{center}
{\bf REFERENCES}
\end{center}
{\small

Akcigit, Ufuk , Salom\'e Baslandze, and Stefanie Stantcheva. ``Taxation and the International Mobility of Inventors'',
American Economic Review 106 (10), 2016, 2930--2981 \href{http://elsa.berkeley.edu/~saez/course/Akcigit_Baslandze_Stantcheva_Taxation_Superstars.pdf} {(web)}

Alvaredo, F., T. Atkinson, T. Piketty, E. Saez, G. Zucman \emph{The World Wealth and Income Database},
\href{http://www.wid.world/} {(web)}

Atkinson, A., T. Piketty and E. Saez ``Top Incomes in the Long Run of History'', Journal of Economic Literature, 49(1), 2011, 3-71. \href{http://elsa.berkeley.edu/~saez/atkinson-piketty-saezJEL10.pdf} {(web)}

Department of the Treasury(2012) ``Capital Gains and Taxes Paid on Capital Gains'' \href{http://elsa.berkeley.edu/~saez/course131/taxable-income11.pdf} {(web)}

Fernandes, Nuno, et al. ``Are US CEOs paid more? New international evidence.'', Review of Financial Studies 26.2, 2013, 323-367.\href{http://elsa.berkeley.edu/~saez/course131/CEO13.pdf}{(web)}

%Goolsbee, A. ``What Happens When You Tax the Rich? Evidence from Executive Compensation'', Journal of Political Economy, Vol. 108, 2000, 352-378. \href{http://links.jstor.org/stable/pdfplus/3038281.pdf} {(web)}

%Gordon, R.H. and J. Slemrod ``Are ``Real'' Responses to Taxes Simply Income Shifting Between Corporate and Personal Tax Bases?'', NBER Working Paper No. 6576, 2000. \href{http://www.nber.org/papers/w6576.pdf} {(web)}

IRS, Statistics of Income Division(2013) ``U.S. Individual Income Tax: Personal Exemptions and Lowest and Highest Tax Bracket''\href{http://elsa.berkeley.edu/~saez/course131/taxable-income4.pdf} {(web)}

Kleven, Henrik, Camille Landais, and Emmanuel Saez ``Taxation and International Mobility of Superstars: Evidence from the European Football Market,'' American Economic Review, 103(5), 2013. \href{http://eml.berkeley.edu/~saez/kleven-landais-saezAER13football.pdf} {(web)}

Kleven, Henrik Jacobsen, Camille Landais, Emmanuel Saez, and Esben Anton Schultz. ``Migration and Wage Effects of Taxing Top Earners: Evidence from the Foreigners' Tax Scheme in Denmark.'' Quarterly Journal of Economics 127(1),  (2014).\href{http://elsa.berkeley.edu/~saez/kleven-landais-saez-schultzQJE13danishscheme.pdf} {(web)}

Moretti, Enrico and Daniel Wilson 2017. ``The Effect of State Taxes on the Geographical Location of Top Earners: Evidence from Star Scientists'', American Economic Review 107(7), 1858-1903 \href{http://elsa.berkeley.edu/~saez/course/moretti-wilsonAER17mobility.pdf} {(web)}

Moretti, Enrico and Daniel Wilson 2019. ``Taxing Billionaires: Estate Taxes and the Geographical Location of the Ultra-Wealthy'',
NBER Working Paper No. 26387. \href{https://www.nber.org/papers/w26387.pdf} {(web)} 

Piketty, T. and E. Saez ``Income Inequality in the United States, 1913-1998'', Quarterly Journal of Economics, Vol. 116, (2003): 1-39. \href{http://links.jstor.org/stable/pdfplus/25053897.pdf} {(web)}

Piketty, Thomas, Emmanuel Saez, and Stefanie Stantcheva ``Optimal Taxation of Top Labor Incomes: A Tale of Three Elasticities,'' American Economic Journal: Economic Policy, 6(1), 2014.
\href{http://eml.berkeley.edu/~saez/piketty-saez-stantchevaAEJ14.pdf} {(web)}

Piketty, Thomas, Emmanuel Saez, and Gabriel Zucman,  ``Distributional National Accounts:
Methods and Estimates for the United States'', Quarterly Journal of Economics, 133(2), 553-609, 2018
\href{https://eml.berkeley.edu/~saez/PSZ2018QJE.pdf} {(web)}

Rauh, Joshua, and Ryan J. Shyu. 2019. ``Behavioral Responses to State Income Taxation of High Earners: Evidence from California.'' National Bureau of Economic Research Working Paper No. 26349.
\href{https://www.nber.org/papers/w26349.pdf} {(web)} 

Saez, Emmanuel �Taxing the Rich More: Preliminary Evidence from the 2013 Tax Increase,� Tax Policy and the Economy, ed. Robert Moffitt, (Cambridge: MIT Press), Volume 31, 2017. \href{https://eml.berkeley.edu/~saez/saezTPE17.pdf} {(web)} 

Saez, Emmanuel, Joel Slemrod, and Seth H. Giertz. "The Elasticity of Taxable Income with Respect to Marginal Tax Rates: A Critical Review." Journal of Economic Literature 50(1) (2012): 3-50. \href{http://elsa.berkeley.edu/~saez/saez-slemrod-giertzJEL12.pdf} {(web)}

Saez, Emmanuel and Gabriel Zucman. ``The Rise of Income and Wealth Inequality in America: Evidence from Distributional Macroeconomic Accounts,'' Journal of Economic Perspectives 34(4), Fall 2020, 3-26.
\href{https://eml.berkeley.edu/~saez/SaezZucman2020JEP.pdf}{(web)} 

Young, Cristobal, Charles Varner, Ithai Lurie, Richard Prisinzano, 2016 ``Millionaire Migration and the Taxation of the Elite: Evidence from Administrative Data'', American Sociological Review 81(3), 421--446
\href{http://elsa.berkeley.edu/~saez/course/Millionaire_Migration_and_the_Taxtion_of_the_Elite.pdf} {(web)}




}

\end{slide}

\end{document}
