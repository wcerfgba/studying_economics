\documentclass[landscape]{slides}

\usepackage[landscape]{geometry}

\usepackage{pdfpages}

\usepackage{hyperref}

\def\mathbi#1{\textbf{\em #1}}

\topmargin=-1.8cm \textheight=17cm \oddsidemargin=0cm
\evensidemargin=0cm \textwidth=22cm

\author{131 Undergraduate Public Economics \\ Emmanuel Saez \\ UC Berkeley}
\date{}

\title{Introduction \\ (Chapter 1, Gruber textbook)} \onlyslides{1-300}

\newenvironment{outline}{\renewcommand{\itemsep}{}}

\begin{document}

\begin{slide}
\maketitle
\end{slide}


\begin{slide}
\begin{center}
{\bf PUBLIC ECONOMICS DEFINITION}
\end{center}
Public Economics  = study
of the Role of the Government in the Economy

Government is instrumental in most aspects of economic life:

1) Government in charge of huge \textbf{regulatory} structure

2) \textbf{Taxes:} governments in advanced economies collect 30-50\% of National Income in taxes

3) \textbf{Expenditures:} taxes fund \textbf{public goods} (infrastructure, public order and safety, defense)
and \textbf{social state} (Education, Retirement benefits, Health care, Income support)

4) Macro-economic \textbf{stabilization} through central bank (interest rate, inflation control), fiscal stimulus, bailout policies

$\Rightarrow$ We pool a large share of our incomes through government
\end{slide}


\begin{slide}
\begin{center}
{\bf Bigger view on government (Saez 2021)}
\end{center}
Economists have a narrow minded view of individual behavior: purely selfish and economically rational interacting through markets
$\Rightarrow$ Limitation to fully understand \textbf{public economics}

Social interactions are critical for humans:  we naturally cooperate at many levels: families, workplaces, communities, nation states
with very strong/versatile in-group attachments

We produce in teams and then we have to split production $\Rightarrow$ We are cooperative and sensitive to distribution

Archaic human societies depended on social cooperation for protection and taking care of the young, sick, and old

$\Rightarrow$ Explains best why our modern nation states provide defense and education, health care, and retirement benefits

%$\Rightarrow$ Humans are ``social animals'' (like bees or ants) 

%Humans reveal their social nature from the size of their ``governments'' (informal and formal)

\end{slide}


\begin{slide}
\begin{center}
{\bf More modest role for economists}
\end{center}
Replacing social institutions by markets does not always work:

Education is primarily government funded: student loans work in economic theory but in practice end up being a huge lifetime burden.
For-profit education has a tendency to become a scam

Retirement benefits: Saving for your own retirement works in theory but in practice most people unable to do so unless institutions (government/employers) help them 

Health care: Health care relies heavily on government/employers support everywhere. People are not able to afford or shop rationally for health care 

Economists can still play a useful role in understanding when markets can help and how individualistic forces can undermine institutions

\end{slide}





%\begin{slide}
%\begin{center}
%{\bf Why Study Public Finance?}
%\end{center}
%
%Controversies about the proper role of the government raise the fundamental questions addressed by the branch of economics known as \emph{public finance} or \emph{public economics}.
%
%The goal of public finance is to \emph{understand the proper role of the government in the economy}.
%
%\end{slide}

%1.1 The Four Questions of Public Finance

\begin{slide}
\begin{center}
{\bf Three questions in public economics}
\end{center}

1) When should the government intervene in the economy?

%2) How might the government intervene?

2) What is the effect of those interventions on economic outcomes?

3) Why do governments choose to intervene in the way that they do?

\end{slide}

\begin{slide}
\begin{center}
{\bf When should the government intervene in the economy? Economists' traditional view: }
\end{center}

{\bf 1) Market Failures}: Market economy sometimes fails to deliver an outcome that is efficient

$\Rightarrow$ Government intervention may improve the situation


{\bf 2) Redistribution}: Market economy generates substantial inequality in economic resources
across individuals 

Inequality is an issue because we are ``social beings''

$\Rightarrow$ People willing to pool their resources (through government taxes and transfers) 
to help reduce inequality 

First part of the class focuses on Redistribution

Second part of the class focuses on Market Failures

\end{slide}

\begin{slide}
\begin{center}
{\bf Main Market Failures}
\end{center}

\textbf{1) Externalities:} (example: greenhouse carbon emissions) $\Rightarrow$ require
govt interventions (such as corrective taxation)

\textbf{2) Imperfect competition:} (example: monopoly) $\Rightarrow$  requires
regulation (typically studied in Industrial Organization)

\textbf{3) Imperfect or Asymmetric Information:} (example: health insurance
markets are subject to death spirals)

\textbf{4) Individual failures:} People do not behave as ``fully rational individuals''. This is analyzed
in behavioral economics a field in huge expansion (example:
myopic people may not save enough for retirement)

\end{slide}


%\begin{slide}
%\includepdf[pages={1}]{Gruber2e_ch01_attach.pdf}
%\end{slide}


\begin{slide}
\begin{center}
{\bf Inequality and Redistribution}
\end{center}

Even if market outcome is efficient, society might not be
happy with the market outcome because market equilibrium
might generate very high economic disparity across individuals

Governments use taxes and transfers to redistribute 
from rich to poor and reduce inequality

Redistribution through taxes and transfers might reduce
incentives to work (\textbf{efficiency costs})

$\Rightarrow$ 
Redistribution creates an \textbf{equity-efficiency trade-off}

Income inequality has soared in the
United States in recent decades, and has moved to the forefront
in the public debate (Piketty's 2014 book success, stats from Piketty-Saez-Zucman '18)

\end{slide}


\begin{slide}
\includepdf[pages={27, 29, 28}, scale=.9]{intro_ch01_new_attach.pdf}
\end{slide}

%\begin{slide}
%\begin{center}
%{\bf How Might the Government Intervene?}
%\end{center}
%
%{\bf 1) Tax or Subsidize Private Sale or Purchase:} Tax goods that are overproduced (e.g. carbon tax) and subsidized
%goods underproduced (e.g., flu shots subsidies)
%
%{\bf 2) Restrict or Mandate Private Sale or Purchase:}
%Restrict the private sale or purchase of overproduced goods (e.g. fuel efficiency requirements), or mandate the private purchase of underproduced goods  (e.g., auto insurance)
%
%{\bf 3) Public Provision:}
%The government can provide the good directly, in order to potentially attain the level of consumption that maximizes social welfare (example is National Defense)
%
%{\bf 4) Public Financing of Private Provision:}
%Government pays for the good but private sector supplies it (e.g., privately provided health insurance paid for by US government in Medicare-Medicaid)
%
%\end{slide}

\begin{slide}
\begin{center}
{\bf What Are the Effects of Alternative Interventions?}
\end{center}

{\bf 1) Direct Effects:} The effects of government interventions that would be predicted if individuals did not change their behavior in response to the interventions.

Direct effects are relatively easy to compute

{\bf 2) Indirect Effects:} The effects of government interventions that arise only because individuals change their behavior in
response to the interventions (sometimes called \textbf{unintended effects})

Empirical public economics analysis tries to estimate indirect effects to inform the policy debate

\textbf{Example:} increasing top income tax rates mechanically raises tax revenue but top earners might find ways to evade/avoid taxes, reducing tax revenue relative to mechanical calculation



\end{slide}

%\begin{slide}
%\includepdf[pages={3}]{intro_ch01_new_attach.pdf}
%\end{slide}

\begin{slide}
\begin{center}
{\bf Why Do Governments Do What They Do?}
\end{center}

{\bf Political economy}:
The theory of how the political process produces decisions that affect individuals and the economy

\textbf{Example:} Understanding how the level of taxes and spending is set through voting and voters'
preferences

\textbf{Public choice} is a sub-field of political economy from a Libertarian perspective that focuses on \textbf{government
failures} 

government failures = situations where the government does not act in the benefit of society
(e.g., government captured by a dictator or special interests)

\end{slide}

\begin{slide}
\begin{center}
{\bf Normative vs. Positive Public Economics}
\end{center}

{\bf Normative Public Economics:} Analysis of How Things Should be
(e.g., should the government intervene in health insurance market?
how high should taxes be?, etc.)

{\bf Positive Public Economics:} Analysis of How Things Really Are
(e.g., Does govt provided health care crowd out private health
care insurance? Do higher taxes reduce labor supply?)

Positive Public Economics is a required 1st step before we can
complete Normative Public Economics

Positive analysis is primarily empirical and Normative analysis is
primarily theoretical

%Positive Public Economics overlaps with Labor Economics

%{\bf Political Economy} is a positive analysis of govt outcomes


\end{slide}


%\begin{slide}
%\begin{center}
%{\bf Paternalism vs. Individual Failures}
%\end{center}
%
%In many situations, individuals may not or do not seem to act in
%their best interests [e.g., many individuals are not able to save
%for retirement by themselves]
%
%Two Polar Views on such situations:
%
%1) {\bf Paternalism [Libertarian View]} Individual
%failures do not exist and government wants to impose its
%own preferences against individuals' will
%
%
%2) {\bf Individual Failures [Behavioral Economics View]}
%Individual Failures exist: Self-control problems, Cognitive
%Limitations
%
%Distinguishing the 2 views: Under Paternalism,
%individuals are opposed to government interventions. If individuals understand they have failures, they will
%support govt interventions.
%
%\end{slide}


\begin{slide}
\begin{center}
{\bf Key Facts on Taxes and Spending}
\end{center}
\textbf{1) Government Growth:} Size of government relative to National Income grows dramatically
over the process of development from less than 10\% in less developed economies to 30-50\% in most
advanced economies

\textbf{2) Government Size Stable} in richest countries after 1980

\textbf{3) Government Growth} is due to the expansion of the \textbf{welfare state:} (a) public education, (b) public retirement benefits, (c) public health insurance, (d) income support programs

\textbf{4) Govt spending $>$ Taxes:} Most rich countries run deficits and have significant public debt (relative to GDP),
particularly during Great Recession of 2008-10 and Covid 2020

\end{slide}



\begin{slide}
\includepdf[pages={23, 22}, scale=1.05]{intro_ch01_new_attach.pdf}
\end{slide}


\begin{slide}
\begin{center}
{\bf DIFFERENT LEVELS OF GOVERNMENTS}
\end{center}
US Federal govt raises about 20\% of GDP in taxes
(and can run deficits)

State+Local govts raise about 10\% of GDP in taxes

Decentralized govt = a larger fraction of taxes/spending are decided at local level

%Decentralized govt give additional power to individuals who can also vote with their feet

%Tailors govt to local views and creates competition between local govts: If local govt is inefficient, residents can leave, putting the local govt ``out of business''

Decentralized govt can tailor policy to local views (example: California has more liberal policies than Texas)

Redistribution through taxes and transfers harder to achieve at local level (rich can leave local jurisdiction if local taxes are too high)
$\Rightarrow$ Local govts tend to do less redistribution

$\Rightarrow$ Conservatives/libertarians tend to prefer decentralized states 

\end{slide}

\begin{slide}
\includepdf[pages={20}]{intro_ch01_new_attach.pdf}
\end{slide}


\begin{slide}
\includepdf[pages={11}]{intro_ch01_new_attach.pdf}
\end{slide}


%\begin{slide}
%\begin{center}
%{\bf DISTRIBUTION OF SPENDING}
%\end{center}
%
%{\bf Public goods}:
%Goods for which the investment of any one individual benefits everyone in a larger group
%(examples: defense, police, roads).
%
%{\bf Social spending programs}:
%Government provision of insurance against adverse events to correct inequality and
%address failures in the private insurance market (examples: education, retirement benefits, public health insurance,
%unemployment insurance, disability insurance)
%
%Growth in government since 1900 mostly due to expansion of social spending: public education, public health benefits,
%retirement benefits, and income support programs
%
%\end{slide}

%\begin{slide}
%\includepdf[pages={12}]{intro_ch01_new_attach.pdf}
%\end{slide}

\begin{slide}
\begin{center}
{\bf DISTRIBUTION OF TAXES}
\end{center}

%In 2018, total taxes in the US were 28\% of national income

US Federal govt raises about 2/3 of total taxes, State+Local govt raises 1/3 of total taxes.

Main Federal taxes: (1) Individual income tax (40\% of Fed tax revenue), (2) payroll taxes on earnings (40\%), (3) corporate tax
(15\%)

Main State taxes: (1) real estate property taxes (30\% of state+local tax revenue), (2) sales and excise taxes (30\%), (3) individual and corporate state taxes (30\%)

Key questions: how are these taxes distributed by income groups (Saez-Zucman '19 book)? what impact do they have on the economy?

\end{slide}


\begin{slide}
\includepdf[pages={21}]{intro_ch01_new_attach.pdf}
\end{slide}

\begin{slide}
\begin{center}
{\bf REGULATORY ROLE OF THE GOVERNMENT}
\end{center}

Another critical role the government plays in all nations is that of \emph{regulating economic and social activities}.
Examples:

1) \textbf{Minimum wage} at the Federal level is \$7.25 (States can adopt higher min wages) $\Rightarrow$ 
Potential impact on inequality 

2) The \textbf{Food and Drug Administration (FDA)} regulates the labeling and safety of nearly all food products 
and approves drugs and medical devices to be sold to the public

3) The \textbf{Occupational Safety and Health Administration (OSHA)} is charged with regulating the workplace safety of 
American workers
%-The \textbf{Federal Communications Commission (FCC)} regulates interstate and international communications by radio, television, wire, satellite, and cable.\\

4) The \textbf{Environmental Protection Agency (EPA)} is charged with minimizing dangerous pollutants in the air, water, and food supplies
\end{slide}







\begin{slide}
\begin{center}
{\bf PUBLIC DEBATES OVER TAXES, HEALTH CARE, AND CLIMATE CHANGE}
\end{center}

Taxes, health care, and climate change are each the subject of debate, with both the ``liberal" and ``conservative" positions holding differing views in their approach to each problem.

{\bf Taxes:} Trump administration decreased taxes on corporations and individuals in 2018. 
Biden plans to increases taxes on the rich
%Democratic candidates want to tax the rich more. Some propose new progressive wealth taxes.

%
%{\bf Social Security:}
%Social Security is the single largest government expenditure program. The financing structure of this program is basically that today's young workers pay the retirement benefits of today's old.

{\bf Health Care:}
Up to 2013, 17-18\% of the non-elderly U.S. population not insured.
With Obamacare down to 10\%. Biden plans to strengthen Obamacare further.

%Some candidates propose Medicare for All.


{\bf Climate change:} Carbon emissions are generating global warming with potentially 
devastating future consequences (sea rise, extreme weather, agricultural output).
What should government do? Nothing (Trump) vs. Green New Deal 

\end{slide}

%\begin{slide}
%\includepdf[pages={24}]{intro_ch01_new_attach.pdf}
%\end{slide}

\begin{slide}
\includepdf[pages={25}]{intro_ch01_new_attach.pdf}
\end{slide}

\begin{slide}
\includepdf[pages={26}, scale=1.1]{intro_ch01_new_attach.pdf}
\end{slide}

%\begin{slide}
%\begin{center}
%{\bf CONCLUSION}
%\end{center}
%
%It is clear from the facts presented here that the government plays a central role in the lives of all Americans.
%
%It is also clear that there is ongoing disagreement about whether that role should expand, stay the same, or contract.
%
%The facts and arguments raised in this chapter provide a backdrop for thinking about the set of public finance issues that we explore in the remainder of the lectures.
%\end{slide}

\begin{slide}
\begin{center}
{\bf PROFESSOR SAEZ' RESEARCH}
\end{center}

Most of my research (available on my webpage) is in public economics:

1) Design of optimal tax policies and optimal transfer programs (theory, normative)

2) Analysis of the effects of taxes and transfers on individual behavior (empirical, positive)

3) Analysis of inequality overtime and across countries (empirical, descriptive)

I will discuss some of my research in this course when we cover the relevant topics


\end{slide}

\begin{slide}
\begin{center}
{\bf REFERENCES}
\end{center}
{\small

Jonathan Gruber, Public Finance and Public Policy, Fifth Edition, 2019 Worth Publishers, Chapter 1

%Kleven, Henrik, Claus Kreiner, and Emmanuel Saez ``Why Can Modern Governments Tax So Much? An Agency Model of Firms as Fiscal Intermediaries'', NBER Working Paper No. 15218, August 2009 \href{http://www.nber.org/papers/w15218.pdf}{(web)}

%National Center for Education Statistics ``Highlights from TIMSS 2007: Mathematics and science achievement of US fourth-and eighth-grade students in an international context.'' Institute of Education Sciences, US Department of Education, 2009.\href{http://elsa.berkeley.edu/~saez/course131/TIMSS07.pdf}{(web)}

Piketty, Thomas, \emph{Capital in the 21st Century},  Cambridge: Harvard University Press, 2014, Chapter 13,  
\href{http://piketty.pse.ens.fr/en/capital21c2}{(web)}

Piketty, Thomas, \emph{Capital and Ideology},  Cambridge: Harvard University Press, 2020,  Chapter 10,
\href{http://piketty.pse.ens.fr/en/ideology}{(web)}

%Piketty, Thomas and Emmanuel Saez ``Income Inequality in the United States, 1913-1998'', Quarterly Journal of Economics, 118(1), 2003, 1-39, series updated to 2012 in September 2013 \href{http://www.jstor.org/stable/pdfplus/25053897.pdf}{(web)}

Piketty, Thomas, Emmanuel Saez, and Gabriel Zucman,  ``Distributional National Accounts:
Methods and Estimates for the United States'', Quarterly Journal of Economics, 133(2), 553-609, 2018
\href{https://eml.berkeley.edu/~saez/PSZ2018QJE.pdf} {(web)}

Saez, Emmanuel  ``Public Economics and Inequality: Uncovering Our Social Nature'', AEA Papers and Proceedings, 121, 2021
\href{https://eml.berkeley.edu/~saez/saez-AEAlecture.pdf} {(web)}

Saez, Emmanuel and Gabriel Zucman. The Triumph of Injustice: How the Rich Dodge Taxes and How to Make them Pay, New York: W.W. Norton, 2019. 
\href{http://www.taxjusticenow.org} {(web)}


}
\end{slide}




\end{document}

