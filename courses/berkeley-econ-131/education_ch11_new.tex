\documentclass[landscape]{slides}

\usepackage[landscape]{geometry}

\usepackage{pdfpages}

\usepackage{hyperref}
\usepackage{amsmath}

\def\mathbi#1{\textbf{\em #1}}

\topmargin=-1.8cm \textheight=17cm \oddsidemargin=0cm
\evensidemargin=0cm \textwidth=22cm


\author{131 Undergraduate Public Economics \\ Emmanuel Saez \\ UC Berkeley}
\date{}

\title{Education} \onlyslides{1-300}

\newenvironment{outline}{\renewcommand{\itemsep}{}}

\begin{document}

\begin{slide}
\maketitle
\end{slide}

%\begin{slide}
%\begin{center}
%{\bf OUTLINE}
%\end{center}
%Chapter 11
%
%11.1 Why Should the Government Be Involved in Education?
%
%11.2 How Is the Government Involved in Education?
%
%11.3 Evidence on Competition in Education Markets
%
%11.4 Measuring the Returns to Education
%
%11.5 The Role of the Government in Higher Education
%
%11.6 Conclusion
%\end{slide}

%11.1 Why Should the Government Be Involved in Education?

%\begin{slide}
%\includepdf[pages={1}]{education_ch11_new_attach.pdf}
%\end{slide}

\begin{slide}
\begin{center}
{\bf Education}
\end{center}

Education is one of the 3 largest programs funded by government
(along with retirement and health)

All advanced economies fund most (80\% on average) of education  (pre-K, K-12, higher ed) through government

$\Rightarrow$ Education level highly dependent on govt policy (see OECD stats)

In US, 4.5\% of GDP or 1/7 of total government expenditure 

In US, 80\% of ed spending done at the state and local level

Focus of an extensive body of research in the rapidly expanding field of economics of education
\end{slide}

\begin{slide}
\includepdf[pages={21}]{education_ch11_new_attach.pdf}
\end{slide}

\begin{slide}
\includepdf[pages={19}]{education_ch11_new_attach.pdf}
\end{slide}

\begin{slide}
\includepdf[pages={20}]{education_ch11_new_attach.pdf}
\end{slide}





\begin{slide}
\begin{center}
{\bf Why Should the Government Be Involved in Education?}
\end{center}

\textbf{Fundamental reason:} education is long and costly (teachers+schools, US cost is \$15K/year-kid)
AND everybody needs it in modern economy 

$\Rightarrow$ without govt provision, low income families would not be able to afford it for their kids
(would hurt opportunity)

Governments created mass education in 19th-20th century [mandatory up to certain ages and 
hence publicly provided]

Played a big role in fostering economic development:

1) modern economy requires an educated workforce

2) empowers women and fertility transition in devo countries
	
	
\end{slide}


\begin{slide}
\includepdf[pages={24-26}, scale=.9]{education_ch11_new_attach.pdf}
\end{slide}


\begin{slide}
\begin{center}
{\bf Why Should the Government Be Involved in Education?}
\end{center}

For economists, ex-ante not obvious because education does not look like a public good

1) Returns to education are largely private

2) Education is excludable

$\Rightarrow$ we should expect students to invest roughly the optimal amount in their own education

\end{slide}


\begin{slide}
\begin{center}
{\bf Why Should the Government Be Involved in Education?}
\end{center}
Traditional motives pointed out by economists:

1) Externalities (productivity spillovers, crime, citizenship)

%2) Family failures: Divergence between parent and child preferences (some parents
%may not take good care of their children)

2) Borrowing constraints (poor but talented students may not be able to borrow against
future earnings to get an education)

3) (MOST IMPORTANT) Family and individual failures (to conform to standard econ model):

a) Some parents may not take good care of their children (public education provides opportunity for all)

b) Young adults might not do what is in their long-run interest due to self-control problems or lack of information

\end{slide}

%\begin{slide}
%\begin{center}
%{\bf EFFECT OF EDUCATION LEVELS ON \\ OTHER OUTCOMES}
%\end{center}
%
%A number of studies have assessed the impact of increased education on external benefits. Key findings include the following:\\
%-Higher levels of education are associated with an increased likelihood of participation in the political process.\\
%-Higher levels of education are associated with a lower likelihood of criminal activity.\\
%-Higher levels of education are associated with improved health of the people who received more education and of their children.\\
%-Higher levels of education of parents are associated with higher levels of education of their children.\\
%-Higher levels of education among workers are associated with higher rates of productivity of their coworkers.
%\end{slide}



\begin{slide}
\begin{center}
{\bf 1) Externalities of education on crime and voting}
\end{center}
%\small
\[ Crime_i= \alpha + \beta Educ_i + \varepsilon_i \]
Observational regression comparing
the educated vs. not-educated likely biased 
because propensity to crime $\varepsilon_i$ is negatively correlated with $Educ_i$.

Lochner and Moretti (2004) use as instrument changes in state compulsory attendance laws:
State T increases compulsory attendance from 9 to 10 years at time $t$, State C does not.

Can look at effect on education, and then look at effect on crime using Difference-in-difference

They show that an extra year of schooling reduces incarceration rates significantly (by about 10\%)

%0.1 pct point decline for white males relative to a mean of 1\%

%0.3 pct point decline for black males relative to mean of 3\%

%$\Rightarrow$ Gap in schooling between whites and blacks accounts for more than 1/4 difference in crime rates

%Social return to education exceeds private return by 25\% based purely on reduction in crime

Moretti, Mulligan, Oreopoulos (2003) find positive effects of education on likelihood of voting using same strategy

\end{slide}




%\begin{slide}
%\begin{center}
%{\bf Why Should the Government Be Involved in Education?}
%\end{center}
%
%{\bf 3) Credit Market Failures}
%
%Another market failure that may justify government intervention is the inability of families to borrow to finance education.
%
%In a world without government involvement, families would have to provide the money to buy their children's education from private schools.
%
%{\bf Educational credit market failure}:
%The failure of the credit market to make loans that would raise total social surplus by financing productive education.
%\end{slide}
%
%\begin{slide}
%\begin{center}
%{\bf Why Should the Government Be Involved in Education?}
%\end{center}
%
%{\bf 4) Failure to Maximize Family Utility}
%
%The reason governments may feel that loans are not a satisfactory solution to credit market failures is that they are concerned that parents would still not choose appropriate levels of education for their children.
%
%{\bf 5) Redistribution}
%
%In a privately financed education model, as long as education is a normal good (demand for which rises with income), higher-income families would provide more education for their children than would lower-income families.
%
%\emph{Income mobility}, whereby low-income people have a chance to raise their incomes, has long been a stated goal for most democratic societies, and public education provides a level playing field that promotes income mobility.
%\end{slide}

%11.2 How Is the Government Involved in Education?

%\begin{slide}
%\begin{center}
%{\bf 2) Divergences between parent and child preferences}
%\end{center}
%
%Hard to find direct evidence
%
%Duflo (2003) shows evidence that grandmothers spend more than grandfathers on female grandchildren
%
%Duflo (2003) uses pension reform in 1992 in South-Africa giving all Blacks (65+) a minimum pension
%when household income is low
%(before, only whites could get the pension under Apartheid)
%
%Duflo (2003) finds that pension availability improves the weight for height Z-score of female grandchildren (nutrition improves)
%but only when a grandmother gets the pension (and not when a grandfather does)
%
%$\Rightarrow$ Parents preferences matter for kids outcomes
%
%\end{slide}
%
%\begin{slide}
%\includepdf[pages={2}]{education_ch11_new_attach.pdf}
%\end{slide}


\begin{slide}
\begin{center}
{\bf 2) Borrowing Constraints: effects of loans}
\end{center}
If there are no borrowing constraints (and individuals are rational), current resources
should not matter for educational decisions: invest in education only if PDV benefits $>$ costs

Empirical evidence shows that availability of loans do matter suggesting that borrowing constraints
are an issue

Solis (2017) studies the effects of guaranteed loans on college attendance in Chile

Guaranteed loan is available if test score of student (equivalent of SAT for Chile) is above threshold
equal to 475.

Regression discontinuity design: discontinuity in loan availability translates into discontinuity in college
attendance

$\Rightarrow$ Very compelling evidence that loan availability matters 

\end{slide}

\begin{slide}
\includepdf[pages={13}]{education_ch11_new_attach.pdf}
\end{slide}

\begin{slide}
\includepdf[pages={27}]{education_ch11_new_attach.pdf}
\end{slide}


%\begin{slide}
%XX ADD SLIDE ON STUDENT LOANS VS GOVT FUNDING
%
%FREE COLLEGE FOR ALL POLICY PROPOSAL
%\end{slide}

%\begin{slide}
%\begin{center}
%{\bf 3) Behavioral motives (individual failures): high-school}
%\end{center}
%
%Rational education decision should be based on comparing returns to education (higher wage later in life) vs. costs of education
%(tuition and time) $\Rightarrow$ Requires that young individuals know the return to education
%
%Jensen (2010) shows that simply presenting information about rates of return to education changes behavior
%%(individual failures likely large)
%
%1) He uses survey data for eighth-grade boys in the Dominican
%Republic
%
%Finds that the perceived returns to secondary school are extremely low, despite high measured returns
%
%2) Then carries out randomized field experiment: Students at randomly selected schools given information on the higher measured returns completed on average 0.20 more
%years of school over the next four years than those who were not.
%\end{slide}
%
%
%
%\begin{slide}
%\includepdf[pages={4}]{education_ch11_new_attach.pdf}
%\end{slide}


\begin{slide}
\begin{center}
{\bf 3) Individual failures (college application tutoring)}
\end{center}

Carrell-Sacerdote (2017) carry out a field experiment in New Hampshire high-schools

College students from Dartmouth help senior high-schoolers to apply to college (weekly meetings
in Winter semester)

Randomization within high schools: select only 50\% of seniors 

Find large positive impact on women (+15 points likelihood of enrolling in college) but small effects 
on men

Also find a cash bonus for applying to colleges without tutorial does not have any impact $\Rightarrow$
Pure econ incentives not enough

$\Rightarrow$ Effects require time intensive tutorials (that parents/teachers typically should be providing)

\small Series of papers by Roland Fryer also show that paying K-12 kids to succeed does not work
(kids don't know how to succeed without guidance)
\end{slide}


%\begin{slide}
%\begin{center}
%{\bf 4) Behavioral motives (individual failures): university}
%\end{center}
%
%Hoxby-Avery '12 shows that high-achieving US students (top 2\% of SAT scores) from disadvantaged backgrounds (bottom 20\%) apply to weaker colleges rich high SAT peers
%
%Even though top schools offer generous financial aid to talented students from disadvantaged
%backgrounds
%
%Mechanism: poor talented kids in ``nowhere'' schools do not get good advice from family/local counselors
%$\Rightarrow$ End up at local college (often paying more than they would at top college)
%
%$\Rightarrow$ Informational failure prevents poor but talented kids to exploit their potential
%
%\small
%Hoxby-Turner '13 does randomized experiment providing personalized mailing info to talented students (relevant suggested
%applications, net-cost calculator) 
%
%$\Rightarrow$ Significant effect on number and quality of applications (but magnitude pretty small)
%
%
%\end{slide}
%
%\begin{slide}
%\includepdf[pages={6-8}]{education_ch11_new_attach.pdf}
%\end{slide}


%\begin{slide}
%\begin{center}
%{\bf Two Approaches to Education Reform}
%\end{center}
%
%\textbf{A) Individual-based interventions}
%
%1) Provide vouchers for K-12 schools
%
%2) Provide subsidies and loans to individuals for college costs
%
%\textbf{B) Improving the production process}
%
%1) Charter schools 
%
%2) Direct improvements in education production function, e.g. teacher quality and personnel policies
%
%3) Letting for-profit schools compete with existing public/non-profit schools
%\end{slide}
%
%
%
%
%
%%\begin{slide}
%%\includepdf[pages={2}]{Gruber2e_ch11_attach.pdf}
%%\end{slide}
%%
%\begin{slide}
%\begin{center}
%{\bf PUBLIC SCHOOLS AND VOUCHERS}
%\end{center}
%K-12 education is provided for free in \textbf{public schools} in the US (funded by taxes)
%
%If parents send child to private school, they have to pay private school tuition and do not get
%refunded for their taxes paid toward public school
%
%$\Rightarrow$ Strong incentives to use public schools
%
%{\bf Educational vouchers}: A fixed amount of money given by the government to families with school-age children, who can spend it at any type of school, public or private.
%
%A voucher effectively refunds parents for taxes paid if they do not use public schools
%$\Rightarrow$ Puts public and private schools in competition
%
%\end{slide}
%%
%%\begin{slide}
%%\includepdf[pages={3}]{Gruber2e_ch11_attach.pdf}
%%\end{slide}
%
%\begin{slide}
%\begin{center}
%{\bf RATIONALES FOR VOUCHERS}
%\end{center}
%
%{\bf (1) Consumer Sovereignty:} Vouchers allow families to more closely match their educational choices with their tastes.
%
%{\bf (2) Competition:} Vouchers allow the education market to benefit from the competitive pressures that make private markets function efficiently.
%\end{slide}
%
%%\begin{slide}
%%\begin{center}
%%{\bf PROBLEMS WITH EDUCATIONAL VOUCHERS}
%%\end{center}
%%
%%{\bf Vouchers Will Lead to Excessive School Specialization}
%%
%%The first argument made here for vouchers, that schools will tailor themselves to meet individual tastes, threatens to undercut the benefits of a common program.
%%
%%By trying to attract particular market segments, schools could give less attention to what are viewed as the central elements of education.
%%\end{slide}
%
%
%\begin{slide}
%\begin{center}
%{\bf PROBLEMS WITH EDUCATIONAL VOUCHERS}
%\end{center}
%
%{\bf 1) Vouchers Will Lead to Segregation:}
%Vouchers have the potential to reintroduce segregation along many dimensions, such as race, income, or child ability.
%
%{\bf 2) Vouchers Benefit kids from richer background:}
%The government would pay a portion of the private school costs that students and their families are currently paying themselves
%
%{\bf 3) The Education Market May not Be Competitive:} A large fraction of parents do not actively search the best possible
%school for their kids
%
%
%\end{slide}
%
%
%
%%11.3 Evidence on Competition in Education Markets
%
%\begin{slide}
%\begin{center}
%{\bf Estimating the Effects of Voucher Programs}
%\end{center}
%\small
%Rouse (1998) studied the effect of the Milwaukee voucher program on the achievement of students who used their vouchers to finance a move to private schools
%
%1)  She noted that one cannot directly compare students who do and do not use vouchers, since they may differ along many dimensions $\Rightarrow$ This selective use of vouchers would bias any comparison between the groups.
%
%2) Oversubscribed schools had to select randomly from all applicants, using a lottery (generates a quasi-experiment)
%$\Rightarrow$ Comparing lottery winners to losers, finds slight improvement in math scores (no difference in reading)
%
%In the United States, about 10\% of students are enrolled in private schools, a proportion that doubles or triples in the low-income developing world
%
%Angrist et al. 2002 shows that lottery voucher program had strong positive effects on education
%in Colombia
%
%External validity issue: voucher lottery strategy estimates effects of vouchers on families motivated to use them
%(entered the lottery). Unmotivated parents might not be affected by vouchers.
%\end{slide}
%
%%\begin{slide}
%%\includepdf[pages={4}]{Gruber2e_ch11_attach.pdf}
%%\end{slide}
%
%\begin{slide}
%\begin{center}
%{\bf Charter Schools}
%\end{center}
%
%Some school districts have not offered vouchers for private schools but have instead allowed students to choose freely among public schools.
%
%%{\bf Magnet schools}:
%%Special public schools set up to attract talented students or students interested in a particular subject or teaching style.
%
%{\bf Charter schools}:
%Schools financed with public funds that are not usually under the direct supervision of local school boards or subject to all state regulations for schools. Have more flexibility to recruit teachers / adjust hours / curriculum
%\end{slide}
%
%\begin{slide}
%\begin{center}
%{\bf Estimating the Effects of Charter Schools}
%\end{center}
%\small
%
%Oversubscribed charter schools also use a lottery to assign admissions
%
%Generates randomized experiment allowing to estimate the causal effect of charter schools
%by comparing lottery winners and lottery losers
%
%Angrist, Pathak, Walters AEJ'13 carry out a comprehensive analysis of charter schools effects
%in Massachusetts
%
%Find that urban charter schools boost
%achievement well beyond that of urban public school students, while non-urban charters
%reduce achievement from a higher baseline
%
%$\Rightarrow$ Charter schools can have a positive or negative impact depending on what they
%do
%
%Most effective approach to education: focus on instruction time, pupil comportment,
%selective teacher hiring, and focus on traditional
%math and reading skills.
%
%\end{slide}
%
%
%
%\begin{slide}
%\begin{center}
%{\bf School Accountability}
%\end{center}
%
%Making schools accountable for student performance can provide incentives for schools to increase the quality of the education they offer.
%
%Accountability programs can have two unintended effects:\\
%1) they can lead schools and teachers to ``teach to the test.''\\
%2) schools can manipulate the pool of test takers and the conditions under which they take tests to maximize success.
%
%{\bf No Child Left Behind} (Key Bush administration program in education):
%
%Evidence that it had small positive effects on test-scores but this could be due primarily to ``teach to the test'' effects
%
%In 2015, program turned over to the states in a more flexible form
%\end{slide}

%\begin{slide}
%\begin{center}
%{\bf BOTTOM LINE ON VOUCHERS AND \\ SCHOOL CHOICE}
%\end{center}
%
%There is also little evidence to support the notion that \textbf{public school choice} has major beneficial effects on outcomes.
%
%There is some evidence that vouchers improve the academic performance of students who move to private schools, particularly in nations/places where public schools are low quality
%
%The United States is currently in a phase of experimentation with both choice and accountability that will provide further evidence on the most effective way to improve elementary and secondary education.
%\end{slide}

%11.4 Measuring the Returns to Education

%\begin{slide}
%\begin{center}
%{\bf MEASURING THE RETURNS TO EDUCATION}
%\end{center}
%
%{\bf Returns to education}:
%The benefits that accrue to society when students get more schooling or when they get schooling from a higher-quality environment.
%
%{\bf Effects of education levels on productivity}
%
%There is a large literature that shows that more education leads to higher wages in the labor market:
%\[ Earnings_i= \alpha + \beta \cdot Education_i + \varepsilon_i \]
%There is substantial controversy, however, over the implications of this correlation $(\beta>0)$.
%\end{slide}


\begin{slide}
\begin{center}
{\bf Effects of Education on Earnings}%title?
\end{center}
Higher educated people earn more. Two explanations:

{\bf 1) Education as Human Capital Accumulation}

{\bf human capital}:
A person's stock of skills, which may be increased by education

In that scenario, education raises earnings because it improves productivity
of the educated person

{\bf 2) Education as a Screening Device}

{\bf screening}:
A model that suggests that education provides only a means of separating high-ability from low-ability individuals and does not actually improve skills.

In that scenario, education raises \textbf{individual} earnings but it does not improve productivity (rat-race)
\end{slide}

\begin{slide}
\begin{center}
{\bf MEASURING THE RETURNS TO EDUCATION}
\end{center}

{\bf Policy Implications}

Under the human capital model, government would want to support education or at least provide loans to individuals so that they can get more education and raise their productivity.

Under the screening model, however, the government would \textbf{not} want to support more education for any given individual.

{\bf Differentiating the theories}

Economists' findings: Most of the returns to education reflect accumulation of human capital rather than screening

Example: Clark-Martorell '14 show that barely getting a high-school degree in Texas has no visible impact on later earnings 

\end{slide}

\begin{slide}
\includepdf[pages={14,15}]{education_ch11_new_attach.pdf}
\end{slide}

\begin{slide}
\begin{center}
{\bf Causal Effect of Education on Earnings}
\end{center}

Basic observational approach:
\[ Earnings_i= \alpha + \beta \cdot Education_i + \varepsilon_i \]
Amounts to comparing the earnings of high vs. low ed people

Issue: ability to earn $\varepsilon_i$ might be correlated with education 

Two methods try to control for this bias in estimating the true human capital effects of education

\small
1)  Control for underlying ability by adding variables (e.g. SAT score) in the regression so that any remaining effect of education represents true productivity effects (omitted variable bias remains a concern)

2) Find exogenous variation in education (e.g., policy change induces more education for some group but not for another
group)

Although all of these approaches have some limitations, the result of the analysis is surprisingly consistent: each additional year of education raises wages by 7-10\% (4-year BA degree increases earnings by 35\% relative to High School)
\end{slide}


\begin{slide}
\begin{center}
{\bf Example: Causal Effect of Majoring in Economics}
\end{center}
Descriptive: Economics majors BAs earn more (\$90K) than non-econ BAs (\$66K) at age 40. Is this causal?

Bleemer and Mehta (2021) use GPA threshold requirement (2.8 in Econ 1 and 2) to major in economics at UC Santa Cruz to
estimate the causal effect of majoring in economics

Regression Discontinuity Design: compare students just below vs. just above 2.8 threshold 

1) Crossing the threshold increases Econ major likelihood by 36 points

2) Crossing the threshold increases wage earnings 5 years after graduation from \$47K to \$55K

$\Rightarrow$ Causal effect of majoring in economics is (\$55K-\$47K)/.36=\$22K which is an almost
50\% earnings premium

\end{slide}


\begin{slide}
\includepdf[pages={28,29}]{education_ch11_new_attach.pdf}
\end{slide}


\begin{slide}
\begin{center}
{\bf THE IMPACT OF SCHOOL QUALITY}
\end{center}

A number of approaches have been taken to estimate the impact of school quality on student test scores.

Two approaches have been used to address this issue: experimental data, and quasi-experimental using
policy changes

Findings suggest that the outcomes of efforts to improve school quality can be very dependent on the approach taken to improvements
\end{slide}

\begin{slide}
\begin{center}
{\bf Estimating the Effects of Class Size}
\end{center}


\textbf{Experimental example:} The state of Tennessee implemented Project STAR in 1985, randomly assigning 11,000 students (grades K--3) to small classes (13--17 students) or regular classes (22--25 students)

Krueger and Whitmore 2001 shows positive effects of small class size on test scores

Chetty et al. 2011 linked students to college enrollment and adult earnings data: finds small positive effects on college
enrollments and adult earnings.

Note: kids and teachers also randomly assigned across classes: strong class effects are visible (due to teachers or peers) and they have long-term effects on college and earnings

%\textbf{Quasi-experimental example:}  By the mid-1990s, California had the largest class sizes in the nation (29 students per class on average). The California state government in 1996 provided strong financial incentives for schools to reduce their class size to 20 students per class: not much effects on outcomes found but controversial
\end{slide}

\begin{slide}
\begin{center}
{\bf Estimating the Effects of Charter Schools}
\end{center}


\textbf{Charter schools} not subject to all state regulations for schools (flexibility to recruit teachers / adjust hours / curriculum)

Oversubscribed charter schools use lottery for admissions

Creates randomized experiment to estimate the causal effect of charter schools
by comparing lottery winners vs. losers

Angrist, Pathak, Walters AEJ'13 carry out a comprehensive analysis of charter schools effects
in Massachusetts

Find that urban charter schools boost
achievement well beyond that of urban public school students, while non-urban charters
reduce achievement from a higher baseline

\small
$\Rightarrow$ Charter schools can have a positive or negative impact depending on what they
do

Most effective approach to education: focus on instruction time, pupil comportment,
selective teacher hiring, and focus on traditional
math and reading skills.

\end{slide}



%\begin{slide}
%\includepdf[pages={6}]{Gruber2e_ch11_attach.pdf}
%\end{slide}

%11.5 The Role of the Government in Higher Education



%\begin{slide}
%\begin{center}
%{\bf Current Government Role in Higher Education}
%\end{center}
%
%{\bf 1. State Provision: } The primary form (over 80\%) of government financing of higher education is direct provision of higher education through locally and state-supported colleges and universities.
%
%{\bf 2. Pell Grants:} Subsidy to higher education administered by the federal government that provides grants to low-income families to pay for their educational expenditures.
%%\end{slide}
%
%{\bf 3. Loans:} {\bf (a) direct student loans}:
%Loans taken directly from the Department of Education.
%{\bf (b) guaranteed student loans}:
%Loans taken from private banks for which the banks are guaranteed repayment by the govt.
%
%{\bf 4. Tax Relief:} Tax credits for higher education tuition costs
%
%%The final way in which the government finances higher education is through a series of tax breaks for college-goers and their families.
%
%
%%For students who qualify on income and asset grounds, the government subsidizes the loan cost to students by\\
%%(a) Guaranteeing a low interest rate.\\
%%(b) Allowing students to defer repayment of the loan until they have graduated.
%\end{slide}

%\begin{slide}
%\includepdf[pages={9}]{education_ch11_new_attach.pdf}
%\end{slide}

%\begin{slide}
%\begin{center}
%{\bf What Is the Market Failure in Higher Education?}
%\end{center}
%If individuals are rational, the borrowing constraint market failure can be addressed solely with government supported
%loans
%
%However, if individuals are not rational (self-control problems, myopia, lack of information), even government supported
%loans might not be enough to motivate individuals to acquire higher education
%
%$\Rightarrow$ Direct tuition subsidies might be more effective
%
%$\Rightarrow$ Direct help with applications
%
%
%\end{slide}
%



%\begin{slide}
%\begin{center}
%{\bf Effects of tuition aid on attending college}
%\end{center}
%
%Dynarski (2003) studies elimination of Social Security Administration (SSA) program to provide tuition aid to students with parents deceased or disabled SSA beneficiaries in 1982
%
%DD analysis: Compare college attendance of kids with deceased father (Treatment group) to kids with father alive
%(control group),
%before 1982 vs. after 1982
%
%Finds very large 20 percentage point impact of program on college attendance
%$\Rightarrow$ Small changes in tuition costs have dramatic impact suggesting that borrowing constraints
%matter
%
%\end{slide}
%
%
%
%\begin{slide}
%\includepdf[pages={3}]{education_ch11_new_attach.pdf}
%\end{slide}

%\begin{slide}
%\begin{center}
%{\bf Effects of cash allowance on attending college in France}
%\end{center}
%
%Fack and Grenet (2014) study the effects of aid to students based on parental income in France
%
%Level of aid is a discontinuous function of parental income 
%
%Regression discontinuity design: does the discontinuity in aid translate into a discontinuity in college
%attendance? YES
%
%$\Rightarrow$ Very compelling evidence that financial aid for higher education matters
%
%\end{slide}
%
%\begin{slide}
%\includepdf[pages={11}]{education_ch11_new_attach.pdf}
%\end{slide}
%
%\begin{slide}
%\includepdf[pages={12}]{education_ch11_new_attach.pdf}
%\end{slide}
%




\begin{slide}
\begin{center}
{\bf Role of Government in supply of Higher Education}
\end{center}
Private non-profit universities have inelastic supply (e.g., fixed student bodies at top schools such as Harvard)

Historically, expansion of supply was carried out by public institutions (state universities and
community colleges): Example: 1960 Master plan for California with 3-tier system (Community, State, University of California)

\small
Government push also central to increase attendance: GI Bill after WWII/Korea War increased college attendance by 15-20
points for men born 1921-1933 %(Stanley QJE'03)
\normalsize

Recently, states have retreated, student loans have exploded and supply has been provided by for-profit schools (get about 10\% of total
enrollment today)

\small
Deming-Goldin-Katz '12 show that for-profit schools provide little benefits, charge a lot, and are savvy
at exploiting Fed Pell Grants and saddle students with debt. Worse when for-profits are taken over by private equity.

$\Rightarrow$ Symptom of market failure due to individual failures/lack of information
\end{slide}

\begin{slide}
\includepdf[pages={10}]{education_ch11_new_attach.pdf}
\end{slide}


\begin{slide}
\begin{center}
{\bf Role of Higher Education in Intergenerational Mobilily}
\end{center}
Chetty et al. '20 compile college level statistics  on parental income and student earnings outcomes. 
Data online at \href{http://www.equality-of-opportunity.org/data/index.html#college}{(web)}. %Four key findings:

\textbf{1) Access:} Huge variation in access across schools: Ivy league has more kids from top 1\% families than from bottom 50\% 
\small Giving poor kids an SAT point boost in admissions (as done for legacy students) could close  gap and increase
intergenerational mobility \normalsize

\textbf{2) Trends:} fraction poor kids stagnated in top schools (in spite of more financial aid) and dropped at best public schools and community
colleges 
 
\textbf{3) Outcomes:} Within good colleges, outcomes of poor vs. rich kids are similar $\Rightarrow$ college is the ticket to opportunity

\textbf{4) Mobility rates:} Large discrepancies across colleges in fraction of students who come from bottom 20\% and reach top 20\% (=mobility rate) 

\end{slide}

\begin{slide}
\includepdf[pages={22, 23}]{education_ch11_new_attach.pdf}
\end{slide}



\begin{slide}
\includepdf[pages={18, 16-17}]{education_ch11_new_attach.pdf}
\end{slide}



%\begin{slide}
%\begin{center}
%{\bf CONCLUSION}
%\end{center}
%
%The provision of education, an impure public good, is one of the most important governmental functions in the United States and around the world.
%
%The optimal amount of government intervention in education markets depends on the extent of market/individual
%failures in private provision of education and on the public/private returns to education
%
%\end{slide}


\begin{slide}
\begin{center}
{\bf REFERENCES}
\end{center}
{\small

Jonathan Gruber, Public Finance and Public Policy, Fourth Edition, 2019 Worth Publishers, Chapter 11

Angrist, Joshua, et al. ``Vouchers for Private Schooling in Colombia: Evidence from a Randomized Natural Experiment.'' American Economic Review 92(5), (2002), 1535--1558.\href{http://elsa.berkeley.edu/~saez/course131/angristetalAER02.pdf}{(web)}

Angrist, Joshua, Parag A. Pathak, and Christopher R. Walters.``Explaining Charter School Effectiveness.'' American Economic Journal: Applied Economics, 5.4 (2013): 1-27.\href{http://elsa.berkeley.edu/~saez/course131/Angrist-Pathak-Walters13.pdf}{(web)}

Bleemer, Zachary  and Aashish Mehta. 2021. ``Will Studying Economics Make You Rich? A Regression Discontinuity Analysis of the Returns to College Major''. American Economic Journal: Applied Economics, forthcoming.
\href{http://elsa.berkeley.edu/~saez/course131/bleemerAEJ21.pdf}{(web)}

Carrell, Scott E., and Bruce Sacerdote. ``Why Do College Going Interventions Work?'' American Economic Journal: Applied, 9(3), 124-151\href{http://elsa.berkeley.edu/~saez/course131/carrell-sacerdoteAEJ17.pdf}{(web)}

Chetty, Raj, John N. Friedman, Nathaniel Hilger, Emmanuel Saez, Diane Whitmore Schanzenbach, and Danny Yagan. ``How does your kindergarten classroom affect your earnings? Evidence from Project STAR.'' Quarterly Journal of Economics 126, no. 4 (2011): 1593-1660.\href{http://elsa.berkeley.edu/~saez/chettyetalQcarrE11star.pdf}{(web)}

Chetty, Raj, John Friedman, Emmanuel Saez, Nicholas Turner, Danny Yagan. ``Income Segregation and Intergenerational Mobility Across Colleges in the United States'' Quarterly Journal of Economics (2020).\href{https://eml.berkeley.edu/~saez/chettyetalQJE20college.pdf}{(web)}

Clark, Damon  and Paco Martorell ``The Signaling Value of a High School Diploma'' 
\emph{Journal of Political Economy} 122(2), 282-318.
\href{http://www.jstor.org/stable/pdf/10.1086/675238.pdf}{(web)}

Dancy, Kim, and Ben Barrett. "Living on Credit? An Overview of Student Borrowing for Non-Tuition Expenses." New America (Washington, DC, August 2018). \href{http://elsa.berkeley.edu/~saez/course131/LivingonCredit18.pdf}{(web)}

Deming, David, Claudia Goldin, and Lawrence F. Katz. ``The For-Profit Postsecondary School Sector: Nimble Critters or Agile Predators?'' Journal of Economic Perspectives (2012), vol. 26(1), pages 139-64. \href{http://www.nber.org/papers/w17710.pdf}{(web)}

Duflo, Esther, ``Grandmothers and Granddaughters: Old?Age Pensions and Intrahousehold Allocation in South Africa'' World Bank Economic Review, 17(1) (2003): 1--25.\href{http://elsa.berkeley.edu/~saez/course131/duflo03.pdf}{(web)}

Dynarski, Susan M. ``Does Aid Matter? Measuring the Effect of Student Aid on College Attendance and Completion.'' American Economic Review 93.1 (2003): 279-288.\href{http://www.jstor.org/stable/pdfplus/3132174.pdf}{(web)}

Eaton, Charlie, Sabrina T. Howell, and Constantine Yannelis. ``When Investor Incentives and Consumer Interests Diverge: Private Equity in Higher Education.'' Review of Financial Studies 33, no. 9 (2020): 4024-4060.
\href{http://elsa.berkeley.edu/~saez/course131/Eatonetal2020privateequity.pdf}{(web)}

Fack, Gabrielle, and Julien Grenet, ``Improving College Access and Success for Low-Income Students: Evidence from a Large French Need-based Grant Program,'' American Economic Journal: Applied Economics, 2014.
\href{http://elsa.berkeley.edu/~saez/course131/fack-grenet.pdf}{(web)}

Goldin, Claudia,  Lawrence Katz, Ilyana Kuziemko ``The Homecoming of American College Women: The Reversal of the College Gender Gap'', Journal of Economic Perspectives, 2006.
\href{http://elsa.berkeley.edu/~saez/course131/gkk_jep.pdf}{(web)}

Hoxby, Caroline, and Christopher Avery. ``The Missing``One-Offs'': The Hidden Supply of High-Achieving, Low Income Students.'' No. w18586. National Bureau of Economic Research, 2012.\href{http://www.nber.org/papers/w18586.pdf}{(web)}

Hoxby, Caroline, and Sarah Turner. ``Expanding College Opportunities for High-Achieving, Low Income Students.'' SIEPR Discussion Paper No. 12-014 (2013).\href{http://elsa.berkeley.edu/~saez/course131/Hoxby-Turner13.pdf}{(web)}

Jensen, Robert. ``The (perceived) returns to education and the demand for schooling.'' The Quarterly Journal of Economics 125.2 (2010): 515-548.\href{http://elsa.berkeley.edu/~saez/course131/Jensen10.pdf}{(web)}

Krueger, Alan B., and Diane M. Whitmore. ``The effect of attending a small class in the early grades on college-test taking and middle school test results: Evidence from Project STAR.'' Economic Journal 111.468 (2001): 1-28.\href{http://elsa.berkeley.edu/~saez/course131/Krueger-Whitmore01.pdf}{web}

Lochner, Lance, and Enrico Moretti. ``The Effect of Education on Crime: Evidence from Prison Inmates, Arrests, and Self-Reports.'' American Economic Review 94.1 (2004): 155-189.\href{http://www.jstor.org/stable/pdfplus/3592774.pdf}{(web)}

Milligan, Kevin, Enrico Moretti, and Philip Oreopoulos. ``Does education improve citizenship? Evidence from the United States and the United Kingdom.'' Journal of Public Economics 88.9 (2004): 1667-1695.\href{http://elsa.berkeley.edu/~saez/course131/Moretti-Mulligan-Oreopoulos03.pdf}{(web)}

Rouse, Cecilia Elena. ``Private school vouchers and student achievement: An evaluation of the Milwaukee parental choice program.'' Quarterly Journal of Economics 113.2 (1998): 553-602.\href{http://elsa.berkeley.edu/~saez/course131/rouse98.pdf}{(web)}

Saez, Emmanuel  ``Public Economics and Inequality: Uncovering Our Social Nature'', AEA Papers and Proceedings, 121, 2021
\href{https://eml.berkeley.edu/~saez/saez-AEAlecture.pdf} {(web)}


Solis, Alex ``Credit Access and College Enrollment'', Journal of Political Economy 125(2), 2017: 562-622.
\href{http://elsa.berkeley.edu/~saez/course131/solis.pdf}{(web)}

Stanley, Marcus. ``College education and the midcentury GI Bills.'' Quarterly Journal of Economics 118.2 (2003): 671-708.\href{http://www.jstor.org/stable/pdfplus/25053917.pdf}{(web)}

}

\end{slide}

\end{document}
