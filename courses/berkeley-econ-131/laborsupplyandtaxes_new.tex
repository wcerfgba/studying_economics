\documentclass[landscape]{slides}

\usepackage[landscape]{geometry}

\usepackage{pdfpages}

\usepackage{hyperref}
\usepackage{amsmath}

\def\mathbi#1{\textbf{\em #1}}

\topmargin=-1.8cm \textheight=17cm \oddsidemargin=0cm
\evensidemargin=0cm \textwidth=22cm


\author{131 Undergraduate Public Economics \\ Emmanuel Saez \\ UC Berkeley}
\date{}


\title{Labor Supply Responses to Taxes and Transfers} \onlyslides{1-300}

\newenvironment{outline}{\renewcommand{\itemsep}{}}

\begin{document}

\begin{slide}
\maketitle
\end{slide}

\begin{slide}
\begin{center}
{\bf MOTIVATION}
\end{center}
1) Labor supply responses to taxation are of fundamental
importance for income tax policy [efficiency costs and optimal tax
formulas]

2) Labor supply responses along many dimensions:

(a) Intensive: hours of work on the job, intensity of work,
occupational choice [including education]

(b) Extensive: whether to work or not [e.g., retirement and
migration decisions]

3) Reported earnings for tax purposes can also vary due to (a) tax
avoidance [legal tax minimization], (b) tax evasion [illegal
under-reporting]

4) Different responses in short-run and long-run: long-run
response most important for policy but hardest to estimate
\end{slide}


\begin{slide}
\begin{center}
{\bf STATIC MODEL: SETUP}
\end{center}
Baseline model (same as previous lecture):

Let $c$ denote consumption and $l$ hours worked, utility $u(c,l)$
increases with $c$, and decreases with $l$

Individual earns wage $w$ per hour (net of taxes) and has $R$ in
non-labor income

Individual solves \[\max_{c,l} u(c,l) \text{ subject to } c=wl+R \]
\end{slide}


\begin{slide}
\begin{center}
{\bf LABOR SUPPLY BEHAVIOR}
\end{center}
FOC: $w \partial u/ \partial c + \partial u/ \partial l =0$ defines uncompensated (Marshallian)  labor
supply function $l^u(w,R)$

Uncompensated elasticity of labor supply: $\varepsilon^u = (w/l)
\partial l^u/ \partial w$ [\% change in hours when net wage $w
$ increases by 1\%]

Income effect parameter: $\eta = w \partial l / \partial R \leq
0$: \$ increase in earnings if person receives \$1 extra in
non-labor income

Compensated (Hicksian) labor supply function $l^c(w,u)$ which
minimizes cost $w l - c$ st to constraint $u(c,l) \geq u$.

Compensated elasticity of labor supply: $\varepsilon^c = (w/l)
\partial l^c/ \partial w>0$

Slutsky equation: $\partial l/ \partial w = \partial l^c/ \partial
w +l  \partial l /\partial R$ $\Rightarrow$ $\varepsilon^u =
\varepsilon^c + \eta $

\end{slide}




%\begin{slide}
%\begin{center}
%{\bf IMPORTANT SPECIAL CASE: NO INCOME EFFECTS}
%\end{center}
%Quasi-linear utility function $u(c,l)=c - h(l)$
%
%$\max_l wl+y -h(l)$ $\Rightarrow$ $h'(l)=w$
%
%$\Rightarrow$ Marshallian $l^u(w,y)=l(w)$ labor supply independent
%of $y$
%
%$\Rightarrow$ Hicksian  $l^c(w,u)=l(w)$ labor supply independent
%of $y$ [parallel indifference curves]
%
%$\Rightarrow$ Identical uncompensated and compensated labor supply
%
%$\Rightarrow$  $\eta=0$ and $\epsilon^c = \epsilon^u >0$
%
%Iso-elastic utility function: $u(c,l)=c-a\frac{l^{1+1/\varepsilon }}{%
%^{1+1/\varepsilon }}$ $\Rightarrow$ $w =C \cdot l^{\varepsilon}$
%\end{slide}

\begin{slide}
\begin{center}
{\bf BASIC CROSS SECTION ESTIMATION}
\end{center}
Data on hours or work, wage rates, non-labor income started
becoming available in the 1960s when first micro surveys and
computers appeared:

Simple OLS (Ordinary Least Square) regression:
$$l_i = \alpha + \beta w_i + \gamma R_i + X_i \delta +
\epsilon_i$$

$w_i$ is the net-of-tax wage rate

$R_i$ measures non-labor income [including spousal earnings for
couples]

$X_i$ are demographic controls [age, experience, education, etc.]

$\beta$ measures uncompensated wage effects, and $\gamma$ measures income
effects [can be converted to $\varepsilon^u$, $\eta$]
\end{slide}

\begin{slide}
\begin{center}
{\bf BASIC CROSS SECTION RESULTS}
\end{center}

{\bf 1. Male workers} [primary earners when married] (Pencavel,
1986 survey):

a) Small effects $\varepsilon^u=0$, $\eta=-0.1$, $\varepsilon^c=0.1$
with some variation across estimates

{\bf 2. Female workers} [secondary earners when married]
(Killingsworth and Heckman, 1986):

Much larger elasticities on average, with larger variations across
studies. Elasticities go from zero to over one. Average around
0.5. Significant income effects as well

Female labor supply elasticities have declined overtime as women
become more attached to labor market (Blau-Kahn JOLE'07)
\end{slide}


\begin{slide}
\begin{center}
{\bf ISSUE WITH OLS REGRESSION: \\ $w_i$ correlated with tastes for work $\epsilon_i$}
\end{center}
\[ l_i = \alpha + \beta w_i + \epsilon_i \]
Identification is based on cross-sectional variation in $w_i$:
comparing hours of work of highly skilled individuals (high $w_i$)
to hours of work of low skilled individuals (low $w_i$)

If highly skilled workers have more taste for work (independent of
the wage effect), then $\epsilon_i$ is positively correlated with
$w_i$ leading to an upward bias in OLS regression

Plausible scenario: hard workers acquire better education and
hence have higher wages

Controlling for $X_i$ can help but can never be sure that we have
controlled for all the factors correlated with $w_i$ and tastes
for work: {\bf Omitted variable bias} $\Rightarrow$ Tax changes
provide more compelling identification
\end{slide}




%\begin{slide}
%\begin{center}
%{\bf Natural Experiment Labor Supply Literature}
%\end{center}
%Literature exploits variation in taxes/transfers to estimate Hours
%and Participation Elasticities
%
%1) Large literature in labor/Public economics estimates effects of
%taxes and wages on hours worked and participation
%
%2) Now discuss some estimates from this older literature
%\end{slide}

\begin{slide}
\begin{center}
{\bf Negative Income Tax (NIT) Experiments}
\end{center}

1) Best identification method: exogenously
increase the tax rate / non-labor income with a {\bf randomized experiment}

2)  NIT experiment conducted in 1960s/70s in Denver, Seattle, and
other cities

3) First major social experiment in U.S. designed to test proposed
transfer policy reform

4) Lump-sum transfers $G$ combined with a steep
phaseout rate $\tau $ (50\%-80\%) [based on family earnings] for 3 or 5 years.

5) Analysis by Rees (1974), Munnell (1986) book, Ashenfelter and\
Plant JOLE'90, and others

6) Several groups, with randomization within each; approx. N = 75
households in each group
\end{slide}

\begin{slide}
\includepdf[pages={1}]{laborsupplyandtaxes_new_attach.pdf}
\end{slide}

\begin{slide}
\includepdf[pages={59-60}]{laborsupplyandtaxes_new_attach.pdf}
\end{slide}

%\begin{slide}
%\begin{center}
%{\bf NIT Experiments: Ashenfelter and Plant 1990}
%\end{center}
%1) Present non-parametric evidence of labor supply effects
%
%2) Compare implied benefit payments to treated vs. control
%households
%
%3) Difference in benefit payments reflects aggregates hours and
%participation responses
%
%4) This is the relevant parameter for expenditure calculations and
%potentially for welfare analysis (revenue method of calculating
%DWL)
%
%%5) Shortcoming: approach does not decompose estimates into income
%%and substitution effects and intensive vs. extensive margin
%%
%%6) Hard to identify the key elasticity relevant for policy
%%purposes and predict labor supply effect of other programs
%\end{slide}
%
%\begin{slide}
%\includepdf[pages={44}]{laborsupply_attach.pdf}
%\end{slide}


\begin{slide}
\begin{center}
{\bf NIT Experiments: Findings}
\end{center}

1) Significant labor supply response but small overall

2) Implied earnings elasticity for males around 0.1

3) Implied earnings elasticity for married women around 0.5

4) Response of married women is concentrated along the extensive margin

5) Earnings of treated married women who were working before the
experiment did not change much

\end{slide}

%\begin{slide}
%\begin{center}
%{\bf Problems with NIT Experimental Design}
%\end{center}
%Estimates from NIT\ not considered fully credible due to several
%shortcomings:
%
%1) {\bf Self reported earnings:} Treatments had financial
%incentives to under-report earnings $\Rightarrow$ Lesson: need to
%match with administrative records [Greenberg and Halsey JOLE'83]
%
%2) {\bf Selective attrition:}
%
%After initial year, data collected based on voluntary income
%reports by families $\Rightarrow$ Those in control group
%had much less incentive to stick around and keep
%reporting
%$\Rightarrow$ No longer a random sample of treatment + controls
%[Ashenfelter-Plant JOLE'90]
%
%3) Response might be smaller than real reform because of {\bf General
%Equilibrium} effects
%\end{slide}


\begin{slide}
\begin{center}
{\bf From true experiment to ``natural experiments'': \\ Estimating income effects with lottery winnings}
\end{center}
True experiments are costly to implement and hence rare

However, real economic world (nature) provides variation that can
be exploited to estimate behavioral responses $\Rightarrow$
``Natural Experiments''

Natural experiments sometimes come very close to true experiments:
Imbens, Rubin, Sacerdote AER '01 did a survey of lottery winners
and non-winners matched to Social Security administrative data to
estimate income effects

Lottery generates random assignment conditional on playing

Find significant but relatively small income effects: $\eta = w
\partial l/\partial R$ between -0.05 and -0.10

Identification threat: differential response-rate among groups
\end{slide}

\begin{slide}
\includepdf[pages={2,3}]{laborsupplyandtaxes_new_attach.pdf}
\end{slide}


\begin{slide}
\begin{center}
{\bf Labor Supply Substitution Effects: \\ Tax Free Second Jobs in Germany}
\end{center}

In 2003, Germany made secondary jobs (paying less than 400 Euros/month) tax free:
amounts to a 20-60\% subsidy on second job earnings: substitution labor supply effect

Tazhitdinova '20 uses social security admin monthly earnings data

Fraction of population holding second jobs increased sharply (from 2.5\% to 6-7\%)
with bigger response overtime

Finds no offsetting effect on primary earnings $\Rightarrow$ People did work more

Looks like a big labor supply response but likely happened because employers 
willing to create lots of mini-jobs to accommodate supply

\end{slide}

\begin{slide}
\includepdf[pages={52}]{laborsupplyandtaxes_new_attach.pdf}
\end{slide}




%\begin{slide}
%\begin{center}
%{\bf Tax Reform Variation (Eissa 1995)}
%\end{center}
%1) Modern studies use tax changes as \textquotedblleft natural
%experiments\textquotedblright
%
%2) Representative example:\ Eissa (1995)
%
%3) Uses the Tax Reform Act of 1986 to identify the effect of MTRs
%on labor force participation and hours of married women
%
%4) TRA 1986 cut top income MTR from 50\% to 28\% from 1986 to 1988
%but did not significantly change tax rates for the middle class
%
%5) Substantially increased incentives to work of wives of high
%income husbands relative wives of middle income husbands
%
%\end{slide}
%
%\begin{slide}
%\includepdf[pages={4}]{laborsupplyandtaxes_new_attach.pdf}
%\end{slide}
%
%\begin{slide}
%\begin{center}
%{\bf Difference-in-Difference (DD) Methodology:}
%\end{center}
%{\bf Step 1: Simple Difference}
%
%Outcome: $LFP$ (labor force participation)
%
%Two groups: Treatment group (T) which faces a change [women
%married to high income husbands] and control group (C) which does
%not [women married to middle income husband]
%
%Simple Difference estimate: $D=LFP^T-LFP^C$ captures treatment
%effect if absent the treatment, $LFP$ equal across 2 groups
%
%Note: assumption always holds when $T$ and $C$ status is randomly
%assigned
%
%Test for : Compare $LFP$ before treatment happened
%$D_B=LFP_B^T-LFP_B^C$
%
%\end{slide}
%
%
%\begin{slide}
%\begin{center}
%{\bf Difference-in-Difference (DD) Methodology:}
%\end{center}
%{\bf Step 2: Difference-in-Difference (DD)}
%
%If $D_B \neq 0$, can estimate DD: $$DD=D_A-D_B=LFP_A^T-LFP_A^C -
%[LFP_B^T-LFP_B^C]$$ (A = after reform, B = before reform)
%
%DD is unbiased if {\bf parallel trend} assumption holds:
%
%Absent the change, difference across $T$ and $C$ would have stayed
%the same before and after
%
%DD most convincing when groups are very similar to start with
%[closer to randomized experiment]
%
%Should always test DD using data from more periods and plot the two time
%series to check parallel trend assumption [as ub Imbens et al. 2001]
%
%\end{slide}
%
%
%
%
%
%\begin{slide}
%\includepdf[pages={5-7}]{laborsupplyandtaxes_new_attach.pdf}
%\end{slide}
%
%
%\begin{slide}%
%\begin{center}
%{\bf Eissa 1995: Results}
%\end{center}
%1) Participation elasticity around 0.4 but large standard errors
%
%2) Hours elasticity of 0.6
%
%3) Total elasticity (unconditional hours) is $0.4+0.6=1$
%
%\end{slide}
%
%\begin{slide}
%\begin{center}
%{\bf Eissa 1995: Caveats}
%\end{center}
%1) Does the parallel trend assumption hold? Potential story biasing
%the result:
%
%Trend toward \textquotedblleft power couples\textquotedblright\
%and thus DD might not be due to taxes: In 1983-1985, professionals
%had non-working spouses, In 1989-1991, professionals married to
%professionals [and no change for middle class]
%
%2) Liebman and Saez (2006) plot full time-series CPS plot and show
%that Eissa's results are not robust using admin data (SSA\ matched
%to\ SIPP) [unfortunately, IRS public tax data does not break down
%earnings within couples]
%
%\end{slide}
%
%\begin{slide}
%\includepdf[pages={8}]{laborsupplyandtaxes_new_attach.pdf}
%\end{slide}
%


\begin{slide}%
\begin{center}
{\bf Responses to Low-Income Transfer Programs}
\end{center}

1) Particular interest in treatment of low incomes in a
progressive tax/transfer system: are they responsive to incentives?

2) Complicated set of transfer programs in\ US

a) In-kind: food stamps (SNAP), Medicaid, public housing, job training,
education subsidies

b) Cash:\ TANF, EITC, SSI

US government (fed+state and local) spent \$1000bn in 2016 on
income-tested programs

a) About 50\% is health care (Medicaid)

b) Only \$250 billion in cash

\end{slide}


%\begin{slide}
%\begin{center}
%{\bf Overall Costs of Anti Poverty Programs}
%\end{center}
%1) US government (fed+state and local) spent \$1000bn in 2016 on
%income-tested programs
%
%a) About 5\% of GDP but 15\% of \$6 Trillion govt budget
%(fed+state+local).
%
%b) About 50\% is health care (Medicaid)
%
%2) Only \$250 billion in cash (1.3\% of GDP, or 25\% of transfer
%spending)
%
%\end{slide}




\begin{slide}%
\begin{center}
{\bf 1996 US Welfare Reform}
\end{center}
1) Largest change in welfare policy

2) Reform modified AFDC\ cash welfare program to provide more
incentives to work (renamed TANF)

a) Requiring recipients to go to job training or work

b) Limiting the duration of benefits (5 year max lifetime)

c) Reducing phase out rate of benefits

3) States got welfare waivers from Federal government to experiment
during 1992-1996 before Federal welfare reform
%Variation across states because Fed govt. gave block grants
%with guidelines

4) EITC\ also expanded during this period:\ general shift from
welfare to \textquotedblleft workfare\textquotedblright

%States and Canada did various welfare reform experiments

Did welfare reform and EITC increase labor supply?

\end{slide}

\begin{slide}
\includepdf[pages={64}]{laborsupplyandtaxes_new_attach.pdf}
\end{slide}

\begin{slide}
\includepdf[pages={63}]{laborsupplyandtaxes_new_attach.pdf}
\end{slide}




%\begin{slide}
%\includepdf[pages={11}]{laborsupplyandtaxes_new_attach.pdf}
%\end{slide}


\begin{slide}
\begin{center}
{\bf Randomized welfare experiment: \\SSP Welfare Demonstration in Canada}
\end{center}

Canadian Self Sufficiency Project (SSP):
randomized experiment that gave welfare recipients an earnings subsidy for 36 months in 1990s
(but need to start working by month 12 to get it)

3 year temporary participation tax rate cut from average rate of 74.3\% to 16.7\%
[get to keep 83 cents for each \$ earned instead of 26 cents]


Card and Hyslop (EMA 2005) provide classic analysis. Two results:

1) Strong effect on employment rate during experiment (peaks at 14 points)

2) Effect quickly vanishes when the subsidy stops after 36 months (entirely gone by month 52)

\end{slide}


\begin{slide}
\includepdf[pages={9}]{laborsupplyandtaxes_new_attach.pdf}
\end{slide}



\begin{slide}
\begin{center}
{\bf Earned Income Tax Credit (EITC) program}
\end{center}
%Hotz-Scholz (2003) and Eissa-Hoynes '06 detailed surveys

1) EITC started small in the 1970s but was expanded in 1986-88,
1994-96, 2008-09: today, largest means-tested cash transfer
program [\$75bn in 2019, 30m families recipients]

2) Eligibility: families with kids and low earnings.

3) Refundable Tax credit: administered through income tax as
annual tax refund received in Feb-April, year $t+1$ (for earnings
in year $t$)

4) EITC has flat pyramid structure with phase-in (negative MTR),
plateau, (0 MTR), and phase-out (positive MTR)

5) Theoretically, EITC should encourage labor force participation
(extensive labor supply margin)

Kleven (2019) who looks at participation of single women (aged 20-50)
with kids (treatment) vs without kids (control)

%5) States have added EITC components to their income taxes [in
%general a percentage of the Fed EITC, great source of natural
%experiments, understudied bc CPS too small]
\end{slide}

%\begin{slide}
%\includepdf[pages={13}]{laborsupplyandtaxes_new_attach.pdf}
%\end{slide}

\begin{slide}
\includepdf[pages={37, 38}]{laborsupplyandtaxes_new_attach.pdf}
\end{slide}




%\begin{slide}
%\begin{center}
%{\bf Welfare Reform and EITC Expansion: \\ Two Empirical Questions}
%\end{center}
%1) Incentives:\ did welfare reform actually increase labor supply?
%
%a) Test whether EITC\ expansions affect labor supply
%
%b) Try and disentangle effect of EITC vs. effect of cut in traditional welfare
%
%2) Benefits:\ did removing many people from transfer system reduce
%their welfare? How did consumption change?
%
%Focus on single mothers, who were most impacted by reform
%
%Use Kleven (2018) who looks at participation of single women (aged 20-50)
%with kids (treatment) vs without kids (control)
%
%
%\end{slide}


\begin{slide}
\includepdf[pages={39-45}]{laborsupplyandtaxes_new_attach.pdf}
\end{slide}


\begin{slide}
\begin{center}
{\bf Welfare Reform and EITC Expansion: Labor supply}
\end{center}

Kleven (2019) who looks at participation of single women (aged 20-50)
with kids (treatment) vs without kids (control)

Large increase in labor force participation of single mothers during the 1990s
during welfare reform and EITC expansion 

Unlikely that the EITC can explain it fully because other EITC changes haven't generated
such large effects

Sociological evidence shows that welfare reform ``scared'' single mothers into working

Single moms in the US were suddenly expected to work

Maybe a unique combination of EITC reform, welfare reform,
economic upturn, and changing social norms lead to this shift


\end{slide}




%\begin{slide}%
%\begin{center}
%{\bf Welfare reform and consumption: \\ Meyer and\ Sullivan 2004}
%\end{center}
%
%1) Examine the consumption patterns of single mothers and their
%families from 1984--2000 using CEX\ data
%
%2) Question:\ did single mothers' consumption fall because they
%lost welfare benefits and were forced to work?
%
%Results:
%
%1) Material conditions of single mothers did not decline in 1990s,
%either in absolute terms or relative to single childless
%women
%
%2) In most cases, evidence suggests that the material conditions
%of single mothers have improved slightly
%
%%During Great Recession: SNAP (food stamps) was the critical program
%%helping those without work
%
%\end{slide}
%
%
%\begin{slide}
%\includepdf[pages={34}]{laborsupplyandtaxes_new_attach.pdf}
%\end{slide}

%\begin{slide}%
%\begin{center}
%{\bf Meyer and\ Sullivan: Results}
%\end{center}
%
%1) Material conditions of single mothers did not decline in 1990s,
%either in absolute terms or relative to single childless
%women
%
%2) In most cases, evidence suggests that the material conditions
%of single mothers have improved slightly
%
%3) Question: is this because economy was booming in 1990s?
%
%4) Is workfare approach more problematic in recession?
%
%Households getting SNAP (food stamps) surged from 12M in '07 to 20M in '10 
%
%But households getting TANF increased only slightly from 1.7M in '07 to 1.85M in '10
%
%\end{slide}%



%\begin{slide}%
%\begin{center}
%{\bf Extensive EITC Labor Supply Response: \\ Eissa and Liebman QJE' 96}
%\end{center}
%
%1) Pioneering study of labor force participation of single mothers
%before/after 1986-7 EITC expansion using CPS data
%
%2) Limitation: this expansion was relatively small
%
%3) Diff-in-Diff strategy:
%
%a) Treatment group: single women with kids
%
%b) Control group: single women without kids
%
%c) Comparison periods: 1984-1986 vs. 1988-1990
%\end{slide}
%
%\begin{slide}
%\includepdf[pages={14-16}]{laborsupplyandtaxes_new_attach.pdf}
%\end{slide}
%
%\begin{slide}
%\includepdf[pages={17}]{laborsupplyandtaxes_new_attach.pdf}
%\end{slide}
%
%\begin{slide}%
%\begin{center}
%{\bf Eissa and Liebman 1996: Results}
%\end{center}
%1) Find a small but significant DD effect: 2.4\% (larger DD effect
%4\% among women with low education) $\Rightarrow$ Translates into
%substantial participation elasticities above 0.5
%
%2) Note the labor force participation for women with/without
%children are not great comparison groups (70\% LFP vs. +90\%):
%time series graph is only moderately convincing
%
%3) Subsequent studies have used much bigger EITC expansions of the
%mid 1990s and also find positive effects on labor force
%participation of single women/single mothers (but contaminated
%by AFDC to TANF reform)
%
%\end{slide}



\begin{slide}
\begin{center}
{\bf Theoretical Behavioral Responses to the EITC}
\end{center}
{\bf Extensive margin:} EITC makes work more attractive (relative to non-work)
$\Rightarrow$ positive effect on Labor Force
Participation

%Meyer-Rosenbaum (2001) find that 60\% of LFP increase of
%single mothers in 1990s due to EITC expansion.

{\bf Intensive margin:} earnings conditional on working;

1) Phase in: (a) Substitution effect: work more due to 40\% increase
in net wage, (b) Income effect: work less $\Rightarrow$ Net
effect: ambiguous; probably work more

2) Plateau: Pure income effect (no change in net wage)
$\Rightarrow$ Net effect: work less

3) Phase out: (a) Substitution effect: work less, (b) Income
effect: also work less $\Rightarrow$ Net effect: work less

%We also should expect \textbf{bunching} at EITC kink points
\end{slide}

\begin{slide}
\includepdf[pages={61-62}]{laborsupplyandtaxes_new_attach.pdf}
\end{slide}


\begin{slide}%
\begin{center}
{\bf EITC and Intensive Labor Supply Response: \\ Bunching at Kinks}
\end{center}
1) Basic labor supply theory predicts that we should observe
bunching of individuals at the EITC kink points:

Some individuals find it worthwhile to work more when subsidy rate is 40\% (2 kids)
but not when subsidy rate falls to 0\% $\Rightarrow$ Utility maximizing labor supply is to
be exactly at the kink 

2) Amount of bunching is proportional to compensated elasticity:
if labor supply is inelastic, then kinks in the budget set are irrelevant and 
do not create bunching

Saez AEJ'10 finds bunching around 1st kink point of EITC but only
for the self-employed $\Rightarrow$ likely due to cheating
to maximize tax refund (and not labor supply)
\end{slide}

\begin{slide}
\includepdf[pages={18,19}]{laborsupplyandtaxes_new_attach.pdf}
\end{slide}

%\begin{slide}
%\includepdf[pages={4}]{laborsupply_attach.pdf}
%\end{slide}

%\begin{slide}
%\includepdf[pages={19}]{pdf/chetty-harvard.pdf}
%\end{slide}

\begin{slide}
\includepdf[pages={20-23}, scale=1.3]{laborsupplyandtaxes_new_attach.pdf}
\end{slide}


\begin{slide}%
\begin{center}
{\bf EITC Empirical Studies}
\end{center}
Some evidence of response along extensive margin but little
evidence of response along intensive margin (except for
self-employed)

$\Rightarrow$ Possibly due to lack of understanding
of the program

Qualitative surveys show that:

Low income families know about EITC and understand that they get a
tax refund if they work

However very few families know whether tax refund increases or
decreases with earnings

Such confusion might be good for the government as the EITC
induces work along participation margin without discouraging work
along intensive margin
\end{slide}


\begin{slide}%
\begin{center}
{\bf Chetty, Friedman, Saez AER'13 EITC information}
\end{center}
Use US population wide tax return data since 1996

1) Substantial heterogeneity fraction of EITC recipients bunching (using self-employment)
across geographical
areas 

$\Rightarrow$ Information about EITC varies across areas

2) Places with high self-employment EITC bunching display
{\bf wage earnings} distribution more concentrated around plateau

$\Rightarrow$ Evidence of wage earnings response to EITC along intensive
margin

3) Omitted variable test: use birth of first child to test
causal effect of EITC on wage earnings

\end{slide}

\begin{slide}
\includepdf[pages={24-33}]{laborsupplyandtaxes_new_attach.pdf}
\end{slide}


%\begin{slide}
%\begin{center}
%{\bf ADVANCE EITC}
%\end{center}
%Recipients get EITC with tax refund in a single annual refund in
%Feb year $t+1$ which seems suboptimal: (a) free interest loan to
%govt and (b) harder to smooth consumption [surveys show that
%primary use of tax refund is to pay overdue bills]
%
%Tax filers have option to use Advance EITC to get part of EITC in
%the paycheck by filing a W5 form with employer [reverse of tax
%withholding]: take up extremely low ($<$2\%)
%
%Possible explanation: (a) Information, (b) Lack of employer
%cooperation, (c) Risk of owing taxes if not EITC eligible, (d) Tax
%filers like big refunds, (e) Inertia (default is no Advance EITC)
%
%\end{slide}
%
%\begin{slide}
%\begin{center}
%{\bf ADVANCE EITC}
%\end{center}
%Jones AEJ-AP'10 carries a randomized experiment with large
%employer to encourage take-up and gets significant but very small
%take-up effect suggesting that (a) [Information] and (b) [Employer
%cooperation] cannot explain low take-up
%
%(d) [Love of refunds] seems plausible but (1) not supplied by
%market absent refunds [employers could also pay part of wages as
%annual lumpsum], (2) A-EITC use has not $\uparrow$ with EITC
%expansions
%
%(c) [Risk of owing taxes] and (e) [Inertia] are likely part of the
%explanation
%
%Interesting research topic: Have big tax refunds fueled low
%income credit [tax refund loans, payday loans, etc.]? Are big
%refunds useful forced saving mechanisms?
%\end{slide}



\begin{slide}
\begin{center}
{\bf Long-term effects of Redistribution: \\ Evidence from the Israeli Kibbutz}
\end{center}
Abramitzky (2018) book based on series of academic papers

Kibbutz are egalitarian and socialist voluntary communities in Israel, thrived for
almost a century within a capitalist society

1) Social sanctions on shirkers effective in small communities with limited privacy

2) Deal with brain drain exit using communal property as a bond

3) Deal with adverse selection in entry with screening and trial period

4) Perfect sharing in Kibbutz has negative effects on high school students performance
but effect is small in magnitude %(concentrated among kids with low educ parents)

\end{slide}

\begin{slide}
\begin{center}
{\bf Long-term effects of Redistribution: \\ Evidence from the Israeli Kibbutz}
\end{center}
Abramitzky-Lavy ECMA'14 show that
high school students study harder once their
kibbutz shifts away from equal sharing

They use a DD strategy: pre-post reform and comparing reform Kibbutz to non-reform Kibbutz.
They find that

1) Students are 3 percentage points more likely to graduate

2)  Students are 6 points more likely to achieve a matriculation
certificate that meets university entrance requirements

%3)  Students get an average of 3.6 more points in their exams

Effect is driven by students whose parents have
low schooling; larger for males; stronger in
kibbutz that reformed to greater degree

\end{slide}

\begin{slide}
\begin{center}
{\bf Culture of Welfare across Generations}
\end{center}
Conservative concern that welfare promotes a culture of dependency: kids growing up
in welfare supported families are more likely to use welfare

Correlation in welfare use across generations is obviously not necessarily causal 

Dahl, Kostol, Mogstad QJE'14 analyze causal effect of parental use of Disability Insurance (DI)
on children use (as adults) of DI in Norway

Identification uses random assignment of judges to denied DI applicants who appeal [some judges severe, others
lenient] 

Find evidence of causality: parents on DI increases odds of kids on DI over next 5 years by 6 percentage points

\small
Mechanism seems to be learning about DI availability rather than reduced stigma from using DI [because no effect on other
welfare programs use]

\end{slide}

\begin{slide}
\includepdf[pages={36}]{laborsupplyandtaxes_new_attach.pdf}
\end{slide}


\begin{slide}
\begin{center}
{\bf SOCIAL DETERMINANTS OF LABOR SUPPLY}
\end{center}
Concern that taxes funding social state could discourage work

\textbf{Standard econ view:} labor supply $l(w,R)$ coming out of \\ $\max u(\underset{+}{c},\underset{-}{l})$ st $c=wl+R$
is highly incomplete

\textbf{Social determinants of labor supply:}

a) Youth labor is regulated by labor laws/education

b) Old age labor regulated by retirement programs

c) Female market labor driven by norms + child care policy

d) Hours of work regulated by overtime + vacation mandates

Social labor supply with disutility for youth, old, overtime labor

\end{slide}

\begin{slide}
\includepdf[pages={53-57}]{laborsupplyandtaxes_new_attach.pdf}
\end{slide}

\begin{slide}
\includepdf[pages={49, 50, 51}]{laborsupplyandtaxes_new_attach.pdf}
\end{slide}

\begin{slide}
\includepdf[pages={58}]{laborsupplyandtaxes_new_attach.pdf}
\end{slide}



%\begin{slide}
%\begin{center}
%{\bf Social Determinants of Labor Supply}
%\end{center}
%Strong evidence that labor supply is not purely an individual decision based on
%standard invariant utility $u(c,l)$
%
%Social norms play large role: e.g. women's market labor supply 
%
%US female labor force participation during World War II: 50\% increase from '40 to '45
%(2/3 reversed afterwards)
%
%Child penalties in female earnings vary a lot across countries (Kleven et al. AEA PP'19) and are not due solely
%to monetary incentives but also to norms about working moms
%
%Responses to taxes and transfers likely affected by social norms
%\end{slide}
%
%\begin{slide}
%\includepdf[pages={49, 50, 51}]{laborsupplyandtaxes_new_attach.pdf}
%\end{slide}


\begin{slide}
\begin{center}
{\bf REFERENCES}
\end{center}
{\small

Abramitzky, Ran ``The Mystery of the Kibbutz: How Socialism Succeeded.''[Book] Princeton University Press,
2018. \href{http://elsa.berkeley.edu/~saez/course/Abramitzky_book_presentation.pdf} {(web)}

Abramitzky, Ran and Victor Lavy, 2014 ``How Responsive is Investment in Schooling to Changes in Redistributive Policies and in Returns?'', Econometrica, 82(4), 1241-1272 \href{http://elsa.berkeley.edu/~saez/course/Abramitzky-Lavy14.pdf} {(web)}

Ashenfelter, O. and M. Plant ``Non-Parametric Estimates of the Labor Supply Effects of Negative Income Tax Programs'', Journal of Labor Economics, Vol. 8, 1990, 396-415. \href{http://links.jstor.org/stable/pdfplus/2535218.pdf} {(web)}

%Card, David, and Dean R. Hyslop. ``Estimating the Effects of a Time-Limited
%Earnings Subsidy for Welfare-Leavers'' Econometrica, 73.6 (2005): 1723-70.
%\href{http://www.jstor.org/stable/pdfplus/3598750.pdf} {(web)}

Chetty, R., J. Friedman and E. Saez ``Using Differences in Knowledge Across Neighborhoods
to Uncover the Impacts of the EITC on Earnings'', 
American Economic Review, 2013, 103(7), 2683-2721 \href{http://eml.berkeley.edu/~saez/chetty-friedman-saezAER13EITC.pdf} {(web)}

Congressional Research Service. 2019.  ``The Temporary Assistance for Needy Families (TANF) Block Grant: Responses to Frequently Asked Questions'', Congressional Research Service
\href{https://fas.org/sgp/crs/misc/RL32760.pdf} {(web)}

Dahl, Gordon B., Andreas Ravndal Kostol, Magne Mogstad ``Family Welfare Cultures''
Quarterly Journal of Economics, 129(4), 2014, 1711-52 \href{http://elsa.berkeley.edu/~saez/course/dahl-kostol-mogstadQJE14.pdf} {(web)} 

%Eissa, N. ``Taxation and Labor Supply of Married Women: The Tax Reform Act of 1986 as a Natural Experiment'', NBER Working Paper No. 5023, 1995. \href{http://www.nber.org/papers/w5023.pdf} {(web)}

Eissa, N. and J. Liebman ``Labor Supply Response to the Earned Income Tax Credit'', Quarterly Journal of Economics, Vol. 111, 1996, 605-637. \href{http://links.jstor.org/stable/pdfplus/2946689.pdf} {(web)}

Heckman, J. and M. Killingsworth ``Female Labor Supply: A Survey'' Handbook of Labor Economics, Vol. I, Chapter 2, 1986. \href{http://elsa.berkeley.edu/~saez/course/Heckman and Killingsworth_Handbook.pdf} {(web)}

Imbens, G.W., D.B. Rubin and B.I. Sacerdote ``Estimating the Effect of Unearned Income on Labor Earnings, Savings, and Consumption: Evidence from a Survey of Lottery'', American Economic Review, Vol. 91, (2001), 778-794. \href{http://links.jstor.org/stable/pdfplus/2677812.pdf} {(web)}

Kleven, Henrik 2019. ``The EITC and the Extensive Margin: A Reappraisal'', NBER working
paper No. 26405.
\href{http://www.nber.org/papers/w26405.pdf} {(web)} 

Kleven, Henrik, Camille Landais, Johanna Posch, Andreas Steinhauer, and Josef Zweimuller. 2019 ``Child penalties across countries: Evidence and explanations.'' AEA Papers and Proceedings, 109, 122-26. 
\href{https://pubs.aeaweb.org/doi/pdfplus/10.1257/pandp.20191078} {(web)}

%Meyer, B. and D. Rosenbaum ``Welfare, the Earned Income Tax Credit, and the Labor Supply of Single Mothers'', Quarterly Journal of Economics, Vol. 116 (2001), 1063-1114. \href{http://links.jstor.org/stable/pdfplus/2696426.pdf} {(web)}

%Meyer, B. and X. Sullivan ``The effects of welfare and tax reform: the material well-being of single mothers in the 1980s and 1990s'', Journal of Public Economics, Vol. 88 (2004), 1387-1420. \href{http://elsa.berkeley.edu/~saez/course/Meyer and Sullivan_JPubE(2004).pdf} {(web)}

Munnell, Alicia H. "Lessons from the income maintenance experiments." Proceedings of a conference sponsored by the Federal Reserve Bank of Boston and the Brookings Institution, Melvin Village, NH. 1986. \href{http://elsa.berkeley.edu/~saez/course/Munnell(1986)book.pdf} {(web)}

Pencavel, J. ``Labor Supply of Men: A Survey'', Handbook of Labor Economics, 	vol. 1, chapter 1, 1986. \href{http://elsa.berkeley.edu/~saez/course/Pencavel_Handbook.pdf} {(web)}

Saez, E. ``Do Taxpayers Bunch at Kink Points?'', American Economic Journal: Economic Policy, Vol. 2, 2010, 180-212. \href{http://elsa.berkeley.edu/~saez/course/Saez_AEJ(2010).pdf} {(web)}

Saez, Emmanuel  ``Public Economics and Inequality: Uncovering Our Social Nature'', AEA Papers and Proceedings, 121, 2021
\href{https://eml.berkeley.edu/~saez/saez-AEAlecture.pdf} {(web)} 

Tazhitdinova, Alisa. 2020 ``Increasing Hours Worked: Moonlighting Responses to a Large Tax Reform'',
NBER Working Paper No. 27726, forthcoming AEJ: Economic Policy. \href{http://www.nber.org/papers/w27726.pdf} {(web)}

}
\end{slide}


\end{document}
