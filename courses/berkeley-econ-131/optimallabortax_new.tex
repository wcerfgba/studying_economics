\documentclass[landscape]{slides}

\usepackage[landscape]{geometry}

\usepackage{pdfpages}

\usepackage{hyperref}
\usepackage{amsmath}

\def\mathbi#1{\textbf{\em #1}}

\topmargin=-1.8cm \textheight=17cm \oddsidemargin=0cm
\evensidemargin=0cm \textwidth=22cm


\author{131 Undergraduate Public Economics \\ Emmanuel Saez \\ UC Berkeley}
\date{}


\title{Optimal Labor Income Taxation \\ (follows loosely Chapters 20-21 of Gruber)}
\onlyslides{1-300}

\newenvironment{outline}{\renewcommand{\itemsep}{}}


\begin{document}

\begin{slide}
\maketitle
\end{slide}

\begin{slide}
\begin{center}
{\bf TAXATION AND REDISTRIBUTION}
\end{center}

{\bf Key question:} By how much should government reduce inequality using
taxes and transfers?

1) Governments use {\bf taxes} to raise revenue

2) This revenue funds {\bf transfer} programs:

a) Universal Transfers: Public Education, Health Care Benefits
(only 65+ in the US), Retirement and Disability Benefits,
Unemployment benefits

b) Means-tested Transfers: In-kind (Medicaid, public housing, foodstamps in the US) and cash benefits

Modern governments raise large fraction of GDP in taxes (30-45\%)
and spend significant fraction of GDP on transfers

\end{slide}

\begin{slide}
\begin{center}
{\bf FACTS ON US TAXES AND TRANSFERS}
\end{center}
{\bf References:} Comprehensive description in:

http://www.taxpolicycenter.org/taxfacts/

{\bf A) Taxes:} (1) individual income tax (fed+state), (2) payroll
taxes on earnings (fed, funds Social Security+Medicare), (3)
corporate income tax (fed+state), (4) sales taxes (state)+excise
taxes (state+fed), (5) property taxes (state)

{\bf B) Means-tested Transfers:} (1) refundable tax credits (fed),
(2) in-kind transfers (fed+state): Medicaid, public housing, nutrition
(SNAP), education, (3) cash welfare: TANF for single parents
(fed+state), SSI for old/disabled (fed)

\end{slide}


\begin{slide}
\begin{center}
{\bf FEDERAL US INCOME TAX}
\end{center}

US income tax assessed on {\bf annual} {\bf family} income (not
individual) [most other OECD countries have shifted to individual
assessment]

Sum all cash income sources from family members (both from labor
and capital income sources) = called {\bf Adjusted Gross Income
(AGI)}

Main exclusions: fringe benefits (health insurance, pension
contributions and returns), imputed rent of homeowners, undistributed corporate profits, unrealized capital gains

$\Rightarrow$ AGI base is only 70\% of national income

\end{slide}


\begin{slide}
\begin{center}
{\bf FEDERAL US INCOME TAX}
\end{center}

Taxable income = AGI - deductions

%personal exemptions = \$4K * \# family members (in 2016)

deduction is max of standard deduction or itemized deductions

Standard deduction is a fixed amount 
(\$12K for singles, \$24K for married couple)

Itemized deductions: (a) state and local taxes paid (up to \$10K), (b) mortgage interest payments, (c) charitable giving,  various small other items

[about 10\% of AGI lost through itemized deductions, called tax
expenditures]

\end{slide}


\begin{slide}
\begin{center}
{\bf FEDERAL US INCOME TAX: TAX BRACKETS}
\end{center}

Tax $T(z)$ is piecewise linear and continuous function of taxable
income $z$ with constant marginal tax rates (MTR) $T'(z)$ by
brackets

In 2018+, 6 brackets with MTR 10\%,12\%,22\%,24\%,32\%,35\%, 37\% (top
bracket for $z$ above \$600K), indexed on price inflation

Lower preferential rates (up to a max of 20\%) apply to dividends
(since 2003), realized capital gains [in part to offset double
taxation of corporate profits].

%20\% of business profits are exempt since 2018

Tax rates change frequently over time. Top MTRs have declined
drastically since 1960s (as in many OECD countries)
\end{slide}

\begin{slide}
\includepdf[pages={15, 16, 1}]{optimallabortax_new_attach.pdf}
\end{slide}

\begin{slide}
\begin{center}
{\bf FEDERAL US INCOME TAX: TAX CREDITS}
\end{center}

%{\bf Alternative minimum tax (AMT)} is a parallel tax system
%(quasi flat tax at 28\%) with fewer deductions: actual tax =$\max
%(T(z),AMT)$ (hits 2-3\% of tax filers in upper middle class)

{\bf Tax credits:} Additional reduction in taxes

(1) {\bf Non refundable} (cannot reduce taxes below zero): foreign
tax credit, child care expenses, education credits, energy credits, and many others

(2) {\bf Refundable} (can reduce taxes below zero, i.e., be net
transfers): EITC (earned income tax credit, up to \$3.5K, \$5.7K, \$6.5K for working
families with 1, 2, 3+ kids), Child Tax Credit (\$2K per kid, partly
refundable)

Refundable tax credits are now the largest means-tested cash transfer for
low income families
\end{slide}

\begin{slide}
\includepdf[pages={14}, scale=1.1]{optimallabortax_new_attach.pdf}
\end{slide}

\begin{slide}
\includepdf[pages={47}, scale=1.1]{optimallabortax_new_attach.pdf}
\end{slide}


\begin{slide}
\begin{center}
{\bf FEDERAL US INCOME TAX: TAX FILING}
\end{center}
Taxes on year $t$ earnings are withheld on paychecks during year
$t$ (pay-as-you-earn)

Income tax return filed in Feb-April 15, year $t+1$ [filers use
either software or tax preparers, huge private industry, most OECD
countries provide pre-populated returns]

Most tax filers get a tax refund as withholdings larger than taxes
owed in general

Payers (employers, banks, etc.) send income information to govt
(3rd party reporting)

3rd party reporting + withholding at source is key for successful
enforcement

\end{slide}

\begin{slide}
\begin{center}
{\bf MAIN MEANS-TESTED TRANSFER PROGRAMS}
\end{center}
1) {\bf Traditional transfers:} managed by welfare agencies, paid
on monthly basis, high stigma and take-up costs $\Rightarrow$ low
take-up rates (often only around 50\%)

Main programs: Medicaid (health insurance for low incomes), SNAP
(former food stamps), public housing, TANF (welfare), SSI
(aged+disabled)

2) {\bf Refundable income tax credits:} managed by tax
administration, paid as an annual lumpsum in year $t+1$, low
stigma and take-up cost $\Rightarrow$  high take-up rates

Main programs: EITC and Child Tax Credit [large expansion since
the 1990s] for low income working families with children
\end{slide}

%\begin{slide}
%\begin{center}
%{\bf BOTTOM LINE ON ACTUAL TAXES/TRANSFERS}
%\end{center}
%1) Based on current income, family situation, and disability
%(retirement) status $\Rightarrow$ Strong link with {\bf current
%ability to pay}
%
%2) Some allowances made to reward / encourage certain behaviors:
%charitable giving, home ownership, savings, energy conservation,
%and more recently work (refundable tax credits such as EITC)
%
%3) Provisions pile up overtime making tax/transfer system more and
%more complex until significant simplifying reform happens (such as
%US Tax Reform Act of 1986)
%
%\end{slide}



\begin{slide}
\begin{center}
{\bf KEY CONCEPTS FOR TAXES/TRANSFERS}
\end{center}
Draw budget $(z,z-T(z))$ which integrates taxes and transfers

1) Transfer benefit with zero earnings $-T(0)$ [sometimes called
demogrant or lumpsum grant]

2) Marginal tax rate (or phasing-out rate) $T'(z)$: individual
keeps $1-T'(z)$ for an additional \$1 of earnings (intensive labor
supply response)

3) Participation tax rate $\tau_p=[T(z)-T(0)]/z$: individual keeps
fraction $1-\tau_p$ of earnings when moving from zero earnings to
earnings $z$ (extensive labor supply response): 
\[ z-T(z)=-T(0) + z \cdot
(1-\tau_p) \] 

4) Break-even earnings point $z^*$: point at which $T(z^*)=0$
\end{slide}

\begin{slide}
\includepdf[pages={17, 18}]{optimallabortax_new_attach.pdf}
\end{slide}



\begin{slide}
\includepdf[pages={2}]{optimallabortax_new_attach.pdf}
\end{slide}


\begin{slide}
\includepdf[pages={3}]{optimallabortax_new_attach.pdf}
\end{slide}

\begin{slide}
\begin{center}
{\bf Profile of Current Means-tested Transfers}
\end{center}
Traditional means-tested programs reduce incentives to work for low income
workers

Refundable tax credits have significantly increased incentive to work
for low income workers

However, refundable tax credits cannot benefit those with zero earnings

Trade-off: US chooses to reward work more than most European countries (such as France) but therefore
provides smaller benefits to those with no earnings

\end{slide}


\begin{slide}
\begin{center}
{\bf Optimal Taxation: Case with No Behavioral Responses}
\end{center}

Utility $u(c)$ strictly increasing and concave

Same for everybody where $c$ is after tax income.

Income $z$ is fixed for each individual, $c=z-T(z)$ where
$T(z)$ is tax/transfer on $z$ (tax if $T(z)>0$, transfer if $T(z)<0$)

$N$ individuals with fixed incomes $z_1<...<z_N$

Government maximizes {\bf Utilitarian} objective:
\[ SWF = \sum_{i=1}^N u(z_i-T(z_i)) \]
subject to {\bf budget constraint} $\sum_{i=1}^N T(z_i) = 0$ (taxes need to fund transfers)
%(multiplier $\lambda$)

\end{slide}

\begin{slide}
\begin{center}
{\bf Simple Model With No Behavioral Responses}
\end{center}
Replace $T(z_1)=- \sum_{i=2}^N T(z_i)$ from budget constraint:
\[ SWF= u \left (z_1 + \sum_{i=2}^N T(z_i)  \right ) +  \sum_{i=2}^N u(z_i-T(z_i)) \]
%Form lagrangian: $L=\sum_{i=1}^N u(z_i-T(z_i)) + \lambda [\sum_{i=1}^N T(z_i) - E] $
First order condition (FOC) in $T(z_j)$ for a given $j=2,..,N$:
\[ 0= \frac{\partial SWF}{\partial T(z_j) } = u' \left (z_1 + \sum_{i=2}^N T(z_i)  \right ) 
-u'(z_j-T(z_j))=0 \Rightarrow   \]
$u'(z_j-T(z_j))=u'(z_1-T(z_1))
\Rightarrow$ $z_j-T(z_j)=$ constant across $j=1,..,N$

Perfect equalization of after-tax income = 100\% tax rate and redistribution [draw graph]

Utilitarianism with decreasing marginal utility leads to perfect
egalitarianism [Edgeworth, 1897]

%Mathematically equivalent to perfect insurance result with risk aversion
%and no moral hazard
\end{slide}

\begin{slide}
\begin{center}
{\bf Simpler Derivation with just 2 individuals}
\end{center}
\[ \max SWF = u(z_1-T(z_1)) + u(z_2-T(z_2)) \text{ s.t. } T(z_1)+T(z_2)=0 \]
Replace $T(z_1)=- T(z_2)$ in $SWF$ using budget constraint:
\[ SWF= u \left (z_1 + T(z_2)  \right ) +   u(z_2-T(z_2)) \]
%Form lagrangian: $L=\sum_{i=1}^N u(z_i-T(z_i)) + \lambda [\sum_{i=1}^N T(z_i) - E] $
First order condition (FOC) in $T(z_2)$:
\[ 0= \frac{d SWF}{d T(z_2) } = u' \left (z_1 + T(z_2)  \right ) 
-u'(z_2-T(z_2))=0 \Rightarrow   \]
$u'(z_1+T(z_2))=u'(z_2-T(z_2)) \Rightarrow u'(z_1-T(z_1))=u'(z_2-T(z_2))$

$\Rightarrow$ $z_1-T(z_1)=z_2-T(z_2)$ constant across the 2 individuals

Perfect equalization of after-tax income = 100\% tax rate and redistribution [see graph]

%Mathematically equivalent to perfect insurance result with risk aversion
%and no moral hazard
\end{slide}


\begin{slide}
\includepdf[pages={20}]{optimallabortax_new_attach.pdf}
\end{slide}



\begin{slide}
\begin{center}
{\bf ISSUES WITH SIMPLE MODEL}
\end{center}

1) {\bf No behavioral responses:} Obvious missing piece: 100\%
redistribution would destroy incentives to work and thus the
assumption that $z$ is exogenous is unrealistic

$\Rightarrow$ Optimal income tax theory incorporates behavioral
responses

2) {\bf Issue with Utilitarianism:} Even absent behavioral
responses, many people would object to 100\% redistribution
[perceived as confiscatory]

$\Rightarrow$ Citizens' views on fairness impose {\bf bounds} on
redistribution govt can do [political economy / public choice
theory]
\end{slide}

%\begin{slide}
%\begin{center}
%{\bf 2ND WELFARE THEOREM FALLACY}
%\end{center}
%Suppose individuals differ in their ability to earn
%
%{\bf 2nd Welfare Theorem:} Any Pareto Efficient outcome can be
%reached by (1) Suitable redistribution of initial endowments
%[individualized {\bf lump-sum} taxes based on ability and not
%behavior], (2) Then letting markets work freely
%
%$\Rightarrow$ No conflict between efficiency and equity
%
%In reality, redistribution of initial endowments is not feasible
%(information pb) and govt needs to use {\bf distortionary} taxes
%and transfers based on income and consumption to redistribute
%
%$\Rightarrow$ Real conflict between efficiency and equity
%
%\end{slide}


\begin{slide}
\begin{center}
{\bf EQUITY-EFFICIENCY TRADE-OFF}
\end{center}

Taxes can be used to raise revenue for transfer programs which can
reduce inequality in disposable income 

$\Rightarrow$ Desirable if
society feels that inequality is too large

Taxes (and transfers) reduce incentives to work 

$\Rightarrow$ High
tax rates create economic inefficiency if individuals respond to
taxes

Size of behavioral response limits the ability of government to
redistribute with taxes/transfers

$\Rightarrow$ Generates an equity-efficiency trade-off

Empirical tax literature estimates the size of behavioral
responses to taxation
\end{slide}


\begin{slide}
\begin{center}
{\bf Labor Supply Theory}
\end{center}
Individual has utility over labor supply $l$ and consumption $c$: $u(c,l)$
\textbf{increasing} in $c$ and \textbf{decreasing} in $l$ [= increasing in leisure]
\[ \max_{c,l} u(c,l) \quad \text{subject to} \quad c = w \cdot l + R \]
with $w=\bar{w} \cdot (1-\tau)$ the net-of-tax wage ($\bar{w}$ is before tax wage rate and $\tau$ is tax rate), and $R$ non-labor income

FOC $w \frac{\partial u}{\partial c}+\frac{\partial u}{\partial l}=0$ defines Marshallian labor supply
$l=l(w,R)$
\[ \textbf{Uncompensated labor supply elasticity:} \quad \varepsilon^u=\frac{w}{l} \cdot \frac{\partial l}
{\partial w} \]
\[ \textbf{Income effects:} \quad \eta=w \frac{ \partial l}{\partial R} \leq 0 \quad \text{(if leisure is a normal good)}\] 


\end{slide}

\begin{slide}
\includepdf[pages={22}]{optimallabortax_new_attach.pdf}
\end{slide}

\begin{slide}
\includepdf[pages={24, 25, 26}]{optimallabortax_new_attach.pdf}
\end{slide}


\begin{slide}
\begin{center}
{\bf Labor Supply Theory}
\end{center}
\textbf{Substitution effects:} Hicksian labor supply: $l^c(w,u)$
minimizes cost needed to reach $u$ given slope $w$ $\Rightarrow$
\[ \mathrm{Compensated\:\:elasticity} \quad \varepsilon^c=\frac{w}{l} \cdot \frac{\partial l^c}
{\partial w} > 0  \]
\[\mathrm{Slutsky\:\:equation} \quad \frac{\partial l}{\partial w} = \frac{\partial l^c}{\partial w}
+ l \frac{ \partial l}{\partial R} \Rightarrow \varepsilon^u = \varepsilon^c + \eta\]

Tax rate $\tau$ discourages work through substitution effects (work pays less at the margin)

Tax rate $\tau$ encourages work through income effects (taxes make you poorer and hence in more need of income)

Net effect ambiguous (captured by sign of $\varepsilon^u$)

\end{slide}

\begin{slide}
\includepdf[pages={23, 27-32}]{optimallabortax_new_attach.pdf}
\end{slide}

\begin{slide}
\begin{center}
{\bf General nonlinear income tax [draw graph]}
\end{center}
With no taxes: $c= z$ (consumption = earnings)

With taxes $c=z-T(z)$ (consumption = earnings - net taxes)

$T(z) \geq 0$ if individual pays taxes on net, $T(z) \leq 0$ if individual receives transfers on net

$T'(z)>0$ reduces net wage rate and reduces labor supply through substitution effects

$T(z) >0$ reduces disposable income and increases labor supply through income effects

$T(z) < 0$ increases disposable income and decreases labor supply through income effects

Transfer program such that $T(z) <0$ and $T'(z)>0$ always discourages labor supply

\end{slide}


\begin{slide}
\includepdf[pages={35-37}]{optimallabortax_new_attach.pdf}
\end{slide}


\begin{slide}
\begin{center}
{\bf OPTIMAL LINEAR TAX RATE: LAFFER CURVE}
\end{center}
$c=(1-\tau)\cdot z + R$ with $\tau$ linear tax rate and $R$ fixed universal transfer funded
by taxes $R=\tau \cdot Z$ with $Z$ average earnings

Individual $i=1,..,N$ chooses $l_i$ to max $u^i((1-\tau)\cdot w_i l_i + R,l_i)$

Labor supply choices $l_i$ determine individual earnings $z_i=w_i l_i$ $\Rightarrow$ Average
earnings $Z=\sum_i z_i/N$ depends (positively) on net-of-tax rate $1-\tau$.

Tax Revenue per person $R(\tau)=\tau \cdot Z(1-\tau)$ is inversely U-shaped
with $\tau$: $R(\tau=0)=0$ (no taxes) and $R(\tau=1)=0$ (nobody
works): called the Laffer Curve

\end{slide}

\begin{slide}
\includepdf[pages={19}]{optimallabortax_new_attach.pdf}
\end{slide}

\begin{slide}
\begin{center}
{\bf OPTIMAL LINEAR TAX RATE: LAFFER CURVE}
\end{center}

Top of the Laffer Curve is at $\tau^*$
maximizing tax revenue:
\[0=R'(\tau^*)=Z - \tau^* \frac{d Z}{d(1-\tau)} \Rightarrow \frac{\tau^*}{1-\tau^*} \cdot \frac{1-\tau^*}{Z} \frac{d Z}{d(1-\tau)} = 1  \]
\[ \text {Revenue maximizing tax rate: }  \tau^*=\frac{1}{1+e} \text{ with } e = \frac{1-\tau}{Z}\frac{dZ}{d(1-\tau)} \] $e$
is the
elasticity of average income $Z$ with respect to the net-of-tax rate $1-\tau$ [empirically estimable]

Inefficient to have $\tau>\tau^*$ because decreasing $\tau$ would make taxpayers better off (they pay less taxes)
and would increase tax revenue for the government [and hence univ. transfer $R$]

If government is \textbf{Rawlsian} (maximizes welfare of the worst-off person with no earnings) then $\tau^*=1/(1+e)$ is optimal
to make transfer $R(\tau)$ as large as possible
\end{slide}


\begin{slide}
\begin{center}
{\bf OPTIMAL LINEAR TAX RATE: FORMULA}
\end{center}
Government chooses $\tau$ to maximize \textbf{utilitarian} social welfare
\[ SWF = \sum_i u^i((1-\tau)w_i l_i+\tau \cdot Z(1-\tau),l_i)  \]
taking into account that labor supply $l_i$ responds to taxation and hence
that this affects the tax revenue per person $\tau \cdot Z(1-\tau)$ that is redistributed
back as transfer to everybody

Government first order condition: (using the envelope theorem as $l_i$ maximizes $u^i$):
\[ 0 = \frac{dSWF}{d\tau} =  \sum_i \frac{ \partial u^i}{\partial c} \cdot
\left [-z_i + Z - \tau \frac{dZ}{d(1-\tau)} \right ],\]

\end{slide}

\begin{slide}
\begin{center}
{\bf OPTIMAL LINEAR TAX RATE: FORMULA}
\end{center}

Hence, we have the following optimal linear income tax formula
\[
\tau= \frac{1-\bar{g}}{1-\bar{g} + e} \quad \mathrm{with} \quad
\bar{g}= \frac{ \sum_i z_i \cdot  \frac{ \partial u^i}{\partial c}   }{Z \cdot \sum_i \frac{ \partial u^i}{\partial c} }
\]
$0\leq \bar{g} <1$ as $\frac{ \partial u^i}{\partial c}$ lower when income $z_i$ is high (marginal utility
falls with consumption)

$\tau$ decreases with elasticity $e$ [efficiency] and with $\bar{g}$ [equity]

Formula captures the \textbf{equity-efficiency trade-off}

$\bar{g}$ is low and $\tau$ close to Laffer rate $\tau^*=1/(1+e)$ when

(a) inequality is high

(b) marginal utility decreases fast with income


\end{slide}


\begin{slide}
\begin{center}
{\bf OPTIMAL TOP INCOME TAX RATE \\ (Diamond and Saez JEP'11)}
\end{center}
In practice, individual income tax is progressive with brackets with increasing
marginal tax rates. What is the optimal top tax rate?

Consider constant MTR $\tau$ above fixed $z^*$. Goal is to
derive optimal $\tau$

In the US in 2018+, $\tau=37\%$ and $z^*\simeq \$600,000$ ($\simeq$ top 1\%)

Denote by
$z$ average income of top bracket earners [depends on net-of-tax rate
$1-\tau$], with elasticity $e=[(1-\tau)/z]\cdot
d z/d(1-\tau)$

Suppose the government wants to maximize tax revenue collected
from top bracket taxpayers (marginal utility of consumption of
top 1\% earners is small)

\end{slide}


\begin{slide}
\includepdf[pages={7,8}]{optimallabortax_new_attach.pdf}
\end{slide}

\begin{slide}
\begin{center}
{\bf OPTIMAL TOP INCOME TAX RATE}
\end{center}
Consider small $d\tau>0$ reform above $z^*$.

1) {\bf Mechanical increase} in tax revenue:
\[ dM=
[z-z^*] d\tau \]

2) {\bf Behavioral response} reduces tax revenue:
\[ dB= \tau dz= - \tau \frac{dz}{d(1-\tau)}d\tau = -
 \frac{\tau}{1-\tau} \cdot  e \cdot z \cdot d\tau \]
\[ dM+dB=d\tau \left \{ [z-z^*]   - e
\frac{\tau}{1-\tau} z \right \}\]
Optimal $\tau$ such that
$dM+dB=0$ 
\[ \Rightarrow \quad \frac{\tau}{1-\tau}=\frac{1}{e} \cdot \frac{z-z^*}{z} \Rightarrow 
\tau=\frac{1}{1+a \cdot e} \quad \mathrm{with} \quad
a=\frac{z}{z-z^*}\]


\end{slide}
\begin{slide}
\begin{center}
{\bf OPTIMAL TOP INCOME TAX RATE}
\end{center}
$$\text{Optimal top tax rate: } \tau=\frac{1}{1+a \cdot e} \quad \mathrm{with} \quad
a=\frac{z}{z-z^*}$$

Optimal $\tau$ decreases with $e$ [efficiency]

Optimal $\tau$ decrease with $a$ [thinness of top tail]

Empirically $a \simeq 1.5$, easy to estimate using distributional data
[mean income above $z^*=\$.5$m is about \$1.5m in the US]

Empirically $e$ is harder to estimate [controversial]

Example: If $e=.25$ then $\tau=1/(1+1.5\cdot 0.25)=1/1.375=73\%$

\end{slide}

%\begin{slide}
%\includepdf[pages={9}]{optimallabortax_new_attach.pdf}
%\end{slide}

\begin{slide}
\begin{center}
{\bf REAL VS. TAX AVOIDANCE RESPONSES}
\end{center}
Behavioral response to income tax comes not only from reduced
labor supply but from tax avoidance or tax evasion

Tax avoidance: legal means to reduce tax liability (exploiting tax loopholes)

Tax evasion: illegal under-reporting of income 

%
%shifts to other forms of income or
%activities: (untaxed fringe benefits, shift to corporate
%income tax base, shift toward tax favored capital gains, etc.)

Labor supply vs. tax avoidance/evasion distinction matters because:

1) If people work less when tax rates increase, there is not much the government can do about 
it

2) If people avoid/evade more when tax rates increase, then the govt can reduce tax avoidance/evasion 
opportunities [closing tax loopholes, broadening the tax base, increasing tax enforcement, etc.]

%Government can control tax avoidance through other tools: closing loopholes,
%broadening the tax base $\Rightarrow$ Elasticity $e$ is lower with no loopholes
%
%2) Most tax avoidance responses create ``fiscal externalities'' in the
%sense that tax revenue increases at other time periods or in other tax bases:
%
%e.g., US top tax rate increased in 2013, taxpayers likely shifted part of their income to 2012
%to avoid higher tax rates


\end{slide}

\begin{slide}
\begin{center}
{\bf REAL VS. AVOIDANCE RESPONSES}
\end{center}
{\bf Key policy question:} Is it possible to eliminate avoidance responses using base broadening, etc.? or would new avoidance schemes keep popping up?

a) Some forms of tax avoidance are due to \textbf{poorly designed tax codes} (preferential treatment for some income forms or some deductions)

b) Some forms of tax avoidance/evasion can only be addressed with \textbf{international cooperation} (off-shore
tax evasion in tax havens)

c) Some forms of tax avoidance/evasion are due to technological limitations of tax collection
(impossible to tax informal cash businesses)

\end{slide}



\begin{slide}
\begin{center}
{\bf OPTIMAL PROFILE OF TRANSFERS}
\end{center}
If individuals respond to taxes only through \textbf{intensive} margin (how much they work rather than
whether they work), optimal transfer at bottom takes the
form of a ``Negative Income Tax'':

1) Lumpsum grant $-T(0)>0$ for those with no earnings

2) High marginal tax rates (MTR) $T'(z)$ at the bottom to phase-out the lumpsum grant
quickly

Intuition: high MTR at bottom are efficient because:

(a) they target transfers to the most needy

(b) earnings at the bottom are low to start with $\Rightarrow$ intensive
labor supply response does not generate large output losses

But US system with zero MTR at bottom justified if society sees people
with zero income as \textbf{less deserving} than average 
\end{slide}

\begin{slide}
\includepdf[pages={43-45}]{optimallabortax_new_attach.pdf}
\end{slide}



\begin{slide}
\begin{center}
{\bf Optimal Transfers: Participation Responses}
\end{center}
Empirical literature shows that participation labor supply
responses [whether to work or not] are large at the bottom
[much larger and clearer than intensive responses]

Participation depends on participation tax rate:
\[ \tau_p=[T(z)-T(0)]/z \]

Individual keeps fraction $1-\tau_p$ of
earnings when moving from zero earnings to earnings $z$:
$z-T(z)=-T(0) + z \cdot (1-\tau_p)$

{\bf Key result:} in-work subsidies with $T'(z)<0$
are optimal when labor supply responses are concentrated along
extensive margin and govt cares about low income workers.

\end{slide}

\begin{slide}
\includepdf[pages={18}]{optimallabortax_new_attach.pdf}
\end{slide}


\begin{slide}
\includepdf[pages={38-42}]{optimallabortax_new_attach.pdf}
\end{slide}

\begin{slide}
\begin{center}
{\bf OPTIMAL PROFILE OF TRANSFERS: SUMMARY}
\end{center}

1) If society views \textbf{zero earners} as less deserving than average [conservative view 
that substantial fraction of zero earners are ``free loaders''] then low lumpsum grant
combined with low phasing out rate at bottom is optimal 

2) If society views \textbf{low income workers} as more deserving than average [typically bipartisan view]
and labor supply responses concentrated along extensive margin (work vs. not) then low phasing out rate at
bottom is optimal 

3) Generous lumpsum grant with high MTR at bottom justified only if society views non workers as deserving
and no strong response along the extensive margin (work vs. not)

\end{slide}



\begin{slide}
\begin{center}
{\bf ACTUAL TAX/TRANSFER SYSTEMS}
\end{center}

1) Means-tested transfer programs used to be of the traditional form with high
phasing-out rates (sometimes above 100\%) $\Rightarrow$ No
incentives to work (even with modest elasticities)

Initially designed for
groups not expected to work [widows in the US] but later attracting
groups who could potentially work [single mothers]

2) In-work benefits have been introduced and expanded in OECD
countries since 1980s (US EITC, UK Family Credit, etc.) and have
been politically successful $\Rightarrow$ (a) Redistribute to low
income workers, (b) improve incentives to work

\end{slide}


\begin{slide}
\begin{center}
{\bf Basic Income vs. Means-tested transfer: Mankiw quiz}
\end{center}

Consider an economy in which average income is \$50,000 but with much income inequality. To provide a social safety net, two possible policies are proposed.

A.  A universal transfer of \$10,000 to every person, financed by a 20-percent flat tax on income.

B.  A means-tested transfer of \$10,000.  The full amount goes to someone without any income.  The transfer is then phased out: You lose 20 cents of it for every dollar of income you earn.  These transfers are financed by a tax of 20 percent on income above \$50,000.

Which would you prefer?

\end{slide}

\begin{slide}
\begin{center}
{\bf Basic Income vs. Means-tested transfer}
\end{center}
Basic income definition: all people receive an unconditional sum of money (every year) regardless of
how much they earn

This is the $R$ of the linear tax system $c= (1-\tau) \cdot z + R$ 

Or the $-T(0)>0$ of the nonlinear tax system $c=z-T(z)$

%In practice, most of the basic income is ``in-kind'' such as Universal health care, public education. Cash component is 
%administered through welfare programs (and phased-out with income)

Basic income for everybody + higher taxes to fund it is \textbf{economically equivalent} to means-tested transfer phased out with earnings

Pro basic income: less stigmatizing than means-tested transfer

Cons: basic income requires higher ``nominal'' taxes (that are then rebated back)
%But means-tested program is often much more stigmatizing so 

Countries provide ``in-kind'' basic income in the form of universal health care (not the US)
and public education
\end{slide}

\begin{slide}
\includepdf[pages={33}]{optimallabortax_new_attach.pdf}
\end{slide}





\begin{slide}
\begin{center}
{\bf IN-KIND REDISTRIBUTION}
\end{center}
Most means-tested transfers are in-kind and often
rationed (health care, child care, public education, public housing, nutrition
subsidies) [care not cash San Francisco reform]

1) {\bf Rational Individual perspective:}

(a) If in-kind transfer is {\bf tradeable} at market price
$\Rightarrow$ in-kind equivalent to cash

(b) If in-kind transfer {\bf non-tradeable} $\Rightarrow$ in-kind
inferior to cash

Cash transfer preferable to in-kind transfer from individual perspective

\end{slide}

\begin{slide}
\includepdf[pages={46}]{optimallabortax_new_attach.pdf}
\end{slide}



\begin{slide}
\begin{center}
{\bf IN-KIND REDISTRIBUTION}
\end{center}
2) {\bf Social perspective:} 3 justifications:

a) Commodity Egalitarianism: some goods (education, health,
shelter, food) seen as {\bf rights} and ought to be provided to
all in a just society

b) Paternalism: society imposes its preferences on recipients
[recipients prefer cash]

c) Behavioral: Recipients do not make choices in their best
interests (self-control, myopia) [recipients understand that
in-kind is better for them]

%d) Efficiency: It could be efficient to give in-kind benefits if it can prevent those who don't really need them 
%from getting them (i.e., force people to queue to get free soup kitchen)

\end{slide}


\begin{slide}
\begin{center}
{\bf FAMILY TAXATION: MARRIAGE AND CHILDREN}
\end{center}
Two important issues in policy debate:

1) Marriage: What is the optimal taxation of couples vs. singles?

2) Children: What should be the net transfer (transfer or tax
reduction) for family with children (as a function of family
income and structure)?

%Theoretical literature is not great in part because utilitarian
%framework is not satisfactory
\end{slide}

\begin{slide}
\begin{center}
{\bf TAXATION OF COUPLES}
\end{center}
Three potentially desirable properties:

(1) income tax should be based on resources (i.e., family income if
families fully share their income)

(2) income tax should be marriage neutral: no higher/lower tax when two
single individuals marry

(3) income tax should be progressive (i.e., higher incomes pay a larger
fraction of their income in taxes)

It is \textbf{impossible} to have a tax system that satisfies all 3 conditions simultaneously:

Income tax that is based on family income and marriage neutral has to satisfy:
$T(z^h+z^w)=T(z^h)+T(z^w)$ and hence be linear i.e. $T(z)=\tau\cdot z$

\end{slide}

\begin{slide}
\begin{center}
{\bf TAXATION OF COUPLES}
\end{center}
(1) If couples share their incomes, then family taxation is better. 
If couples don't share their incomes, then individualized tax is better 

(2) If marriage responds to tax/transfer differential $\Rightarrow$
better to reduce marriage penalty, i.e., move toward
individualized system

Particularly important when cohabitation is close
substitute for marriage (as in Scandinavian countries)

(3) If labor supply of secondary earners more elastic than labor
supply of primary earner $\Rightarrow$ Secondary earnings should
be taxed less (Boskin-Sheshinski JpubE'83)

Labor supply elasticity differential between primary and secondary earners is decreasing over time as earnings
gender gap decreases (Blau and Kahn 2007)

\end{slide}

\begin{slide}
\begin{center}
{\bf TRANSFERS OR TAX CREDITS FOR CHILDREN}
\end{center}
1) Children reduce {\bf normalized family income} $\Rightarrow$ Children
increase marginal utility of consumption $\Rightarrow$ Transfer
for children $T_{kid}$ should be positive

In practice, transfers for children are always positive

2) Should $T_{kid}(z)$ increase with income $z$?

Pro: rich spend more on their kids than lower income families

%they reduce normalized income most for upper earners [e.g.,
%France computes taxes as $N \cdot T(z/N)$ where $N$ is \# family
%members, kids count as .5 $\Rightarrow$ $T_{kid}(z) \uparrow z$].

Cons: Lower income families need child transfers most

In practice, $T_{kid}(z)$ is fairly constant with $z$

\small
Europe has much more generous pre-kindergarten child care benefits,
US has more generous cash tax credits for working families with children

Strong evidence that govt provided child care (Europe) more egalitarian and cheaper overall than private
provision (US)

%[most OECD countries
%have means-tested transfers conditional on number of kids
%$\Rightarrow$ $T_{kid}(z) \downarrow z$, US has  $T_{kid}(z)$
%inverted U-shape due to EITC and Child Tax Credit]

\end{slide}


\begin{slide}
\begin{center}
{\bf REFERENCES}
\end{center}
{\small

Jonathan Gruber,Public Finance and Public Policy, Fifth Edition, 2016 Worth Publishers, Chapter 20 and Chapter 21

Blau, F. and L. Kahn ``Changes in the Labor Supply Behavior of Married Women: 1980-2000'', Journal of Labor Economics, Vol. 25, 2007, 393-438. \href{http://www.jstor.org/stable/pdfplus/10.1086/513416.pdf} {(web)}

Boskin, Michael J., and Eytan Sheshinski. ``Optimal tax treatment of the family: Married couples.'' Journal of Public Economics 20.3 (1983): 281-297.\href{http://elsa.berkeley.edu/~saez/course131/Boskin-Sheshinski83.pdf} {(web)}

Diamond, P. and E. Saez ``From Basic Research to Policy Recommendations:
The Case for a Progressive Tax'', Journal of Economic Perspectives, 25.4, (2011): 165-190.
\href{http://elsa.berkeley.edu/~saez/diamond-saezJEP11full.pdf} {(web)}

IRS, Statistics of Income Division ``U.S. Individual Income Tax: Personal Exemptions and Lowest and Highest Tax Bracket''(2013) \href{http://elsa.berkeley.edu/~saez/course131/taxable-income4.pdf} {(web)}

%Kleven, Henrik Jacobsen, Claus Thustrup Kreiner, and Emmanuel Saez. ``The optimal income taxation of couples.'' Econometrica 77.2 (2009): 537-560.\href{http://www.jstor.org/stable/pdfplus/40263874.pdf}{(web)}

Piketty, Thomas and Emmanuel Saez ``Optimal Labor Income Taxation,'' Handbook of Public Economics, Volume 5, Amsterdam: Elsevier-North Holland, 2013. \href{http://www.nber.org/papers/w18521.pdf} {(web)}

Saez, Emmanuel. ``Optimal income transfer programs: intensive versus extensive labor supply responses.'' The Quarterly Journal of Economics 117.3 (2002): 1039-1073.\href{http://elsa.berkeley.edu/~saez/botqje.pdf}{(web)}

}
\end{slide}

\end{document}
