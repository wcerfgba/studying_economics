\documentclass[landscape]{slides}

\usepackage[landscape]{geometry}

\usepackage{pdfpages}
\usepackage{eurosym}
\usepackage{amssymb}
\usepackage{amsmath}
\usepackage{float}

\usepackage{hyperref}

\def\mathbi#1{\textbf{\em #1}}

\topmargin=-1.8cm \textheight=17cm \oddsidemargin=0cm
\evensidemargin=0cm \textwidth=22cm

\author{Emmanuel Saez}

\date{Berkeley}

\title{131: Public Economics \\
Taxes on Capital and Savings} \onlyslides{1-300}

\newenvironment{outline}{\renewcommand{\itemsep}{}}

\begin{document}

\begin{slide}
\maketitle
\end{slide}

\begin{slide}
\begin{center}
{\bf MOTIVATION}
\end{center}
1) Capital income is about 25-30\% of national income (labor income
is 70-75\%) but distribution of capital income is much more unequal
than labor income

Capital income inequality is due to differences in savings
behavior but also inheritances received

$\Rightarrow$ Equity suggests it should be taxed more than labor

2) Capital Accumulation correlates strongly with growth [although
causality link is not obvious] and capital accumulation might be
sensitive to the net-of-tax return.

$\Rightarrow$ Efficiency cost of capital taxation might be high.

\end{slide}

\begin{slide}
\begin{center}
{\bf MOTIVATION}
\end{center}

3) Capital more mobile internationally than labor
%$\Rightarrow$ Incidence of capital taxation might fall on workers:

Key distinction is \textbf{residence} vs. \textbf{source} base capital taxation:

\textbf{Residence: } Tax based on residence of owner of capital.

Most individual income tax systems are residence based (with credits for taxes
paid abroad)

%Incidence falls on the owner $\Rightarrow$ can only escape
%tax through tax evasion (offshore tax heavens) or changing residence (mobility)
%
%Tax evasion through tax heavens is a very serious concern (Zucman's book ``Scourge of Tax Heavens'' 2015)

\textbf{Source:} Tax based on location of capital 

Real estate property tax and corporate income tax are source based

%(exception US corporate tax used to apply to worldwide profits of US corporations but only upon rapatriation. US corp tax source based since 2018 but with min tax on foreign profits)

%Incidence is then partly shifted to labor if capital is mobile

%Mechanism: tax on capital, capital flees the country, hurts the wage of domestic workers (as
%workers are less productive with less capital) $\Rightarrow$ Workers bear part of the burden

%\small
%Example: Open economy with fully mobile capital and source taxation: 
%
%Net-of-tax rate of return
%is fixed by the international rate of return $r^*$ so that domestic pre-tax rate of return $r$ adjusts so
%that $r \cdot (1-\tau)=r^*$
%
%Capital owners get the same return as before but capital stock has
%come down hurting wages of workers $\Rightarrow$ Workers bear full burden
%
%\normalsize

%$(1-\tau)\frac{\partial F}{\partial K}(L,K)=r^*$ where $K$ is capital stock and $L$ labor.


4) Capital taxation is extremely complex and provides many tax
avoidance opportunities particularly for multinational firms 
\end{slide}

%\begin{slide}
%\begin{center}
%{\bf MACRO FRAMEWORK}
%\end{center}
%
%Constant return to scale aggregate production:
%
%$Y = F(K,L) = r K + w L$ = output = income
%
%$K$ = capital stock (wealth), $L$ = labor input
%
%$r$ = rate of return on capital, $w$ is wage rate
%
%$rK$ = capital income, $wL$ = labor income
%
%$\alpha=rK/Y$ = capital income share (constant $\alpha$ when
%$F(K,L)=K^{\alpha} L^{1-\alpha}$ Cobb-Douglas), $\alpha \simeq
%30\%$
%
%$\beta=K/Y$ = wealth to annual income ratio, $\beta \simeq 5-6$
%
%$r= (rK/Y) \cdot (Y/K) = \alpha/\beta$, $r=5-6\%$
%
%%Infinite horizon model: $U=\sum_t u(c_t)/(1+\delta)^t$
%%$\Rightarrow$ $r=\delta$ (discount rate) and $\beta=\alpha/\delta$
%\end{slide}

%\begin{slide}
%\begin{center}
%{\bf SAVING FLOWS}
%\end{center}
%Saving is a flow and wealth or net worth is a stock
%
%Three saving flows:
%
%1) {\bf Personal saving:} individual income less individual
%consumption [fell dramatically in the US since 1980s, recent
%$\uparrow$ since 2008]
%
%2) {\bf Corporate Saving:} retained earnings =  after tax profits
%- distributions to shareholders
%
%3) {\bf Government Saving:} Taxes - Expenditures [federal, state
%and local]
%
%Taxes on savings might affect different savings flows differently:
%savings subsidy through a tax credit can $\uparrow$ individual
%savings but $\downarrow$ govt saving [if govt spending stays
%constant]
%\end{slide}

%\begin{slide}
%\begin{center} {\bf FACTS ABOUT WEALTH AND CAPITAL INCOME} \end{center}
%{\bf Definition:} Capital Income = Returns from Wealth Holdings
%
%Aggregate US {\bf Private} Wealth $\simeq$ 5$\times$Annual National  Income
%
%{\bf Housing:} residential real estate (land+buildings)
%[income = rents] net of mortgage debt  
%
%{\bf Unincorporated business assets:} value of sole proprietorships and partnerships
%[income = individual business profits]
%
%{\bf Corporate equities:} Value of corporate stock [income = dividends + retained earnings]
%
%{\bf Fixed claim assets:} Currency, deposits, bonds [income = interest income] minus
%debts [credit card, student loans]
%
%{\bf Pension funds:} Substantial amount of equities and fixed claim assets held indirectly through pension funds
%
%
%\end{slide}


\begin{slide}
\begin{center} {\bf FACTS ABOUT WEALTH} \end{center}
{\bf Definition:} Capital Income = Return from Wealth 

Wealth arises from expected future income and value of assets

Private wealth includes real estate (land+buildings), corporate and business equity, fixed claimed assets (bonds+deposits), net
of debts (mortgage, student loans, consumer credit)

Aggregate US {\bf Private} Wealth $\simeq$ 5$\times$Annual National  Income

Total wealth reflects both capital stock accumulated through savings and pure price effects

\small
Example 1: house can increase in value because it is improved (capital) or because
local prices go up (pure price effect)

Example 2: greater monopoly power makes a business more valuable to owners (but at the expense of consumers)

Recent increase in US private wealth mostly due to price effects
%More capital stock is good but more private wealth is not always good

\end{slide}

\begin{slide}
\includepdf[pages={26}]{taxsavings_new_attach.pdf}
\end{slide}

%\begin{slide}
%\includepdf[pages={16,15}]{taxsavings_new_attach.pdf}
%\end{slide}


\begin{slide}
\includepdf[pages={1-2}]{taxsavings_new_attach.pdf}
\end{slide}

\begin{slide}
\includepdf[pages={3-4}]{taxsavings_new_attach.pdf}
\end{slide}


%\begin{slide}
%\begin{center} {\bf CAPITAL INCOME IN NATIONAL ACCOUNTS} \end{center}
%
%Gross capital income (before depreciation) is about 40\% of GDP
%
%Net capital income (after depreciation) is about 25-30\% of personal
%income
%
%The capital income share in total income is relatively stable in
%the long-run (but with some short term fluctuations)
%
%Average real rate of return of capital around 5-6\%, varies
%greatly from year to year
%\end{slide}

\begin{slide}
\begin{center} {\bf FACTS ABOUT WEALTH AND CAPITAL INCOME} \end{center}

Wealth = $W$, Return = $r$, Capital Income = $rW$
$$W_t=W_{t-1}+r_t W_{t-1}+ E_t + I_t - C_t $$
where $W_t$ is wealth at age $t$, $C_t$ is consumption, $E_t$
labor income earnings (net of taxes), $r_t$ is the average (net)
rate of return on investments and $I_t$ net inheritances (gifts
received and bequests - gifts given).

Differences in Wealth and Capital income due to: \small

1) Age

2) past earnings, and past saving behavior $E_t-C_t$ [life
cycle wealth]

3) Net Inheritances received $I_t$ [transfer wealth]

4) Rates of return $r_t$

\end{slide}


\begin{slide}
\begin{center} {\bf Wealth Inequality (Saez and Zucman '16)} \end{center}

Wealth inequality is very large (always much higher than income inequality)

In the US in 2019: Top 1\% wealthiest families get 40\% of total wealth, Next 9\% get
about 35\%, next 40\% get 25\%, bottom 50\% get about 0\%

Wealth inequality \textbf{decreases} from 1929 to 1980: wealth democratization due
to rise in homeownership and pensions

Wealth inequality increases sharply since 1980 fueled by increases in \textbf{income}
inequality and \textbf{savings} inequality [bottom 90\% saves zero in net since 1990]

US public underestimates extent of wealth inequality and thinks the
ideal wealth distribution should be a lot more equal [Norton-Ariely '11]
\end{slide}

%\begin{slide}
%\includepdf[scale=.95,pages={11-13}]{taxsavings_new_attach.pdf}
%\end{slide}
%
%\begin{slide}
%\includepdf[pages={17, 14}]{taxsavings_new_attach.pdf}
%\end{slide}
%
%\begin{slide}
%\includepdf[pages={18}]{taxsavings_new_attach.pdf}
%\end{slide}

\begin{slide}
\includepdf[pages={33}]{taxsavings_new_attach.pdf}
\end{slide}

\begin{slide}
\includepdf[pages={32}]{taxsavings_new_attach.pdf}
\end{slide}

\begin{slide}
\includepdf[pages={34}]{taxsavings_new_attach.pdf}
\end{slide}

\begin{slide}
\includepdf[pages={5}]{taxsavings_new_attach.pdf}
\end{slide}

\begin{slide}
\begin{center} {\bf FACTS ON US CAPITAL INCOME TAXATION} \end{center}

1) {\bf Corporate Income Tax} (fed+state)
on profits of corporations [complex rules with many industry
specific provisions]: effective tax rate only 16\% of corporate profits in 2018

%[Full expensing means deducting investment costs instead of
%depreciation, amounts to exempting normal rate of return from tax]

2) {\bf Individual Income Tax} (fed+state): taxes many forms of
capital income

\small
Realized capital gains and dividends receive preferential treatment (to lower double
taxation of corporate profits)

Imputed rent of home owners and returns on pension funds are exempt
\normalsize

3) {\bf Estate tax:} tax on very large estates (40\% tax above \$11m) bequeathed to heirs (small and poorly enforced)

4) {\bf Property taxes} (local) on real estate (old tax):

\small
Tax varies across jurisdictions. About 0.5\% of market value on
average
\normalsize

5) {\bf Wealth tax} on total net worth of rich families (does not currently exist, proposed by Warren and Sanders, and in CA)

\end{slide}


\begin{slide}
\includepdf[pages={27-28}]{taxsavings_new_attach.pdf}
\end{slide}



\begin{slide}
\begin{center}
{\bf LIFE CYCLE VS. INHERITED WEALTH}
\end{center}
Economists divide existing wealth into 2 categories:

\textbf{1) Life-cycle wealth} is wealth from savings earlier in your life
%(e.g.,  pension contributions out of earnings, paying down a home mortgage, etc.)

\textbf{2) Inherited wealth} is wealth from inheritances received
% (e.g., receiving a house or a trust fund from parents)

Distinction matters for taxation because individuals are responsible for life-cycle wealth
but not inherited wealth %[meritocracy vs. aristocracy]

Inherited wealth used to be very large in Europe (before World-War I), became small in post-World War II period,
but is growing in recent decades (especially in Europe) [Piketty' 14]

Same trend in the US but less pronounced but poor data quality (Alvaredo-Piketty-Garbinti '17)

Piketty '14: return on wealth bigger than growth rate ($r>g$) $\Rightarrow$ wealth concentration and inherited wealth increases



\end{slide}



%\begin{slide}
%\includepdf[pages={6}]{taxsavings_new_attach.pdf}
%\end{slide}



\begin{slide}
\includepdf[pages={23}]{taxsavings_new_attach.pdf}
\end{slide}

%\begin{slide}
%\begin{center}
%{\bf Piketty (2014) book: Capital in the 21st Century}
%\end{center}
%Analyzes income, wealth, inheritance data over the long-run:
%
%1) Growth rate $g$ = population growth + growth per capita. Population growth will converge
%to zero, growth per capita for frontier economies is modest (1-1.5\%) $\Rightarrow$ long-run $g \simeq 1-1.5\%$
%
%2) Long-run aggregate wealth to income ratio ($\beta$) = savings rate ($s$) / annual growth ($g$):
%
%Proof: $W_{t+1}=(1+g) \cdot W_t = W_t + s\cdot Y_t \Rightarrow W_t/Y_t = s /g$
%
%With $s=8\%$ and $g=2\%$, $\beta=400\%$ but with $s=8\%$ and $g=1\%$, $\beta=800\%$
%$\Rightarrow$ Wealth will become important
%
%\end{slide}


%\begin{slide}
%\begin{center}
%{\bf Piketty (2014) book: Capital in the 21st Century}
%\end{center}
%
%3) Rate of return on wealth $r \simeq 5\%$ significantly larger than $g$ [except exceptional period of 1940s-1960s]
%
%With $r >> g$, role of inheritance in wealth grows and wealth inequality increases [past swallows the future]
%
%\small
%Explanation: Rentier who saves all his return on wealth accumulates wealth at rate $r$ bigger than $g$ and hence
%his wealth grows relative to the size of the economy. The bigger $r-g$, the easier it is for wealth to ``snowball'': 
%fortunes are created faster and last longer
%\normalsize
%
%$\Rightarrow$ Capital income taxation reduces $r$ to $r \cdot (1-\tau_K)$ $\Rightarrow$ reduces wealth concentration and relative weight of inherited wealth
%
%
%\end{slide}
%
%\begin{slide}
%\includepdf[pages={10}]{taxsavings_new_attach.pdf}
%\end{slide}


%\begin{slide}
%\begin{center} {\bf KEY ELEMENTS OF DEBATE ON CAPITAL INCOME
%TAXATION}\end{center}
%
%Economic debate:
%
%1) Distributional concerns: capital income accrues
%disproportionately to higher income families
%
%2) Efficiency concerns: capital tax distorts savings, business
%creation, capital mobility across countries
%
%Public policy debate:
%
%3) Should we tax income vs. consumption? [Fundamental tax reform
%debate]
%
%4) Should we encourage savings by cutting tax on capital income or
%with tax favored savings vehicles?
%\end{slide}

\begin{slide}
\begin{center} {\bf  LIFE-CYCLE MODEL} \end{center}
Individual lives for 2 periods, works $l$, earns $w l$, consumes $c_1$ in period $1$, consumes $c_2$ in period $2$:
$$U=u(c_1,l)+ \delta \cdot v(c_2)$$
Start with case with no taxes

Savings $s=w l - c_1$, $c_2=(1+r)s$. Capital income $r s$
\[ \text{Intertemporal budget:} \quad
c_1+\frac{c_2}{1+r} \leq w l \]
\[ \max_{l,c_2} u \left (wl- \frac{c_2}{1+r} ,l \right )+ \delta \cdot v(c_2) \]
\[ \text{First order condition labor Supply: } \quad  w  \frac{\partial u}{\partial c_1} +  \frac{\partial u}{\partial l}=0 \]
\[ \text{First order condition savings: } \quad  \frac{\partial u}{\partial c_1} = \delta \cdot  (1+r)  \frac{\partial v}{\partial c_2} \]


\end{slide}

\begin{slide}
\begin{center} {\bf TAXES IN LIFE-CYCLE MODEL} \end{center}
1) Budget with consumption tax at rate $t_c$: $$(1+t_c) [ c_1+c_2/(1+r)] \leq
w l $$ Budget with labor income tax at rate $\tau_L$:
$$c_1+c_2/(1+r)  \leq (1-\tau_L)w l $$
2) Consumption and labor income tax are equivalent if
$$1+t_c=1/(1-\tau_L)$$ Both taxes distort only labor supply and not savings
%\[ \text{Labor Supply: } \quad w(1-\tau_L)  \frac{\partial u}{\partial c_1} +  \frac{\partial u}{\partial l}=0 \]

\end{slide}

\begin{slide}
\begin{center} {\bf TAXES IN LIFE-CYCLE MODEL} \end{center}
3) With capital income tax at rate $\tau_K$: $c_2=(1+r(1-\tau_K)) \cdot s \Rightarrow$
$$c_1+c_2/(1+r(1-\tau_K))
\leq w l $$ $\tau_K$ distorts only
savings choice (and not labor supply)

%\[ \text{Optimal savings: } \quad  \frac{\partial u}{\partial c_1} = \delta \cdot  (1+r(1-\tau_K))  \frac{\partial u}{\partial c_2} \]

4) With comprehensive income tax $\tau$ on both labor and capital income: $c_1=w(1-\tau)l-s, \quad c_2=(1+r(1-\tau))s$
$$c_1+c_2/(1+r(1-\tau)) \leq (1-\tau) w l$$ $\tau$ distorts both labor supply and savings

$\tau$ imposes ``double'' tax: on (1) earnings AND on
(2) savings
\end{slide}

\begin{slide}
\begin{center} {\bf EFFECT OF CAPITAL TAX ON SAVINGS} \end{center}
Consider simpler model (fixed earnings $w$ in period 1)
\[ \max_{c_1,c_2} u(c_1) + \delta \cdot u(c_2) \quad \text{subject to} \quad c_1 + \frac{c_2}{1+r(1-\tau_K)} \leq w \]

Recall that $c_1=w-s$ and $c_2= [1+r(1-\tau_K)] \cdot s$ [draw graph]

Suppose $\tau_K$ increases and hence $1/[1+r(1-\tau_K)]$ $\uparrow$

\textbf{1) Substitution effect:} price of $c_2$ $\uparrow$ $\Rightarrow$
$c_2 \downarrow$, $c_1$ $\uparrow$ $\Rightarrow$  savings
$s=w-c_1$ decrease

\textbf{2) Income effect:} consumer is
poorer $\Rightarrow$  both
$c_1$ and $c_2$ $\downarrow$ $\Rightarrow$ savings $s$ increase

Total net effect is theoretically ambiguous $\Rightarrow$ $\tau_K$
has ambiguous effects on $s$
\end{slide}


\begin{slide}
\includepdf[pages={19, 20, 21}]{taxsavings_new_attach.pdf}
\end{slide}

\begin{slide}
\begin{center} {\bf Fundamental tax reform: Shift to consumption taxation} \end{center}
Current US tax system is an income tax taxing both earnings and capital income

Some conservatives advocate shifting to consumption tax

Consumption tax is economically equivalent to taxing only labor earnings

But shift from labor tax to consumption tax generates double taxation of transitional
generation (who have paid labor tax when working and need to pay consumption tax when old)

Shift to consumption tax also generates a one time wealth tax (as accumulated wealth can now buy less)

Actual consumption taxes (such as value-added taxes) are regressive on an annual basis as
rich save a lot more than the poor (relative to income)

\end{slide}



\begin{slide}
\begin{center} {\bf OPTIMAL CAPITAL INCOME TAXATION}
\end{center}
Two broad types of models:

1) Life-cycle models: wealth is due solely to life-cycle savings

2) Models with bequests: wealth is due solely to inheritances

\end{slide}


\begin{slide}
\begin{center} {\bf Optimal Tax in Life-Cycle model} \end{center}

Government can use both a progressive labor income tax $T(wl)$ and a linear capital income tax $\tau_K$

Individuals live 2 periods, earn in period 1, retired in period 2
\[ \max_{c_1,c_2,l} u(c_1) -h(l) + \delta u(c_2) \quad \text{s.t.} \quad c_1 + \frac{c_2}{1+r(1-\tau_K)} \leq w l - T(wl) \]

Individuals differ only according to their earning ability $w$

Government maximizes social welfare function based on individual utilities

\textbf{Atkinson-Stiglitz JpubE'76 theorem:} The optimal tax $\tau_K$ on capital income should be zero.
Using a labor tax on earnings $T(wl)$ is sufficient.

\end{slide}


\begin{slide}
\begin{center} {\bf Optimal Tax in Life-Cycle model} \end{center}

Atkinson-Stiglitz' theorem shows that life-time savings should not be
taxed, tax only labor income

Key intuition: in basic life-cycle model, inequality in life-time resources is due solely
to differences in earnings ability. This inequality can be addressed with labor income
taxation. Capital income taxation needlessly distorts saving behavior.

From justice view: seems fair to not discriminate against savers
if labor earnings is the only source of inequality.
\end{slide}


\begin{slide}
\begin{center} {\bf LIMITS OF LIFE-CYCLE MODEL} \end{center}

In reality, capital income inequality also due

(1) difference in rates of returns across individuals

(2) inheritances

And distinction between labor income and capital income
is hard to make in practice
%shifting of labor income into capital income


%(4) tax evasion through off-shore accounts

\end{slide}

\begin{slide}
\begin{center} {\bf SHIFTING OF LABOR / CAPITAL INCOME} \end{center}
In practice, difficult to distinguish between capital and labor
income [e.g., small business profits, professional traders]

Differential tax treatment can induce shifting

(1) Carried interest in the US: hedge fund and private equity fund
managers receive fraction of profits of assets they manage for
clients. Those profits are really labor income but are taxed as
realized capital gains

(2) Finnish Dual income tax system: taxes separately capital
income at preferred rates since 1993: Pirttila and Selin SJE'11
show that it induced shifting from labor to capital income
especially among self-employed

With income shifting, taxing capital income becomes desirable to curb this tax
avoidance opportunity
\end{slide}


\begin{slide}
\begin{center} {\bf Difference in Rates of Returns Across Individuals} \end{center}
Rate of return on wealth varies significantly over time and across individuals

Example: stock market can gain 30\% in some years or lose 20\% in others

Specific stocks can increase much faster for successful start-ups (Google) or collapse entirely
for bankrupt firms (Enron)

In general, richer individuals are able to invest in higher return assets due to ability to take risks and
scale effects in financial advice [e.g., large University endowments get a larger return than smaller ones, Piketty 2014, Chapter 12]

$\Rightarrow$ Taxing capital income is a way to mitigate such inequality

\end{slide}




\begin{slide}
\begin{center} {\bf Inheritance: Estate Taxation in the United States} \end{center}

Estate federal tax imposes a tax on estates above \$11M exemption (less than  .1\% of deceased
liable), tax rate is 40\% above exemption (in 2018+)

Charitable and spousal giving are fully exempt from the tax

E.g.: if Bill Gates / Warren Buffet give all their wealth to charity, they won't pay estate tax

Popular support for estate tax is pretty weak (``death tax'') but public does not know that estate tax affects
only richest

Support for estate tax increase shots up from 17\% to 53\% when survey respondents are informed that only richest pay it
(Kuziemko-Norton-Saez-Stantcheva AER'15 do an online Mturk survey experiment)


\end{slide}

\begin{slide}
\includepdf[pages={7}]{taxsavings_new_attach.pdf}
\end{slide}


\begin{slide}
\begin{center} {\bf Taxation of Inheritances: Welfare Effects} \end{center}
%Definitions: donor is the person giving, donee is the person
%receiving

Inheritances (or gifts from living parents) raise difficult issues of social justice [see Kaplow 2001]:

(1) Inequality in inheritances contributes to economic inequality and individuals
not responsible for inheritances they receive:

$\Rightarrow$ seems fair to redistribute from those who received inheritances to
those who did not 

(2) However, it seems unfair to tax the parents who worked
hard (and already paid tax on income) to pass on wealth to children

Liberals emphasize (1) [taxing heirs] while conservative emphasize (2) [death tax]

%$\Rightarrow$ Double welfare effect: inheritance tax hurts donor
%(if donor altruistic to donee) and donee (which receives less) [Kaplow, '01]
\end{slide}


\begin{slide}
\begin{center} {\bf Taxation of Inheritances: Behavioral Responses} \end{center}

Potential behavioral response effects of inheritance tax:

(1) reduces wealth accumulation of altruistic parents (and hence
tax base) [no very good empirical evidence, Kopczuk-Slemrod 2001 suggest small effects]

(2) reduces labor supply of altruistic parents (less motivated to
work if cannot pass wealth to kids) [no good evidence]

(3) induces inheritors to work more through income effects because they receive smaller inheritances (Carnegie
effect, decent evidence from Holtz-Eakin,Joulfaian,Rosen QJE'93)

Critical to understand why there are inheritances for optimal inheritance tax policy. 3 models of bequests: (a)
accidental, (b) altruistic bequests, (c) social/family pressure 
\end{slide}


\begin{slide}
\begin{center} {\bf (a) ACCIDENTAL BEQUESTS} \end{center}

People die with a stock of wealth they intended to spend on
themselves (or that they accumulated out of love for wealth, Carroll '98): 

Bequest taxation has no distortionary effect on behavior of parent
and can only increase labor supply of inheritors (through income
effects) $\Rightarrow$ strong case for taxing bequests heavily

Surveys show that bequest motives are not the main driver of wealth accumulation
(Kopczuk-Lupton '07):

Only 1/3 of people surveyed say that the main reason they accumulate wealth is for
bequests to their children

\end{slide}

\begin{slide}
\begin{center} {\bf (b) ALTRUISTIC BEQUESTS (Piketty and Saez 2013)} \end{center}

Utility $u(c)-h(l)+ \delta v(b^{\text{left}})$ where $c$ is own consumption, $l$ is
labor supply, and $b^{\text{left}}$ is net-of-tax bequests left to next
generation and $v(b^{\text{left}})$ is utility of leaving bequests for donor

Individual receives $b^{\text{received}}$, works and earns $w l-T(w l)$, consumes
$c$, saves $s= w l-T(w l)+b^{\text{received}} - c$, which translates into $b^{\text{left}} = s (1+r) (1-\tau_B)$
for heir ($\tau_B$ is bequest tax rate)

Bequests provide an additional source of life-income:
\[c+\frac{b^{\text{left}}}{(1-\tau_B)(1+r)}=w l-T(w l)+b^{\text{received}} \]
In this model, Atkinson-Stiglitz breaks down and using bequest taxation is desirable
to supplement labor income taxation

$\Rightarrow$ Two-dimensional inequality (labor,bequests) requires two-dimensional tax policy tool
(labor tax, bequest tax)

\end{slide}



%\begin{slide}
%\begin{center} {\bf (c) MANIPULATIVE BEQUESTS} \end{center}
%
%Parents use potential bequest to extract favors from children
%
%Empirical Evidence: Bernheim-Shleifer-Summers JPE '85 show that
%number of visits of children to parents is correlated with
%bequeathable wealth but not annuitized wealth of parents
%
%
%[Annuitized wealth is wealth that disappears at death such as a pension or
%an annuity]
%\[ \text{Visits}_i = \alpha + \beta \cdot \text{Bequeathable Wealth}_i + \gamma \cdot \text{Annuitized wealth}_i + \varepsilon_i \]
%In regression, they find $\beta>0$ and $\gamma=0$ (but causality not clear)
%
%$\Rightarrow$ Bequest becomes one additional form of labor income
%for inheritor and one consumption good for parent
%
%$\Rightarrow$ Inheritances should be counted and taxed as labor
%income for donees
%
%\end{slide}

\begin{slide}
\begin{center} {\bf (c) SOCIAL-FAMILY PRESSURE BEQUESTS} \end{center}

Parents may not want to leave bequests but feel compelled to by
pressure of heirs or society: bargaining between parents and
children

With estate tax, parents do not feel like they need to give as
much $\Rightarrow$ parents are made better-off by the estate tax
$\Rightarrow$ Case for estate taxation stronger

Empirical evidence:

Aura JpubE'05: reform of private pension annuities in the US in
1984 requiring both spouses signatures when worker decides to get
a single annuity or couple annuity: reform increases sharply
couple annuities choice

Equal division of estates [Wilhelm AER'96, Light-McGarry '04]: estates are
very often divided equally probably to avoid conflicts [gifts before death are not as equally
split]
\end{slide}


\begin{slide}
\begin{center} {\bf US WEALTH TAX DEBATE} \end{center}

Recent proposals for progressive wealth tax (Warren, Sanders, CA). Various justifications from center left to radical left:

%(0) A well enforced wealth tax would reduce wealth concentration

\textbf{(1) Revenue:} US wealth is top heavy $\Rightarrow$ well enforced
wealth tax can raise substantial revenue 

\textbf{(2) Tax fairness:} super-rich do not need to ``realize'' income and hence pay fairly low
taxes relative to their true incomes (Warren Buffett example, Saez-Zucman '19)

\textbf{(3) Oligarchy risk:} wealth at the top is power.
Evidence from Robber Barons US 19th century and devo countries that entrenched wealth stifles growth (Acemoglu-Robinson '10)

\textbf{Concerns of opponents:} Wealth tax will be easy to avoid/evade. If not, wealth tax will
discourage entrepreneurs.

\end{slide}


\begin{slide}
\begin{center} {\bf US WEALTH TAX DEBATE} \end{center}

\textbf{Politically:} wealth tax is easy for public to understand as a tax on the rich
(and polls well even among republicans)

\textbf{Economically:} wealth tax powerful because 

(1) wealth tax goes after the stock while a capital income tax goes after the flow:
example if rate of return is $r=5\%$, a wealth tax at rate 5\% is like taxing capital income at 100\%

(2) wealth tax builds overtime: for billionaires, wealth tax mechanically reduces wealth by $(1-\tau_W)$ after 1 year,
$(1-\tau_W)^2$ after 2 years, ..., $(1-\tau_W)^t$ after t years, etc.

$\Rightarrow$ Billionaires can still arise but don't stay billionaires as long

\end{slide}

\begin{slide}
\includepdf[pages={27-28}]{taxsavings_new_attach.pdf}
\end{slide}

\begin{slide}
\includepdf[pages={30-31}]{taxsavings_new_attach.pdf}
\end{slide}


\begin{slide}
\begin{center}
{\bf COULD A WEALTH TAX BE ENFORCED?}
\end{center}
%Concerns: (a) can a wealth tax be properly enforced? [offshore evasion and valuation of businesses] (b) will it induce rich people to leave the US? (c) will it discourage entrepreneurs? [see Zucman-Saez '19b discussion]

Wealth taxes have been used in Europe but have been scaled down (and never
raised much revenue, except Switzerland). Suffered from 2 issues:

1) Tax competition concerns through offshore tax evasion and mobility of the
rich: could evade easily or move out to avoid

2) Exemption threshold too low (like \$1m) creating hardship for illiquid millionaires
(led to inefficient illiquid asset exemptions or tax limits based on reported income)

%3) Reliance on self-assessment (making enforcement hard)

Both weaknesses could be remedied:

1) Fight offshore tax evasion (FATCA) and tax expatriates 

2) Set high exemption threshold (\$50m rather than \$1m)

%3) Develop systematic information reporting

\end{slide}


\begin{slide}
\includepdf[pages={28-29}]{taxsavings_new_attach.pdf}
\end{slide}



\begin{slide}
\begin{center}
{\bf WEALTH IN TAX HAVENS}
\end{center}
Official statistics substantially underestimate the net foreign asset positions of rich countries 
bc they do not capture most of the assets held by households in off-shore tax havens

%Example: Wealthy US individual opens a Cayman Islands account and buys
%mutual fund shares (composed of US corporate stock): Cayman Islands record a liability
%but US do not record an asset (because this is not reported in the US) 

$\Rightarrow$ Total world liabilities are larger than world total assets

Zucman QJE'13 compiles international financial stats and estimates that around 8\% of the global financial wealth of households is held in tax
havens (3/4 of which is unrecorded = 6\%)

%If top 1\% hold about 50\% of total financial wealth, then about 12\% of financial wealth of the rich is hidden in tax heavens

Alstadsaeter-Johannesen-Zucman '19 link data from HSBC leak of accounts to Norwegian tax data

$\Rightarrow$ offshore evasion super concentrated among wealthy and pretty large at the very top

\end{slide}

\begin{slide}
\includepdf[pages={25, 24}]{taxsavings_new_attach.pdf}
\end{slide}


\begin{slide}
\begin{center}
{\bf CURBING OFF-SHORE TAX EVASION}
\end{center}
Offshore tax evasion possible because of bank secrecy. Example: France cannot get a list
of French individuals owning Swiss bank accounts from Switzerland

$\Rightarrow$ No 3rd party reporting makes tax enforcement very difficult

In principle, problem could be solved with exchange of information across
countries BUT need all countries to cooperate

Johannesen-Zucman AEJ-EP'14 analyze tax haven crackdown: 
G20 countries forced a number of tax havens to sign bilateral treaties
on bank information sharing

Key result: Instead of repatriating funds, tax evaders
shifted deposits to havens not covered by treaty with home country.

\end{slide}


\begin{slide}
\begin{center}
{\bf CURBING OFF-SHORE TAX EVASION}
\end{center}

\textbf{FATCA'10 US regulations} impose information exchange for all entities dealing with US: 

If foreign bank B
does not provide list of all its US account holders, any financial transaction between B and US will carry
30\% tax withholding $\Rightarrow$ Interesting to see what it will do

\small
Leaks like HSBC and Panama papers make tax evasion riskier
\normalsize

\textbf{Long-term solution} will require:

a) Systematic registration of assets to ultimate owners [already exists within countries for domestic
tax enforcement]

b) Systematic information exchange between tax countries with no exceptions for tax heavens 

$\Rightarrow$ Could be enforced with tariffs threats on tax heavens [Zucman JEP'14 and book '15]

\end{slide}



\begin{slide}
\begin{center}
{\bf REFERENCES}
\end{center}
{\small

Alstadsaeter, Annette, Niels Johannesen, and Gabriel Zucman 2019 ``Tax Evasion and Inequality'', American Economic Review forthcoming.
\href{http://elsa.berkeley.edu/~saez/course/AJZ2017.pdf} {(web)}

Alvaredo, Facundo, Bertrand Garbinti, and Thomas Piketty. 2017, ``On the share of inheritance in aggregate wealth: Europe and the USA, 1900-2010." Economica 84(334), 239-260.
\href{http://elsa.berkeley.edu/~saez/course/Alvaredoetal2017Economica.pdf} {(web)}

Atkinson, A.B. and J. Stiglitz ``The Design of Tax Structure: Direct Versus Indirect Taxation'', Journal of Public Economics, Vol. 6, 1976, 55-75. \href{http://elsa.berkeley.edu/~saez/course/AtkinsonStiglitz_JPubE(1976).pdf} {(web)}

Aura, S. ``Does the Balance of Power Within a Family Matter? The Case of the Retirement Equity Act'', Journal of Public Economics, Vol. 89, 2005, 1699-1717. \href{http://elsa.berkeley.edu/~saez/course/Aura_JPubE(2005).pdf} {(web)}

Bernheim, B. D., A. Shleifer, and L. Summers ``The Strategic Bequest Motive'', Journal of Political Economy, Vol. 93, 1985, 1045-76. \href{http://links.jstor.org/stable/pdfplus/1833175.pdf} {(web)}

Carroll, C. ``Why Do the Rich Save So Much?'', NBER Working Paper No. 6549, 1998. \href{http://www.nber.org/papers/w6549.pdf} {(web)}

%Christiansen, Vidar and Matti Tuomala ``On taxing capital income with income shifting'', International Tax and Public Finance, Vol. 15, 2008, 527-545. \href{http://elsa.berkeley.edu/~saez/course/christiansen-tuomalaITAX08capitalincomeshifting.pdf} {(web)}

%Davies, J. and A. Shorrocks, Chapter 11 The distribution of wealth, In: Anthony B. Atkinson and Francois Bourguignon, Editor(s), Handbook of Income Distribution, Elsevier, 2000, Vol. 1, 605-675. \href{http://elsa.berkeley.edu/~saez/course/Davies,Shorrocks(2000).pdf} {(web)}

%DeLong, J.B. ``Bequests: An Historical Perspective,'' in A. Munnell, ed., \emph{The Role and Impact of Gifts and Estates}, Brookings Institution, 2003\href{http://elsa.berkeley.edu/~saez/course/DeLong(2003)_MunnellBook.pdf} {(web)}

%Diamond, Peter and Emmanuel Saez ``The Case for a Progressive Tax: From Basic Research to Policy Recommendations'', Journal of Economic Perspectives, 25(4), Fall 2011, 165-190.\href{http://elsa.berkeley.edu/~saez/diamond-saezJEP11full.pdf} {(web)}

%Finkelstein A. and J. Poterba, ``Adverse Selection in Insurance Markets: Policyholder Evidence from the U.K. Annuity Market'', Journal of Political Economy, Vol. 112, 2004, 183-208. \href{http://links.jstor.org/stable/pdfplus/3555197.pdf} {(web)}
%
%Finkelstein A. and J. Poterba, ``Selection Effects in the United Kingdom Individual Annuities Market'', The Economic Journal, Vol. 112, 2002, 28-50. \href{http://www.jstor.org/stable/pdfplus/798430.pdf} {(web})


%Gordon, R.H. and J. Slemrod ``Are ``Real'' Responses to Taxes Simply Income Shifting Between Corporate and Personal Tax Bases?,'' NBER Working Paper, No. 6576, 1998. \href{http://www.nber.org/papers/w6576} {(web)}

Gravelle, Jane. ``The Economic Effects of Taxing Capital Income.'' [Book] MIT Press, 1994.

Holtz-Eakin, D., D. Joulfaian and H.S. Rosen ``The Carnegie Conjecture: Some Empirical Evidence'', Quarterly Journal of Economics Vol. 108, May 1993, pp.288-307. \href{http://links.jstor.org/stable/pdfplus/2118337.pdf} {(web)}

Johannesen, Niels  and Gabriel Zucman ``The End of Bank Secrecy? An Evaluation of the G20 Tax Haven Crackdown,''
forthcoming in American Economic Journal: Economic Policy, 2013.
\href{http://elsa.berkeley.edu/~saez/course/johannesen-zucmanAEJ13.pdf} {(web)}

Kaplow, L. ``A Framework for Assessing Estate and Gift Taxation'', in
Gale, William G., James R. Hines Jr., and Joel Slemrod (eds.) {\em Rethinking estate and gift taxation} Washington, D.C.: Brookings Institution Press, 2001.
\href{http://www.nber.org/papers/w7775.pdf} {(web)}

%Kennickell, A. ``Ponds and streams: wealth and income in the U.S., 1989 to 2007'', Board of Governors of the Federal Reserve System (U.S.), Finance and Economics Discussion Series: 2009-13, 2009. \href{http://elsa.berkeley.edu/~saez/course/Kennickell(2009).pdf} {(web)}

%King, M. ``Savings and Taxation'', in G. Hughes and G. Heal, eds., Public Policy and the Tax System (London: George Allen Unwin, 1980), 1-36. \href{http://elsa.berkeley.edu/~saez/course/King(1980).pdf} {(web)}

%Kopczuk, Wojciech ``The Trick Is to Live: Is the Estate Tax Social Security for the Rich?'', The Journal of Political Economy, Vol. 111, 2003, 1318-1341. \href{http://www.jstor.org/stable/pdfplus/10.1086/378529.pdf} {(web)}

Kopczuk, Wojciech and Joseph Lupton 2007. ``To Leave or Not to Leave: The Distribution
of Bequest Motives,'' Review of Economic Studies, 74(1), 207-235.
\href{http://www.jstor.org/stable/pdfplus/4123242.pdf} {(web)}

%Kopczuk, Wojciech and Emmanuel Saez ``Top Wealth Shares in the United States, 1916-2000: Evidence from Estate Tax Returns'', National Tax Journal, 57(2), Part 2, June 2004, 445-487.\href{http://elsa.berkeley.edu/~saez/course/kopczuksaez04.pdf} {(web)}

Kopczuk, Wojciech and Joel Slemrod, ``The Impact of the Estate Tax on the Wealth Accumulation and Avoidance Behavior of Donors'', in William G. Gale, James R. Hines Jr., and Joel B. Slemrod (eds.), \emph{Rethinking Estate and Gift Taxation,} Washington, DC: Brookings Institution Press, 2001, 299-343.
\href{http://www.nber.org/papers/w7960.pdf} {(web)}

Kuziemko, Ilyana, Michael I. Norton, Emmanuel Saez, and Stefanie Stantcheva ``How Elastic are Preferences for Redistribution? Evidence from Randomized Survey Experiments,''  American Economic Review, 105(4), 2015, 1478-1508 \href{https://eml.berkeley.edu/~saez/kuziemko-norton-saez-stantchevaAER15.pdf} {(web)}

Light, Audrey and Kathleen McGarry. ``Why Parents Play Favorites: Explanations For Unequal Bequests,'' American Economic Review, 2004, v94(5,Dec), 1669-1681.
\href{http://www.jstor.org/stable/pdfplus/3592839.pdf} {(web)}

%Modigliani, F. ``The Role of Intergenerational Transfers and Lifecyle Savings in the Accumulation of Wealth'', Journal of Economic Perspectives, Vol. 2, 1988, 15-40. \href{http://links.jstor.org/stable/pdfplus/1942847.pdf} {(web)}

Norton, M. and D. Ariely ``Building a Better America--One Wealth Quintile at a Time'',
Perspectives on Psychological Science 2011 6(9).
\href{http://elsa.berkeley.edu/~saez/course/norton-ariely11.pdf} {(web)}

%Piketty, T. ``On the Long-Run Evolution of Inheritance: France 1820-2050'', Quarterly Journal of Economics, 126(3), 2011, 1071-1131. \href{http://elsa.berkeley.edu/~saez/course/PikettyQJE11.pdf} {(web)}

%Piketty, T. ``Wealth and Inheritance in the Long-Run'', Handbook of Income Distribution, Volume 2, Elsevier-North Holland,in preparation (slides April 2013)\href{http://elsa.berkeley.edu/~saez/course/Piketty2013HIDSlides.pdf} {(web)}

Piketty, Thomas, \emph{Capital in the 21st Century},  Cambridge: Harvard University Press, 2014,  
\href{http://piketty.pse.ens.fr/en/capital21c2}{(web)}

Piketty, Thomas, \emph{Capital and Ideology},  Cambridge: Harvard University Press, 2020, 
\href{http://piketty.pse.ens.fr/en/ideology}{(web)}

Piketty, T. and E. Saez ``A Theory of Optimal Inheritance Taxation'', Econometrica, 81(5), 2013, 1851-1886.
\href{http://elsa.berkeley.edu/users/saez/piketty-saezECMA13.pdf} {(web)} 

Piketty, Thomas, Emmanuel Saez, and Gabriel Zucman,  ``Distributional National Accounts:
Methods and Estimates for the United States'', NBER Working Paper No. 22945, 2016.
\href{http://www.nber.org/papers/w22945.pdf} {(web)}

Piketty, Thomas and Gabriel Zucman,  ``Capital is Back: Wealth-Income Ratios in Rich Countries, 1700-2010'',  \emph{Quarterly Journal of Economics} 129(3), 2014, 1255-1310
\href{http://www.econ.berkeley.edu/~saez/PikettyZucman2014QJE.pdf} {(web)}

Pirttila, Jukka and Hakan Selin, ``Income Shifting Within a Dual Income Tax System: Evidence from the Finnish Tax Reform,'' Scandinavian Journal of Economics, 113(1), 120-144, 2011.
\href{http://elsa.berkeley.edu/~saez/course/pirttila-selinSJE11.pdf} {(web)}

%Saez, Emmanuel and Stefanie Stantcheva, �A Simpler Theory of Optimal Capital Taxation�, NBER Working Paper No. 22664, 2016 \href{http://www.nber.org/papers/w7960.pdf} {(web)}

Saez, Emmanuel and Gabriel Zucman, ``Wealth Inequality in the United States since 1913: Evidence from Capitalized Income Tax Data'', \emph{Quarterly Journal of Economics}  131(2), 2016, 519-578
\href{http://eml.berkeley.edu/~saez/SaezZucman2016QJE.pdf}{(web)}

Saez, Emmanuel and Gabriel Zucman. 2019. The Triumph of Injustice: How the Rich Dodge Taxes and How to Make them Pay, New York: W.W. Norton, 
\href{http://www.taxjusticenow.org} {(web)}

Saez, Emmanuel and Gabriel Zucman. 2019b. ``Progressive Wealth Taxation''
Brookings Paper
\href{https://eml.berkeley.edu/~saez/saez-zucmanBPEAoct19.pdf}{(web)}

Wilhelm, Mark O.  ``Bequest Behavior and the Effect of Heirs' Earnings: Testing the Altruistic Model of Bequests,'' American Economic Review, 86(4), 1996, 874-892. \href{http://www.jstor.org/stable/pdfplus/2118309.pdf} {(web)}

Zucman, G. ``The Missing Wealth of Nations: Are Europe and the US Net
Debtors or Net Creditors'', Quarterly Journal of Economics, 2013, 1321-1364. \href{http://elsa.berkeley.edu/~saez/course/zucmanQJE13.pdf} {(web)}

Zucman, G. \emph{The Hidden Wealth of Nations}, September 2015, University of Chicago Press.
\href{http://gabriel-zucman.eu/hidden-wealth/} {(web)}

Zucman, G.  ``Taxing across Borders: Tracking
Personal Wealth and Corporate Profits'' Journal of Economic Perspectives, 28(4), 2014, 121-148.
\href{http://elsa.berkeley.edu/~saez/course/Zucman2014JEP.pdf} {(web)}

}



\end{slide}



\end{document}



