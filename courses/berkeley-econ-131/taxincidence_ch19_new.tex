\documentclass[landscape]{slides}

\usepackage[landscape]{geometry}

\usepackage{pdfpages}

\usepackage{hyperref}
\usepackage{amsmath}

\def\mathbi#1{\textbf{\em #1}}

\topmargin=-1.8cm \textheight=17cm \oddsidemargin=0cm
\evensidemargin=0cm \textwidth=22cm


\author{131 Undergraduate Public Economics \\ Emmanuel Saez \\ UC Berkeley}
\date{}

\title{Incidence and Efficiency Costs of Taxation \\ (Chapters 19-20 of Gruber's textbook)}
\onlyslides{1-300}

\newenvironment{outline}{\renewcommand{\itemsep}{}}

\begin{document}

\begin{slide}
\maketitle
\end{slide}

%\begin{slide}
%\begin{center}
%{\bf OUTLINE}
%\end{center}
%Chapters 19-20
%
%
%19.1 The Three Rules of Tax Incidence
%
%19.2 Tax Incidence Extensions
%
%19.3 General Equilibrium Tax Incidence
%
%19.4 The Incidence of Taxation in the United States
%
%20.1 Efficiency Costs of Taxation
%
%20.2 Optimal Commodity Taxation
%
%
%\end{slide}


%\begin{slide}
%\includepdf[pages={1}]{taxincidence_ch19_new_attach.pdf}
%\end{slide}


\begin{slide}
\begin{center}
{\bf TAX INCIDENCE}
\end{center}

Tax incidence is the study of the effects of tax policies on prices
and the economic welfare of individuals

What happens to market prices when a tax is introduced or changed?


Example: what happens when impose \$1 per pack tax on cigarettes?

Effect on price $\Rightarrow$ distributional effects on smokers,
profits of producers, shareholders, farmers, etc.


This is positive analysis: typically the first step in policy
evaluation; it is an input to later thinking about what policy maximizes
social welfare.

\end{slide}


\begin{slide}
\begin{center}
{\bf TAX INCIDENCE}
\end{center}

Tax incidence is not an accounting exercise but an analytical
characterization of changes in economic equilibria when taxes are changed.

Key point: Taxes can be shifted: taxes affect directly prices, 
which affect quantities because of behavioral responses, which affect
indirectly the price of other goods.

If prices are constant economic incidence would be the same as
legislative incidence.

\small
Example: Liberals favor capital income taxation because capital income
is concentrated at the high end of the income distribution. Taxing capital
means taxing disproportionately the rich.

Conservatives respond: if
people save less because of capital taxes, capital stock may go down driving
also the wages down and hurting workers. The capital tax might be shifted partly on
workers.


\end{slide}



%\begin{slide}
%\begin{center}
%{\bf TAX INCIDENCE}
%\end{center}
%
%Ideally, we want to know the effect of a tax change on utility levels
%of all agents in the economy.
%
%Realistically, we usually look at impacts on prices or income, rather
%than utility
%
%Useful simplification is to aggregate economic agents into a few
%groups. Examples:
%
%\small
%1)  gas tax: producers vs consumers
%
%2) income tax: rich vs poor
%
%3) property tax: region or county
%
%4) social security: across generations
%
%
%\end{slide}

\begin{slide}
\begin{center}
{\bf Partial Equilibrium Tax Incidence}
\end{center}

Partial Equilibrium Model:

Simple model goes a long way to showing main results.

Government levies an excise tax on good $x$

Excise means it is levied on a quantity (gallon, pack, ton, ...).
Typically fixed in nominal terms (e.g, \$1 per pack) \

[ad-valorem tax is a fraction of prices (e.g. 5\% sales tax)]
 
Let $p$ denote the pretax price of $x$ (producer price)

Let $p^c=p+t$ denote the tax
inclusive price of $x$ (consumer price) 

Draw graph on blackboard
\end{slide}

\begin{slide}
\includepdf[pages={41-43}]{taxincidence_ch19_new_attach.pdf}
\end{slide}


\begin{slide}
\begin{center}
{\bf TAX INCIDENCE}
\end{center}
Demand for good $x$ is $D(p^c)$ decreases with $p^c=p+t$

Supply for good $x$ is $S(p)$ increases with $p$

Equilibrium condition with tax $t$: $Q=S(p)=D(p+t)$

Start from $t=0$ and $S(p)=D(p)$

We want the effect of a small tax $dt$ on
price $p$: $dp/dt$:

Change $dt$ generates change $dp$ so that equilibrium holds:
\small
\[ S(p+dp)=D(p+dp+dt) \Rightarrow \] \[ S(p)+S'(p)dp = D(p)+D'(p)(dp+dt) \Rightarrow \]
\[  S'(p)dp = D'(p)(dp+dt) \Rightarrow \]
\normalsize
\[
\frac{dp}{dt}=\frac{D'(p)}{S'(p)-D'(p)}
\]



\end{slide}

\begin{slide}
\begin{center}
{\bf TAX INCIDENCE FOR SMALL TAX $dt$}
\end{center}
Elasticities useful in economics because they are
unit free

\textbf{Elasticity:} percentage change in quantity when price changes by one
percent

$\varepsilon _{D}= \frac{ p^c} {D} \frac{ dD} {dp^c} = \frac{p^c D'(p^c)}{D(p^c)}<0$ denotes
the price elasticity of demand

%(consumer faces price $q=p+t$)

$\varepsilon _{S}=\frac{ p} {S} \frac{ dS} {dp} =  \frac{pS'(p)}{S(p)}>0$ denotes
the price elasticity of supply
\[
\frac{dp}{dt}=\frac{D'(p)}{S'(p)-D'(p)} = \frac{pD'(p)/D(p)}{pS'(p)/S(p)-pD'(p)/D(p)}  = \frac{\varepsilon _{D}}{\varepsilon _{S}-\varepsilon _{D}}
\]
\[ -1 \leq \frac{dp}{dt} \leq 0 \quad \text{and} \quad 0 \leq \frac{dp^c}{dt}=1+\frac{dp}{dt} \leq 1\]

\end{slide}





\begin{slide}
\begin{center}
{\bf TAX INCIDENCE}
\end{center}
\[
\frac{dp}{dt}=\frac{\varepsilon _{D}}{\varepsilon _{S}-\varepsilon _{D}}
\]
When do consumers bear the entire burden of the tax? ($dp/dt=0$ and $dp^c/dt=1$)

\small
1) $\varepsilon _{D}=0$ [inelastic demand]

example: short-run demand for gasoline inelastic (need to drive to work)

2) $\varepsilon _{S}=\infty $ [perfectly elastic supply]

example: perfectly competitive industry

\normalsize
When do producers bear the entire burden of the tax? ($dp/dt=-1$ and $dp^c/dt=0$)

\small
1) $\varepsilon _{S}=0$ [inelastic supply]

example: fixed quantity supplied

2) $\varepsilon _{D}=-\infty $ [perfectly elastic demand]

example: there is a close substitute, and demand shifts to this
substitute if price changes.


\end{slide}


\begin{slide}
\includepdf[pages={3,4,5}]{taxincidence_ch19_new_attach.pdf}
\end{slide}

\begin{slide}
\begin{center}
{\bf TAX INCIDENCE: KEY RESULTS}
\end{center}
1) statutory incidence not equal to economic incidence

2) equilibrium is independent of who nominally pays the tax

3) more inelastic factor bears more of the tax

These are robust conclusions that hold with more complicated models

\end{slide}





%\begin{slide}
%\begin{center}
%{\bf TAX INCIDENCE: EXTENSIONS}
%\end{center}
%
%1) Market rigidities (suppose there is a minimum or maximum price) then
%standard analysis does not carry over
%
%\small
%Example: minimum wage. Social security taxes 7.65\% on employer and
%7.65\% on employee. In principle the share of each should not matter as long
%as total is constant but minimum wage is computed on net wage (gross wage -
%employer tax = net wage + employee tax).
%\normalsize
%
%
%%2) Monopolistic competition $\Rightarrow$ Changes basic incidence analysis
%
%2) Effects on other markets:
%
%\small
%Example: Suppose tax on cigarettes increases, if people substitute
%cigarettes for cigars then price of cigars increases and part of the burden
%is shifted to the cigar market and cigarette demand curves will move.
%
%Revenue effects on other markets: tax increases, I am poorer, I have
%less to spend on other markets.
%
%For small, narrow markets such as cigarettes, partial eq. analysis is a
%reasonable approximation (although effects on substitutes could be
%important).
%
%\end{slide}

%\begin{slide}
%\includepdf[pages={7}]{taxincidence_ch19_new_attach.pdf}
%\end{slide}


%\begin{slide}
%\begin{center}
%{\bf Efficiency Costs of Taxation}
%\end{center}
%Thus far, we have focused on the incidence of government policies:
%how price interventions affect equilibrium prices and factors returns:
%how policies affect the \textbf{distribution} of the pie
%
%A second general set of questions is how taxes affect the \textbf{size} of the
%pie.
%
%Example: income taxation
%
%Government raises taxes to raise revenue to finance public goods or
%to redistribute income from rich to poor.
%
%But raising tax revenue generally has an efficiency cost: to generate \$1
%of revenue, need to reduce welfare of the taxed individuals by more
%than \$1
%
%Efficiency costs come from distortion of behavior
%
%\end{slide}

\begin{slide}
\begin{center}
{\bf Efficiency Costs of Taxation}
\end{center}

Deadweight burden (also called excess burden) of taxation is defined
as the welfare loss (measured in dollars) created by a tax over and above
the tax revenue generated by the tax

In the simple supply and demand diagram, welfare is measured by the sum
of the consumer surplus and producer surplus

The welfare loss of taxation is measured as change in consumer+producer surplus
minus tax collected: it is the triangle on the figure

\small
The inefficiency of any tax is determined by the extent to which consumers and producers change their behavior to avoid the tax; deadweight loss is caused by individuals and firms making inefficient consumption and production choices in order to avoid taxation.

If there is no change in quantities consumed, the tax has no efficiency costs

\end{slide}

\begin{slide}
\includepdf[pages={44}]{taxincidence_ch19_new_attach.pdf}
\end{slide}


\begin{slide}
\begin{center}
{\bf Efficiency Costs of Taxation}
\end{center}
Deadweight burden (or deadweight loss) of small tax $dt$ (starting from zero tax) is measured
by the \textbf{Harberger Triangle}: \\ \smallskip
\[ DWB = \frac{1}{2} dQ \cdot dt =  \frac{1}{2} S'(p) \cdot  dp \cdot dt =  \frac{1}{2} \frac{pS'(p)}{S(p)} \cdot  \frac{Q}{p} \cdot dp \cdot  dt \]
\smallskip [recall that $Q=S(p)$ and hence $dQ=S'(p)dp$]

Recall that $dp/dt=\varepsilon_{D}/(\varepsilon _{S}-\varepsilon _{D})$, hence:
\[ DWB = \frac{1}{2}  \cdot \frac{\varepsilon_S \cdot \varepsilon_D }{\varepsilon _{S}-\varepsilon _{D}} \cdot \frac{Q}{p} (dt)^2 \]
\end{slide}



\begin{slide}
\begin{center}
{\bf Efficiency Costs of Taxation}
\end{center}
\[ DWB = \frac{1}{2}  \cdot \frac{\varepsilon_S \cdot \varepsilon_D }{\varepsilon _{S}-\varepsilon _{D}} \cdot \frac{Q}{p} (dt)^2 \]
1) $DWB$ increases with the absolute size of elasticities $\varepsilon _{S}>0$ and
$-\varepsilon _{D}>0$

\small $\Rightarrow$ More efficient to tax relatively inelastic goods \normalsize

2) $DWB$ increases with the square of the tax rate $t$: small taxes have relatively small efficiency costs, large taxes
have relatively large efficiency costs

\small
$\Rightarrow$ Better to spread taxes across all goods to keep each tax rate low

$\Rightarrow$ Better to fund large one time govt expense (such as a war) with debt and repay slowly afterwards
than have very high taxes only during war

 \normalsize
3) Pre-existing distortions (such as an existing tax) makes the cost
of taxation higher: move from the triangle to trapezoid


\end{slide}

\begin{slide}
\includepdf[pages={9,10}]{taxincidence_ch19_new_attach.pdf}
\end{slide}



\begin{slide}
\begin{center}
{\bf Application: Optimal Commodity Taxation}
\end{center}
Ramsey (1927) asked by Pigou to solve the following problem:

Consider one consumer who consumes $K$ different goods

What are the tax rates $t_1,..,t_K$ of each good that raise a given amount of revenue while minimizing
the welfare loss to the individual?

Uniform tax rates $t=t_1=..=t_K$ is not optimal if the individual has more elastic demand for some goods than
for others

Optimum is called the \textbf{Ramsey tax rule}: optimal tax rates are such that the marginal DWB for last dollar of tax
collected is the same across
all goods 

$\Rightarrow$ Tax more the goods that have inelastic demands [and tax less the goods that have elastic demands]

\small
Note: this abstracts from redistribution and focuses solely on efficiency
\end{slide}

\begin{slide}
\begin{center}
{\bf Tax Incidence: Empirical Application}
\end{center}

European countries have large taxes on consumption: Value Added Tax (VAT)

Normal VAT rates are high (15-25\%) but some goods/services have lower rates (or are exempt)

Benzarti et al. (2020) study the effects of VAT rates $\uparrow$ and $\downarrow$

Nice illustrative case study: hairdressers in Finland got a VAT cut of 14 points in Jan 2007 that was repealed in Jan 2012

Provide a basic graphical \textbf{difference-in-difference} analysis of prices of hairdressers (treatment) with beauty salons (control)

\small

$\Rightarrow$ Find that tax decreases are only 50\% passed on consumers while tax increases are almost fully passed on consumers. 

Most likely explanation: producers pocket tax cut bc consumers are inattentive to taxes.
Producers pass tax increase because they can justify the price increase to consumers.

$\Rightarrow$ Price determination does not work like basic model 
 
\end{slide}

\begin{slide}
\includepdf[pages={45}]{taxincidence_ch19_new_attach.pdf}
\end{slide}



%\begin{slide}
%\begin{center}
%{\bf Tax Incidence: Empirical Application}
%\end{center}
%
%Doyle and Sampatharank (2008) study the Gas Tax Holidays in Indiana (IN) and Illinois (IL).
%
%%Does it make sense to cut gas taxes to help families in recessions?
%
%Are gas tax cuts passed through to consumers? or do producers pocket the
%tax cut and leave consumer price unchanged?
%
%Study this question using state-level gas tax reforms
%
%\small
%Gas prices spike above \$2.00 in 2000
%
%IN suspends 5\% gas tax on July 1. Reinstated on Oct 30.
%
%IL suspends 5\% gas tax on July 1. Reinstated on Dec 31.
%\end{slide}
%
%
%\begin{slide}
%\begin{center}
%{\bf Tax Incidence: Empirical Application}
%\end{center}
%
%Empirical approach in paper: difference-in-difference (DD), compare treated states with
%neighboring states (MI, OH, MO, IA, WI) before and after tax change
%
%Graphical evidence is most transparent. Findings:
%
%1) 10 cent increase in gas tax $\Rightarrow$ 7 cent increase in price paid by consumers
%
%2) Consumers bear 70\% of incidence of the gas tax (and conversely, get 70\% of the benefit of a gas tax cut)
%
%
%\end{slide}
%
%\begin{slide}
%\includepdf[pages={12-14}]{taxincidence_ch19_new_attach.pdf}
%\end{slide}

%\begin{slide}
%\includepdf[pages={21,23}]{HoynesLec1-Tax-Incidence.pdf}
%\end{slide}


%\begin{slide}
%\begin{center}
%{\bf Tax Incidence: Empirical Application}
%\end{center}
%
%
%
%3) Test predictions of incidence theory further by checking if incidence on consumers is higher in areas near borders between states
%
%Idea: demand more elastic in border towns $\Rightarrow$ expect higher incidence of gas tax on suppliers near border.
%
%\end{slide}
%
%
%\begin{slide}
%\includepdf[pages={19}]{Chetty_lec_1_2012.pdf}
%\end{slide}


\begin{slide}
\begin{center}
{\bf Difference-in-Difference (DD) methodology}
\end{center}

Two groups: Treatment group (T) which faces a change [hairdressers] and control group (C) which does not [beauty salons]

Compare the evolution of T group (before and after change) to the evolution of the C group (before and after change)

DD identifies the \textbf{treatment effect} if the parallel trend assumption holds:

Absent the change, $T$ and $C$ would have evolved in parallel 

DD most convincing when groups are very similar to start with

Should always test DD using data from more periods and plot the two time
series to check parallel trend assumption

\end{slide}

\begin{slide}
\includepdf[pages={49-51}]{taxincidence_ch19_new_attach.pdf}
\end{slide}

\begin{slide}
\begin{center}
{\bf General Equilibrium Tax Incidence}
\end{center}
Examples so far have focused on \textbf{partial equilibrium} incidence which considers impact of a tax on one market in isolation

\textbf{General equilibrium} models consider the effects on related markets of a tax imposed on one market

E.g. imposition of a tax on cars may reduce demand for steel $\Rightarrow$ additional effects on prices in equilibrium beyond car market.

\end{slide}

\begin{slide}
\begin{center}
{\bf General Equilibrium Tax Incidence:\\ Example: Soda Tax in Berkeley}
\end{center}

Consider the market for Soda beverages in Berkeley 

Berkeley imposes a Soda tax since 2015: \$.01 per ounce (=\$.12/can)

Goal was to reduce soda consumption for better health (people overdrink).
See Allcott et al. '18 for merits of soda tax. 

Here narrower question: Who bears the incidence?

If soda demand in Berkeley is inelastic, consumers bear burden 

Demand for Soda in Berkeley is likely to be elastic: if price of Soda in Berkeley goes up, you consume less Soda [intention of the tax] or you buy Soda elsewhere

Consider extreme case of perfectly elastic demand 

\end{slide}

\begin{slide}
\includepdf[pages={16}]{taxincidence_ch19_new_attach.pdf}
\end{slide}


\begin{slide}
\begin{center}
{\bf General Equilibrium Tax Incidence: \\ Example: Soda Tax}
\end{center}
If Soda demand perfectly elastic then:

1) Berkeley Soda sellers (supermarkets, restaurants) cannot charge more and hence bear the full burden of the tax.

2) But Soda sellers are not self-contained entities

\small
Companies are just a technology for combining capital and labor to produce an output.

Capital: land, physical inputs like building, kitchen equipment, etc.

Labor: cashier staff, cooks, waitstaff, etc.
\normalsize

3) Ultimately, these two factors (capital or labor) must bear the loss in profits due to the tax
[if consumer demand is perfectly elastic]


\end{slide}

\begin{slide}
\begin{center}
{\bf General Equilibrium Tax Incidence: \\ Example: Soda Tax}
\end{center}

Incidence is ``shifted backward'' to capital and labor.

Assume that labor supply is perfectly elastic because Berkeley restaurant/supermarket workers can always go and work in Oakland if they get paid less in Berkeley

Capital, in contrast, is perfectly inelastic in short-run: you cannot pick up the shop and move it in the short run.

In short run, capital bears tax because it is completely inelastic $\Rightarrow$ Soda business owners lose (not consumers or workers)

In the longer-run, the supply of capital is also likely to be highly elastic: Investors can close or sell the shop, take their money, and invest it elsewhere.

\end{slide}

%\begin{slide}
%\includepdf[pages={17}]{taxincidence_ch19_new_attach.pdf}
%\end{slide}

%\begin{slide}
%\begin{center}
%{\bf General Equilibrium Tax Incidence: \\ Example: Soda Tax}
%\end{center}
%
%In short run, capital bears tax because it is completely inelastic $\Rightarrow$ Soda tax owners lose (not consumers or workers)
%
%In the longer-run, the supply of capital is also likely to be highly elastic.
%
%Investors can close or sell the shop, take their money, and invest it elsewhere.
%
%%There are many good substitutes for investing in a particular restaurant in a particular town.
%
%\end{slide}


\begin{slide}
\begin{center}
{\bf General Equilibrium Tax Incidence: Long-run effects}
\end{center}

If both labor and capital are highly elastic in the long run, who bears the tax?

The one additional inelastic factor is land.

%\small
The supply is clearly fixed.

When both labor and capital can avoid the tax, the only way Soda sellers will remain in Berleley is if they pay a lower rent on their land.
%\normalsize

$\Rightarrow$ Soda tax ends up hurting Berkeley landowners in general equilibrium [if Soda demand,
labor and capital are fully elastic]

This is of course an idealized example, in practice, demand, labor, and capital are not fully elastic
so that incidence is shared 
\end{slide}


%\begin{slide}
%\begin{center}
%{\bf Asset Price Approach to Incidence}
%\end{center}
%
%General Equilibrium Incidence is hard to calculate empirically because of large number of effects in equilibrium
%
%One potential solution: look at asset prices, e.g. the value of stocks or houses (��capitalization��)
%
%Consider an increase in tax on car companies
%
%Incidence could partly be shifted to consumers, workers, etc.
%
%Can easily summarize overall net effect on GM by looking at how its stock price changes when tax is announced.
%
%Limitation of asset price approach: can only be used for capital owners (Applications: corporate tax, environmental policy)
%
%
%\end{slide}

%\begin{slide}
%\begin{center}
%{\bf Asset Price Approach to Incidence \\ Empirical Application: Costs of Crime}
%\end{center}
%Rockoff and Linden (2008) apply asset price incidence approach to estimate costs of crime
%
%Idea: look at how house prices change when a registered sex offender moves into a neighborhood
%
%Data: public records on offender's addresses and property values in North Carolina.
%
%
%
%\end{slide}
%
%\begin{slide}
%\includepdf[pages={20-22}]{taxincidence_ch19_new_attach.pdf}
%\end{slide}
%
%
%\begin{slide}
%\begin{center}
%{\bf Asset Price Approach to Incidence \\ Empirical Application: Costs of Crime}
%\end{center}
%Finding: house prices fall by 4\% (\$5500) when a sex offender is located within 0.1 miles of a house
%
%Implied cost of an offense (given probabilities of repeat offense): \$1.2 mil.
%
%$\Rightarrow$  Suggests that cost of such crimes is far higher than what is used by Department of Justice
%
%Caveats: are you really measuring cost of the crime or a psychological overreaction?
%Why does price fall only within 0.1 mile radius?
%
%\end{slide}



%\begin{slide}
%\begin{center}
%{\bf Mandated Benefits}
%\end{center}
%
%Now consider incidence of a mandated benefit instead of a tax
%
%\small
%Examples: (a) requirement that employers pay for healthcare (employers
%with 50+ employees are required to do that under Obamacare or pay a fine),
%(b) workers compensation benefits [for injuries on the job]
%\normalsize
%
%Affects firms like a tax
%
%But effect of mandated benefits on equilibrium wages and employment differ from a tax (Summers 1989)
%because workers value the mandated benefit
%
%Suppose workers value \$1 of mandated benefit at \$ $\alpha \geq 0$
%
%\small
%Could have $\alpha<1$ if benefit not as valuable as cash
%
%Could have $\alpha>1$ if benefit more valuable than cash (e.g., can't buy
%health insurance on individual market)
%\normalsize
%
%If $\alpha=1$ then no change in employment
%
%\end{slide}
%
%\begin{slide}
%\includepdf[pages={23-25}]{taxincidence_ch19_new_attach.pdf}
%\end{slide}


\begin{slide}
\begin{center}
{\bf Tax Salience: A  New Theory}
\end{center}

Traditional model assumes that all individuals are fully aware of taxes that they pay

Is this true in practice? May be not be because many taxes are not fully salient.

\small
Do you know your exact marginal income tax rate?
Do you think about it when choosing a job?

Do you know the sales tax you have to pay in addition to posted prices at cash register?

\normalsize

Chetty, Looney, Kroft AER '09: test this assumption in the context of commodity taxes
and develop a theory of taxation with inattentive consumers

\end{slide}

\begin{slide}
\begin{center}
{\bf Tax Salience: A  New Theory}
\end{center}
Chetty, Looney, Kroft AER'09 develop two empirical strategies to test whether
salience matters for sales tax incidence

Sales tax is paid at the cash register and not displayed on price tags in stores

\textbf{1) Randomized field experiment} with  supermarket stores

\small
In one treatment store: they display new price tags
showing the level of sales tax and total price 
on a \textbf{subset} of products

Compare shopping behavior for treated products vs. control products in treated store, before and
after new tags are implemented (this is called difference-in-difference [DD] strategy)

Repeat the analysis in control stores as a placebo DD strategy
\normalsize


\textbf{2) Policy experiment} using changes in beer excise and sales taxes across states

\small
Excise tax is salient because built into posted price while sales tax is not salient because
it is not included in posted price

\end{slide}


\begin{slide}
\includepdf[pages={26, 27,28,29,30}]{taxincidence_ch19_new_attach.pdf}
\end{slide}


\begin{slide}
\begin{center}
{\bf Tax Salience: A  New Theory}
\end{center}
Key Empirical Result: \textbf{Salience matters}

1) Posting sales taxes reduces demand for those goods

2) Beer consumption is elastic to excise tax rate (built in posted price)
but not to the sales tax rate (not built in the posted price)

$\Rightarrow$ If tax is not salient to consumers, they are less elastic,
and hence more likely to bear the tax burden 

A number of recent empirical studies show that individuals are not fully
informed and fully rational and this has large consequences for policy

\end{slide}


\begin{slide}
\begin{center}
{\bf TAX PROGRESSIVITY IN THE US}
\end{center}
Saez-Zucman (2019) distribute taxes by factor. At \href{https://taxjusticenow.org/}{Taxjusticenow.org}, you can explore
changing the current tax system.

1) Labor taxes (payroll taxes and individual income taxes) assigned to corresponding workers (whether tax remitted by the workers or employers)

2) Consumption taxes (excise and sales) assigned to corresponding consumers 

3) Capital taxes (corporate tax, property tax, taxes on capital income) assigned to corresponding owners of the capital assets

This distribution by factor does not capture ultimate incidence nor DWB if taxes are shifted through incidence

\end{slide}


\begin{slide}
\includepdf[pages={48, 46, 47}]{taxincidence_ch19_new_attach.pdf}
\end{slide}


\begin{slide}
\begin{center}
{\bf US TAX PROGRESSIVITY BY TYPE OF TAX}
\end{center}

1. \textbf{Individual Income tax} is progressive (exempts the bottom 50\% and increasing rates by brackets)

2. \textbf{Payroll taxes} on earnings are a constant tax rate of 15\% but only up to \$137K of earnings $\Rightarrow$ Regressive at the
top

3. \textbf{Excise and sales taxes} are regressive because share of income devoted to consumption of goods falls with income

4. \textbf{Corporate tax} is progressive because corporate owners tend to be at the top (\textbf{property tax} somewhat progressive)

5. \textbf{Estate tax} on large fortunes at death progressive but small

Federal taxes are more progressive than state+local taxes. Official stats from CBO focus on federal taxes only

\end{slide}




\begin{slide}
\begin{center}
{\bf Is Distribution by Factor Close to True Incidence? }
\end{center}
1) Labor taxes borne by workers if wages set as in  competitive model and
labor supply less elastic than labor demand

\small
In practice, wages are rigid in short-run so employer vs. employee payroll tax don't
have the same effect (evidence from France and Greece). In long-run incidence likely
on wages (as employer payroll taxes haven't reduced macro capital share)
\normalsize

2) Consumption taxes borne by consumers if prices set competitively
and demand for goods less elastic than supply 

\small
VAT evidence and salience evidence show non-standard incidence in short and medium-run
but long-run incidence likely on consumers
\normalsize

3) Capital taxes borne by owners of capital if supply of capital (savings) less elastic
than demand for capital (investment) 

\small
Evidence here is most disputed. Official CBO statistics shift 1/4 of corporate tax on workers without
much evidence (see corp tax lecture)

\normalsize

%Bottom line: standard economic model does not do a great job at predicting the incidence
%of tax changes at least in short and medium-run



\end{slide}

%\begin{slide}
%\begin{center}
%{\bf CBO TAX INCIDENCE ASSUMPTIONS}
%\end{center}
%
%The Congressional Budget Office (CBO) analysis considers the incidence of the full set of taxes levied by the \emph{federal government}. Their key assumptions follow:
%
%1. \textbf{Individual Income taxes} are borne fully by the households that pay them.
%
%2. \textbf{Payroll taxes} are borne fully by workers, regardless of whether these taxes are paid by the workers or by the firm
%
%3. \textbf{Excise taxes} are fully shifted to prices and so are borne by individuals in proportion to their consumption of the taxed item.
%
%4. \textbf{Corporate taxes} are allocated 75\% to owners of capital (not only shareholders but owners of capital in general) in proportion to capital income
%and 25\% to labor in proportion to labor income [controversial and to be discussed later]
%
%\end{slide}


%\begin{slide}
%\includepdf[pages={31-32}]{taxincidence_ch19_new_attach.pdf}
%\end{slide}


%\begin{slide}
%\includepdf[pages={35, 33, 34}, scale=.9]{taxincidence_ch19_new_attach.pdf}
%\end{slide}





\begin{slide}
\begin{center}
{\bf REFERENCES}
\end{center}
{\small

Jonathan Gruber, Public Finance and Public Policy, Fifth Edition, 2019 Worth Publishers, Chapter 19

Allcott, Hunt, Benjamin B. Lockwood, and Dmitry Taubinsky. ``Should We Tax Soda?
An Overview of Theory and Evidence'', forthcoming Journal of Economic Perspectives, 2019.
\href{http://elsa.berkeley.edu/~saez/course131/AllcottetalJEPSodaTax.pdf}{(web)}

Benzarti, Youssef, Dorian Carloni, Jarkko Harju, and Tuomas Kosonen. ``What Goes Up May Not Come Down: Asymmetric Incidence of Value-Added Taxes.'' Journal of Political Economy 128(12), 2020.\href{http://elsa.berkeley.edu/~saez/course131/benzartietalJPE2020VAT.pdf}{(web)}

Bozio, Antoine, Thomas Breda, Julien Grenet. 2019 ``Does Tax-Benefit Linkage Matter for the Incidence of
Social Security Contributions? Evidence from France''. \href{http://elsa.berkeley.edu/~saez/course131/bozioetal19payrolltax.pdf}{(web)} 

Chetty, Raj, Adam Looney, and Kory Kroft. 2009. ``Salience and Taxation: Theory and Evidence.'' American Economic Review 99(4): 1145-1177.\href{http://www.jstor.org/stable/pdfplus/25592504.pdf}{(web)}

Doyle Jr, Joseph J., and Krislert Samphantharak ``\$2.00 Gas! Studying the effects of a gas tax mobratorium.'' Journal of Public Economics 92.3 (2008): 869-884.\href{http://www.nber.org/papers/w12266.pdf}{(web)}

ITEP (Institute on Taxation and Economic Policy). 2018. ``Who Pays: A Distributional Analysis of the Tax Systems in All 50 States'', 6th edition. 
\href{https://itep.org/whopays/}{(web)}

%Linden, Leigh, and Jonah E. Rockoff. ``Estimates of the impact of crime risk on property values from Megan's laws.'' The American Economic Review 98.3 (2008): 1103-1127.\href{http://www.jstor.org/stable/pdfplus/29730108.pdf}{(web)}

Piketty, Thomas, Emmanuel Saez, and Gabriel Zucman,  ``Distributional National Accounts:
Methods and Estimates for the United States'', Quarterly Journal of Economics, 133(2), 553-609, 2018
\href{https://eml.berkeley.edu/~saez/PSZ2018QJE.pdf} {(web)}

Ramsey, Frank P. ``A Contribution to the Theory of Taxation.'' The Economic Journal 37.145 (1927): 47-61.\href{http://elsa.berkeley.edu/~saez/course131/Ramsey27.pdf}{(web)}

Saez, Emmanuel, Manos Matsaganis, and Panos Tsakloglou. ``Earnings determination and taxes: Evidence from a cohort-based payroll tax reform in Greece.'' The Quarterly Journal of Economics 127, no. 1 (2012): 493-533.\href{https://eml.berkeley.edu/~saez/saez-matsaganis-tsakloglouQJE11greecetax.pdf}{(web)}
 
Saez, Emmanuel and Gabriel Zucman. The Triumph of Injustice: How the Rich Dodge Taxes and How to Make them Pay, New York: W.W. Norton, 2019. 
\href{http://www.taxjusticenow.org} {(web)}

%Summers, Lawrence H. ``Some simple economics of mandated benefits.'' The American Economic Review 79.2 (1989): 177-183.\href{http://elsa.berkeley.edu/~saez/course131/Summers89.pdf}{(web)}

US Congressional Budget Office, 2016. ``The Distribution of Household Income and Federal Taxes, 2013''
\href{http://elsa.berkeley.edu/~saez/course131/CBOtaxincidence.pdf}{(web)}

}

\end{slide}
\end{document}
