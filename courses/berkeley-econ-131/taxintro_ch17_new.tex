\documentclass[landscape]{slides}

\usepackage[landscape]{geometry}

\usepackage{pdfpages}

\usepackage{hyperref}
\usepackage{amsmath}

\def\mathbi#1{\textbf{\em #1}}

\topmargin=-1.8cm \textheight=17cm \oddsidemargin=0cm
\evensidemargin=0cm \textwidth=22cm


\author{131 Undergraduate Public Economics \\ Emmanuel Saez \\ UC Berkeley}
\date{}

\title{Introduction to \\
Inequality, Poverty, Taxes and Transfers \\ (loosely follows Gruber Chapters 17-18)} \onlyslides{1-300}

\newenvironment{outline}{\renewcommand{\itemsep}{}}

\begin{document}

\begin{slide}
\maketitle
\end{slide}


\begin{slide}
\begin{center}
{\bf Recall: Two General Rules for Government Intervention}
\end{center}

\textbf{1) Market Failures:} Government intervention can help
if there are market failures

\textbf{2) Redistribution:} Free market generates inequality. Public cares about
economic disparity. Govt taxes
and spending can reduce inequality 

\end{slide}

\begin{slide}
\begin{center}
{\bf Role 2: Redistribution}
\end{center}

Even with no market failures, free market outcome might generate
substantial inequality

Inequality matters because humans are social beings: people evaluate their economic well-being
relative to others, not in absolute terms  $\Rightarrow$ Public cares about inequality

In advanced economies, people pool 30-50\% of their income through their government
to fund many transfer programs

%Government uses taxes and transfers to redistribute
Do taxes and transfers affect economic behavior?

$\Rightarrow$ Generates an
efficiency and equity trade-off (size of economic pie vs. distribution of the economic pie)



\end{slide}


\begin{slide}
\begin{center}
{\bf Income Inequality: Labor vs. Capital Income}
\end{center}
Economic production happens with labor and capital

Individuals derive market income (before tax) from {\bf labor} (work) and
{\bf capital} (ownership): $z=wl+rk$ where $w$ is wage, $l$ is labor supply,
$k$ is capital, $r$ is rate of return on capital

1) {\bf Labor income inequality} is due to differences in working abilities
(education, talent, physical ability, etc.), work effort (hours of
work, effort on the job, etc.), and luck (labor effort might
succeed or not)

2) {\bf Capital income inequality} is due to differences in
wealth $k$ (due to past saving behavior and inheritances
received), and in rates of return $r$ 

Capital Income (or wealth) is much more concentrated than Labor
Income

%Top 1\% wealth holders have 1/3 of total wealth. Bottom 50\% wealth holders own zero wealth.
%Top 1\% labor income earners have about 15\% of total labor income. [Top
%1\% incomes have 20\% of total income]

\end{slide}



\begin{slide}
\begin{center}
{\bf Macro-aggregates: Labor vs. Capital Income}
\end{center}

National Income = GDP - depreciation of capital + net foreign income

Labor income $wl \simeq$ 75\% of market income $z$

Capital income $ rk \simeq$ 25\% of market income $z$

Private wealth  $k \simeq 400-500\%$ of market income $z$
(increased a lot in the US in last decade to 550\% in 2020)

Rate of return on wealth $ r \simeq 5-6\%$

In GDP, gross capital share is higher (35\%) because it includes
depreciation of capital


\end{slide}

\begin{slide}
\includepdf[pages={22}]{taxintro_ch17_new_attach.pdf}
\end{slide}

\begin{slide}
\includepdf[pages={23}]{taxintro_ch17_new_attach.pdf}
\end{slide}



\begin{slide}
\begin{center}
{\bf Income Inequality Measurement}
\end{center}
Inequality can be measured by indexes such as Gini coefficient, 
quantile income shares which are functions of the income
distribution $F(z)$

Most famous inequality index: \textbf{Gini coefficient}

Gini = 2 * area between 45 degree line and Lorenz curve

Lorenz curve $L(p)$ at percentile $p$ is fraction of total income
earned by individuals below percentile $p$

$0 \leq L(p) \leq p$

Gini=0 means perfect equality

Gini=1 means complete inequality (top person has all the income)


\end{slide}

\begin{slide}
\includepdf[pages={1}]{taxintro_ch17_new_attach.pdf}
\end{slide}


\begin{slide}
\begin{center}
{\bf Key Empirical Facts on Income Inequality}
\end{center}
1) In the US, labor income inequality has increased substantially
since 1970: debate between skilled biased technological progress
view vs. institution view (min wage and Unions) [Autor-Katz'99]

2) Gender gap has decreased but remains substantial especially at the very top

3) In the US, top income shares dropped dramatically from 1929 to
1950 and increased dramatically since 1980 

4) Bottom 50\% pre-tax income per adult have stagnated since 1980
in spite of a 60\% increase in average national income 

%3) Top incomes used to be primarily capital income. Now, top
%incomes are divided 50/50 between labor and capital income (due to
%explosion of top labor incomes with stock-options, bonuses, etc.)

4) Fall in top income shares from 1900-1950 happened in most
OECD countries. Surge in top income shares has happened primarily
in English speaking countries, not as much in Continental Europe
and Japan [Atkinson, Piketty, Saez JEL'11]

\end{slide}

\begin{slide}
\includepdf[pages={2}]{taxintro_ch17_new_attach.pdf}
\end{slide}

\begin{slide}
\includepdf[pages={41}]{taxintro_ch17_new_attach.pdf}
\end{slide}


\begin{slide}
\includepdf[pages={33}, scale = .9]{taxintro_ch17_new_attach.pdf}
\end{slide}

\begin{slide}
\includepdf[pages={35}, scale = 1]{taxintro_ch17_new_attach.pdf}
\end{slide}

\begin{slide}
\includepdf[pages={48, 49}, scale = 1]{taxintro_ch17_new_attach.pdf}
\end{slide}


\begin{slide}
\includepdf[pages={57}]{taxintro_ch17_new_attach.pdf}
\end{slide}

\begin{slide}
\includepdf[pages={47, 46, 45, 44, 43}]{taxintro_ch17_new_attach.pdf}
\end{slide}



%\begin{slide}
%\includepdf[pages={8}, scale=1.1]{taxintro_ch17_new_attach.pdf}
%\end{slide}

%\begin{slide}
%\includepdf[pages={9}]{taxintro_ch17_new_attach.pdf}
%\end{slide}

%\begin{slide}
%\begin{center}
%{\bf POVERTY RATE}
%\end{center}
%
%A popular measure of disparity used is the poverty rate defined as the fraction of households below
%the poverty line
%
%Poverty line can be defined in absolute terms as in the US, or in relative terms (such as 50\% of median
%household income) as in European countries
%
%US poverty line developed in 1964 by Molly Orshansky as 3 times the cost of an adequate diet bundle. Poverty line depends
%on size of household.
%
%Since then, numbers have just been updated from inflation
%
%Income definition used is incomplete: fails to include non-cash transfers, and is before income tax and income
%tax credits
%
%\end{slide}


\begin{slide}
\begin{center}
{\bf POVERTY RATE DEFINITIONS}
\end{center}
1) {\bf Absolute:} Fraction of population with disposable income
(normalized by family size) below {\bf poverty threshold} $z^*$
fixed in real terms (e.g., World Bank uses \$1.90/day in 2011 dollars)

2) {\bf Relative:} Fraction of population with disposable income
(normalized by family size) below {\bf poverty threshold} $z^*$
fixed relative to median (European Union uses 60\% of
median)

\small
Absolute poverty falls in the long run with economic growth
[nobody in the US is World Bank poor] but relative poverty does
not 

Absolute poverty captures both growth and inequality effects while
relative poverty captures only inequality effects

The fact that inequality stays in the debate in spite of huge growth since 1800
shows that relative income is the relevant concept (see e.g. Luttmer 2005 for an empirical study)

$\Rightarrow$ Health measures (mortality, stunting) are the only relevant
absolute measures of deprivation in the long-run

%A recent study by Luttmer (2005) finds that individuals' self-reported well-being rises as their own income rises, but falls as their neighbors' incomes rise, suggesting that it is relative income, and not absolute income, that determines well-being.
\end{slide}


\begin{slide}
\includepdf[pages={42}]{taxintro_ch17_new_attach.pdf}
\end{slide}

\begin{slide}
\begin{center}
{\bf Poverty Rate Disposable Income Definition}
\end{center}
Most intuitive notion of poverty is based on consumption $c$ [not
pre-tax income $z$] $$c=z-T(z)+B(z)+E-s$$ where $T(z)$ is tax, $B(z)$ govt
transfers, $E$ net private transfers (charity, family, friends),
$s$ is net savings/borrowing

{\bf Consumption} $c$ is difficult to measure

{\bf Disposable Income} $z-T(z)+B(z)$  [post-tax income] measured
in traditional Current Population Survey (CPS) 
%[but does not fully capture in-kind
%elements of $B(z)$ such as Medicaid]

\end{slide}

\begin{slide}
\begin{center}
{\bf FAMILY SCALE}
\end{center}
Ideally, poverty should be defined at the individual level based
on individual consumption [e.g., kids better off when mother or
grandmother controls income instead of father, Duflo '03]

However, many consumption goods are shared within the family
[e.g., housing, joint meals, etc.] and it is difficult to measure consumption
at individual level

Measured poverty is therefore based on consumption or disposable
income at the family level [or unit sharing resources] and
everybody within the family has same poverty status

Bigger families need more resources but economies of scale in
consumption: scale disposable income by family size

\end{slide}

%\begin{slide}
%\begin{center}
%{\bf US POVERTY RATE DEFINITION}
%\end{center}
%http://www.census.gov/hhes/www/poverty/poverty.html
%
%Definition developed in 1963 by Molly Orshansky (at Social
%Security Administration) as 3 times amount required to buy a
%``thrifty food plan''
%
%[3 times because 1955 survey showed that low income families spent
%1/3 of income on food]
%
%Poverty threshold afterward indexed on official Consumer Price
%Index and adopted by Administration as Official Poverty Index
%
%Family scaling was based on 1955 survey
%
%\end{slide}



\begin{slide}
\begin{center}
{\bf US POVERTY RATE DEFINITION}
\end{center}
Based on {\bf money income} = market income before taxes + cash
govt transfers + cash private transfers

In-kind market income and transfers (employer health insurance,
Medicaid, nutrition, public housing) do NOT count

Income and employee payroll taxes are NOT deducted, Income tax
credits (EITC, Child Tax Credit) are NOT added

Threshold depends on household size/structure: e.g., \$20K/year
for single parent with 2 kids

Thresholds adjusted annually using the official CPI

In 2020: \$13K for single adult, \$17K family of 2, \$22K for family of 3, \$26K for 4

\end{slide}


%\begin{slide}
%\includepdf[pages={10}]{taxintro_ch17_new_attach.pdf}
%\end{slide}



\begin{slide}
\includepdf[pages={27, 11}]{taxintro_ch17_new_attach.pdf}
\end{slide}


\begin{slide}
\begin{center}
{\bf Factors Explaining Evolution of Poverty}
\end{center}
Based on Hoynes-Page-Stevens JEP'06

1) Increasing pre-tax inequality: stagnant bottom wages in spite
of economic growth per capita [large effect]

2) Changes in family structure: single parent families $\uparrow$
from 7\% in 1967 to 14.4\% in 2003 $\Rightarrow$ Increases poverty
rate by 4 pts [large effect]

3) Increase in female labor force participation $\Rightarrow$
Reduces poverty rate [significant effect only since 1980]

4) Immigration: accounts for about 0.7 points in the poverty rate
increase from 1969 to 1999 [small effect]

5) Means-tested transfers [medium effect because they are
concentrated below poverty line]
\end{slide}

%\begin{slide}
%\includepdf[pages={12-16}]{taxintro_ch17_new_attach.pdf}
%\end{slide}

\begin{slide}
\begin{center}
{\bf ISSUES WITH US POVERTY RATE DEFINITION}
\end{center}
Definition was close to disposable income when measuring poverty started but no longer:

1) In-kind transfers have grown substantially [Medicaid]

2) Payroll tax and Income tax credits (EITC, Child Tax Credit)
have grown substantially for low income families

3) Official CPI overstates inflation [and understates real economic growth] because
it is not chained [i.e., does not take into account that relative price changes lead to changes in
consumption]

% for the average consumer, low income consumer price
%might evolve differently (Walmart) [also geographical variation]

Politically difficult to change definition
\end{slide}


%\begin{slide}
%\begin{center}
%{\bf Recomputing Poverty Rate: Meyer-Sullivan NBER'09}
%\end{center}
%%\small
%
%1) Change the scaling for family size (no strong effect)
%
%2) Change the price index: shift to CPI-U-RS instead of official
%CPI-U (large legitimate effect, CPI-U-RS better index)
%
%3) Shift to households [people living in same unit] instead of
%family [people in same unit related by blood/adoption]: not clear
%which is best, depends on sharing [some effect]
%
%4) Shift to after-tax income [deduct income/payroll taxes, add tax
%credits]: large legitimate effect
%
%5) Add non-cash benefits [nutrition, housing, health insurance]:
%tiny net effect [medicaid $\uparrow$, other programs $\downarrow$]
%
%6) Shift to consumption [modest effect on poverty rate, huge
%effect on deep poverty]
%\end{slide}
%
%
%
%\begin{slide}
%\includepdf[pages={17}]{taxintro_ch17_new_attach.pdf}
%\end{slide}
%





\begin{slide}
\begin{center}
{\bf Measuring Intergenerational Income Mobility}
\end{center}
Strong consensus that children's success should not depend too much on parental income

Studies linking adult children to their parents can measure link between children and parents income

Simple measure: average income rank of children by income rank of parents (Chetty et al. '14)

\small

1) US has less mobility than European countries (especially Scandinavian countries such as Denmark)

2) Substantial heterogeneity in mobility across cities in the US

3) Places with low segregation, low income inequality, good K-12 schools, high social capital, high family stability 
tend to have high mobility [this is a correlation not necessarily causal]

4) Substantial racial disparity in mobility (Chetty et al. 2020)

\end{slide}


\begin{slide}
\includepdf[pages={25,26,29-32}]{taxintro_ch17_new_attach.pdf}
\end{slide}

\begin{slide}
\includepdf[pages={50, 51}]{taxintro_ch17_new_attach.pdf}
\end{slide}

\begin{slide}
\begin{center}
{\bf Govt Redistribution with Taxes and Transfers}
\end{center}

Govt taxes individuals based on income and consumption and
provides transfers: $z$ is pre-tax income, $y=z-T(z)+B(z)$ is
post-tax income

1) If inequality in $y$ is less than inequality in $z$
$\Leftrightarrow$ tax and transfer system is redistributive (or
progressive)

2) If inequality in $y$ is more than inequality in $z$
$\Leftrightarrow$ tax and transfer system is regressive

a) If $y=z \cdot (1-t)$ with constant $t$, tax/transfer system is
neutral

b) If $y=z \cdot (1-t)+G$ where $G$ is a universal 
transfer, then tax/transfer system is progressive 

Actual tax/transfer systems in rich countries roughly like b) with $G$ welfare
state transfers [education, health, retirement]
\end{slide}

\begin{slide}
\begin{center}
{\bf US Distributional National Accounts}
\end{center}
Piketty-Saez-Zucman (2018) distribute both pre-tax and post-tax US \textbf{national income} across adult individuals 

National income = GDP - depreciation of capital + net foreign income = broadest measure of income

Pre-tax income is income before taxes and transfers: $z$

Post-tax income is income net of all taxes and adding all transfers and public good spending: $y=z-T(z)+G$

Both concepts add up to national income and provide a comprehensive view of the mechanical impact
of government redistribution

\end{slide}

%\begin{slide}
%\includepdf[pages={38}, scale=.95]{taxintro_ch17_new_attach.pdf}
%\end{slide}



\begin{slide}
\includepdf[pages={36}, scale=.9]{taxintro_ch17_new_attach.pdf}
\end{slide}

\begin{slide}
\includepdf[pages={58}, scale=.9]{taxintro_ch17_new_attach.pdf}
\end{slide}



\begin{slide}
\begin{center}
{\bf Federal US Tax System (2/3 of total taxes)}
\end{center}
1) Individual income tax (on both labor+capital income)
[progressive](40\% of fed tax revenue)

2) Payroll taxes (on labor income) financing social security programs [regressive] (40\% of revenue)

3) Corporate income tax (on capital income) [progressive] (15\% of revenue)

4) Estate taxes (on capital income) [very progressive] (1\% of
revenue)

5) Minor excise taxes (on consumption) [very regressive] (3\%
of revenue)

Fed agencies (CBO, Treasury, Joint Committee on Taxation) and think-tanks (Tax Policy Center)
provide distributional Fed tax tables

\end{slide}


\begin{slide}
\begin{center}
{\bf State+Local Tax System (1/3 of total taxes)}
\end{center}
Decentralized governments can experiment, be tailored to local views, create tax competition and make redistribution harder (famous Tiebout 1956 model)
hence favored by conservatives

1) Individual + Corporate income taxes
[progressive] (1/3 of state+local tax revenue)

2) Sales taxes + Excise taxes (tax on consumption) [very regressive] (1/3 of
revenue)

3) Real estate property taxes (on capital income)
[slightly progressive] (1/3 of revenue)

See ITEP (2018) ``Who Pays'' for systematic state level distributional tax tables

US Census provides Census of Government data

%http://www.census.gov/govs/www/qtax.html

\end{slide}


%\begin{slide}
%\includepdf[pages={19,20}]{taxintro_ch17_new_attach.pdf}
%\end{slide}


\begin{slide}
\begin{center}
{\bf US tax/transfer System: Progressivity and Evolution}
\end{center}
{\bf 0) US Tax/Transfer system is progressive overall:} pre-tax national income is less
equally distributed than post-tax/post-transfer national income 

{\bf 1) Medium Term Changes:} US Tax Progressivity has declined
since 1950 (Saez and Zucman 2019) but govt redistribution through transfers has increased (Medicaid, Social Security retirement, DI, UI
various income support programs)

{\bf 2) Long Term Changes:} Before 1913, US taxes were primarily
tariffs, excises, and real estate property taxes [slightly
regressive], minimal welfare state (and hence small govt)

%http://www.treasury.gov/education/fact-sheets/taxes/ustax.shtml
\end{slide}

%\begin{slide}
%\includepdf[pages={21}]{taxintro_ch17_new_attach.pdf}
%\end{slide}


\begin{slide}
\includepdf[pages={53, 54, 52}, scale=1]{taxintro_ch17_new_attach.pdf}
\end{slide}



\begin{slide}
\begin{center}
{\bf Plan for Lectures on Taxation/Redistribution}
\end{center}

1) Tax incidence, efficiency costs of taxation, optimal commodity taxation

2) Taxation of labor income:

Optimal design of labor income taxation and means-tested transfers

Empirical analysis of tax and transfer programs on labor supply and earnings

3) Taxation of capital income (savings, wealth, and corporate profits)



\end{slide}


\begin{slide}
\begin{center}
{\bf REFERENCES}
\end{center}
{\small

Jonathan Gruber, Public Finance and Public Policy, Fifth Edition, 2016 Worth Publishers, Chapter 17 and Chapter 18

Alvaredo, F., Atkinson, A., T. Piketty, E. Saez, and G. Zucman \emph{World Inequality Database},
\href{http://www.wid.world/} {(web)}

Alvaredo, F., Atkinson, A., T. Piketty, E. Saez, and G. Zucman, 2018. \emph{World Inequality Report},
\href{https://wir2018.wid.world/} {(web)}

Atkinson, Anthony B., Thomas Piketty, and Emmanuel Saez. ``Top Incomes in the Long Run of History.'' Journal of Economic Literature 49.1 (2011): 3-71.\href{http://elsa.berkeley.edu/~saez/atkinson-piketty-saezJEL10.pdf}{(web)}

Blanden, J and Machin, S (2008) ``Up and down the generational income ladder in Britain: Past changes and future prospects'' \emph{National Institute Economic Review} 205 (1). 101--116. 

Boserup,  Simon, Wojciech Kopczuk, and Claus Kreiner "Stability and Persistence of Intergenerational Wealth Formation: Evidence from Danish Wealth Records of Three Generations", October 2014
\href{http://elsa.berkeley.edu/~saez/course131/WealthAcrossGen.pdf}{(web)}

Chetty, Raj, Nathan Hendren, Patrick Kline, and Emmanuel Saez, ``Where is the Land of Opportunity? The Geography of Intergenerational Mobility in the United States,'' \emph{Quarterly Journal of Economics}, 129(4), 2014, 1553-1623.
\href{http://eml.berkeley.edu/~saez/chetty-friedman-kline-saezQJE14mobility.pdf}{(web)}

Chetty, Raj, Nathan Hendren, Patrick Kline, Emmanuel Saez,  and Nicholas Turner ``Is the United States Still a Land of Opportunity? Trends in Intergenerational Mobility Over 25 Years,'' American Economic Review, Papers and Proceedings, 104(5), 2014, 141-147 \href{https://eml.berkeley.edu/~saez/chettyetalAERPP2014.pdf}{(web)}

Chetty, Raj, Nathaniel Hendren, Maggie R. Jones, and Sonya R. Porter. ``Race and economic opportunity in the United States: An intergenerational perspective.'' Quarterly Journal of Economics, 2020  \href{https://opportunityinsights.org/wp-content/uploads/2018/04/race_paper.pdf}{(web)}

Corak, Miles, and Andrew Heisz, ``The Intergenerational Earnings and Income
Mobility of Canadian Men: Evidence from Longitudinal Income Tax Data,''
Journal of Human Resources, 34, no. 3 (1999), 504--533.
\href{http://www.jstor.org/stable/pdfplus/146378.pdf} {(web)}

Duflo, Esther. ``Grandmothers and Granddaughters: Old-Age Pensions and Intrahousehold Allocation in South Africa'', The World Bank Economic Review
Vol. 17, 2003, 1-25 \href{http://www.jstor.org/stable/pdfplus/3990043.pdf} {(web)}

Hoynes, Hilary W., Marianne E. Page, and Ann Huff Stevens. ``Poverty in America: Trends and explanations.'' The Journal of Economic Perspectives 20.1 (2006): 47-68.\href{http://www.nber.org/papers/w11681.pdf}{(web)}

Katz, Lawrence F., and David H. Autor. ``Changes in the wage structure and earnings inequality.'' Handbook of Labor Economics 3 (1999): 1463-1555. \href{http://elsa.berkeley.edu/~saez/course131/Katz-Autor99.pdf}{(web)}

Kopczuk, Wojciech, Emmanuel Saez, and Jae Song. ``Earnings inequality and mobility in the United States: evidence from social security data since 1937.'' The Quarterly Journal of Economics 125.1 (2010): 91-128.\href{http://elsa.berkeley.edu/~saez/kopczuk-saez-songQJE10mobility.pdf}{(web)}

Luttmer, Erzo FP. ``Neighbors as negatives: Relative earnings and well-being.'' Quarterly Journal of Economics 120.3 (2005): 963-1002.\href{http://elsa.berkeley.edu/~saez/course131/Luttmer05}{(web)}

Meyer, Bruce D., and James X. Sullivan. ``Consumption and Income Inequality in the US: 1960-2008.'' (2010).\href{http://elsa.berkeley.edu/~saez/course131/Meyer-Sullivan09}{(web)}

Piketty, Thomas, and Emmanuel Saez. ``Income inequality in the United States, 1913-1998.'' The Quarterly Journal of Economics 118.1 (2003): 1-39.\href{http://elsa.berkeley.edu/~saez/pikettyqje.pdf}{(web)}

%Piketty, Thomas, and Emmanuel Saez. ``How Progressive is the US Federal Tax System? A Historical and International Perspective.'' The Journal of Economic Perspectives 21.1 (2007): 3-24.\href{http://elsa.berkeley.edu/~saez/piketty-saezJEP07taxprog.pdf}{(web)}

Piketty, Thomas, Emmanuel Saez, and Gabriel Zucman,  ``Distributional National Accounts:
Methods and Estimates for the United States'', Quarterly Journal of Economics, 133(2), 553-609, 2018
\href{https://eml.berkeley.edu/~saez/PSZ2018QJE.pdf} {(web)}

Saez, Emmanuel and Gabriel Zucman. The Triumph of Injustice: How the Rich Dodge Taxes and How to Make them Pay, New York: W.W. Norton, 2019. 
\href{http://www.taxjusticenow.org} {(web)}

Saez, Emmanuel and Gabriel Zucman. ``The Rise of Income and Wealth Inequality in America: Evidence from Distributional Macroeconomic Accounts,'' Journal of Economic Perspectives 34(4), Fall 2020, 3-26.
\href{https://eml.berkeley.edu/~saez/SaezZucman2020JEP.pdf}{(web)} 


US Census Bureau, 2020. ``Income and Poverty in the United States: 2019'', report P60-263.
\href{http://elsa.berkeley.edu/~saez/course131/CPSpoverty.pdf}{(web)} 

}


\end{slide}


\end{document}






\end{document}
