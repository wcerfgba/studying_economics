\documentclass[landscape]{slides}

\usepackage[landscape]{geometry}

\usepackage{pdfpages}

\usepackage{hyperref}
\usepackage{amsmath}

\def\mathbi#1{\textbf{\em #1}}

\topmargin=-1.8cm \textheight=17cm \oddsidemargin=0cm
\evensidemargin=0cm \textwidth=22cm


\author{131 Undergraduate Public Economics \\ Emmanuel Saez \\ UC Berkeley}
\date{}

\title{Cost-Benefit Analysis} \onlyslides{1-300}

\newenvironment{outline}{\renewcommand{\itemsep}{}}

\begin{document}

\begin{slide}
\maketitle
\end{slide}

\begin{slide}
\begin{center}
{\bf OUTLINE}
\end{center}
Chapter 8

8.1 Measuring the Costs of  Public Projects

8.2 Measuring the Benefits of  Public Projects

8.3 Putting It All Together

8.4 Conclusion
\end{slide}

\begin{slide}
\begin{center}
{\bf DEFINITION}
\end{center}

{\bf Cost-benefit analysis}
The comparison of costs and benefits of public goods projects to decide if they should be undertaken.
\end{slide}

%8.1 Measuring the Costs of  Public Projects

\begin{slide}
\includepdf[pages={1}]{costbenefit_ch08_new_attach.pdf}
\end{slide}

\begin{slide}
\begin{center}
{\bf MEASURING CURRENT COSTS}
\end{center}

{\bf Cash-flow accounting}:
Accounting method that calculates costs solely by adding up what the government pays for inputs to a project, and calculates benefits solely by adding
up income or government revenues
generated by the project.

{\bf Opportunity cost}:
The social marginal cost of any resource is the value of that resource in its next best use.
\end{slide}


\begin{slide}
\begin{center}
{\bf MEASURING CURRENT COSTS}
\end{center}

General rule: Economic costs are only those costs associated with diverting the resource from its next best use

\small
{\bf Perfectly Competitive Markets}\\
Social Cost = Price (true for labor and material)

{\bf Imperfectly Competitive Markets}\\
{\bf A. Monopoly}: (suppose asphalt is produced by monopoly) \\
Price = Marginal cost + Monopoly Marginal Profit $>$ Marginal cost\\
On efficiency grounds, Social cost = Marginal cost \\
Profit is a transfer from government (taxpayers) to monopoly (this
matters for redistribution but not efficiency)

{\bf B. Labor market with unemployment}:
Suppose a minimum wage set at %typo: 'a' changed to 'at'
\$10 creates involuntary unemployment\\
The unemployed would be willing to work for \$6 on average but cannot find jobs\\
Govt provides jobs paying \$10/hour. Hired unemployed get a surplus of \$4/hour.\\
Social Cost = \$6 = \$10 (wage) - \$4 (surplus value of jobs for workers)
\end{slide}

\begin{slide}
\begin{center}
{\bf MEASURING FUTURE COSTS}
\end{center}

{\bf Present discounted value (PDV)}:
A dollar next year is worth $1+r$ times less than a dollar now because the dollar could earn $r$
in interest if invested.

Maintenance cost of $F$ in perpetuity as PDD of
\[ \frac{F}{1+r} +  \frac{F}{(1+r)^2} + \frac{F}{(1+r)^3} + ... = \frac{F}{r} \]
Government uses public debt with interest $r$ to borrow
(example: $r=6\%$ nominal, $r$-inflation=3\%)

Using debt $r$ makes sense for short-run projects but not necessarily for long-run

Problematic predictions for the long-run: r=3\% $\Rightarrow$ \$1 in 100 years = $1/(1+r)^{100}=$ \$.052 today $\Rightarrow$ Long-run costs (such as global warming) are heavily discounted

\end{slide}


\begin{slide}
\begin{center}
{\bf LONG-RUN SOCIAL DISCOUNTING}
\end{center}
{\bf Social discount rate}:
The appropriate value of $r$ to use in computing PDV for social investments.

2 reasons for discounting \$1 in future relative to \$1 today

\textbf{1) Absolute discounting:} people prefer \$1 now than \$1 in one year.
But on ethical grounds, not clear why we should do absolute discounting of future generation
relative to current generation (except
for meteorite end of world risk)

\textbf{2) Economic growth} makes future generations richer so \$1 extra means
less for them than for us $\Rightarrow$ Even with zero absolute discounting,
we want to discount future.

In ideal world those two effects are embodied in interest rate $r$ so we
just need to take current $r$ to discount

\end{slide}


\begin{slide}
\begin{center}
{\bf LONG-RUN SOCIAL DISCOUNTING}
\end{center}

Problem is that we don't know how growth (and hence $r$) are going to evolve over next 100 years

If economy collapses due to global warming, future people will be poor
and we
don't want to discount 
\small
Example:

Zero growth: 50\% probability: $r=0\%$: \$1 in 100 years = \$1 now

Normal growth: 50\% probability: $r=3\%$: \$1 in 100 year = \$.052 now

\$1 in 100 years worth on average now: $.5 \cdot \$1 + .5 \$ \cdot .052= \$.552 $

Implied discount rate $\bar{r}$ such $(1+\bar{r})^{-100}=.552$ $\Rightarrow$ $\bar{r}= .6\%$

\normalsize
$\Rightarrow$ We should use low discounting for distant future is there is a chance that
growth will stop (Weitzman 1998)

\end{slide}

\begin{slide}
\includepdf[pages={2}]{costbenefit_ch08_new_attach.pdf}
\end{slide}



\begin{slide}
\begin{center}
{\bf VALUING DRIVING TIME SAVED}
\end{center}

{\bf 1.Using Market-Based Measures to Value Time: Wages}

If individuals optimize their labor supply decision, each individual is indifferent between
working one hour more and getting paid its wage
$\Rightarrow$ hourly wage = value of one extra hour of leisure \\
$\Rightarrow$ The value of saving time can be measured using wages (whether people use the saved time to work more or enjoy more leisure)

This theoretical proposition runs into some problems in practice:
\small

1) Individuals may not be able to freely trade off leisure and hours of work; jobs may come with hours restrictions

2) One hour sitting in traffic is worse than losing one hour of leisure $\Rightarrow$ value of reducing traffic higher than time saved
\end{slide}


\begin{slide}
\begin{center}
{\bf Using Survey-Based Measures to Value Time: Contingent Valuation}
\end{center}

{\bf Contingent valuation}:
Asking individuals to value an option they are not now choosing or do not have the opportunity to choose.

Only feasible method to value situations where there is no market price:
Value of saving endangered species, keeping the Arctic pristine, etc.

Popular among environmentalists to argue that causes were worthy

\end{slide}


\begin{slide}
\begin{center}
{\bf The Problems of Contingent Valuation}
\end{center}

The structure of contingent valuation surveys can lead to widely varying responses (Diamond
and Hausman)

\small
Examples of issues:

0) People don't have to pay so they can easily exaggerate value

1) Isolation of issues matters (asking 1 thing vs many)

2) Order of issues matters

3) The ``embedding effect'' matters (preserving 1 lake, vs 3 lakes)
\normalsize


Can only make rational allocation decision by looking at all the issues at the same time: allocate a budget among all causes.

Government is best placed to make this allocation.

Asking people cause by cause does not make sense for evaluating benefits for public policy decisions
\end{slide}

\begin{slide}
\begin{center}
{\bf Using Revealed Preference to Value Time}
\end{center}

{\bf Revealed preference}:
Letting the actions of individuals reveal their valuation (also called hedonic approach)

\small
Examples:

1) How much people are willing to pay to avoid queues: gas price controls of 1970s
generated queues but small mom and pop stations exempted from price controls (could charge
more and had smaller queues): can compare the difference in prices relative to queue length:

Save \$10 by queuing 1 hour $\Rightarrow$ 1 hour is worth \$10

2) How much people are willing to pay for fast highway lanes (e.g., FasTrak lanes
in Bay Area)

In all cases, it is not just time saved, but avoiding unpleasant queuing or traffic

These studies estimate the value of time for the marginal person (i.e. the person indifferent
between paying vs. spending time) not necessarily the same as the average person

\end{slide}


\begin{slide}
\begin{center}
{\bf VALUING SAVED LIVES}
\end{center}

Valuing human lives is the single most difficult issue in cost-benefit analysis and raises ethical issues

%Many would say that human life is priceless, that we should pay any amount of money to save a life. By this argument, valuing life is a reprehensible activity; there is no way to put a value on such a precious commodity.

However, virtually any government expenditure has some odds of saving a life (e.g.,
making roads safer, health care, etc.)

$\Rightarrow$ we need to be able to place some value on a statistical human life.

Contrast between statistical life (fewer accidents)  and a real life (one specific person at risk):
Possible to set a value on a statistical life but not on a real life
\end{slide}

%\begin{slide}
%\includepdf[pages={3,4}]{costbenefit_ch08_new_attach.pdf}
%\end{slide}

%\begin{slide}
%\begin{center}
%{\bf VALUING SAVED LIVES}
%\end{center}
%
%{\bf Using Wages to Value a Life}
%
%As with valuing time, the market-based approach to valuing lives is to use wages: life's value is the present discounted value of the lifetime stream of earnings.
%
%{\bf Contingent Valuation}
%
%Another approach to valuing a life uses contingent valuation. One way to do this is to ask individuals what their lives are worth.
%\end{slide}

\begin{slide}
\begin{center}
{\bf VALUING SAVED LIVES}
\end{center}

{\bf Revealed Preference:} As with valuing time savings, the method preferred by economists for valuing life is to use revealed preferences. We can value life by estimating how much individuals are willing to pay for something that reduces their odds of dying.

{\bf Compensating differentials:}
Additional (or reduced) wage payments to workers to compensate them for the negative (or positive) amenities of a job, such as increased risk of mortality (or a nicer location).

\small
Example: bonus of \$10K needed to recruit soldiers during Afghanistan-Irak wars (relative to peacetime).
Afghanistan-Irak wars carries an extra 1/1000 odd of dying $\Rightarrow$ Value of life would be
\$10K/.001=\$10m

Limitations: Requires people to be rational and measures value of life for marginal person

\normalsize
US studies show that revealed value of life is \$9.3 million on average currently
\end{slide}

\begin{slide}
\includepdf[pages={5}]{costbenefit_ch08_new_attach.pdf}
\end{slide}

\begin{slide}
\begin{center}
{\bf Trading-off time saved and value of life: speeding limits}
\end{center}
Speeding limits reduce traffic fatalities but increase travel time

Ashenfelter and Greenstone JPE'04 analyze speed limits:

In 1987, the federal government allowed states to raise the speed limit from 55 mph to 65 mph in rural highways
$\Rightarrow$ 21 states adopted higher speed limit

The 65 mph limit increased speeds by approximately 3.5\%, and increased fatality rates by roughly 35\%
$\Rightarrow$ 125,000 hours of travel time were saved per lost life

Valuing the time saved at the average hourly wage implies that adopting states were willing to accept risks that resulted in a savings of \$1.54 million (1997\$) per fatality

$\Rightarrow$ Those states were valuing a life saved at \$1.54 million at most

\end{slide}




%\begin{slide}
%\begin{center}
%{\bf MEASURING THE BENEFITS OF PUBLIC PROJECTS: OTHER PROBLEMS}
%\end{center}
%
%{\bf Discounting Future Benefits}
%
%A particularly thorny issue for cost-benefit analysis is that many projects have costs that are mostly immediate and benefits that are mostly long-term.
%
%{\bf Cost-Effectiveness Analysis}
%
%For projects that have unmeasurable benefits, or are viewed as desirable regardless of the level of benefits, we can compute only their costs and choose the most cost-effective project.
%\end{slide}

%8.3 Putting It All Together

\begin{slide}
\includepdf[pages={6}]{costbenefit_ch08_new_attach.pdf}
\end{slide}

\begin{slide}
\begin{center}
{\bf OTHER ISSUES IN COST-BENEFIT ANALYSIS}
\end{center}

{\bf Common Counting Mistakes:}
When analyzing costs and benefits, a number of common mistakes arise, such as:

\small

-Counting secondary benefits (e.g., more commerce activity around new highway comes at the
expense of other places)

-Counting labor as a benefit (e.g., labor is a cost, jobs created means those workers do not produce
something else)

-Double-counting benefits (e.g., rise in house values due to reduced commuting time)

\normalsize

{\bf Distributional Concerns: }
The costs and benefits of a public project do not necessarily accrue to the same individuals.

{\bf Uncertainty: }
The costs and benefits of public projects are often highly uncertain.
\end{slide}

%8.4 Conclusion

%\begin{slide}
%\begin{center}
%{\bf CONCLUSION}
%\end{center}
%
%Government analysts at all levels face a major challenge in attempting to turn the abstract notions of social costs and benefits into practical implications for public project choice.
%
%What at first seems to be a simple accounting exercise becomes quite complicated when resources cannot be valued in competitive markets.
%
%Nevertheless, economists have developed a set of tools that can take analysts a long way toward a complete accounting of the costs and benefits of public projects.
%\end{slide}

\begin{slide}
\begin{center}
{\bf REFERENCES}
\end{center}
{\small

Jonathan Gruber, Public Finance and Public Policy, Fourth Edition, 2012 Worth Publishers, Chapter 8

Ashenfelter, Orley, and Michael Greenstone. ``Estimating the Value of a Statistical Life: The Importance of Omitted Variables and Publication Bias.'' The American Economic Review 94.2 (2004): 454-460.\href{http://www.nber.org/papers/w10401.pdf}{(web)}

Diamond, Peter A., and Jerry A. Hausman. ``Contingent valuation: Is some number better than no number?.'' The Journal of economic perspectives 8.4 (1994): 45-64.\href{http://elsa.berkeley.edu/~saez/course131/diamond-hausman94.pdf}{(web)}

Weitzman, M.L., 1998. ``Why the far-distant future should be discounted at its lowest
possible rate.'' Journal of Environmental Economics and Management 36 (3),
201�208.\href{http://elsa.berkeley.edu/~saez/course131/weitzman98.pdf}{(web)}

}

\end{slide}


\end{document}
