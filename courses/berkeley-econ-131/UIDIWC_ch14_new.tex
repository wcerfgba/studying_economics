\documentclass[landscape]{slides}

\usepackage[landscape]{geometry}

\usepackage{pdfpages}

\usepackage{hyperref}
\usepackage{amsmath}

\def\mathbi#1{\textbf{\em #1}}

\topmargin=-1.8cm \textheight=17cm \oddsidemargin=0cm
\evensidemargin=0cm \textwidth=22cm


\author{131 Undergraduate Public Economics \\ Emmanuel Saez \\ UC Berkeley}
\date{}

\title{Unemployment Insurance, Disability Insurance, and Workers' Compensation} \onlyslides{1-300}

\newenvironment{outline}{\renewcommand{\itemsep}{}}

\begin{document}

\begin{slide}
\maketitle
\end{slide}

%\begin{slide}
%\begin{center}
%{\bf OUTLINE}
%\end{center}
%Chapter 14
%
%
%14.1 Institutional Features of Unemployment Insurance, Disability Insurance, and Workers' Compensation
%
%14.2 Unemployment Insurance
%
%14.3 Disability Insurance
%
%14.4 Workers' Compensation
%
%\end{slide}

%14.1 Institutional Features of Unemployment Insurance, Disability Insurance, and Workers' Compensation


\begin{slide}
\begin{center}
{\bf INSTITUTIONAL FEATURES}
\end{center}

Unemployment insurance, workers' compensation, and disability insurance are three social insurance programs in the United States, and they share many common features.

{\bf Unemployment insurance (UI)}:
A federally mandated, state-run program in which payroll taxes are used to pay benefits to unemployed workers laid off by companies.

{\bf Disability insurance (DI):} A federal program in which a portion
of the Social Security payroll tax is used to pay benefits to
workers who have suffered a medical impairment that leaves
them permanently unable to work.

{\bf Workers' compensation (WC):} State-mandated insurance, which
firms generally buy from private insurers, that pays for medical
costs and lost wages associated with an on-the-job injury.

\end{slide}

\begin{slide}
\includepdf[pages={1}]{UIDIWC_ch14_new_attach.pdf}
\end{slide}


% use Raj chetty UI lecture

\begin{slide}
\begin{center}
{\bf Unemployment Insurance}
\end{center}

Unemployment insurance is a major social insurance program in the U.S.

Spending size: \$50bn/year in normal times (up to \$150bn/year during Great Recession,
around \$500bn in 2020 due to COVID)

Macroeconomic importance in stabilization/stimulus

Like other social programs, triggered by an event

In this case, involuntary job loss

Controversial debate about unemployment benefits

Benefit: helps people in a time of need

Cost: reduces incentive to search for work while unemployed

%What is the optimal design of UI system given this tradeoff?

\end{slide}

\begin{slide}
\includepdf[pages={31}]{UIDIWC_ch14_new_attach.pdf}
\end{slide}


\begin{slide}
\begin{center}
{\bf Institutional Features of Unemployment Insurance}
\end{center}

UI is a federally mandated, state-run program

Although UI is federally-mandated, each state sets its own parameters on the program.

This creates a great deal of variation across states

Useful as a ``laboratory'' for empirical work

$\Rightarrow$ UI is a heavily studied program

In 2020 crisis, most state systems unable to cope with volume and new expanded rules  $\Rightarrow$ Weakness of 
decentralized system
\end{slide}

\begin{slide}
\begin{center}
{\bf Financing of UI Benefits}
\end{center}

1) UI is financed through a payroll tax on employers:

$\Rightarrow$ an employee will not see a deduction for UI on his or her paycheck.

This payroll tax averages 1-2\% of earnings

2) UI is partially experience-rated on firms

$\Rightarrow$ the tax that finances the UI program rises as firms have more layoffs, but not on a one-for-one basis
\end{slide}

\begin{slide}
\begin{center}
{\bf Eligibility Requirements and Benefits}
\end{center}

1) Individuals must have earned a minimum amount over the previous year.

2) Unemployment spell must be a result of a layoff, rather than from quitting or getting fired for cause
(easy to check)

3) Individual must be actively seeking work and willing to accept a job comparable to the one lost
(hard to check)

These eligibility requirements mean that not all of the unemployed actually collect benefits.

Even among eligible, 50\% do not takeup the UI benefit
(Lack of information about eligibility, stigma from collecting a government handout, or transaction costs)

Take-up typically lower in good times and depends on how hard
states make enrollment (e.g. Florida makes it hard)


\end{slide}

\begin{slide}
\includepdf[pages={30}]{UIDIWC_ch14_new_attach.pdf}
\end{slide}



\begin{slide}
\begin{center}
{\bf UI  Benefits}
\end{center}

UI benefits are a function of previous earnings

These benefits vary by state.

The replacement rate is the amount of previous earnings that is replaced by the UI system.

$R = B/W$

Replacement rates vary from 35\% to 55\% of earnings

In 2020 coronavirus crisis, CARES increased weekly benefits by \$600 across the board for 4 months,
and expands eligibility to self-employed and lower earners (+\$300 in Jan-Sep 2021)

Average UI benefit jumped up from \$400 to \$1000/week. Person on \$10/hour wage making \$400/week now
makes more on UI. Uniform \$600/week done bc of admin simplicity


\end{slide}

\begin{slide}
\includepdf[pages={2}]{UIDIWC_ch14_new_attach.pdf}
\end{slide}


\begin{slide}
\begin{center}
{\bf UI  Benefits Duration}
\end{center}

In general, one can collect UI for 6 months.

In recessions, benefits are automatically extended to 9 months or 12 months

In deep recessions, benefits can be further extended (23 months in
2008-13)

Duration of UI benefits typically much higher in European countries

In 2020 COVID crisis, UI extended to Sept 2021 for all the unemployed (including
the previously self-employed).

EU countries tend to have more generous and longer benefits
\end{slide}

%\begin{slide}
%\includepdf[pages={3}]{UIDIWC_ch14_new_attach.pdf}
%\end{slide}

\begin{slide}
\begin{center}
{\bf Analysis of Optimal Unemployment Insurance}
\end{center}

Optimal UI trades-off insurance value vs. efficiency costs

In principle, provide full insurance (perfect consumption smoothing) with 100\% 
replacement rate if there is no moral hazard

With moral hazard, 100\% replacement rate would eliminate incentives to find a job

$\Rightarrow$ Optimal replacement rate should be less than 100\% %(partial insurance)

Optimal replacement rate depends negatively on the size of moral hazard and positively
on how much people value insurance 

Empirical work examines size of moral hazard and value of UI for consumption smoothing


\end{slide}


\begin{slide}
\begin{center}
{\bf Expected Utility Model}
\end{center}

Individual's expected utility:
\[ EU = (1-p) u(c_e) + p u(c_u) = (1-p)u(w-t) + p u(b) \]
$p$: probability of being unemployed

$c_e$ = consumption when employed,

$c_u$ = consumption when unemployed

$w$ = wage when working

$t$ = tax used to finance program,

$b$ = UI benefit

Government needs to balance budget (taxes fund benefits):
\[(1-p) \cdot t = p \cdot b \quad  \Rightarrow \quad t= (p/(1-p)) \cdot b \]

\end{slide}

\begin{slide}
\begin{center}
{\bf Optimal UI with no moral hazard}
\end{center}
No moral hazard means that $p$ is not affected by UI

Plugging in govt. budget constraint, rewrite individual's expected utility as:
\[ EU = (1-p)u(w-(p/(1-p))b) + p u(b) \]

Government's problem: find $b$ that maximizes $EU$.

Optimal benefit $b^*$ will be $b$ such that: $c_u=c_e$

This is \textbf{full insurance} (as we saw earlier in class)
\end{slide}


\begin{slide}
\begin{center}
{\bf Optimal UI with moral hazard}
\end{center}
With moral hazard, $p$ increases with $b$ as more generous benefits
deter job search and hence increase unemployment

Government now chooses $b$ to maximize $EU$ but taking into account
that $p$ is a function of $b$ in the budget constraint
\[ EU = (1-p)u(w-[p(b)/(1-p(b))]b) + p u(b) \]
Get new formula:
\[ \frac{u'(c_u)-u'(c_e)}{u'(c_e)} = \frac{1}{1-p} \varepsilon_{p,b}   \text{ with }
\varepsilon_{p,b}  = \frac{b}{p} \cdot \frac{dp}{db}\]
$\varepsilon_{p,b}>0$ is the  elasticity of unemployment rate with respect to benefits (captures
size of moral hazard effects)

Now $0<c_u<c_e<w$: partial insurance is optimum. Optimum level increases with curvature of
$u(.)$ but decreases with elasticity $\varepsilon_{p,b}$.

\end{slide}

\begin{slide}
\begin{center}
{\bf Empirical Estimation of Effects of UI}
\end{center}

Moral hazard in UI manifests itself in the duration of the unemployment spell

Economists ask whether the unemployed find jobs more slowly when benefits are higher

Key challenge: need to use quasi-experiments to identify these effects

One common empirical approach (Meyer 1990): difference-in-difference

Exploit changes in UI laws that affect a ``treatment'' group and compare to a ``control'' group

\end{slide}

\begin{slide}
\includepdf[pages={4}]{UIDIWC_ch14_new_attach.pdf}
\end{slide}

\begin{slide}
\begin{center}
{\bf Empirical Estimation of Effects of UI: Evidence}
\end{center}

Meyer (1990) and many other implement this method using data on unemployment durations in the U.S. and state-level reforms

General finding: benefit elasticity of 0.4-0.6

10\% rise in unemployment benefits leads to about a 4-6\% increase in unemployment durations.

More recent empirical approach:
\textbf{regression discontinuity}

Card-Chetty-Weber (2007) use the fact that in Austria, you get up to 30 weeks of benefits when you have
been employed for 36+ months in last 5 years (instead of up to 20 weeks)

Can look at duration of unemployment based on how long you have worked in last 5 years
$\Rightarrow$ Finds somewhat smaller elasticity around 0.3

\end{slide}

\begin{slide}
\includepdf[pages={5}]{UIDIWC_ch14_new_attach.pdf}
\end{slide}

\begin{slide}
\begin{center}
{\bf Evidence on Consumption-Smoothing}
\end{center}

Difference-in-difference strategy has been used to examine how UI benefits affects consumption

Gruber (1997) finds that consumption falls on average when people lose their job by about 10-15\%

\$1 increase in UI benefits increases consumption by 30 cents

Much less than 1-1 because savings behavior changes, spousal labor supply, borrowing from friends, etc.
(this is called self-insurance)

Recent study by Ganong-Noel AER'19 uses bank account data to follow people through UI spell
$\Rightarrow$ Finds big effects of UI benefit exhaustion on consumption especially for groups with high replacement rates or low wealth

\end{slide}

\begin{slide}
\includepdf[pages={21-23}]{UIDIWC_ch14_new_attach.pdf}
\end{slide}


\begin{slide}
\begin{center}
{\bf Does UI have Long-Term Benefits?}
\end{center}

Another potential benefit of UI, neglected in simple model above: improvements in \textbf{match quality}

Are people forced to take worse jobs because they have to rush back to work to put food on the table?

E.g. engineer starts working at McDonalds.

Can examine this using similar data

Look at whether people who got higher benefits and took longer to find a job are better off years later

Card-Chetty-Weber (2007) exploit again the \textbf{regression discontinuity} and find no long-term match benefit
on subsequent wage or subsequent job duration
\end{slide}

\begin{slide}
\includepdf[pages={6,7}]{UIDIWC_ch14_new_attach.pdf}
\end{slide}

\begin{slide}
\begin{center}
{\bf Summary of Empirical Findings on UI}
\end{center}

1. Higher benefit level $\Rightarrow$ longer unemployment durations (moral hazard cost)

2. Higher benefit level $\Rightarrow$ more consumption while unemployed (consumption smoothing benefit)

3. UI benefits have no beneficial effects on long-term job outcomes

$\Rightarrow$ Model implies that providing some UI is desirable but UI replacement rate should
be only around 50\% based on those empirical findings

\end{slide}

%\begin{slide}
%\begin{center}
%{\bf UI and Firm Behavior: Experience Rating}
%\end{center}
%
%Effect of UI on firms comes from experience rating
%
%Perfect experience rating for firms: if firm A lays a person off, firm A pays
%taxes to make up for the costs firm A imposes on the UI system
%
%Employers with a lot of layoffs get taxed more
%
%In practice, we have partial experience rating
%
%Firm does not fully pay an additional tax each time it lays off a worker.
%
%Payroll taxes rise less than one-for-one with layoffs because of cap on UI tax
%
%$\Rightarrow$ Empirical studies find that imperfect experience rating leads to more
%layoffs
%\end{slide}
%
%\begin{slide}
%\includepdf[pages={8}]{UIDIWC_ch14_new_attach.pdf}
%\end{slide}
%
%
%\begin{slide}
%\begin{center}
%{\bf Partial Experience Rating and Temporary Layoffs}
%\end{center}
%
%Partial experience rating subsidizes firms with high layoff rates.
%
%Firms and workers may make a joint decision whether to place the worker on temporary layoff, with a promise of being hired back later.
%
%UI system makes this a partially paid vacation.
%
%With partial experience rating, government ends up sharing in the cost of the vacation.
%\end{slide}
%

%\begin{slide}
%\begin{center}
%{\bf Experience Rating in Canada}
%\end{center}
%
%UI systems in other countries create even more moral hazard inefficiency on the firm side than in the U.S.
%
%In Canada, UI is financed through a flat payroll tax, unrelated to actual layoff behavior.
%
%In Canada, workers only have to work 10 weeks to qualify for 42 weeks of UI with a replacement rate of 60\%.
%\end{slide}
%
%\begin{slide}
%\begin{center}
%{\bf Example: Distortions in Canadian Fishing}
%\end{center}
%Consequences of imperfect experience rating:
%
%You and four friends buy a fishing boat, and can catch \$40,000 in fish over 10 weeks, or \$8,000 per person.
%
%In the absence of UI, \$8,000 is not enough for you or your friends to want to operate this business.
%
%With Canada's UI system, you work for 10 weeks and are then ``laid off''. Your earnings were \$800 per week, of which 60\% is replaced by UI for the remainder of the year.
%
%Your benefit from UI is 0.6*800*42, or \$20,160.
%
%With UI, each person gets \$28,160 for only 10 weeks of work.
%
%UI induces inefficiently large number of fisherman, construction workers, etc.
%\end{slide}
%

%\begin{slide}
%\begin{center}
%{\bf Partial Experience Rating and Layoffs: Evidence}
%\end{center}
%
%Empirical studies have examined state systems with different degrees of experience rating using difference-in-difference methods
%
%They find that partial experience rating increases the rate of temporary layoffs.
%
%Partial experience rating alone can account for one- third of all temporary layoffs in the U.S.
%\end{slide}
%
%\begin{slide}
%\begin{center}
%{\bf The Benefits of Partial Experience Rating}
%\end{center}
%
%What is the benefit of partial experience rating?
%
%Fully experience rated UI would ``hit firms while they are down.''
%
%Similar to ``consumption smoothing'' for workers
%
%Conceptually, tradeoff is similar to that we discussed for individual workers
%
%But for firms, smoothing benefits are weaker
%
%Firms have collateral - should be easier to get a loan
%
%Most economists agree that UI should be fully
%experience rated\end{slide}
%

\begin{slide}
\begin{center}
{\bf Should UI Benefits be Extended during Recessions?}
\end{center}
US extends UI benefits during recessions. Extensions ended in 2014 (controversial policy debate)

\textbf{1) Social Justice:} Harder to find jobs in recessions $\Rightarrow$ being unemployed is less of a choice 

$\Rightarrow$ Extending benefits is desirable for fairness

\textbf{2)  Efficiency:} In recessions, the job market is too slack [harder to find jobs, easier for firms to find workers]
$\Rightarrow$ discouraging search effort in recessions is not as problematic. 

Furthermore, UI benefits support spending and hence the economy (through short-term macro effects) 

$\Rightarrow$ Extending benefits is desirable for efficiency

%If longer UI benefits decrease slack in labor market then longer UI benefits desirable [this is the case if UI benefits stimulate aggregate demand or if job seekers compete for a fixed number of jobs in recession, this is the left-wing view]

%b) If longer UI benefits increase slack in labor market then shorter UI benefits desirable [this is the case if longer UI benefits
%increase the bargaining power of workers and hence increase wages further reducing labor demand, this is the right-wing
%view]

%\normalsize
%Economists try to tell apart a) from b) using empirical evidence



\end{slide}



\begin{slide}
\begin{center}
{\bf DISABILITY INSURANCE}
\end{center}
Disability is conceptually close to retirement: some people
become unable to work before old age (due to accidents, medical
conditions, etc.)

All advanced countries offer public disability insurance almost
always linked to the public retirement system

Disability insurance allows people to get Social Security retirement benefits before the ``Early
Retirement Age'' if they are unable to work due to disability

\end{slide}

\begin{slide}
\begin{center}
{\bf US DISABILITY INSURANCE}
\end{center}
1) Federal program funded by OASDI payroll tax, pays
SS benefits to disabled workers under retirement age (similar computation
of benefits based on past earnings)

2) Program started in 1956 and became more generous overtime
(age 50+ condition removed, definition of disability liberalized,
replacement rate has grown)

3) Eligibility: Medical proof of being unable to work for at least a year,
Need some prior work experience, 5 months waiting period with no earnings
required (screening device)

4) Social security examiners rule on applications. Appeal possible
for rejected applicants. Imperfect process with big type I and II errors
(Parsons AER'91) $\Rightarrow$ Scope for Moral Hazard

5) DI tends to be an absorbing state (most beneficiaries never leave DI program). Can earn up of \$1200/month while on DI.
\end{slide}



\begin{slide}
\begin{center}
{\bf US DISABILITY INSURANCE}
\end{center}
1) In 2018, about 10.2m DI beneficiaries (not counting
widows+children), about 5-6\% of working age (20-64) population

2) Very rapid growth: In 1960, less than 1\% of working age population was on DI

3) Growth particularly strong during recessions: early 90s, late 00s.
Slight decline from 11m in 2013 to 10.2m in 2018

\textbf{Key empirical question:} Are DI beneficiaries unable to work? or are DI beneficiaries
not working because of DI.

\end{slide}


\begin{slide}
\includepdf[pages={9,10}]{UIDIWC_ch14_new_attach.pdf}
\end{slide}


\begin{slide}
\begin{center}
{\bf US DISABILITY INSURANCE}
\end{center}

Detecting disability is challenging, particularly for back injuries and mental health conditions

One way to quantify difficulty in assessment: audit study

Take a set of disability claims that was initially reviewed by a state panel

One year later, resubmit them to the panel as anonymous new claims.

Compare decisions on the \textbf{same} cases

$\Rightarrow$ Substantial evidence of Type I errors (incorrect rejection of a disabled person) and Type II errors
(letting a non-disabled person on the program)

\end{slide}


\begin{slide}
\includepdf[pages={11}]{UIDIWC_ch14_new_attach.pdf}
\end{slide}


%\begin{slide}
%\includepdf[pages={12}]{UIDIWC_ch14_new_attach.pdf}
%\end{slide}
%
%
%\begin{slide}
%\begin{center}
%{\bf DI Empirical Effects: Observational Studies}
%\end{center}
%Parallel growth of DI recipients and non-participation rates among men aged 45-54
%but causality link not clear
%
%\textbf{Cross-Sectional Evidence (Parsons '80):} Does potential DI replacement rate
%have an impact on labor force participation (LFP) decision?
%
%Uses cross-sectional variation in potential replacement rates
%
%Survey data on men aged 45-59 from 1966-69
%
%OLS regression
%\[ NLFP_i=\alpha+ \beta DIreprate_i + \varepsilon_i \]
%Large effect that can fully explain decline in LFP among men 45+
%
%\end{slide}
%
%\begin{slide}
%\begin{center}
%{\bf DI EMPIRICAL EFFECTS: OBSERVATIONAL STUDIES}
%\end{center}
%\textbf{Issues with Cross-Sectional Evidence:}
%
%1) $DIreprate_i$ depends on wages (higher for low wage earners)
%and likely to be correlated with $\varepsilon_i$ (likelihood to become truly disabled)
%
%2) Impossible to control fully for wages in regression because all variation
%in $DIreprate_i$ is due to wages 
%
%3) Bound AER'89 replicates Parson's regression on sample that never applied to
%DI and obtains similar effects implying that the OLS correlation not driven by UI
%
%\end{slide}

\begin{slide}
\begin{center}
{\bf DI EMPIRICAL EFFECTS: REJECTED APPLICANTS}
\end{center}
Bound AER'89 proposes a technique to bound effect of DI on LFP rate

Uses data on LFP on (small sample of) rejected applicants as a counterfactual

\textbf{Idea:} If rejected applicants do not work, then surely DI recipients
would not have worked
$\Rightarrow$
Rejected applicants' LFP rate is an upper bound for LFP rate of DI
recipients absent DI

\textbf{Results:} Only 30\% of rejected applicants return to work
and they earn less than half of the mean non-DI wage

$\Rightarrow$ at most 1/3 of the trend in male LFP decline can be
explained by shift to DI

Von Waechter-Manchester-Song AER'11 replicate Bound using full pop SSA admin data
and confirm his results
\end{slide}

\begin{slide}
\includepdf[pages={13}]{UIDIWC_ch14_new_attach.pdf}
\end{slide}

\begin{slide}
\begin{center}
{\bf DI EMPIRICAL EFFECTS: REJECTED APPLICANTS}
\end{center}
Maestas-Mullen-Strand AER'13 obtain causal effect of DI
on LFP using natural variation in DI examiners' stringency and
large SSA admin data linking DI applicants and examiners

\textbf{Idea:} (a) Random assignment of DI appplicants to examiners and
(b) examiners vary in the fraction of cases they reject $\Rightarrow$ Valid
instrument of DI receipt

\textbf{Result 1:} DI benefits reduce LFP of applicants by 28 points $\Rightarrow$ DI has an impact
but fairly small (consistent with Bound AER'89)

\textbf{Result 2:} DI has heterogeneous impact: small effect on
those severely impaired but big effect on less severly impaired

\small Tough judges marginal cases unlikely to work without DI, lenient
judges marginal cases somewhat likely to work without DI

% XX DI allowance rate means "fraction of applicants admitted in DI, high means DI is lenient"
\end{slide}

\begin{slide}
\includepdf[pages={20}]{UIDIWC_ch14_new_attach.pdf}
\end{slide}


\begin{slide}
\begin{center}
{\bf DI Generosity Effects: Regression Kink Design (RKD)}
\end{center}
DI benefits calculated like SS benefits: AIME formula 
based on average life-time earnings creates a ``kinked'' relationship 

Ideal setting for an RKD (Card et al. 2015): test whether outcome such as earnings or mortality is also ``kinky''

1) Test first for no sorting of DI recipients around kink to validate RKD design [similar to RDD validation]

2) RKD estimate: Change in slope of outcome at kink / Change in slope of benefits at kink

a) Gelber et al. '17 analyze effects on earnings of DI generosity and find
an income effect of -\$0.2 per dollar of benefits

b) Gelber et al. 18 analyze effects on mortality: at lower bend point, \$1K extra DI/year reduces annual mortality by .25 points (1 out of 400 lives saved)
\end{slide}

\begin{slide}
\includepdf[pages={28, 27, 25, 24}]{UIDIWC_ch14_new_attach.pdf}
\end{slide}


%\begin{slide}
%\begin{center}
%{\bf Moral Hazard in Disability Insurance}
%\end{center}
%Bound AER'89 evidence is suggestive that DI is not solely responsible for reduction in labor supply over time
%
%But does not tell us how big an effect DI has
%
%Strategy 3: Difference-in-difference
%
%Cannot be implement in U.S.
%
%Gruber (2000): studies Candian experience, comparing Quebec with the rest of Canada
%
%1987: reform that sharply increased benefit in rest of Canada while Quebec was unchanged
%
%Finds an elasticity of labor force participation w.r.t. DI benefit rate of 0.2
%
%\end{slide}
%
%\begin{slide}
%\includepdf[pages={14,15}]{UIDIWC_ch14_new_attach.pdf}
%\end{slide}




%\begin{slide}
%\begin{center}
%{\bf Moral Hazard vs. Benefits of DI}
%\end{center}
%Gruber study finds an elasticity of labor force participation w.r.t. DI benefit rate of 0.2
%
%Important to note that this is not in itself evidence that DI is ``bad''
%
%May simply be helping people who have a very high disutility of labor and were forced to work to survive
%
%This is why it is critical to compare costs of taking people out of labor force with benefits (relieving need to work for those who are disabled)
%
%No good evidence on latter issue yet $\Rightarrow$ unclear whether DI benefit is too high or low.
%
%
%\end{slide}






\begin{slide}
\begin{center}
{\bf Workers Compensation: Institutional Features}
\end{center}

Workers compensation is insurance
for injuries on the job, mainly temporary injuries that prevent work (short-term)

Workers Compensation is a state-level program

Two components: medical and indemnity

Indemnity payment replaces roughly two-thirds of lost wages.

Unlike UI, WC payments are untaxed, leading to a higher replacement that is near 90\% on average.

Substantial variation across states in benefit levels


\end{slide}

%\begin{slide}
%\includepdf[pages={16}]{UIDIWC_ch14_new_attach.pdf}
%\end{slide}



\begin{slide}
\begin{center}
{\bf Workers Compensation (WC): Institutional Features}
\end{center}
1) Workers comp is a mandated benefit; no explicit tax but firms required by law to provide this benefit to workers

Most firms choose to buy coverage from private insurers

Premiums are more tightly experience rated than UI because they are determined by private sector

Insurance companies charge high-risk firms more.

2) Important feature of WC: no-fault insurance.

When there is a qualifying injury, WC benefits paid regardless of whether the injury was the worker's or the firm's fault.

Idea: reduce inefficiency of tort system (legal costs) by having fixed rules and not worrying about liability


\end{slide}

\begin{slide}
\begin{center}
{\bf Moral Hazard in Workers? Compensation}
\end{center}
Moral hazard in WC can manifest itself in reported injuries, injury durations, and types of injuries reported.

E.g. easier to report back pain--very hard to verify

Huge issue in CA--companies paid high workers comp rates

Governor Schwarzenegger reform in 2004 cut benefits sharply, claiming to reduce injuries and ``open CA for business''

Is it true that there is substantial moral hazard?

Again, consider several pieces of evidence

Strategy 1: Timing of injuries. ``Monday effect'' (faking week-end injuries into
work injuries)

\end{slide}

\begin{slide}
\includepdf[pages={17}]{UIDIWC_ch14_new_attach.pdf}
\end{slide}


\begin{slide}
\begin{center}
{\bf Moral Hazard in Workers? Compensation}
\end{center}
Strategy 2: examine effect of workers comp benefit levels on durations using a diff-in-diff strategy (Meyer, Viscusi, Durbin 1995)

Reforms in Kentucky and Michigan that increased benefits for high-earning workers (but not low-earning workers) in late 1980s

Compare changes in injury durations and medical costs for high-earners vs. low earners in those states before and after reform


\end{slide}

\begin{slide}
\includepdf[pages={29}]{UIDIWC_ch14_new_attach.pdf}
\end{slide}

\begin{slide}
\includepdf[pages={18,19}]{UIDIWC_ch14_new_attach.pdf}
\end{slide}


\begin{slide}
\begin{center}
{\bf Moral Hazard in Workers' Compensation}
\end{center}
Result: 10\% increase in WC benefit raises out-of-work duration due to injury by 4\%

Again, need to weigh this against benefits to reach policy conclusions

Give people more time to heal after injury without rushing them back to work

Higher consumption while out of work

No evidence yet on these issues


\end{slide}


\begin{slide}
\begin{center}
{\bf CONCLUSION}
\end{center}

Individuals clearly value the consumption smoothing provided by social insurance programs

In each case there are moral hazard costs associated with the provision of the insurance

Empirical analyses of all three programs can be used to inform policy makers' decisions as program reforms move forward
\end{slide}



\begin{slide}
\begin{center}
{\bf REFERENCES}
\end{center}
{\small

Jonathan Gruber, Public Finance and Public Policy, Fifth Edition, 2016 Worth Publishers, Chapter 14

Bound, John. ``The Health and Earnings of Rejected Disability Insurance Applicants.'' American Economic Review 79.3 (1989): 482-503.\href{http://www.jstor.org/stable/pdfplus/1806858.pdf}{(web)}

Card, David, Raj Chetty, and Andrea Weber. ``The Spike at Benefit Exhaustion: Leaving the Unemployment System or Starting a New Job?.'' The American Economic Review 97.2 (2007): 113-118.\href{http://www.jstor.org/stable/pdfplus/30034431.pdf}{(web)}

Card, David, David S. Lee, Zhuan Pei, and Andrea Weber. 2015. ``Inference on Causal Effects in a
Generalized Regression Kink Design.'' Econometrica 83 (6): 2453-83.
\href{https://www.jstor.org/stable/pdf/43866417.pdf} {(web)}


Card, David, and Brian P. McCall. ``Is Workers' Compensation Covering Uninsured Medical Costs? Evidence from the`` Monday Effect''.'' Industrial and Labor Relations Review (1996): 690-706.\href{http://elsa.berkeley.edu/~saez/course131/Card-McCall96}{(web)}


Ganong, Peter  and Pascal Noel. 2019. ``Consumer Spending During Unemployment: Positive and Normative Implications.''  American Economic Review
\href{http://elsa.berkeley.edu/~saez/course131/ganong-noelAER19UI.pdf}{(web)}


Gelber, Alex, Timothy Moore, and Alexander Strand. 2018. ``The Effect of Disability Insurance Payments on Beneficiaries' Earnings,'' American Economic Journal: Economic Policy, 9(3),  229-261
\href{http://elsa.berkeley.edu/~saez/course/gelbermoorestrandDI17earnings.pdf} {(web)} 


Gelber, Alex, Timothy Moore, and Alexander Strand. 2018
``Disability Insurance Income Saves Lives'' UC Berkeley Working Paper
\href{http://elsa.berkeley.edu/~saez/course/gelbermoorestrandDI18mortality.pdf} {(web)} 

Gruber, Jonathan. ``The Consumption Smoothing Benefits of Unemployment Insurance.'' The American Economic Review (1997).\href{http://www.nber.org/papers/w4750.pdf}{(web)}

Gruber, Jonathan. ``Disability Insurance Benefits and Labor Supply.'' The Journal of Political Economy 108.6 (2000): 1162-1183.\href{http://www.jstor.org/stable/pdfplus/10.1086/317682.pdf}{(web)}

Gruber, Jonathan. ``Covering the Uninsured in the United States.'' Journal of Economic Literature, 43.3 (2008): 571-606.\href{http://www.nber.org/papers/w13758.pdf}{(web)}

Maestas, Nicole, Kathleen Mullen and Alexander Strand
``Does Disability Insurance Receipt Discourage Work?
Using Examiner Assignment to Estimate Causal Effects of SSDI Receipt'', American Economic Review, 103(5), 2013,
1797-1829.
\href{http://elsa.berkeley.edu/~saez/course/maestas-mullen-strandAER13.pdf} {(web)} 

Meyer, Bruce D. ``Unemployment Insurance and Unemployment Spells.'' Econometrica 58.4 (1990): 757-782.\href{http://www.jstor.org/stable/pdfplus/2938349.pdf}{(web)}

Meyer, Bruce D., W. Kip Viscusi, and David L. Durbin. ``Workers' compensation and injury duration: evidence from a natural experiment.'' The American Economic Review (1995): 322-340.\href{http://www.jstor.org/stable/pdfplus/2118177.pdf}{(web)}

Parsons, Donald O. ``The decline in male labor force participation.'' The Journal of Political Economy (1980): 117-134.\href{http://www.jstor.org/stable/pdfplus/1830962.pdf}{(web)}

Parsons, Donald O. ``The health and earnings of rejected disability insurance applicants: comment.'' The American Economic Review 81.5 (1991): 1419-1426.\href{http://www.jstor.org/stable/pdfplus/2006930.pdf}{(web)}

Smith, Richard Thomas, and Abraham M. Lilienfeld. ``The social security disability program: an evaluation study.''[Book] No. 39. US Social Security Administration, Office of Research and Statistics, 1971.

Von Wachter, Till, Jae Song, and Joyce Manchester. ``Trends in employment and earnings of allowed and rejected applicants to the social security disability insurance program.'' American Economic Review 101.7 (2011): 3308.\href{http://elsa.berkeley.edu/~saez/course131/vonwachterAER11}{(web)}





}
\end{slide}



\end{document}
