\documentclass[landscape]{slides}

\usepackage[landscape]{geometry}

\usepackage{pdfpages}

\usepackage{hyperref}
\usepackage{amsmath}

\def\mathbi#1{\textbf{\em #1}}

\topmargin=-1.8cm \textheight=17cm \oddsidemargin=0cm
\evensidemargin=0cm \textwidth=22cm


\author{131 Undergraduate Public Economics \\ Emmanuel Saez \\ UC Berkeley}
\date{}


\title{Social Security and Retirement} \onlyslides{1-300}

\newenvironment{outline}{\renewcommand{\itemsep}{}}

\begin{document}

\begin{slide}
\maketitle
\end{slide}

%\begin{slide}
%\begin{center}
%{\bf OUTLINE}
%\end{center}
%Chapter 13
%
%{\bf Social Security}:
%A federal program that taxes workers to provide income support to the elderly.
%
%13.1 What Is Social Security and How Does It Work?
%
%13.2 Consumption-Smoothing Benefits of Social Security
%
%13.3 Social Security and Retirement
%
%13.4 Social Security Reform
%
%13.5 Conclusion
%\end{slide}
%


%13.1 What Is Social Security and How Does It Work?


\begin{slide}
\begin{center}
{\bf RETIREMENT PROBLEM}
\end{center}

{\bf Life-Cycle:} Individuals ability to work declines with aging and
continue to live after they are unwilling/unable to
work

{\bf Standard Life-Cycle Model Prediction:} Absent any government program,
rational individual would save while working to consume savings
while retired [Modigliani life cycle graph] 
%earnings, wealth,
%consumption: $T=60$ adult life, $R=40$ working life, $T-R=20$
%retirement life]

Optimal saving problem is extremely complex: uncertainty in
returns to saving, in life-span, in future ability/opportunities
to work, in future tastes/health

{\bf In practice:} When govt was small $\Rightarrow$ Many people
worked till unable to (often till death) and then were taken care
of by family members 

{\bf Today:} Govt is taxing workers to provide for retirees through
social security retirement systems




\begin{slide}
\includepdf[pages={24}]{socialsecurity_ch13_new_attach.pdf}
\end{slide}

\end{slide}
\begin{slide}
\includepdf[pages={37}]{socialsecurity_ch13_new_attach.pdf}
\end{slide}

%\begin{slide}
%\includepdf[pages={35, 34, 30-33}]{socialsecurity_ch13_new_attach.pdf}
%\end{slide}

\begin{slide}
\includepdf[pages={35, 34}]{socialsecurity_ch13_new_attach.pdf}
\end{slide}

\begin{slide}
\includepdf[pages={39}]{socialsecurity_ch13_new_attach.pdf}
\end{slide}

\begin{slide}
\begin{center}
{\bf GOVT INTERVENTION IN RETIREMENT POLICY}
\end{center}

{\bf Actual Retirement Programs:} All OECD countries implement
substantial government funded retirement programs (substantial share of GDP around
6-10\%, US smaller around 5\%), started in first part of 20th
century and have been growing. 

\textbf{Common structure:}

Individuals pay social security contributions (payroll taxes) while
working and receive retirement benefits when they stop working
till the end of their life (annuity)

Extension of the earlier family model: it's no longer your own working
kids who take care of you in old age but all workers in the country

In the United States, the public retirement program is called
\textbf{Social Security}

\end{slide}

\begin{slide}
\includepdf[pages={40}]{socialsecurity_ch13_new_attach.pdf}
\end{slide}


\begin{slide}
\begin{center}
{\bf SOCIAL SECURITY: PROGRAM DETAILS}
\end{center}

{\bf How Is Social Security Financed?}

Almost all workers in the United States pay the Federal Insurance Contributions Act (FICA) tax on their earnings.

Tax is 12.4\% of earnings (6.2\% paid by employer, 6.2\% paid by employees) up to a cap of \$142,800 in 2021

{\bf Who Is Eligible to Receive Social Security?}

A person must have worked and paid this payroll tax for 40 quarters (10 years) over their lifetime, and must be of age 62 or older.
\end{slide}

\begin{slide}
\begin{center}
{\bf SOCIAL SECURITY: PROGRAM DETAILS}
\end{center}

{\bf How Are Social Security Benefits Calculated?}

{\bf Annuity payment}:
A payment that lasts until the recipient's death.

The amount of this annuity payment is a progressive function of the recipient's average (taxable) earnings over the person's 35 highest earning years where each month's earnings are expressed in today's dollars (AIME = average indexed monthly earnings)

Once benefits start for a given person, they are indexed to price inflation once every year (``real'' annuity)

Higher earners live longer. Progressivity of benefits formula roughly offsets this (but life expectancy gap between rich and poor is increasing)
\end{slide}





\begin{slide}
\includepdf[pages={3}]{socialsecurity_ch13_new_attach.pdf}
\end{slide}


\begin{slide}
\includepdf[pages={41}, scale=.9]{socialsecurity_ch13_new_attach.pdf}
\end{slide}

\begin{slide}
\begin{center}
{\bf How Are Social Security Benefits Paid Out?}
\end{center}

{\bf Full Benefits Age (FBA)}:
The age at which a Social Security recipient receives full retirement benefits (Primary Insurance Amount):
currently 67 if born 1960+ (used to be 65)

{\bf Early Entitlement Age (EEA)}:
The earliest age at which a Social Security recipient can receive reduced benefits: currently 62

If you claim benefits 1 year before FBA, you get 8\% less in annual benefits (permanently), if you claim
2 years before FBA, you get 16\% less in annual benefits (permanently), etc.

You get 8\% more in benefits if you claim 1 year after FBA. Benefits automatically paid at 70.

\end{slide}

\begin{slide}
\begin{center}
{\bf SOCIAL SECURITY: PROGRAM DETAILS}
\end{center}

{\bf Can You Work and Receive Social Security?}\\
The \emph{earnings test} reduces benefits of the 62 to 66-year old by \$0.50 for each dollar of earnings they have above
about \$15K

Not really a tax because later benefits are increased (as if you had retired later) but most people don't
understand the system and perceive the earnings test as a pure tax 

$\Rightarrow$ Bunching at earnings test kink at ages 62-65 (Gelber-Jones-Sacks '19)

{\bf Are There Benefits for Family Members?}\\
-Spouses of claimants (get own benefits or 50\% of primary earner benefits, whichever is biggest)\\
-Children of deceased workers.\\
-Spouses who survive a Social Security recipient

\end{slide}

\begin{slide}
\includepdf[pages={36}]{socialsecurity_ch13_new_attach.pdf}
\end{slide}

\begin{slide}
\includepdf[pages={38}]{socialsecurity_ch13_new_attach.pdf}
\end{slide}

\begin{slide}
\begin{center}
{\bf SOURCES OF RETIREMENT INCOME IN THE US}
\end{center}
1) Govt provided retirement benefits (US Social Security):
For 2/3 of retirees, SS is more than 50\% of income. 1/3 of
elderly households depend almost entirely on SS.

2) Home Ownership: 75\% of US elderly are homeowners

3) Employer pensions (tax favored): 40-45\% of elderly US households have employer pensions.
Two types:

a) Traditional: Defined Benefit (DB) and mandatory: {\bf employer} carries full
risk [in sharp decline, many in default]

b) New: Defined Contribution (DC) and elective: 401(k)s, {\bf employee}
carries full risk

%60\% of workers have access to empl. pensions, 45\% contribute

%4) Supplementary individual elective pensions (tax favored): IRAs and Keoghs
%(Keoghs for the self-employed)

4) Extra savings through non-tax favored instruments:
significant only for wealthy minority [=10\% of retirees]

\textbf{Key lesson:} Bottom 90\% wealth is (a) housing (net of mortgage debt), (b) pensions,
(c) minus other debts (consumer credit, student loans) 

All 3 components are heavily affected by government policy (education finance), institutions (such as employers), financial regulations (mortgage refinance, credit card and loans)

\end{slide}



\begin{slide}
\begin{center}
{\bf FUNDED VS. UNFUNDED PROGRAMS}
\end{center}
Two forms of retirement programs:

{\bf 1) Unfunded (pay-as-you-go):} benefits of current retirees are
paid out of contributions from current workers [generational link]

current benefits = current contributions

{\bf 2) Funded:} workers contributions are invested in financial
assets and will pay for benefits when they retire [no generational
link]

current benefits = past contributions + market returns on past contributions

Social security (as most public retirement systems) is unfunded

Most private pension plans (such as 401(k)s) are funded

\end{slide}

\begin{slide}
\begin{center}
{\bf FUNDED VS UNFUNDED SYSTEMS}
\end{center}

{\bf 1) Funded system:} each generation gets a market return $r$ on
contributions: benefits=tax you paid $\cdot (1+r)$

{\bf 2) Unfunded system:} 1st generation of retirees gets free benefits when the
system starts

For later generations: pay tax (for older generation) and you get benefits from younger
generation

%benefits received/taxes paid = $(1+n) \cdot (1+g)$ 

Implicit return on taxes is the sum of population growth $n$ and real wage growth (per worker) $g$

benefits=tax paid $\cdot (1+n)(1+g) \simeq$ tax paid $\cdot (1 + n + g)$

\end{slide}

%\begin{slide}
%\includepdf[pages={4}]{socialsecurity_ch13_new_attach.pdf}
%\end{slide}





\begin{slide}
\begin{center}
{\bf FUNDED VS UNFUNDED SYSTEMS}
\end{center}

Unfunded system is always desirable when $n+g>r$ (Diamond 1965): an economy
with $n+g>r$ is called \textbf{dynamically inefficient} and introducing an unfunded system
makes a Pareto improvement

US economy: Annual $n=1\%$ and $g=1\%$ [$n+g$ was higher in
1940-1970]. $r \simeq 5\%$. In general $r>n+g$ in practice.

Note that $r$ is much more risky than $n+g$: risk adjusted market
rate of return should be lower than average market rate $r$ but still higher than $n+g$

Funded system delivers higher returns because it does not deliver
a free lunch to 1st generation

Choice between funded vs. unfunded system is an \textbf{inter-generational redistribution trade-off}


\end{slide}







%\begin{slide}
%\begin{center}
%{\bf How Does Social Security Redistribute in Practice?}
%\end{center}
%
%{\bf Social Security Wealth (SSW)}:
%The expected present discounted value of a person's future Social Security payments minus the expected present discounted value of a person's payroll tax payments.
%
%SSW is computed as follows:\\
%-Calculate the entire future stream of benefits that a person expects to receive before he or she dies.\\
%-Use a discount rate to calculate the present discounted value (PDV) of that stream of benefits.\\
%-Calculate the entire future stream of social security taxes that a person expects to pay before he or she dies.\\
%-Compute the PDV of that stream of taxes.\\
%-Take the difference between these two to get the SSW.
%\end{slide}
%
%\begin{slide}
%\includepdf[pages={5,6}]{socialsecurity_ch13_new_attach.pdf}
%\end{slide}

%13.2 Consumption-Smoothing Benefits of Social Security

%\begin{slide}
%\begin{center}
%{\bf RATIONALES FOR SOCIAL SECURITY}
%\end{center}
%
%{\bf A. Individual Failure (MOST IMPORTANT)}
%
%Without a public program, people won't save enough for their own retirement because of myopia, self-control problems, information (how much to save, how to invest savings)
%
%Popularity of Social Security suggests that people understand their own failures and the need for government intervention
%
%{\bf B. Adverse selection in the annuities market}
%
%The longer a person lives, the less money the insurer makes from an annuity contract
%
%$\Rightarrow$ People with short life expectancy less likely to buy annuities
%
%This could lead to such a high price for annuities that most potential buyers would not want to buy them
%\end{slide}


\begin{slide}
\begin{center}
{\bf MODEL: RATIONAL VS. MYOPIC SAVERS}
\end{center}
Most important reason for social security: many people are unable to save rationally
for retirement (due to myopia, self-control problems, lack of information, etc.)

1) Rational individuals: [draw graph]


$\max_{c_1,c_2} u(c_1) + \delta u(c_2)$ st $c_1+s=w$ and $c_2=s\cdot(1+r)$

$\Rightarrow$ $c_1+c_2/(1+r)=w$

FOC: $u'(c_2)/u'(c_1)=1/[(1+r)\delta]$, let $s^*$ be optimal
saving

Example: If $\delta=1$ and $r=0$ then $c_1=c_2=w/2$ and $s^*=w/2$ 

2) Myopic individuals:

$\max_{c_1,c_2} u(c_1)$ st $c_1+s=w$ and
$c_2=s\cdot(1+r)$ $\Rightarrow$ $c_1=w$ and $s=c_2=0$

\end{slide}

\begin{slide}
\includepdf[pages={25}]{socialsecurity_ch13_new_attach.pdf}
\end{slide}

\begin{slide}
\begin{center}
{\bf MODEL: RATIONAL VS. MYOPIC SAVERS}
\end{center}

Social welfare is always $u(c_1) + \delta u(c_2)$

Govt imposes forced saving tax $\tau$ such that $\tau=s^*$
and benefits $b=\tau \cdot (1+r)$. Cannot borrow against $b$ [as in
current Social Security]

1) Rational individual unaffected: adjusts $s$ one-to-one so that
outcome unchanged [rational unaffected as long as $\tau \leq
s^*$]: 100\% crowding out of private savings by forced savings

\small
$c_1=w-(s^*+s')$ and and $c_2=(s^*+s')\cdot(1+r)$ $\Rightarrow$ choosing
$s'$ is equivalent to choosing $s=s^*+s'$, rational person chooses $s'=0$
\normalsize

2) Myopic individual affected (0\% crowding out): new outcome
maximizes Social Welfare

Forced savings is a good solution: does not affect those responsible,
affects the myopic individuals in socially desired way

\end{slide}

\begin{slide}
\includepdf[pages={26}]{socialsecurity_ch13_new_attach.pdf}
\end{slide}


\begin{slide}
\begin{center}
{\bf MODEL: COMMENTS}
\end{center}

{\bf 1) Universal vs. Means-Tested Program:} Universal forced savings is better than means-tested program financed by tax on everybody [Samaritan's dilemma]. With means-test program, two drawbacks:

a) Responsible individuals subsidize myopic individuals 

b) Incentives to under-save to get means-tested pension

{\bf 2) Heterogeneity in $w$}: Forced saving should be
proportional to $w$ (as long as govt does not care about
redistribution)

%{\bf 2) Adding labor Supply Responses:}
%
%$u(c_1)-h(l_1)+\delta u(c_2)$ with $c_1=(1-\tau)wl_1-s$ and
%$c_2=(1+r)(s+\tau w l_1)$ $\Rightarrow$ $c_1+c_2/(1+r)=wl_1$
%$\Rightarrow$
%
%a) $l_1$ of the rational individuals not affected [as benefits are
%{\bf actuarially fair}]
%
%b) $l_1$ of myopic is distorted downward: $\max_{l_1}
%u((1-\tau)wl_1)-h(l_1)$ as they perceive the tax but not the
%future benefits

\end{slide}



%\begin{slide}
%\begin{center}
%{\bf Does Social Security Smooth Consumption?}
%\end{center}
%
%All that Social Security may be doing is crowding out the savings that individuals would otherwise set aside for their retirement.
%
%Social Security might crowd out private savings by allowing people to count on a government transfer to support their income in old age. The larger this crowd-out is, the less consumption smoothing Social Security provides for retired individuals.
%\end{slide}


\begin{slide}
\begin{center}
{\bf Crowd-Out Effect of Social Security on Savings}
\end{center}

The effect of Social Security on private savings has been the subject of a large number of studies over the past 30 years

To measure the impact of Social Security on savings, there must be a way to compare people with different levels of Social Security benefits who are otherwise identical

\small
In the United States, Social Security is a national program that applies to almost all workers; very similar people usually have very similar benefits. Recent studies have provided evidence on the impact of Social Security-like programs on private savings in Italy.

Italian Reforms in 1992 substantially reduced the benefits, and thus future SSW, for younger workers in the public sector, while reducing much less the benefits of older workers and those in the private sector.

Studies estimate that about 30--40\% of the reduction in SSW was offset by higher private savings.
\end{slide}




%13.3 Social Security and Retirement

\begin{slide}
\begin{center}
{\bf Evidence for Myopia and adequate savings}
\end{center}

1) Diamond JpubE 1977: old age poverty has fallen as Social Security expanded.
Poverty for other groups has not fallen
nearly as much

2) Fall in consumption {\bf during} retirement: Hamermesh (1984)
shows that consumption falls by 5\% per year for the elderly
[consumption is not smooth but not necessarily suboptimal]

3) Fall in consumption {\bf at} retirement: Bernheim, Skinner,
Weinberg (2001) show that drop in consumption is significant for all groups
except the wealthiest [consistent with
myopia]

\end{slide}

\begin{slide}
\includepdf[pages={7}]{socialsecurity_ch13_new_attach.pdf}
\end{slide}



\begin{slide}
\includepdf[pages={8}]{socialsecurity_ch13_new_attach.pdf}
\end{slide}



%\begin{slide}
%\begin{center}
%{\bf Consumption drop at retirement: Aguiar-Hurst JPE05}
%\end{center}
%
%Starting point: Empirically, consumption falls with
%retirement...but studies use expenditures as measure of consumption
%
%Aguiar-Hurst JPE05 show that it is important to differentiate
%between consumption and expenditures. Further, the paper provides
%new information on the complementarity of consumption and leisure
%after retirement.
%
%1) Confirm that expenditure on food falls by 17\% at retirement but
%
%2) time spent on home production rises by 60\%
%
%3) all measures of caloric intake, vitamin intake, meat quality, etc. do not drop at retirement
%(find that caloric intake falls when getting unemployed, hard to believe but suggestive)
%\end{slide}
%
%\begin{slide}
%\includepdf[pages={9}]{socialsecurity_ch13_new_attach.pdf}
%\end{slide}



\begin{slide}
\begin{center}
{\bf SOCIAL SECURITY AND RETIREMENT: THEORY}
\end{center}

If a 62-year-old worker works until 63, instead of retiring at 62 and claiming her Social Security benefits, three things happen through the Social Security system:

1) She pays an extra year of payroll taxes on her earnings.

2) She receives one year less of Social Security benefits.

3) She gets a higher Social Security benefit level through the actuarial adjustment (8\% extra permanently per year of delay)

Adjustment is called {\bf actuarially fair} if those 3 effects cancel out in PDV (US system has been reformed to be close to fair on average)
\end{slide}


\begin{slide}
\begin{center}
{\bf SOCIAL SECURITY AND RETIREMENT: THEORY}
\end{center}

Three key elements of a social security system may affect retirement behavior:

1) Availability of benefits at \textbf{Early Retirement Age} (EEA): (62 in US)

\small
Those effects arise because of myopia or lack of information [a rational individual is
not affected by EEA because he/she can use own savings while retired till he/she reaches age 62]

\normalsize
2) Non-actuarially fair adjustments of benefits for those retiring after the EEA:

\small
If benefits are not adjusted in a fair way, they can create a huge implicit tax on work (US used
to have very little adjustment)

\normalsize
3) Social norm created by retirement benefits: govt calling some age the ``Normal Retirement Age'' (NRA) can affect decisions in spite of no underlying economic incentives (see Seibold '21 for such effects in Germany)

\end{slide}



\begin{slide}
\includepdf[pages={11-12, 14}]{socialsecurity_ch13_new_attach.pdf}
\end{slide}


%\begin{slide}
%\begin{center}
%{\bf Focal points and social norms}
%\end{center}
%SS programs have a normal retirement age (NRA) but such NRAs are not always associated with real economic incentives (e.g. US SS is actuarially fair so NRA should be irrelevant)
%
%Siebold '17 shows that in Germany, 30\% of workers retire at statutory retirement ages even with no specific underlying economic incentives and not option by default
%
%Siebold '17 shows that statutory age bunching much larger than bunching around kinks of lifetime budget constraints created by retirement system
%
%$\Rightarrow$ Cannot be explained within standard model
%
%$\Rightarrow$ NRA perceived as a social injunction obeyed by workers
%
%$\Rightarrow$ Nominal NRAs can potentially be a powerful govt tool to change the retirement
%age
%
%
%\end{slide}



\begin{slide}
\begin{center}
{\bf Social Security and Retirement: Implications}
\end{center}

Evidence suggests that it is potentially very costly to design Social Security systems that
allow very early retirement and/or penalize additional work beyond the retirement age.

Adjusting systems to more fairly reward work at old ages can increase labor supply of elderly

It seems better to have an early retirement age that is not too low and provide disability benefits to those who truly cannot work and haven't yet reached the early retirement age


``Normal retirement age'' labelling can also have an impact through social norms and focal points (as in Germany as shown in Seibold '21)


\end{slide}


\begin{slide}
\includepdf[pages={39}]{socialsecurity_ch13_new_attach.pdf}
\end{slide}

%Social Security Reform



\begin{slide}
\begin{center}
{\bf Social Security Reform: Problems with Current System}
\end{center}

Rate of return $n+g$ has declined from over 3\% to about 2\% due
to:

1) $n$: Retirement of baby boom large cohorts born 1945-1965:
%3.3 workers/beneficiary, 2030: 2 workers/beneficiaries

2) Increase in life expectancy at retirement age

\small
Note: top half of individuals  (in terms
of lifetime earnings)
has seen large life expectancy gains while bottom half life expectancy
has stagnated in recent decades

\normalsize

1)+2) imply number of elderly per working age person increases from .15
in 1960 to .35 in 2030

3) $g$: Slower productivity growth since 1975 (from 2\% to 1\%)

System requires adjusting taxes or benefits to remain in balance

\end{slide}

\begin{slide}
\includepdf[pages={17}]{socialsecurity_ch13_new_attach.pdf}
\end{slide}

\begin{slide}
\begin{center}
{\bf 1983 GREENSPAN COMMISSION}
\end{center}

Demographic changes are predictable, so 1st reform was implemented
in 1983 (designed to solve budget problems over next 75 years)

1) Increased payroll taxes to build a trust-fund

2) Increased retirement age in the future (from age 65 to 67)

Trust fund invested in Treasury Bills (Fed gov debt):
\[ TF_{t+1}=TF_{t} \cdot (1+i)+SSTax_t-SSBen_t \]

Trust fund peaked at \$2.8T in 2013 and will be exhausted by
2034, taxes will then cover about 75\%
of promised benefits

%Requires additional adjustment: can fix it for next 75 years by increasing payroll tax rate now by 1.7 percentage points or wait till 2035 and then increase tax by 3.5 pp (not huge)

Requires additional adjustment: increasing payroll tax rate now by 1.7 percentage points or wait
till 2035 and then increase tax by 3.5 pp (from 12.4\% to 15.9\%, not huge)

\end{slide}


\begin{slide}
\includepdf[pages={18}]{socialsecurity_ch13_new_attach.pdf}
\end{slide}


\begin{slide}
\begin{center}
{\bf Social Security Small Reform Options}
\end{center}

1) Increase contributions: increase tax rate or earnings cap

2) Reduce benefits: straight cut not politically feasible: a) Index retirement age to
life expectancy, b) Index benefits to chained-CPI instead of
CPI after retirement, c) Make benefits fully taxable for income tax



3) Means-tested benefits: bad for savings incentives and could
make program politically unstable [a program for the poor is a
poor program]. Explains conservatives support.

%4) Social security privatization: create individual accounts (like a forced 401k for all).


Key issue is distributional: low income earners have seen income and life expectancy stagnate
but they have increased for high income earners 
%Issue: Socialism (or lobbying and
%corruption in investment choices). Could leave investment
%choices to independent board

%5) Major reform: privatization

\end{slide}

\begin{slide}
\begin{center}
{\bf Major reform: Social Security Privatization}
\end{center}

Two components:

1) Funding the system

2) Create individual accounts (like private employer 401k pensions)

current benefits = past contributions + market return

Controversial academic and policy debate, a number of countries have privatized
their social security systems (Chile, Mexico, UK)

Main proponent: Feldstein (Harvard), Main critic: Diamond (MIT). Bush attempted reform in 2005
but it went nowhere

Pro argument: get higher return on contributions $r>n+g$, increase capital stock and future wages.

\end{slide}

\begin{slide}
\begin{center}
{\bf SOCIAL SECURITY PRIVATIZATION ACCOUNTING}
\end{center}
Exactly the reverse of pay-as-you-go calculations:

1) First generation loses as they need to fund current retirees
and own contributions. All future generations gain
[generational redistribution]

2) If govt increases debt to pay for current retirees: future
generations get higher return on contributions but need to re-pay
higher govt debt $\Rightarrow$ Complete wash for all generations

%tax to pay debt interest = returns on funded contributions -
%older returns on  contributions

$\Rightarrow$ Only way funding generates real changes is by
hurting some transitional generations which have to double pay

\end{slide}

\begin{slide}
\begin{center}
{\bf ADDITIONAL PRIVATIZATION ISSUES}
\end{center}

1) Risk: individuals bear investment risk (stock market fluctuates
too much relative to economy) and cannot count on defined level of benefits
[Privatization needs to include minimum pension provision]

2) Annuitization: hard to impose in privatized system because of
political constraints [hard to force sick person to annuitize her wealth]
$\Rightarrow$ Some people will exhaust benefits before death and
be poor in very old age [looming problem with 401(k)s]

3) Lack of financial literacy: Individuals do not know how to
invest. Complicated choice, govt
can do it for people more efficiently

4) Administrative costs: privatized systems (Chile, UK) admin
costs very high (1\% of assets = 10 times more than Social Security) due to wasteful advertisement
by mutual funds because of lack of financial literacy

\end{slide}


\begin{slide}
\begin{center}
{\bf Evidence on Lack of Financial Literacy }
\end{center}
401(k) private pensions in the US offer strong evidence of lack of financial literacy

\small

1) Default effects: opt-in vs. opt-out have enormous effects on 401(k) enrollment
[Madrian and Shea QJE'01] %and overall savings [Chetty et al. QJE'14]

2) $1/N$ investment choices of 401(k) contributions: many people invest contributions
by dividing them equally into investment options (regardless of the options)

3) People often invest 401(k) in company stock which is extremely risky (Enron).
Strong evidence of default effects in investment choices as well

4) Evidence that financial education and advice has impacts on savings decisions
(Thaler and Benartzi JPE '04: Saving More Tomorrow experiment).

\normalsize
$\Rightarrow$ Much better to force people to save via mandatory social security system
than rely on individual rationality

\end{slide}

\begin{slide}
\includepdf[pages={19-22}]{socialsecurity_ch13_new_attach.pdf}
\end{slide}

\begin{slide}
\begin{center}
{\bf CONCLUSION}
\end{center}

Social Security is the largest social insurance program in the United States, and the largest single expenditure item of the federal government

Key reason for existence of social security programs is the inability of individuals to save adequately for retirement
on their own (individual failure)

Social Security faces a long-run financing problem requiring to increase taxes or cut benefits in the long-run

The question of how to resolve this problem will be one of the most contentious sources of political debate for at least the first part of the twenty-first century
\end{slide}

\begin{slide}
\begin{center}
{\bf REFERENCES}
\end{center}
{\small

Jonathan Gruber, Public Finance and Public Policy, Fifth Edition, 2019 Worth Publishers, Chapter 13

Aguiar, M. and E. Hurst ``Consumption vs Expenditure.'', Journal of Political Economy, Vol. 113, 2005, 919-948.\href{http://www.jstor.org/stable/pdfplus/10.1086/491590.pdf.pdf}{(web)}

Bernheim, B. Douglas, Jonathan Skinner, and Steven Weinberg. ``What accounts for the variation in retirement wealth among US households?.'' American Economic Review (2001): 832-857.\href{http://www.jstor.org/stable/pdfplus/2677815.pdf}{(web)}

Bosworth, Barry, Gary Burtless, and Kan Zhang. 2016. ``What growing life expectancy gaps mean for the promise of social security.'' Economic Studies at Brookings. \href{http://elsa.berkeley.edu/~saez/course131/bosworthatal16longevity.pdf}{(web)}

Choi, James, David Laibson, and Brigitte Madrian. ``The flypaper effect in individual investor asset allocation.'' (2007).\href{http://elsa.berkeley.edu/~saez/course131/Choi-Laibson-Madrian07.pdf}{(web)}

Diamond, P. ``National Debt in a Neoclassical Growth Model'', American Economic Review, Vol. 55, 1965, 1126-1150. \href{http://links.jstor.org/stable/pdfplus/1809231.pdf} {(web)}

Diamond, Peter A. ``A framework for social security analysis.'' Journal of Public Economics 8.3 (1977): 275-298.\href{http://elsa.berkeley.edu/~saez/course131/Diamond77.pdf}{(web)}

Gelber, Alex, Damon Jones, and Dan Sacks
``Estimating Earnings Adjustment Frictions: Method and Evidence from the Earnings Test'', 
AEJ: Applied Economics, forthcoming 2019 
\href{http://elsa.berkeley.edu/~saez/course/gelber-jones-sacks19.pdf} {(web)} 

Gruber, Jonathan, and David A. Wise, eds. ``Social security and retirement around the world.''[Book] University of Chicago Press, 2008.\href{http://papers.nber.org/books/grub99-1}{(web)}

Hamermesh, Daniel S. ``Consumption during retirement: the missing link in the life cycle.'' The Review of Economics and Statistics 66.1 (1984): 1-7.\href{http://www.jstor.org/stable/pdfplus/1924689.pdf}{(web)}

Madrian, Brigitte C., and Dennis F. Shea. ``The power of suggestion: Inertia in 401 (k) participation and savings behavior.'' Quarterly Journal of Economics 116.4 (2001): 1149-1187.\href{http://www.jstor.org/stable/pdfplus/2696456.pdf}{(web)}

Saez, Emmanuel  ``Public Economics and Inequality: Uncovering Our Social Nature'', AEA Papers and Proceedings, 121, 2021, 1-27
\href{https://eml.berkeley.edu/~saez/saez-AEAlecture.pdf} {(web)} 

Seibold, Arthur. 2021 ``Reference Points for Retirement Behavior: Evidence from German
Pension Discontinuities'', American Economic Review forthcoming  \href{http://elsa.berkeley.edu/~saez/course/seiboldAER21.pdf} {(web)} 

Thaler, Richard H., and Shlomo Benartzi. ``Save more tomorrow: Using behavioral economics to increase employee saving.'' Journal of Political Economy 112.S1 (2004): S164-S187.\href{http://www.jstor.org/stable/pdfplus/10.1086/380085.pdf}{(web)}


}

\end{slide}



\end{document}

