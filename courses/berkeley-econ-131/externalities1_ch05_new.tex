\documentclass[landscape]{slides}

\usepackage[landscape]{geometry}

\usepackage{pdfpages}

\usepackage{hyperref}
\usepackage{amsmath}

\def\mathbi#1{\textbf{\em #1}}

\topmargin=-1.8cm \textheight=17cm \oddsidemargin=0cm
\evensidemargin=0cm \textwidth=22cm


\author{131 Undergraduate Public Economics \\ Emmanuel Saez \\ UC Berkeley}
\date{}




\title{Externalities} \onlyslides{1-300}

\newenvironment{outline}{\renewcommand{\itemsep}{}}

\begin{document}

\begin{slide}
\maketitle
\end{slide}

\begin{slide}
\begin{center}
{\bf OUTLINE}
\end{center}

Second part of course is going to cover market failures and show how government interventions can help

1) Externalities and public goods

2) Asymmetric information (social insurance)

3) Individual failures (savings for retirement)

\end{slide}

%\begin{slide}
%\begin{center}
%{\bf EXTERNALITIES: PROBLEMS AND SOLUTIONS}
%\end{center}
%
%In December 1997, representatives from over 170 nations met in Kyoto, Japan, to attempt one of the most ambitious international negotiations ever: an international pact to limit the emissions of carbon dioxide worldwide because of global warming. The nations faced a daunting task.
%
%The cost of reducing the use of fossil fuels is large: Replacing these fossil fuels with alternatives would significantly raise the costs of living
%
%Not curbing carbon emissions will have major impacts on the climate and hence large economic costs down the road
%\end{slide}

%\begin{slide}
%\includepdf[pages={1}]{externalities1_ch05_new_attach.pdf}
%\end{slide}

\begin{slide}
\begin{center}
{\bf EXTERNALITIES}
\end{center}

{\bf Market failure}:
A problem that violates one of the assumptions of the 1st welfare theorem and
 causes the market economy to deliver an outcome that does not maximize efficiency

{\bf Externality}:
Externalities arise whenever the actions of one economic agent \textbf{directly} affect another economic agent
outside the market mechanism

Externality example: a steel plant that pollutes a river used for recreation

Not an externality example: a steel plant uses more electricity and bids up the price of electricity
for other electricity customers

Externalities are one important case of market failure

\end{slide}

%5.1 Externality Theory

\begin{slide}
\begin{center}
%how to add the title?
{\bf EXTERNALITY THEORY: ECONOMICS OF NEGATIVE PRODUCTION EXTERNALITIES}
\end{center}

{\bf Negative production externality}:
When a firm's production reduces the well-being of others who are not compensated by the firm.

{\bf Private marginal cost (PMC)}:
The direct cost to producers of producing an additional unit of a good

{\bf Marginal Damage (MD)}:
Any additional costs associated with the production of the good that are imposed on others
but that producers do not pay

{\bf Social marginal cost (SMC = PMC + MD)}:
The private marginal cost to producers plus marginal damage

Example: steel plant pollutes a river but plant does not face any pollution regulation (and hence
ignores pollution when deciding how much to produce)

\end{slide}

\begin{slide}
\includepdf[pages={2}]{externalities1_ch05_new_attach.pdf}
\end{slide}



\begin{slide}
\begin{center}
{\bf EXTERNALITY THEORY: ECONOMICS OF NEGATIVE CONSUMPTION EXTERNALITIES}
\end{center}
{\bf Negative consumption externality}:
When an individual's consumption reduces the well-being of others who are not compensated by the individual.


{\bf Private marginal benefit (PMB)}:
The direct benefit to consumers of consuming an additional unit of a good by the consumer.

{\bf Social marginal benefit (SMB)}:
The private marginal benefit to  consumers plus any costs associated with the consumption
of the good that are imposed on others

Example: Using a car and emitting carbon contributing to global warming
\end{slide}


\begin{slide}
\includepdf[pages={4}]{externalities1_ch05_new_attach.pdf}
\end{slide}

\begin{slide}
\begin{center}
{\bf Externality Theory: Positive Externalities}
\end{center}

{\bf Positive production externality}:
When a firm's production increases the well-being of others but the firm is not compensated by those others.

Example: Beehives of honey producers have a positive impact on pollination and agricultural output

{\bf Positive consumption externality}:
When an individual's consumption increases the well-being of others but the individual is not compensated by those others.

Example: Beautiful private garden that passers-by enjoy seeing

\end{slide}

\begin{slide}
\includepdf[pages={5}]{externalities1_ch05_new_attach.pdf}
\end{slide}


\begin{slide}
\begin{center}
{\bf Externality Theory: Market Outcome is Inefficient}
\end{center}

With a free market, quantity and price are such that $PMB=PMC$

Social optimum is such that $SMB=SMC$

$\Rightarrow$ Private market leads to an inefficient outcome (1st welfare theorem does not work)

Negative production externalities lead to over production

Positive production externalities lead to under production

Negative consumption externalities lead to over consumption

Positive consumption externalities lead to under consumption


\end{slide}


%\begin{slide}
%\begin{center}
%{\bf EXTERNALITY THEORY: GRAPHICAL ANALYSIS}
%\end{center}
%
%One aspect of the graphical analysis of externalities is knowing which curve to shift, and in which direction. There are four possibilities:
%
%{\small
%\begin{itemize}
%\item{Negative production externality: SMC curve lies above PMC curve}
%\item{Positive production externality: SMC curve lies below PMC curve}
%\item{Negative consumption externality: SMB curve lies below PMB curve}
%\item{Positive consumption externality: SMB curve lies above PMB curve}
%\end{itemize} }
%
%
%The key is to assess which category a particular example fits into. First, you must assess whether the externality is associated with producing a good or with consuming a good. Then, you must assess whether the externality is positive or negative.
%\end{slide}

%5.2 Private-Sector Solutions

\begin{slide}
\begin{center}
{\bf Private-Sector Solutions to Negative Externalities}
\end{center}

Key question raised by Ronald Coase (famous Nobel Prize winner Chicago libertarian economist):

Are externalities really outside the market mechanism?

{\bf Internalizing the externality}:
When either private negotiations or government action lead the price to the party to fully reflect
the external costs or benefits of that party's actions.
\end{slide}


\begin{slide}
\begin{center}
{\bf PRIVATE-SECTOR SOLUTIONS TO NEGATIVE EXTERNALITIES: COASE THEOREM}
\end{center}

{\bf Coase Theorem (Part \textrm{I})}:
When there are well-defined property rights and costless bargaining, then negotiations between the party creating the externality and the party affected by the externality can bring about the socially optimal market quantity.

{\bf Coase Theorem (Part \textrm{II})}:
The efficient quantity for a good producing an externality does not depend on which party is assigned the property rights, as long as someone is assigned those rights.
\end{slide}

\begin{slide}
\begin{center}
{\bf COASE THEOREM EXAMPLE}
\end{center}
Firms pollute a river enjoyed by swimmers. If firms ignore swimmers, there is too much pollution

\textbf{1) Swimmers own river:} If river is owned by swimmers then swimmers can charge firms for polluting the river.
They will charge firms the marginal damage (MD) per unit of pollution.

\small
Why price pollution at MD? If price is above MD, swimmers would want to sell an extra unit of pollution and get hit by pollution damage MD, so price must fall.  MD is the equilibrium efficient price in the newly created pollution market.

\normalsize

\textbf{2) Firms own river:} If river is owned by firms then firms can charge swimmers in exchange of
polluting less. They will also charge swimmers the MD per unit of pollution reduction.

Final level of pollution will be the same in 1) and 2)

\end{slide}

\begin{slide}
\includepdf[pages={6}]{externalities1_ch05_new_attach.pdf}
\end{slide}




\begin{slide}
\begin{center}
{\bf PROBLEMS WITH COASIAN SOLUTION}
\end{center}

In practice, the Coase theorem is unlikely to solve many of the types of externalities that cause market failures.

{\bf 1) The assignment problem}: In cases where externalities affect many agents (e.g. global warming),
assigning property rights is difficult 

\small
$\Rightarrow$ Coasian solutions are likely to be more effective for small, localized externalities (water wells in Southern California, Ostrom 1990 ) than for larger, more global externalities involving large number of people and firms
\normalsize

%{\bf 2) The holdout problem}:
%Shared ownership of property rights gives each owner power over all the others (because joint owners have
%to all agree to the Coasian solution)
%
%As with the assignment problem, the holdout problem would be amplified with an externality involving many parties.
%\end{slide}
%
%\begin{slide}
%\begin{center}
%{\bf PROBLEMS WITH COASIAN SOLUTION}
%\end{center}

%{\bf 3) The Free Rider Problem}:
%When an investment has a personal cost but a common benefit, individuals will underinvest
%(example: a single country is better off walking out of Kyoto protocol for carbon emission controls)

{\bf 2) Transaction Costs and Negotiating Problems}:
The Coasian approach ignores the fundamental problem that it is hard to negotiate when there are large numbers of individuals on one or both sides of the negotiation.

\small This problem is amplified for an externality such as global warming, where the potentially divergent interests of billions of parties on one side must be somehow aggregated for a negotiation.
\end{slide}

\begin{slide}
\begin{center}
{\bf PROBLEMS WITH COASIAN SOLUTION: BOTTOM LINE}
\end{center}
Ronald Coase's insight that externalities can sometimes be internalized was useful.

It provides the competitive market model with a defense against the onslaught of market failures.

It is also an excellent reason to suspect that the market may be able to internalize some small-scale, localized externalities.

It won't help with large-scale, global externalities, where only a ``government'' can successfully
aggregate the interests of all individuals suffering from externality

\end{slide}

%5.3 Public-Sector Remedies

\begin{slide}
\begin{center}
{\bf Public Sector Remedies For Externalities}%title
\end{center}
%The Environmental Protection Agency (EPA) was formed in 1970 to provide public-sector solutions to the problems of externalities in the environment.

Public policy makers employ two types of remedies to resolve the problems associated with negative externalities:

\textbf{1) quantity regulation:} government limits use of externality producing chemicals.
Example CFCs [chlorofluorocarbons] that deplete ozone layer banned in 1990s

\textbf{2) corrective taxation:} corrective tax or subsidy equal to marginal damage per unit.
Example: Carbon tax to fight global warming due to CO2 emissions

1) and 2) can be combined with \textbf{tradable emissions permits} to firms that can then be traded (cap-and-trade for carbon
emissions)

Key advantage of price policy or tradable permits: price of emissions is the same for all which is efficient


\end{slide}

\begin{slide}
\includepdf[pages={7}]{externalities1_ch05_new_attach.pdf}
\end{slide}


%\begin{slide}
%\includepdf[pages={8}]{externalities1_ch05_new_attach.pdf}
%\end{slide}

%\begin{slide}
%\begin{center}
%{\bf PUBLIC SECTOR REMEDIES FOR EXTERNALITIES: REGULATION}
%\end{center}
%
%In an ideal world, Pigouvian taxation and quantity regulation would be identical
%
%Quantity regulation seems more straightforward, hence, it has been the traditional choice for addressing environmental externalities 
%
%In practice, there are complications that may make taxes a more effective means of addressing externalities.
%\end{slide}
%
%%5.4 Distinctions between price and quantity approaches
%
%\begin{slide}
%\includepdf[pages={9}]{externalities1_ch05_new_attach.pdf}
%\end{slide}
%
%\begin{slide}
%\begin{center}
%{\bf MODEL WITH HETEROGENEOUS COSTS}
%\end{center}
%Assume MD of pollution is \$1 per unit of pollution
%
%2 firms with low ($L$) or high ($H$) cost of pollution reduction $q$:
%\[ c_H(q)=1.5 q^2 \Rightarrow MC_H(q)=c'_H(q)=3q \]
%\[c_L(q)=.75 q^2 \Rightarrow  MC_L(q)=c'_L(q)=1.5q \]
%With no taxes, no regulations, firms do $q_L=q_H=0$
%
%Social welfare maximization:
%\[ V=\max_{q_H,q_L} q^H+q^L - c_H(q^H)-c_L(q^L)  \Rightarrow \]
%\[ MC_H = 1, MC_L=1 \Rightarrow q^H=1/3, q^L=2/3 \]
%Optimum outcome is to have the low cost firm do more pollution reduction than the high
%cost firm
%
%\end{slide}
%
%\begin{slide}
%\begin{center}
%{\bf TAX VERSUS REGULATION SOLUTION}
%\end{center}
%Socially optimal outcome can be achieved by \$1 tax per unit of pollution (same tax across
%firms):
%
%Firm $H$ chooses $q_H$ to maximize $q^H - c_H(q^H) \Rightarrow MC_H=1$
%
%Firm $L$ chooses $q_L$ to maximize $q^L - c_L(q^L) \Rightarrow MC_L=1$
%
%Uniform quantity regulation $q^H=q^L=1/2$ is not efficient because firm $H$ has higher
%$MC$ of polluting than firm $L$:
%
%Proof: Firm $H$ would be happy to pay firm $L$ to reduce
%$q^L$ and increase $q^H$ to keep $q^L+q^H=1$, firm $L$ is happier and society has
%same level of pollution
%
%\end{slide}
%
%\begin{slide}
%\begin{center}
%{\bf Quantity Regulation with Trading Permits}
%\end{center}
%
%Suppose start with quantity regulation $q_0^H=q_0^L=1/2$ and allow firms to trade
%pollution reductions as long as $q^H+q^L=1$
%
%Generates a market for pollution reduction at price $p$
%
%Firm $H$ maximizes $pq^H-c_H(q^H)$ $\Rightarrow$ $MC_H=p$ and $q^H=p/3$
%
%Firm $L$ maximizes $pq^L-c_L(q^L)$ $\Rightarrow$ $MC_L=p$ and $q^L=2p/3$
%
%$\Rightarrow$ $q^H+q^L=p$. As $1=q_0^L+q_0^H=q^H+q^L$, in equilibrium $p=1$ and hence
%$q_H=1/3$ and $q_L=2/3$
%
%Final outcome $q_H,q_L$ does not depend on initial regulation $q_0^H,q_0^L$
%
%Quantity regulation with tradable permits is efficient as long as total quantity $q_0^L+q_0^H=1$
%
%
%\end{slide}
%
%
%
%
%%\begin{slide}
%%\includepdf[pages={10}]{externalities1_ch05_new_attach.pdf}
%%\end{slide}
%
%
%\begin{slide}
%\begin{center}
%{\bf MULTIPLE PLANTS WITH DIFFERENT REDUCTION COSTS}
%\end{center}
%
%Policy Option 1: Quantity Regulation (not efficient unless quantity can be based on actual reduction cost for each firm)
%
%Policy Option 2: Price Regulation Through a Corrective Tax (efficient)
%
%Policy Option 3: Quantity Regulation with Tradable Permits (efficient)
%
%\end{slide}
%

\begin{slide}
\begin{center}
{\bf CORRECTIVE TAXES VS. TRADABLE PERMITS}
\end{center}
Two differences between corrective taxes and tradable permits (carbon tax vs. cap-and-trade
in the case of CO2 emissions)

\textbf{1) Initial allocation of permits:}
If the government sells them to firms, this is equivalent to the tax

If the government gives them to current firms for free, this is like the
tax + large transfer to initial polluting firms.


\textbf{2) Uncertainty in marginal costs:} With uncertainty in costs of reducing pollution,
tax cannot target a specific quantity while tradable permits can
$\Rightarrow$ two policies no longer equivalent.

Taxes preferable when MD curve is flat.
Tradable permits are preferable when MD curve
is steep.

%\textbf{3) Uncertainty in price of permits:} Tax is stable and predictable for businesses but pollution qty is
%not.
%Permits

\end{slide}

\begin{slide}
\includepdf[pages={11-12}]{externalities1_ch05_new_attach.pdf}
\end{slide}


%\begin{slide}
%\begin{center}
%{\bf CONCLUSION}
%\end{center}
%
%Externalities are the classic answer to the ``when'' question of public finance: when one party's actions affect another party, and the first party doesn't fully compensate (or get compensated by) the other for this effect, then the market has failed and government intervention is potentially justified.
%
%This naturally leads to the ``how'' question of public finance. There are two classes of tools in the government's arsenal for dealing with externalities: price-based measures (taxes and subsidies) and quantity-based measures (regulation).
%
%Which of these methods will lead to the most efficient regulatory outcome depends on factors such as the heterogeneity of the firms being regulated, the flexibility embedded in quantity regulation, and the uncertainty over the costs of externality reduction.
%\end{slide}


\begin{slide}
\begin{center}
{\bf Empirical Example: Acid Rain and Health}
\end{center}
Acid rain due to contamination by emissions of sulfur dioxide ($SO_2$) and nitrogen oxide ($NO_x$).

{\bf 1970 Clean Air Act}:
Landmark federal legislation that first regulated acid rain-causing emissions by setting maximum standards for atmospheric concentrations of various substances, including $SO_2$.

{\bf The 1990 Amendments and Emissions Trading:}

$SO_2$ allowance system:
The feature of the 1990 amendments to the Clean Air Act that granted plants permits to emit $SO_2$ in limited quantities and allowed them to trade those permits.
\end{slide}

\begin{slide}
\begin{center}
{\bf Empirical Example: Effects of Clean Air Act of 1970}
\end{center}
How does acid rain (or SO2) affect health?

Observational approach: relate mortality in a geographical area to the level of particulates (such as SO2) in the air

Problem: Areas with more particulates may differ from areas with fewer particulates in many other ways, not just in the amount of particulates in the air

Chay and Greenstone (2003) use clean air act of 1970 to resolve the causality problem:

Areas with more particulates than threshold required to clean up air [treatment group]. Areas with less particulates
than threshold are control group.

Compares infant mortality across 2 types of places before and after (DD approach)

\end{slide}

\begin{slide}
\includepdf[pages={14,15}]{externalities1_ch05_new_attach.pdf}
\end{slide}

\begin{slide}
\begin{center}
{\bf Climate Change and CO2 Emissions}
\end{center}
Industrialization has dramatically increased CO2 emissions
and atmospheric CO2 generates global warming

Four factors make this challenging (Wagner-Weitzman 2015): 
%global, irreversible, long-term, uncertain

\textbf{1) Global:} Emissions in one country affect the full world

\textbf{2) Irreversible:} Atmospheric CO2 has long life (centuries) [absent carbon capture tech breakthrough]

\textbf{3) Long-term:} Costs of global warming are decades/centuries away [how should this be discounted?]

\textbf{4) Uncertain:} Great uncertainty in costs of global warming [mitigation vs. amplifying feedback loops]

How fast should we start reducing emissions? [Stern-Weitzman want a fast reduction, Nordhaus advocates a 
slower path]

\end{slide}

\begin{slide}
\includepdf[pages={1}, scale=1.13]{externalities1_ch05_new_attach.pdf}
\end{slide}



\begin{slide}
\begin{center}
{\bf Main costs of global warming}
\end{center}
Enormous variation across geographical areas and economic development. Pace of change
makes adaptation daunting

1) Sea rise will flood low lying coasts and major population centers in many countries (e.g., Miami, Florida; value of
real estate subject to regular flooding has dropped)

2) Impact on bio-diversity (mass extinctions)

3) Agricultural production could be disrupted by climate change and the increased weather variability it generates: 

\small demand for food
is very inelastic in the short-run $\Rightarrow$ Spikes in prices if agricultural output falls $\Rightarrow$ disruption/famines possible in low income countries

\normalsize

4) Droughts and heat waves will make many places less livable 

\small Some societies may collapse and generate mass migration movements

\end{slide}




\begin{slide}
\begin{center}
{\bf Empirical Example: Costs of Global Warming}
\end{center}

Estimating costs of Global warming is daunting because society will adapt and
reduce costs (relative to a scenario with no adaptation)

Example: heat waves and mortality analysis of Barreca et al. (2016)

1) The mortality effect of an extremely hot day ($80^o$F+) declined by about 75\% between 1900-1959 and 1960-2004.

2) Adoption of residential air conditioning (AC) explains the entire decline

3) Worldwide adoption of AC will speed up the rate of climate change (if fossil fuel powered)

\end{slide}

\begin{slide}
\includepdf[pages={19}, scale = 1]{externalities1_ch05_new_attach.pdf}
\end{slide}




%\begin{slide}
%\begin{center}
%{\bf CURBING GLOBAL WARMING:  KYOTO TREATY}
%\end{center}
%
%Kyoto 1997: 35 industrialized nations (but not US) agreed to reduce their emissions of greenhouse gases to 5\% below (depends on country) 1990 levels by the year 2012
%
%Industrialized countries are allowed to trade emissions rights {\bf among themselves}, as long as the total emissions goals are met [=quantity regulation with trading permits]
%
%Developing countries are not in the treaty even though it is cheaper to use fuel efficiently as you develop an industrial base than it is to ``retrofit'' an existing industrial base
%
%\end{slide}


\begin{slide}
\begin{center}
{\bf Global Warming:  Economists' Narrow View}
\end{center}
Economists view global warming as a classical externality

CO2 emissions impose a global warming externality $\Rightarrow$ Solution is to impose a carbon tax
equal to the marginal damage of CO2 emissions and let market forces work their magic

E.g. see \href{https://www.econstatement.org}{recent economists' statement} in favor of carbon
tax (rebated with a fixed carbon dividend)

But what is the marginal damage of CO2? It depends greatly on how you discount the future

\normalsize
Economists use interest rate $r$ to discount future: \$1 today is worth \$$(1+r)^T$ in $T$ years
(long-distance future heavily discounted: e.g., $r=4\%$ and $T=1000$ $\Rightarrow$ $(1+r)^T=10^{17}$)

If interest rate is high (=individual humans are impatient), it is desirable to let global warming happen and societies collapse!

\end{slide}

\begin{slide}
\begin{center}
{\bf Global Warming:  Broader View}
\end{center}
Massive CO2 emissions pose existential civilizational risk (like CFC destroying vital ozone layer)

Only solution is to decarbonize and we need to do it fast (within decades not centuries)

Decarbonization is within sight: renewable electricity (solar/wind) + grid + big batteries could power
most energy needs and replace most fossil fuels

$\Rightarrow$ could be done without killing economic growth and without huge short-term disruptions (less costly
than coronavirus)

Economists' useful point: some sectors are easier to decarbonize than others (e.g. cars easier than planes)
\\ $\Rightarrow$ start decarbonizing easiest sectors first (Sachs 2020)

\end{slide}

\begin{slide}
\begin{center}
{\bf How to Decarbonize? Richer countries}
\end{center}
Must become a clear policy choice that mobilizes society 

Encourage research on renewable technologies both public and private
(King, David et al. 2015)

Plan phase out of carbon in various sectors [industrial policy] and weaken fossil fuel
industry political power (Sachs 2020)

Raising carbon tax could be one tool (but we should not bet everything on it)

Be flexible and compensate low income losers 
(to avoid yellow vests protests as in France with higher gas tax)

\small
In the US, modest Obama moves, undone by Trump

Democrats offer \textbf{Green New Deal} (economic planning and industrial policy ideas coupled with social justice vision)

Biden administration will propose a big green infrastructure deal this week

\end{slide}





\begin{slide}
\begin{center}
{\bf How to Decarbonize? Developing countries}
\end{center}
Disagreement between rich and developing countries on who should bear the cost of curbing greenhouse gas emissions

Rich countries responsible for most of historical CO2 emissions

Poor countries want to develop using the cheapest available technologies (coal power still cheaper than solar power, etc.)

Makes sense for richer countries to encourage/help poorer countries leapfrog carbon in favor of renewable energy

Carrot: R\&D on renewables in rich countries can be adopted in poorer countries, direct subsidies can help

Stick: Impose tariffs on carbon content of imported goods 






%\small Higher price on carbon emissions [through taxes or trading permits] will be needed to curb emissions and global warming. Participation of the US and large devo countries (China, India) will be needed 

\end{slide}





\begin{slide}
\begin{center}
{\bf REFERENCES}
\end{center}
{\small

Jonathan Gruber, Public Finance and Public Policy, 5th Edition, 2019 Worth Publishers, Chapters 5 and 6

Barreca, Alan, et al. ``Adapting to Climate Change: The Remarkable Decline in the US Temperature-Mortality Relationship over the 20th Century.'' Journal of Political Economy 124(1), 2016, 105-159.\href{http://elsa.berkeley.edu/~saez/course131/barrecaetalJPE16.pdf}{(web)}

Chay, K. and M. Greenstone ``Air Quality, Infant Mortality, and the Clean Air Act of 1970,''NBER Working 
Paper No. 10053, 2003.\href{http://www.nber.org/papers/w10053.pdf}{(web)}

Ellerman, A. Denny, ed. ``Markets for clean air: The US acid rain program.'' Cambridge University Press, 2000.\href{http://elsa.berkeley.edu/~saez/course131/Clean-Air00.pdf}{(web)}

Gruber, Jonathan. ``Tobacco at the crossroads: the past and future of smoking regulation in the United States.'' The Journal of Economic Perspectives 15.2 (2001): 193-212.\href{http://www.jstor.org/stable/pdfplus/2696598.pdf?&acceptTC=true&jpdConfirm=true}{(web)}

King, David et al. 2015 ``A Global Apollo Programme to Combat Climate Change'', LSE Report
\href{http://elsa.berkeley.edu/~saez/course131/apollo.pdf}{(web)}

Nordhaus, William D., and Joseph Boyer. ``Warning the World: Economic Models of Global Warming.'' MIT Press (MA), 2000.\href{http://elsa.berkeley.edu/~saez/course131/Warm-World00.pdf}{(web)}

Nordhaus, William D. ``After Kyoto: Alternative mechanisms to control global warming.'' The American Economic Review 96.2 (2006): 31-34.\href{http://www.jstor.org/stable/pdfplus/30034609.pdf?&acceptTC=true&jpdConfirm=true}{(web)}

Nordhaus, William D.  \emph{The Climate Casino: Risk, Uncertainty, and Economics for a Warming World},
Yale University Press, 2013.

Ostrom, Elinor. 1990. \emph{Governing the Commons: The Evolution of Institutions for Collective Action.} Cambridge: Cambridge University Press.

Sachs, Jeffrey. 2019. ``Getting to a Carbon-Free Economy: The urgent is attainable, and at entirely affordable cost.'', The American Prospect.
\href{https://prospect.org/greennewdeal/getting-to-a-carbon-free-economy/}{(web)}

Stern Review, 2007. \emph{The Economics of Climate Change}. Cambridge University Press.

Wagner, Gernot  and Martin L. Weitzman. \emph{Climate Shock: The Economic Consequences of a Hotter Planet.}
Princeton University Press 2015.
}

\end{slide}


\end{document}

