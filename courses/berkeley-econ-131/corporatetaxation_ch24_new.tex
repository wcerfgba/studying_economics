\documentclass[landscape]{slides}

\usepackage[landscape]{geometry}

\usepackage{pdfpages}

\usepackage{hyperref}
\usepackage{amsmath}

\def\mathbi#1{\textbf{\em #1}}

\topmargin=-1.8cm \textheight=17cm \oddsidemargin=0cm
\evensidemargin=0cm \textwidth=22cm


\author{131 Undergraduate Public Economics \\ Emmanuel Saez \\ UC Berkeley}
\date{}


\title{Corporate Taxation} \onlyslides{1-300}

\newenvironment{outline}{\renewcommand{\itemsep}{}}

\begin{document}

\begin{slide}
\maketitle
\end{slide}

%\begin{slide}
%\begin{center}
%{\bf OUTLINE}
%\end{center}
%Chapter 24
%
%
%24.1 What Are Corporations and Why Do We Tax Them?
%
%24.2 The Structure of the Corporate Tax
%
%24.3 The Incidence of the Corporate Tax
%
%24.4 The Consequences of the Corporate Tax for Investment
%
%24.5 The Consequences of the Corporate Tax for Financing
%
%24.6 Treatment of International Corporate Income
%
%24.7 Conclusion
%\end{slide}

%\begin{slide}
%\includepdf[pages={1}]{corporatetaxation_ch24_new_attach.pdf}
%\end{slide}


%\begin{slide}
%\begin{center}
%{\bf MOTIVATION}
%\end{center}
%
%To its detractors, the corporate tax is a drag on the productivity of the  corporate sector, and the reduction in the tax burden on corporations has been a boon to the economy that has led firms to increase their investment in productive assets.
%
%To its supporters, the corporate tax is a major safeguard of the overall progressivity of our tax system. By allowing the corporate tax system to erode over time, supporters of corporate taxation argue, we have enriched capitalists at the expense of other taxpayers.
%\end{slide}

\begin{slide}
\begin{center}
{\bf Basic Definitions}
\end{center}



{\bf Corporation} is a for-profit business owned by shareholders with limited liability
(if business goes bankrupt, share price drops to zero but shareholders not liable for unpaid bills/debt)



{\bf Shareholders}:
Individuals who own the stock of the company.



{\bf Ownership vs. control:} owners are shareholders. Managers (CEO and top executives) in general
do not own the company but run
the corporation on behalf of shareholders



{\bf Agency problem}:
A misalignment of the interests of the owners and the managers of a firm


{\bf Corporation objective:} Economic view is that corporations should 
maximize profits to benefit shareholders.
Corporate social responsibility view is that
corporations should also care about their workers, customers, and community

\end{slide}

%\begin{slide}
%\includepdf[pages={2-5}]{corporatetaxation_ch24_new_attach.pdf}
%\end{slide}

\begin{slide}
\begin{center}
{\bf FIRM FINANCING}
\end{center}

Firms can finance themselves through debt or through equity

{\bf Debt finance}:
The raising of funds by borrowing from lenders such as banks, or by selling corporate bonds.

{\bf Corporate bonds}:
Promises by a corporation to make periodic interest payments, as well as ultimate repayment of principal, to the bondholders (the lenders)

{\bf Equity finance}: The raising of funds by sale of ownership shares in a firm. Shareholders receive
dividends from corporation and capital gain if the share price increases

Bondholders have priority on shareholders for repayment in case of bankruptcy

\end{slide}

\begin{slide}
\begin{center}
{\bf Profits and corporate tax}
\end{center}

Corporations use capital (land, buildings, machines, equipment) and labor (workers) to transform inputs (raw materials) into outputs (goods/services produced and sold to customers).

Profits = revenues from sales - expenses (labor costs, inputs, capital depreciation, interest payments on debt)

Profits are taxed by corporate tax at 21\% (since 2018). After-tax profits can be distributed to shareholders (called payouts)
as dividends or as a share buyback (share repurchase), or retained in the corporation (retained earnings).

\small
{\bf dividend}:
The periodic payment that investors receive from the company, per share owned.

{\bf retained earnings}:
Any net profits that are kept by the company rather than paid out to debt or equity holders.

{\bf capital gain}:
The increase in the price of a share since its purchase. Retained earnings increase the value
of the corporation and hence the share price.

\end{slide}

\begin{slide}
\begin{center}
{\bf Why Do We Have a Corporate Tax?}
\end{center}
Corporations are not people but are ultimately owned by people. In principle, we want to tax people based on their economic resources but:

\small

{\bf 1) Tax collection convenience:} Historically, corporations are more convenient to tax than individuals because
they are large, visible, and have detailed accounts (for transparency for their shareholders). So taxing
corporate income (profits) was attractive

{\bf 2) Taxing foreign owners:} Corporations often have foreign owners. Countries want to tax economic
activity on their territory. E.g., consider developing country with foreign owned mineral/oil extraction companies

{\bf 3) Back-up for individual taxes:} With no corporate tax,
individuals who own shares in corporations could postpone taxes indefinitely if corporations
never pay out their earnings. Individuals could also incorporate their economic activity and
be taxed only when they take their money out
%If corporations paid out those earnings many years later, the present discounted value of the tax burden would be quite low.

{\bf 4) Taxing Pure Profits:} Some firms have market power (e.g., Microsoft) and hence earn pure profits.
Taxing pure profits does not distort behavior because firms maximize profits anyway

\end{slide}

\begin{slide}
\begin{center}
{\bf Corporate tax revenue and progressivity}
\end{center}

\textbf{1) Revenue:} Aggregate corporate tax revenue has fallen sharply since 1950s: in 2018, Fed corporate tax revenue
less than 1\% of national income (was 5\%+ in the 1950s)

\textbf{2) Progressivity:}  Corporate tax is quite progressive because corporate share ownership is concentrated at the top of
distribution (slightly less so in recent decades due to rise of pension funds which democratize share ownership).

Among billionaires, wealth is primarily in the form of corporate stock (Amazon for Bezos, Tesla for Musk, etc.)

Corporate tax was backbone of progressivity in the US in mid-20th century (tax at source of 50\% of real corporate profits)

2018 Trump tax reform cut Fed corporate tax from 35\% to 21\% and lowered revenue by almost half $\Rightarrow$ Explains why
the top 400 face a lower rate than other income groups in 2018



\end{slide}

\begin{slide}
\includepdf[pages={29, 28, 27}]{corporatetaxation_ch24_new_attach.pdf}
\end{slide}



%\begin{slide}
%\begin{center}
%{\bf THE STRUCTURE OF THE CORPORATE TAX}
%\end{center}
%
%The taxes of any corporation are:
%$$\mbox{Taxes} = [\mbox{Revenues}-\mbox{Expenses}] \times \tau -\mbox{Investment tax credit}$$
%
%{\bf Revenues}:
%These are the revenues the firm earns by selling goods and services to the market
%
%
%{\bf Expenses:} include labor costs, intermediate inputs, interest payments to debt holders, and depreciation
%of capital (wear and tear of capital goods such as machines and buildings)
%\end{slide}
%
%
%\begin{slide}
%\begin{center}
%{\bf DEPRECIATION}
%\end{center}
%
%{\bf Depreciation}:
%The rate at which capital investments lose their value over time.
%
%{\bf Economic depreciation}:
%The true deterioration in the value of capital in each period of time.
%
%{\bf Depreciation allowances}:
%The amount of money that firms can deduct from their taxes to account for capital investment depreciation.
%
%Economic depreciation cannot be measured easily so capital goods are classified in categories: 5 year (e.g.
%computers), 10 year (some machines), ..., 30 years (buildings), etc.
%
%Corporations can deduct $1/N$ of cost of capital asset each year for $N$ years
%if asset has a life of $N$ years according to classification
%
%
%\end{slide}

%\begin{slide}
%\includepdf[pages={6}]{corporatetaxation_ch24_new_attach.pdf}
%\end{slide}

%\begin{slide}
%\begin{center}
%{\bf THE STRUCTURE OF THE CORPORATE TAX}
%\end{center}
%
%{\bf Investment tax credit (ITC)}:
%A credit that allows firms to deduct a percentage of their annual qualified investment expenditures from the taxes they owe
%
%Often used as temporary measures to stimulate investment during recessions
%
%This is equivalent to accelerated depreciation
%
%An alternative to depreciation is \textbf{expensing investments} which allows to immediately subtract
%full cost of new investment in the year (most favorable for the corporation). This expensing investment is sometimes
%discussed as a reform option
%
%\end{slide}

\begin{slide}
\begin{center}
{\bf THE INCIDENCE OF THE CORPORATE TAX}
\end{center}
Theoretically, incidence depends on whether capital is \textbf{mobile} internationally and within country because
corporate tax is based on where capital is used

\small
[in contrast, individual income tax is tax
based on where individual owners reside regardless of where their capital is invested]

\textbf{1) Perfectly internationally mobile capital:} returns to capital (after corporate tax rate) in US need to be equal
to return abroad $r^*$ $\Rightarrow$ $r^{US} \cdot (1-\tau_c)=r^*$
$\Rightarrow$ net-of-tax return on US based capital not affected by $\tau_c$ $\Rightarrow$ Corporate tax
is fully borne by labor

\textbf{2) Capital not mobile internationally but fully mobile within country:} net return to corporate capital needs to equal return to non-corporate
capital (non-corporate businesses) $\Rightarrow$ All forms of capital affected by $\tau_c$ as assumed by CBO
incidence calculations

Small open country more likely to be in situation 1), while big country like US is probably still more like in situation 2).

\textbf{3) Capital not even perfectly mobile within country:} Many firms depend on local amenities [pool of workers, other firms]: Apple or Google could not costlessly move away
from Silicon Valley $\Rightarrow$ Such firms bear more of the corporate tax burden

%Unfortunately, we have little convincing empirical evidence on the incidence of corporate taxation

\end{slide}


\begin{slide}
\begin{center}
{\bf Debt vs. Equity financing}
\end{center}
For corporations, financing investment with debt instead of equity is fiscally advantageous because
interest on debt can be deducted from corporate tax base [while dividends payout to shareholders are not
deductible]

However, financing project with debt is more risky, if investment does not pay off, 
firm will be unable to pay back debt and will go bankrupt

2018 tax reform: limits for 5 years deductibility of interest and in exchange allows firms to write off (=expensing)
the value of investment immediately (instead of depreciating investment assets over the course
of asset life)

%Proposal (part of Paul Ryan 2017 tax plan): disallow deducting interest of debt for corporations 
%Problem: need transition rules as many corporations are heavily indebted 

\end{slide}


%\begin{slide}
%\includepdf[pages={7}]{corporatetaxation_ch24_new_attach.pdf}
%\end{slide}

%\begin{slide}
%\begin{center}
%{\bf EFFECTIVE CORPORATE TAX RATE}
%\end{center}
%
%{\bf Effective corporate tax rate}:
%The percentage increase in the rate of pre-tax return to capital that is necessitated by taxation.
%
%More generally, the effective corporate tax rate (ETR) is measured as:
%$$ETR=\frac{MP_k(\mbox{after tax)}-MP_k(\mbox{before tax})}{MP_k(\mbox{after tax)}}$$
%
%With depreciation and the ITC, the effective tax rate is:
%$$ETR=\frac{(t-tz-\alpha)}{(1-tz-\alpha)}$$
%
%\end{slide}
%
%\begin{slide}
%\begin{center}
%{\bf THE CONSEQUENCES OF THE CORPORATE TAX FOR INVESTMENT}
%\end{center}
%
%{\bf Negative Effective Tax Rates}
%
%With a large enough $z$ and ITC $\alpha$, the effective corporate tax rate could be negative.
%
%{\bf Policy Implications of the Impact of the Corporate Tax on Investment}
%
%For any given corporate tax rate, the tax system can be designed to offer very different incentives for investment.
%\end{slide}

%\begin{slide}
%\includepdf[pages={8,9}]{corporatetaxation_ch24_new_attach.pdf}
%\end{slide}

\begin{slide}
\begin{center}
{\bf EVIDENCE ON TAXES AND INVESTMENT}
\end{center}

There is a large literature investigating the impact of corporate taxes on corporate investment decisions.
Two effects:

Price effect: corporate tax increases the cost of investment

Income effect: corporate tax reduces cash available for investment

In principle, income effect should be zero if the corporation is not credit constrained
(= can invest as much as it wants in any profitable project)

Recent studies show significant income effects (cash flow matters), some evidence of price effects
(but mostly shifting around temporary investment tax credits) [see Zwick and Mahon AER'17]

Trump corporate tax cut does not seem to have generated a surge in investment in 2018+ 

%This sizeable elasticity suggests that corporate tax policy can be a powerful tool in determining investment and that the corporate tax is very far from a pure profits tax.
\end{slide}

\begin{slide}
\includepdf[pages={32}]{corporatetaxation_ch24_new_attach.pdf}
\end{slide}
%
%\begin{slide}
%\begin{center}
%{\bf WHY NOT ALL DEBT?}
%\end{center}
%Financing investment with debt instead of equity is fiscally advantageous as $1-\tau_{int} \geq (1-\tau_{corp})(1-\tau_{div})$ 
%
%However, financing everything with debt is risky: 
%
%If investment does not pay off, firm will be unable to pay back debt and will go bankrupt
%
%\end{slide}
%
%
%\begin{slide}
%\begin{center}
%{\bf The Consequences of the Corporate Tax for Financing}
%\end{center}
%
%{\bf The Dividend Paradox}
%
%Empirical evidence supports two different views about why firms pay dividends, as reviewed by Gordon and Dietz (2006):
%
%1.  An agency theory: investors are willing to live with the tax inefficiency of dividends to get the money out of the hands of managers who suffer from the agency problem.
%
%2.  A signaling theory: investors have imperfect information about how well a company is doing, so the managers of the firm pay dividends to signal to investors that the company is doing well.
%
%{\bf How Should Dividends Be Taxed?}
%
%An important ongoing debate in tax policy concerns the appropriate tax treatment of dividend income.
%\end{slide}

%\begin{slide}
%\includepdf[pages={12}]{corporatetaxation_ch24_new_attach.pdf}
%\end{slide}


\begin{slide}
\begin{center}
{\bf Dividend Tax Effects: Empirical Analysis}
\end{center}
Chetty and Saez QJE'05 use large dividend tax cut (for the individual income tax) from 35\% to 15\% in 2003. Key results:

1) \$50 billion increase in dividend payments per year among large publicly traded firms

2) Increase came primarily from firms where ``key players'' had a strong change in tax incentives
(firms with large executive share ownership)

3) No impact on aggregate investment levels [Yagan '15 compares corporations affected by tax cut
to pass-through businesses (S-corporations) not affected by tax cut]

\small
These results are not consistent with the traditional model

Point instead toward an ``agency model'' where executives do what is in their interest, not necessarily what is in the interest of shareholders

\end{slide}

\begin{slide}
\includepdf[pages={13, 15}]{corporatetaxation_ch24_new_attach.pdf}
\end{slide}

\begin{slide}
\includepdf[pages={18}]{corporatetaxation_ch24_new_attach.pdf}
\end{slide}

\begin{slide}
\begin{center}
{\bf CORPORATE TAX INTEGRATION}
\end{center}
Profits from corporations are taxed twice:

1) Corporate income tax on corporate profits

2) Individual income tax on corporate payout to shareholders: dividends and realized capital gains

US reduced tax on dividends in 2003 to alleviate double tax

Problem: not all corporations pay tax on corporate profits because of tax avoidance

Better way to alleviate double taxation is called \textbf{corporate tax integration}

Corporate tax becomes like a withholding pre-paid tax that is refunded when dividends
are paid out to individuals (Europe used to have such a system)

\end{slide}


\begin{slide}
\begin{center}
{\bf How State Corporate Taxes Work}
\end{center}
Most states have specific state corporate taxes (typically in the 5-10\% tax rate range, CA tax rate is 8.83\%)

Many companies operate across various states. Before mid-20th century, firms had to report where they made
their profits across states $\Rightarrow$ Easy to game 

\textbf{Formulary apportionment solution:} Since 1950s, multi-state companies apportion profits across states using formulas based
on payroll, tangible capital, and sales in each state

Most states have switched to using sales only in recent decades: Apple has 20\% of its US sales in CA,
then 20\% of Apple US corporate profits are taxed in CA

Sales only apportionment removes incentives for firms to locate production (workers+capital) outside the
state

\end{slide}




\begin{slide}
\begin{center}
{\bf Multinational companies and taxation}
\end{center}


{\bf Multinational firms}: Firms that operate in multiple countries. Foreign branches of the firm
are called subsidiaries.

%{\bf Subsidiaries}: The production arms of a corporation that are located in other nations.

{\bf Territorial tax system}: Corporations earning income abroad pay tax only to the government of the country in which the income is earned (most countries use this system)

{\bf Global tax system}: Corporations are taxed by their home countries on their income regardless of where it is earned (with tax credit for foreign corporate taxes paid)

\small 
US had global tax system before 2018 (but foreign profits were taxed only when ``repatriated'')

US system in 2018+: territorial system but with modest minimum tax of 10.5\% on foreign profits (with foreign tax credit)

Biden campaign proposed to beef up minimum tax

%(US system before 2018 but foreign profits were taxed only when ``repatriated'')
%
%{\bf Foreign tax credit}:
%U.S.-based multinational corporations may claim a credit against their U.S. taxes for any tax payments made to foreign governments when funds are repatriated to the US parent.

%{\bf Repatriation}:
%The return of income from a foreign country to a corporation's home country.
\end{slide}

\begin{slide}
\begin{center}
{\bf Repatriation Tax Holidays (before 2018)}
\end{center}
\vspace{-0.5cm}

In US pre-2018, owners eventually wanted the income repatriated from abroad and paid out to them as dividends

Corporations paid normal (old) corporate tax  35\% tax on foreign profits upon repatriation

Massive amount of profits accumulated abroad (about \$2.5 Tr by 2018) $\Rightarrow$ Temptation for politicians to offer repatriation tax holiday

\small
American Jobs Creation Act of 2004: Reduced tax rate on repatriated profits from 35\% to 5.25\% for 2005 only: surge in repatriations in 2005 (by \$250bn) followed by reductions in repatriations in subsequent years 

$\Rightarrow$ Net tax loser and no surge in investment
\normalsize


2018 Trump tax reform forces repatriations over 2018-2025 with 15.5\% tax on cash and 8\% on other assets and imposes min tax of 10.5\% on foreign profits with foreign tax credit


\end{slide}


\begin{slide}
\includepdf[pages={25}]{corporatetaxation_ch24_new_attach.pdf}
\end{slide}


%\begin{slide}
%\begin{center}
%{\bf Inversions}
%\end{center}
%
%Other way for U.S. corporations to dodge U.S. corporate tax: change country of incorporation to a tax haven
%
%Cannot just say ``I'm an Irish corporation now.'' Must merge with an Irish corporation first, called ``corporate inversion''
%
%Ex. Medtronic (maker of heart pacemakers) merged with Irish Covidien in 2014 $\rightarrow$ Declared legal headquarters in Ireland $\rightarrow$ Avoided U.S. tax on \$14bn held overseas
%
%
%
%
%Potential rationale for low U.S. corporate tax rate: Corporations will move headquarters/jobs overseas
%
%No evidence though that many actual jobs move (e.g. Medtronic kept operational headquarters in Minnesota)
%\end{slide}



\begin{slide}
\begin{center}
{\bf Tax Avoidance of Multinationals (Zucman '14)}
\end{center}
Share of profits made abroad by US corporations is about 1/3 today (was less than 5\% in the 1930s)

50\% of foreign profits of \textbf{multinationals} are reported in tax havens (such as Ireland)


Multinational companies are particularly savvy to avoid corporate income tax
by reporting most of their profits in low tax countries using \textbf{transfer pricing}: one subsidiary buys/sells to another
at manipulated prices to transfer profits

Example: Google located its search engine algorithm in Bermuda and Google Bermuda leases it to Google US, Google EU, etc.

Profits are moving to tax havens but not workers nor real capital $\Rightarrow$ This is a tax avoidance story

%$\Rightarrow$ Effective corporate tax rate is lower than nominal US Federal tax rate


\end{slide}

\begin{slide}
\includepdf[pages={20}]{corporatetaxation_ch24_new_attach.pdf}
\end{slide}

\begin{slide}
\includepdf[pages={31, 30}]{corporatetaxation_ch24_new_attach.pdf}
\end{slide}

\begin{slide}
\begin{center}
{\bf Issues with new US Corporate Tax System}
\end{center}
Since 2018, US has a very low corporate tax rate of 21\%

$\Rightarrow$ Strong incentives for successful business owners to incorporate 
and keep profits inside the corporation and pay only 21\% (instead of higher top individual tax rate)

$\Rightarrow$ This can undermine the progressive individual income tax

If business is a multinational: profits abroad are taxed at an even lower 10.5\% tax rate (with foreign tax credit)

$\Rightarrow$ Multinationals still have strong incentives to shift profits abroad in tax havens

Declining corporate tax rates across the world due to harmful tax competition (re-inforces inequities created by globalization)

\end{slide}

\begin{slide}
\includepdf[pages={26}]{corporatetaxation_ch24_new_attach.pdf}
\end{slide}

%\begin{slide}
%\includepdf[pages={1}, scale = 1]{materials/TWZ201748.pdf}
%\end{slide}

\begin{slide}
\begin{center}
{\bf Taxing Multinational Companies more Effectively}
\end{center}
Current territorial system where multinationals choose where to report
profits is easy to game. Need a better system: Several possibilities:

1) Tax on global profits (each country taxes its multinationals on global profits with
credit for foreign taxes paid)

2) Minimum tax on foreign profits country-by-country: min tax needs to be high enough to discourage reporting
in tax havens 

3) Apportioning profits based on sales in each country [as states are doing within the US]

Probably need to combination of these and have strong anti-inversion regulations so
that it's hard for multinationals to change nationality [Saez-Zucman 2019 discussion]

\end{slide}


%\begin{slide}
%\begin{center}
%{\bf Taxing Multinational Companies more Effectively}
%\end{center}
%
%\textbf{3) Border adjustment} (Paul Ryan 2017 tax plan):
%
%Idea developed by Auerbach (2010) (Berkeley Professor) [see Auerbach and Hotz-Eakin 2016
%for simple presentation]
%
%Include in corporate tax base the value of all imports and deduct the
%value of all exports 
%
%Allow businesses to expense investment immediately (instead of depreciation over life of each
%investment asset)
%
%Disallow deduction of interest paid on corporate debt
%
%\textbf{Value-added-taxes} [=consumption taxes] used widely (except US) use the same base but also including all labor costs and also have border adjustment 
%
%\end{slide}
%
%\begin{slide}
%\begin{center}
%{\bf DBCFT tax proposal}
%\end{center}
%Destination-based cash-flow tax proposed by P. Ryan in 2016
%
%Economically, proposal is equivalent to:
%
%1) Abolish corporate income tax
%
%2) Introduce a value-added-tax on consumption at 20\% rate
%
%3) subsidize labor earnings at 20\% rate (like a giant payroll tax cut)
%
%1) is very regressive and makes US corporate tax haven
%
%2)+3) is equivalent to a one-time tax on existing wealth (as consumption tax 
%and labor earnings tax are equivalent)
%
%Net effect quite regressive [no impact on trade due to exchange rate adjustment if economists
%are right]
%\end{slide}




%\begin{slide}
%\begin{center}
%{\bf CONCLUSION}
%\end{center}
%\small
%Despite the declining importance of the corporate tax as a source of revenue in the United States, it remains an important determinant of the behavior of corporations in the United States.
%
%The complicated incentives and disincentives that the corporate tax creates for investment appear to be significant determinants of a firm's investment decisions.
%
%Both corporate and personal capital taxation, although not completely, drive a firm's decisions about how to finance its investments.
%
%The United States faces a difficult set of decisions about how to reform its corporate tax system.
%
%Despite repeated calls for ending ``abusive corporate tax shelters,'' there has been little movement to end the types of corporate tax loopholes that cause such activity.
%
%This lack of interest should not be surprising: corporate tax breaks have highly concentrated and powerful supporters, with only the diffuse taxpaying public to oppose them.
%\end{slide}



\begin{slide}
\begin{center}
{\bf REFERENCES}
\end{center}
{\small

Jonathan Gruber, Public Finance and Public Policy, 2019, Worth Publishers, Chapter 24

Auerbach, Alan J. (2010) ``A Modern Corporate Tax.'' Center for American Progress/The Hamilton
Project. \href{http://elsa.berkeley.edu/~saez/course131/auerbach2010corptax.pdf}{(web)}

Auerbach, Alan J. and Douglas Holtz-Eakin (2016) ``The Role of Border Adjustments in International Taxation'', UC Berkeley working paper  \href{http://elsa.berkeley.edu/~saez/course131/auerbach-hotzeakin2016.pdf}{(web)}

Chetty, Raj, and Emmanuel Saez. ``Dividend taxes and corporate behavior: Evidence from the 2003 dividend tax cut.'' Quarterly Journal of Economics 120.3 (2005): 791-833.\href{http://elsa.berkeley.edu/~saez/chetty-saezQJE05dividends.pdf}{(web)}

Chetty, Raj, and Emmanuel Saez. ``The Effects of the 2003 Dividend Tax Cut on Corporate Behavior: Interpreting the Evidence.'' American Economic Review 96.2 (2006): 124-129.\href{http://elsa.berkeley.edu/~saez/chetty-saezAEA06.pdf}{(web)}

Chetty, Raj, and Emmanuel Saez. ``Dividend and corporate taxation in an agency model of the firm.'' American Economic Journal: Economic Policy 2.3 (2010): 1-31.\href{http://emlab.berkeley.edu/users/saez/chetty-saezAEJ10divtaxtheory.pdf}{(web)}

%Gordon, Roger, and Martin Dietz. ``Dividends and taxes.'' No. w12292. National Bureau of Economic Research, 2006.\href{http://www.nber.org/papers/w12292.pdf}{(web)}

Saez, Emmanuel and Gabriel Zucman. The Triumph of Injustice: How the Rich Dodge Taxes and How to Make them Pay, New York: W.W. Norton, 2019. 
\href{http://www.taxjusticenow.org} {(web)}

Torslov, Thomas R., Ludvig S. Wier, and Gabriel Zucman. The missing profits of nations. National Bureau of Economic Research Working Paper
No. 24701, 2018. \href{https://www.nber.org/papers/w24701.pdf} {(web)}

Yagan, Danny. ``Capital tax reform and the real economy: the effects of the 2003 dividend tax cut.'' American Economic Review, 105(12), 3531--3563, 2015.\href{http://elsa.berkeley.edu/~saez/course131/yaganAER15.pdf}{(web)}

Zucman, Gabriel ``Taxing Across Borders: Tracking Personal Wealth and Corporate Profits,'' \emph{Journal of Economic Perspectives}, 2014, 28(4): 121-148.\href{http://elsa.berkeley.edu/~saez/course131/Zucman2014JEP.pdf}{(web)}

Zwick, Eric and James Mahon. ``Tax Policy and Heterogeneous Investment Behavior'', American Economic Review, 107(1): 217-48, 2017.\href{http://elsa.berkeley.edu/~saez/course131/zwick-mahonAER17.pdf}{(web)}


}
\end{slide}

\end{document}
