\documentclass[landscape]{slides}

\usepackage[landscape]{geometry}

\usepackage{pdfpages}
\usepackage{graphics}

\usepackage{hyperref}
\usepackage{amsmath}

\def\mathbi#1{\textbf{\em #1}}

\topmargin=-1.8cm \textheight=17cm \oddsidemargin=0cm
\evensidemargin=0cm \textwidth=22cm


\author{131 Undergraduate Public Economics \\ Emmanuel Saez \\ UC Berkeley}
\date{}


\title{Tools of Budget Analysis \\ (Chapter 4 in Gruber's textbook)} \onlyslides{1-300}

\newenvironment{outline}{\renewcommand{\itemsep}{}}

\begin{document}

\begin{slide}
\maketitle
\end{slide}

%\begin{slide}
%\begin{center}
%{\bf OUTLINE}
%\end{center}
%Chapter 4
%
%4.1 Government Budgeting
%
%4.2 Measuring the Budgetary Position of the Government: Alternative Approaches
%
%4.3 Do Current Debts and Deficits Mean Anything? A Long-run Perspective
%
%4.4 Why Do We Care About the Government's Fiscal Position?
%
%4.5 Conclusion
%\end{slide}

%4.1 Government Budgeting

\begin{slide}
\begin{center}
{\bf GOVERNMENT BUDGETING}
\end{center}

{\bf Debt}:
The amount borrowed by government through bonds to individuals, firms, or foreigners. Debt is a \textbf{stock}

{\bf Deficit}: government's spending + interest payments on debt minus
government revenues in a given year.  A negative
deficit is called a surplus. Deficit is a \textbf{flow}

Evolution of debt from year to year:
\[ \text{Debt}_{t+1}=\text{Debt}_t + \text{Deficit}_t = \text{Debt}_t \cdot (1+r_t) + \text{Spending}_t - \text{Revenue}_t \]
with $r_t$ interest paid on government debt

Primary Deficit = Spending - Revenue

US Federal debt (held outside govt) is \$21Tr around 100\% of GDP, 2020 US deficit huge 16\% (\$3.5T) of GDP 
bc COVID

US government owns assets worth about 80\% of GDP

\end{slide}

\begin{slide}
\includepdf[pages={18}]{budget_ch04_new_attach.pdf}
\end{slide}

\begin{slide}
\includepdf[pages={16}]{budget_ch04_new_attach.pdf}
\end{slide}


\begin{slide}
\includepdf[pages={17}]{budget_ch04_new_attach.pdf}
\end{slide}



%\begin{slide}
%\includepdf[pages={9}]{budget_ch04_new_attach.pdf}
%\end{slide}


\begin{slide}
\begin{center}
{\bf GOVERNMENT DEBT SUSTAINABILITY}
\end{center}
\[ \text{Debt}_{t+1}=\text{Debt}_t + \text{Deficit}_t  \]
Debt/GDP stable when Deficit less than GDP nominal annual growth $g$ 

$g$ around 5\% per year = 2\% price inflation + 1\% population growth + 2\% real growth per capita
\bigskip
\[ \text{Deficit}_t =  r_t \cdot \text{Debt}_t + \text{Spending}_t - \text{Revenue}_t \]
with $r_t$ nominal interest on debt

Debt can snowball when $r_t$ exceed $g_t$

Since 2008, in the US, $r_t \simeq 1.5\%$ much lower than $g_t=5\%$

$\Rightarrow$ US debt sustainable as long as primary deficit Spending-Revenue less than 3\% of GDP
and $r$ stays low
\end{slide}



\begin{slide}
\begin{center}
{\bf GOVERNMENT DEBT IN CLOSED ECONOMY}
\end{center}
Govt borrows from private sector (ultimately individuals). 

Government debt increases private wealth at the expense of public wealth

$\Rightarrow$ No direct effect on national wealth = private wealth + public wealth in closed economy

Govt debt is not borrowing on the back of future generations but rather changing 
the distribution of wealth 

High debt with high interest rate limits spending ability of govt (as taxes must pay first
interest on debt)

Today: US (and most EU countries and Japan) have very low interest rate on govt debt: about 0\% in real terms 

$\Rightarrow$ Makes govt debt more attractive than taxes for spending 



\end{slide}



\begin{slide}
\begin{center}
{\bf GOVERNMENT DEBT IN OPEN ECONOMY}
\end{center}

Govt debt can also be borrowed from abroad

In this case, govt debt is indeed making future generations poorer (indebted to other countries)

1/3 of US debt (\$7T) held abroad but US also owns foreign assets that pay higher returns

US debt held abroad primarily by foreign central banks that use it as reserves

While interest rate is low, this is a good deal for the US

If interest rate increases, it will be perceived in US as a heavy burden to be paid to 
foreigners 

Many developing countries have been caught in cycles of unsustainable govt debt 
borrowed from foreigners
 
\end{slide}



\begin{slide}
\begin{center}
{\bf THE US FEDERAL PROCESS}
\end{center}

Taxes, spending, and debt ceiling are decided by Congress and the President

Any new law requires majority vote both in House and in Senate along with
President's signature (veto power)

In recent years, Senate vote requires 60/100 super-majority (due to filibuster)

Two forms of spending:

{\bf Entitlement spending}:
Mandatory funds for programs for which funding levels are automatically set by the number of eligible recipients (ex: medicare, social security)

{\bf Discretionary spending}:
Optional spending set by  appropriation levels each year, at Congress's discretion (ex: defense)

\vspace{-10pt}

Failure to pass appropriation results in Fed govt shutdown

\end{slide}

%\begin{slide}
%\includepdf[pages={14}]{budget_ch04_new_attach.pdf}
%\end{slide}

\begin{slide}
\begin{center}
{\bf Short-Run Effects of the Govt on the Macroeconomy}
\end{center}

\textbf{Keynesian theory (IS-LM macro model):} 
More government spending or tax cuts stimulates 
the economy in the short-run [and conversely]


{\bf Short-run stabilization}:
Govt can use taxes and spending policies to smooth the peaks and troughs of the business cycle

\small

{\bf Automatic stabilization}:
Policies that automatically alter taxes or spending in response to economic fluctuations to offset changes in household consumption levels (ex: unemployment insurance, progressive taxation, corporate profits tax)

{\bf Discretionary stabilization}:
Policy actions taken by the government in response to business cycle (ex: Fiscal stimulus with Spring 2008 rebate checks, 2009-12 Obama stimulus, COVID care acts)

\normalsize

$\Rightarrow$ Ability to run deficits in recessions is a great tool for short-run business cycle stabilization
%(but need to reduce debt during good times to keep ability to run deficits when needed)

\end{slide}

\begin{slide}
\includegraphics[page=10,scale=1.1]{budget_ch04_new_attach.pdf}

\small \% changes in annual real govt spending and changes in real GDP, 33 EU countries, 2010-11, 2011-2, 2012-3
(=99 dots). Source: Krugman NYtimes blog, January 6, 2015
\end{slide}


\begin{slide}
\begin{center}
{\bf Budget Policies and Deficits at the State Level}
\end{center}
In contrast to Federal govt, States have budget balance requirements
forcing spending to equate tax revenue each year

In downturns, tax revenue falls due to decreased incomes $\Rightarrow$ Forces states
to either cut spending and increase taxes $\Rightarrow$ Further exarcerbates the economic downturn

California had to cut spending drastically during Great Recession 2008-2010
$\Rightarrow$ California established a rainy fund for future hard times but it remains too small

Absent federal help, COVID crisis is forcing states to cut spending as well
(CA tax revenue is doing fine because the very rich have done well and continue to pay taxes)

\end{slide}

%4.2 Measuring the Budgetary Position of the Government: Alternative Approaches

\begin{slide}
\begin{center}
{\bf STATIC VS. DYNAMIC SCORING}
\end{center}
Govts have agencies evaluating effects of proposed reforms on govt deficit (Congressional Budget Office for US fed govt)

{\bf Static scoring}:
A method used by budget modelers that assumes that government policy changes only the distribution of total resources, not the amount of total resources.

{\bf Dynamic scoring}:
A method used by budget modelers that attempts to model the effect of government policy on both the distribution of total resources and the amount of total resources.

\small
Example: tax decreases on the rich, static scoring assumes no effect on GDP, dynamic scoring incorporates effects on growth
\normalsize

Static scoring is safest in the absence of good empirical estimates of growth effects (dynamic scoring can be manipulated by ideologues, see Lynch 2015 for detailed pros/cons)

\end{slide}

%%4.3 Do Current Debts and Deficits Mean Anything? A Long-run Perspective
%\begin{slide}
%\begin{center}
%{\bf Do Current Debts and Deficits Mean Anything? \\ A Long-Run Perspective}
%\end{center}
%
%{\bf Implicit obligation}:
%Financial obligations the government has to the future that are not recognized in the annual budgetary process.
%
%Examples:
%
%1) Medicare costs due to baby boom generation getting to age 65 (Medicare is govt provided
%health care to seniors 65+)
%
%2) Social security benefits for baby boom generation (Social Security is govt provided retirement
%benefits to seniors)
%
%\end{slide}


\begin{slide}
\begin{center}
{\bf Intertemporal Government Budget Constraint}
\end{center}

Policy debates have traditionally focused on the extent to which this year's governmental spending exceeds this year's governmental revenues.

The existence of implicit obligations in the future, however, suggests that this does not capture the full picture

E.g. population aging increases cost of social security and Medicare

{\bf Intertemporal budget constraint}:
An equation relating the Present Discounted Value of the government's obligations to the Present Discounted Value of its revenues.

$PDV$ of Tax Payments  = \\ $PDV$ of All Future Govt Spending + Current Govt Debt
\end{slide}


\begin{slide}
\begin{center}
{\bf BACKGROUND: PRESENT DISCOUNTED VALUE}
\end{center}
For govt, spending $F$ now has the same cost as spending $F \cdot (1+r)$ next year with $r$ interest rate on
government debt

{\bf Present discounted value (PDV)}:
The value of each period's dollar amount in today's terms.

Govt spends $F_1$, $F_2, F_3,...$  in each future year, then the $PDV$ is computed as:
\[PDV=\frac{F_1}{(1+r)}+\frac{F_2}{(1+r)^2}+\frac{F_3}{(1+r)^3}+...\]
If $F_1=F_2=.. =F$ then
\[PDV = \frac{F}{1+r} \cdot \left [1+\frac{1}{(1+r)}+\frac{1}{(1+r)^2} + ... \right ] = \frac{F}{1+r} \cdot \frac{1}{1-\frac{1}{1+r}}
=\frac{F}{r} \]
Paying $F$ in perpetuity is equivalent to paying $F/r$ upfront
\end{slide}



\begin{slide}
\begin{center}
{\bf LONG-RUN FISCAL IMBALANCE}
\end{center}
It is defined as gap between

1) PDV of All Future Govt Spending + Current Govt Debt 

2) PDV of Tax Payments   

If the  government continues with today's policies, how much more will the government spend than it will collect in taxes over the entire future? 

A long-run fiscal imbalance means that policies will have to be adjusted at some point

Some policies can drastically change the long-run fiscal imbalance even if they don't affect the current deficit much

\small Example: In 2003, the government added roughly \$20 trillion to the fiscal imbalance (due to tax cuts and medicare prescription drug benefit of Bush administration)

%Each year, the fiscal imbalance grows by roughly 3--4\%, as the nation accumulates interest obligations on the existing large implicit debt.
\end{slide}

%\begin{slide}
%\begin{center}
%{\bf PROBLEMS WITH LONG-RUN FISCAL MEASURES}
%\end{center}
%
%The fiscal imbalance calculations are fairly tenuous:
%
%1)  They depend critically on many assumptions about future growth rates in costs and incomes, and the interest rate used for discounting 
%
%$\Rightarrow$ Those assumptions become heroic for long-distance future (example: how will health care
%costs evolve?)
%
%2) The calculations also assume that government policy remains unchanged (but if big imbalance arises,
%then government will typically be forced to act and fix it)
%
%$\Rightarrow$ Makes most sense to consider a time window that is longer than 1 year but less than infinity
%
%%US budget discussions (covered by press) focus on a 10 year window
%
%\end{slide}
%
%\begin{slide}
%\includepdf[pages={12}]{budget_ch04_new_attach.pdf}
%\end{slide}
%
%
%\begin{slide}
%\begin{center}
%{\bf PROBLEMS WITH LONG-RUN FISCAL MEASURES}
%\end{center}
%
%Some programs are easier to project than others.
%
%Example: social security retirement benefits are easier
%to project than medicare benefits
%
%Social security benefits depend on demography and longevity (slow moving variables)
%$\Rightarrow$ Social security does fairly reliable 75 year projections
%
%Medicare depends on growth of health care costs that have been growing very fast (before the Great recession)
%$\Rightarrow$ such a rate of growth is not sustainable for ever so making a long-run projection
%based on those rates is not meaningful
%
%CBO makes budget projections over the next 10 years in its official budget projection 
%
%
%\end{slide}

%\begin{slide}
%\includepdf[pages={15}]{budget_ch04_new_attach.pdf}
%\end{slide}
%
%\begin{slide}
%\includepdf[pages={2}]{budget_ch04_new_attach.pdf}
%\end{slide}

%\begin{slide}
%\includepdf[pages={14}]{budget_ch04_new_attach.pdf}
%\end{slide}

%4.4 Why Do We Care About the Government's Fiscal Position?



\begin{slide}
\begin{center}
{\bf LONG-RUN EFFECTS OF GOVERNMENT DEBT}
\end{center}
In the long-run, government debt affects the capital market where savers meet investors

In closed economy: private savings = investment + new govt debt

With more government debt, if private savings do not change, less funds available for investment:
investment decreases  

Two mitigating factors:

1) In an open economy, investment or govt debt can be funded with foreign savings

2) If individuals are forward looking, they understand that higher debt implies high taxes
later on and hence they save more to be able to pay higher taxes later on [Ricardian equivalence
but not much empirical support]



%{\bf Background: Savings and Economic Growth:}
%The earliest economic growth models emphasized a central role for savings as an engine of growth, and this insight remains important for  growth economics today.
%
%{\bf More Capital, More Growth:}
%As there is more capital in an economy, each worker is more productive, and total social product rises. A larger capital stock means more total output for any level of labor supply. Thus, the size of the capital stock might be a primary driver of growth.
%
%{\bf Neo-classical aggregate production function:}
%
%$K$ capital stock, $L$ labor, $A$ technology
%\[ Y=F(K,L)= A \cdot K^{\alpha} L^{1-\alpha} \quad \text{with} \quad \alpha \simeq 30\% \]

\end{slide}


%\begin{slide}
%\begin{center}
%{\bf Savings and Investment market}
%\end{center}
%
%{\bf Interest rate:}
%The rate of return in year $t+1$ on investments made in year $t$.
%
%Save $S$ in period $t$, you get $(1+r) \cdot S$ in period $t+1$
%
%{\bf Supply and Demand for capital:}
%
%Supply of savings (from households) depends positively on $r$ (higher $r$ means bigger returns to savings)
%
%Demand for investment (from firms) depends negatively on $r$ (firms invest only if return on investment is
%at least equal to $r$)
%
%In a competitive market, the equilibrium amount of investment is determined by the intersection of these demand and supply curves: $S(r) = D(r)$
%\end{slide}
%
%\begin{slide}
%\includepdf[pages={8}]{budget_ch04_new_attach.pdf}
%\end{slide}
%
%
%
%\begin{slide}
%\begin{center}
%{\bf Effects of Government Debt on Capital Market}
%\end{center}
%
%If there is a deficit $d$, the government must borrow to finance the difference between its revenues and its expenditures: individual savings need to cover both $d$ and firms' demand for capital
%
%\textbf{Equilibrium:} $S(r) = d + D(r)$ or $S(r)-d = D(r)$ $\Rightarrow$ $r$ increases and $K$ decreases
%
%The government's borrowing may \emph{crowd out} the borrowing of the private sector and lead to a lower level of capital accumulation
%
%In reality, there are a number of complications of how government financing affects interest rates and growth
%
%\end{slide}
%
%\begin{slide}
%\begin{center}
%{\bf INTERNATIONAL CAPITAL MARKETS}
%\end{center}
%With international capital markets, there is a worldwide interest rate $r$ (i.e., supply of savings is
%perfectly elastic for a small open economy and hence horizontal) 
%
%$\Rightarrow$ Government debt has no impact
%on $r$ and $K$
%
%There is a large body of economics literature that has investigated the integration of international capital markets
%
%It has generally concluded that while integration is present (and perhaps growing), it is far from perfect.
%
%As a result, the supply of capital to the United States may not be perfectly elastic, and government deficits could crowd out private savings
%\end{slide}

%\begin{slide}
%\begin{center}
%{\bf RICARDIAN EQUIVALENCE}
%\end{center}
%Standard supply and demand savings model assumed that supply of savings $S(r)$ was not
%affected by government deficit $d$
%
%In reality, if individuals are rational, they recognize that more debt now means higher taxes (or less spending)
%in the future
%
%If individuals are fully rational in their savings decisions, individuals understand that \$1 tax reduction today
%(and \$1 of debt increase)
%
%\small
%$\Rightarrow$ \$1 of increases in PDV of taxes so the inter-temporal budget
%of the individual is unchanged
%
%$\Rightarrow$ Individuals save the extra \$1 of tax reduction and will use it to pay future taxes
%
%$\Rightarrow$ Govt debt increases the supply of individual savings one-for-one so govt debt
%does not affect $r$ and capital
%
%\end{slide}
%
%\begin{slide}
%\begin{center}
%{\bf RICARDIAN EQUIVALENCE}
%\end{center}
%What if increased taxes are on future generations? Will the elderly respond to the tax cut?
%
%Barro (1974) showed that if generations are linked through altruism:
%parents care about kids and can leave
%them bequests, then the
%dynasty behaves as a single individual
%
%\$1 tax cut on elderly leads them to save it and increase their bequests by \$1 so that their kids (or grand kids) can pay the corresponding tax increase down the road (consumption is unaffected)
%
%This model has received very little empirical support in the economics literature
%
%Example: Individuals are not that rational and strongly respond to tax rebates
%\end{slide}
%
%\begin{slide}
%\begin{center}
%{\bf THE FEDERAL BUDGET, INTEREST RATES, AND ECONOMIC GROWTH: EVIDENCE}
%\end{center}
%
%Theory therefore tells us that higher deficits likely lead to higher interest rates and less capital investment, but it does not tell us how much higher and how much less.
%
%Effects of deficits on interest rates depend on circumstances
%
%In normal times, you would expect a positive effect of deficits on interest rates
%
%In recessions (like in the US since 2008), interest rate on govt debt is very low in spite of very
%large deficits (due to the Fed setting $r$ at zero to stimulate the economy)
%
%%The existing empirical literature on this question is somewhat inconclusive, although recent evidence suggests that projected long-term deficits do appear to be reflected to some extent in long-term interest rates
%
%\end{slide}


\begin{slide}
\begin{center}
{\bf CONCLUSION}
\end{center}

The deficit has been a constant source of policy interest and political debate over the last decade

Short-run: should the govt spend more and increase deficit to stimulate the economy?

Long-run: should the govt address long-term deficits by increasing taxes or cutting spending?

International evidence shows that austerity during the Great Recession worsens the recession 

Health care cost growth has slowed down sharply since 2008, substantially improving 
the long-term Federal budget outlook

But 2018 tax reform has worsened the budget situation

\end{slide}


\begin{slide}
\begin{center}
{\bf REFERENCES}
\end{center}
{\small

Jonathan Gruber, Public Finance and Public Policy, 2019 Worth Publishers, Chapter 4

Barro, Robert J. ``Are government bonds net wealth?.'' The Journal of Political Economy 82.6 (1974): 1095-1117.\href{http://elsa.berkeley.edu/~saez/course131/Barro74JPE.pdf}{(web)}

Congressional Budget Office ``The Budget and Economic Outlook: Fiscal Years 2019 to 2029'', August 2019 \href{http://elsa.berkeley.edu/~saez/course131/CBO2020report.pdf}{(web)}

Congressional Budget Office ``An Update to the Budget Outlook: 2020 to 2030'', September 2020 \href{http://elsa.berkeley.edu/~saez/course131/CBO2020report.pdf}{(web)}

Lynch, Robert 2015 ``The benefits and drawbacks of using dynamic scoring in the federal budget'', Equitable Growth \href{https://equitablegrowth.org/benefits-drawbacks-using-dynamic-scoring-federal-budget/}{(web)}

Piketty, Thomas, \emph{Capital in the 21st Century},  Cambridge: Harvard University Press, 2014,  
\href{http://piketty.pse.ens.fr/en/capital21c2}{(web)}


}

\end{slide}



\end{document}

