\documentclass[landscape]{slides}

\usepackage[landscape]{geometry}

\usepackage{pdfpages}

\usepackage{hyperref}
\usepackage{amsmath}

\def\mathbi#1{\textbf{\em #1}}

\topmargin=-1.8cm \textheight=17cm \oddsidemargin=0cm
\evensidemargin=0cm \textwidth=22cm


\author{131 Undergraduate Public Economics \\ Emmanuel Saez \\ UC Berkeley}
\date{}

\title{State and Local Governments} \onlyslides{1-300}

\newenvironment{outline}{\renewcommand{\itemsep}{}}

\begin{document}

\begin{slide}
\maketitle
\end{slide}

%\begin{slide}
%\begin{center}
%{\bf OUTLINE}
%\end{center}
%Chapter 10
%
%10.1 Fiscal Federalism in the United States and Abroad
%
%10.2 Optimal Fiscal Federalism
%
%10.3 Redistribution Across Communities
%
%10.4 Conclusion
%
%\end{slide}

\begin{slide}
\begin{center}
{\bf FISCAL FEDERALISM}
\end{center}

{\bf optimal fiscal federalism}:
The question of which activities should take place at which level of government: federal, state, local

The distribution of government spending has changed dramatically over time in the United States

$\bullet$ Local state and spending have declined considerably.

$\bullet$ Much state and local spending now supported by
intergovernmental grants [transfers from the federal government]

$\bullet$ Many countries are much more centralized than US (e.g. France)

\end{slide}

%10.1 Fiscal Federalism in the United States and Abroad

\begin{slide}
\includepdf[pages={2}]{local_ch10_new_attach.pdf}
\end{slide}

\begin{slide}
\begin{center}
{\bf SPENDING AND REVENUE OF STATE AND \\ LOCAL GOVERNMENTS}
\end{center}

{\bf Property tax}:
The tax on land and any buildings on it, such as commercial businesses or residential homes.

Main source of revenue from local governments due to:

1) History: real estate property is visible and hence taxable even in archaic economies
with informal businesses

2) Immobile tax base: the real estate tax base cannot flee to another jurisdiction (mobility
of the tax base is an issue for local governments)

US today, property tax is about 1/3 of revenue raised by state+local government (rest is 1/3 income tax,
1/3 sales taxes)

\end{slide}

%\begin{slide}
%\includepdf[pages={3}]{local_ch10_new_attach.pdf}
%\end{slide}

%10.2 Optimal Fiscal Federalism

\begin{slide}
\begin{center}
{\bf THE TIEBOUT (1956) MODEL}
\end{center}
Darling model of local public economics among economists

What is it about the private market that generates efficient provision of private goods that is missing for public goods?

Tiebout's insight was that the factors missing from the market for public goods were shopping and competition

The situation is different when public goods are provided at the local level by cities and towns:

Competition will naturally arise because individuals can \emph{vote with their feet}: if they don't like the level or quality of public goods provision in one town, they can move to the next town

This threat of exit can induce efficiency in local public goods production
\end{slide}

\begin{slide}
\begin{center}
{\bf THE TIEBOUT FORMAL MODEL}
\end{center}
We consider a very simple model to illustrate Tiebout's insight and theorem

Suppose there are $2\cdot N$ families with identical income $Y$ and 2 towns with $N$ homes
each

Towns 1 and 2 supply level $G_1, G_2$ of local public schools

There are 2 types of families:

1) $N$ families with kids, with utility $U^K(C,G)$, value private consumption $C$ and schools $G$

2) $N$ elderly families, with utility $U^E(C)$, value only private consumption $C$
\end{slide}

\begin{slide}
\begin{center}
{\bf THE TIEBOUT  EQUILIBRIUM DEFINITION}
\end{center}

Allocation of families across towns is a \textbf{Tiebout Equilibrium} if and only if two conditions are met:

1) In each town, $G$ is decided by median voter and financed equally by town
residents with budget $Y=G/N+C$

$\Rightarrow$ If majority in town is elderly then $G=0$ as this maximizes $U^E(Y-G/N)$

$\Rightarrow$ If majority in town is families with kids then $G=G^*$ that maximizes $U^K(Y-G/N,G)$

2) No 2 families want to exchange locations across towns

\end{slide}

\begin{slide}
\begin{center}
{\bf THE TIEBOUT  THEOREM}
\end{center}

\textbf{Tiebout Theorem Part I:} In equilibrium, families will sort themselves in towns
according to their taste for public good (1 town with elderly only, 1 town with
families with kids only)

\textbf{Proof:} Suppose elderly dominate in town 1 and $G_1=0$, then families with kids dominate in town 2
and $G_2=G^*$. If there is a family with kids in town 1, then there is an elderly
family in town 2 and they are willing to switch $\Rightarrow$ not an equilibrium.


\textbf{Tiebout Theorem Part II:} In each town, the level of local public good is efficient

\textbf{Proof:}  In elderly town, $G=0$ which is efficient as nobody values $G$.

In kids town, $G^*$ maximizes $U^K(Y-G/N,G)$ which is also efficient as it is the preferred
choice of everybody.

%\small Note that Samuelson rule holds: $-U^K_C/N+U^K_G=0 \Rightarrow U^K_G/U^K_C=1/N$
%$\Rightarrow  \sum U^K_G/U^K_C = \sum MRS_{GC} = N/N=1=MC$ which is the Samuelson rule

\end{slide}




\begin{slide}
\begin{center}
{\bf THE TIEBOUT MODEL}
\end{center}

People can vote with their feet by choosing the locality that best fits their tastes and provides the best
public goods given the tax

The main message of the model is that competition across local jurisdictions puts competitive pressure on the provision of local public goods:

1) Public goods need to reflect tastes of local residents

2) Public goods need to be efficiently provided (without waste)

\end{slide}

\begin{slide}
\begin{center}
{\bf Centralized vs. Decentralized Government}
\end{center}
Conservatives/libertarians tend to like decentralized governments over centralized governments 

Conservatives/libertarian dislike redistribution and like individual choice and competition. In Tiebout model:

1)  local governments do not do any redistribution: individuals receive in
local public goods exactly what they are paying in taxes (= benefit principle of taxation)

2) individuals can choose (through their location choice) their preferred mix of public goods
and taxes

3) competition between local govts forces them to provide local public good efficiently

%Even if Tiebout model is not perfect description of reality, it is indeed harder for local governments
%to redistribute than for central governments due to mobility



\end{slide}


\begin{slide}
\begin{center}
{\bf PROBLEMS WITH THE TIEBOUT MODEL}
\end{center}

The Tiebout model is an idealized model that requires a number of assumptions that may not hold perfectly in reality:

1) Individuals can move without any cost across towns

2) Individuals have perfect information on the benefits and taxes paid in each town

%3) Individuals must be able to freely choose among a range of towns that might match their taste for public goods

%4) The provision of some public goods requires sufficient scale or size

3) There must be enough towns so that individuals can sort themselves into groups with similar preferences for public goods

4) No externalities/spillovers of public goods across towns [with spillovers across towns, 
public goods will be under provided in Tiebout model, e.g. pandemic coordination]

5) Local govts can charge ``poll'' taxes (equal payments per person) to residents. In reality, local taxes depend
on property, consumption, and sometimes income.
\end{slide}

%\begin{slide}
%\begin{center}
%{\bf PROBLEMS WITH THE TIEBOUT MODEL}
%\end{center}
%The Tiebout model requires equal financing of the public good among all residents.
%
%{\bf Lump-sum tax}:
%A fixed taxation amount independent of a person's income, consumption of goods and services, or wealth.
%Sometimes called a poll tax.
%
%Towns typically finance their public goods instead through a property tax where rich pay more than poor (because
%they live in nicer houses). The problem that property taxation causes is that the \emph{poor chase the rich}
%(rich also want to be with rich)
%
%\small
%Two mechanisms prevent poor from chasing the rich:
%
%{\bf 1) Housing prices:} places with rich people have high housing prices
%
%{\bf 2) Zoning}: Restrictions that towns place on the use of real estate (e.g., each house must sit on a parcel of
%at least 6000 sq feet)
%
%Zoning regulations protect the tax base of wealthy towns by pricing lower-income people out of the housing market.
%\end{slide}


%\begin{slide}
%\begin{center}
%{\bf PROBLEMS WITH THE TIEBOUT MODEL}
%\end{center}
%
%{\bf No Externalities/Spillovers}
%
%The Tiebout model assumes that public goods have effects only in a given town and that the effects do not spill over to neighboring towns.
%
%Many local public goods have similar externality or spillover features: police; public works; education
%
%With spillovers across towns, public goods will be under provided as each town does not take into account
%positive externality effects of its public good decisions
%\end{slide}

\begin{slide}
\begin{center}
{\bf EVIDENCE ON THE TIEBOUT MODEL}
\end{center}


{\bf Tiebout Sorting: Resident Similarity Across Areas}

A testable implication of the Tiebout model is that when people have more choice of local community, the tastes for public goods will be more similar among residents than when people do not have many choices

This fact is indeed pretty well established

{\bf More Efficiency when there is more Tiebout sorting}

This fact is controversial

\end{slide}

\begin{slide}
\begin{center}
{\bf Evidence on the Tiebout Model: Hoxby (2000)}
\end{center}

Hoxby (2000) considers public school districts in the US. She
compares cities where:

A) There are few large school districts and hence little choice for residents (such as Miami or LA)

B) There are many small school districts and hence a lot of choice for residents (such as Boston)

2 key findings:

I) Cities with few districts have less sorting across neighborhood (in terms of school quality) than
cities with many districts (this result is well established)

II) Cities with many districts have {\bf higher} test scores on average:
this result is controversial (see Rothstein, 2007 critique)
\end{slide}



\begin{slide}
\begin{center}
{\bf Capitalization of Fiscal Differences into House Prices}
\end{center}


{\bf House price capitalization}:
Incorporation into the price of a house of the costs (including local property taxes) and benefits (including local public goods) of living in the house.

$\Rightarrow$ High property taxes (relative to public goods quality) depresses housing prices

$\Rightarrow$ Low property taxes (relative to public goods quality) increases housing prices

\small
Example: Suppose \$1 cut in property tax (in perpetuity) with no change in value of local public good

Capitalized value: $\Delta V = 1 + 1/(1+r) + 1/(1+r)^2 + ... = 1/[1-1/(1+r)]=(1+r)/r = \$21$ if $r=5\%$

Oates (1969) is the classic reference on property tax capitalization

Modern study by Cellini-Ferreira-Rothstein (2010) on school bonds in CA using regression discontinuity in vote share of local bond measures: find positive effects of bonds on house values $\Rightarrow$ under-investment in schools 

\end{slide}

%\begin{slide}
%\includepdf[pages={5,6}]{local_ch10_new_attach.pdf}
%\end{slide}


\begin{slide}
\includepdf[pages={12, 13}]{local_ch10_new_attach.pdf}
\end{slide}

\begin{slide}
\begin{center}
{\bf KEY CONSEQUENCE OF TIEBOUT MODEL}
\end{center}

Hard for a local government to redistribute from rich to poor:

If local redistribution is high $\Rightarrow$

1) Poor flock to the city which provides welfare benefits

2) Rich flee to other cities to avoid paying for redistribution 

$\Rightarrow$ Local redistribution program will break down

Redistribution programs work better if implemented at higher level: state or federal (harder to leave the state or country). At local level, need to have tax-benefit linkage to avoid migration

{\bf Tax-benefit linkages}:
The relationship between the taxes people pay and the government goods and services they get in return.
\end{slide}

%10.3 Redistribution Across Communities

\begin{slide}
\begin{center}
{\bf REDISTRIBUTION ACROSS COMMUNITIES}
\end{center}

There is currently enormous inequality in both the ability of local communities to finance public goods and the extent to which they do so.

Central government can redistribute across communities \textbf{directly} using taxes and spending but also \textbf{indirectly} by giving grants to lower levels of government

Higher levels of government can redistribute across lower levels of government through \textbf{intergovernmental grants}.

We assume in graphical analysis that local community chooses public spending and private spending
according the preferences of Median voter in the community

\end{slide}

%\begin{slide}
%\includepdf[pages={7}]{local_ch10_new_attach.pdf}
%\end{slide}

\begin{slide}
\begin{center}
{\bf Intergovernmental Grants}
\end{center}
Higher level government can provide grants to redistribute across communities and incentivize
communities to spend on public goods

Three main forms of grants:

{\bf 1) Matching grant}:
A grant, the amount of which is tied to the amount of public good spending by the local community.


{\bf 2) Block grant}:
A grant of some fixed amount with no mandate on how it is to be spent.

{\bf 3) Conditional block grant}:
A grant of some fixed amount with a mandate that the money be spent in a particular way.

\end{slide}




\begin{slide}
\includepdf[pages={14-18}]{local_ch10_new_attach.pdf}
\end{slide}

%\begin{slide}
%\includepdf[pages={9}]{local_ch10_new_attach.pdf}
%\end{slide}
%
%\begin{slide}
%\includepdf[pages={10}]{local_ch10_new_attach.pdf}
%\end{slide}

%\begin{slide}
%\includepdf[pages={9}]{Gruber2e_ch10_attach.pdf}
%\end{slide}

\begin{slide}
\begin{center}
{\bf KEY PREDICTION OF THEORY: CROWD-OUT}
\end{center}
In theory: a \$1000 increase in private income has the same effect
as a \$1000 increase in Fed block grant: both shift the budget in the same way
and lead to the same outcome

Example: \$1000 private income increase leads to \$800 more in private consumption
and \$200 more in local taxes and public spending. \$1000 extra fed grant leads to \$200 extra in
public good spending and \$800 cut in local taxes and hence \$800 extra in private consumption

Similarly, with multiple public goods (e.g., schools and police), an extra \$1000 Fed grant for school
has the same effect on schools and police than a \$1000 Fed grant for police

Money is fungible: only total resources matter for the allocation across private good and public goods
at the local level

\end{slide}




\begin{slide}
\begin{center}
{\bf THE FLYPAPER EFFECT}
\end{center}
%\small

Hines and Thaler JEP'95 found that the crowd-out of state spending by federal spending is low and often close to zero

Economist Arthur Okun described this as the flypaper effect because ``the money sticks where it lands'' instead of replacing state spending

But evidence is based on correlation [not necessarily causation as states that get grants maybe the ones that like spending the most]

%
%
%Knight AER'02 tries to get at causation by studying highway grants: highway grants from the federal government to states are determined by the strength of the state's political representatives.
%
%Knight compared the level of spending in treatment states that see increases in the power of their congressional delegations with the level of spending in control states that see decreases in the power of their congressional delegations
%
%He found that each additional \$1 of federal grant money increase due to rising congressional power leads to a \$0.90 reduction in the state's own spending

Recent studies show that there is a flypaper effect in the short-run but that there is substantial crowd-out
from block grants in the long-run


\end{slide}

\begin{slide}
\begin{center}
{\bf REDISTRIBUTION IN ACTION: \\ SCHOOL FINANCE EQUALIZATION}
\end{center}

{\bf School finance equalization}:
Laws that mandate redistribution of funds across communities in a state to ensure more equal financing of schools.

Without school finance equalization, huge disparity in property tax base and hence school funding (per pupil) across areas
(e.g. in Bay Area: Lafayette is very wealthy, Richmond is poor)

Many states (including California) impose equalization: pool local taxes at state level and redistribute them across
districts

Equalization often imposed by courts without thinking carefully about
economic consequences


\end{slide}

\begin{slide}
\begin{center}
{\bf REDISTRIBUTION IN ACTION: \\ SCHOOL FINANCE EQUALIZATION}
\end{center}

{\bf Implicit tax on local government tax revenue}:
For school equalization schemes, for \$1 of extra local taxes, how much the central govt takes
away in reduced transfers to local govt

1) With no equalization, the tax rate is 0\% (local govt keeps all its revenue)

2) With perfect equalization, the tax rate is 100\% (raising local revenue has zero impact on local spending)

\end{slide}



\begin{slide}
\begin{center}
{\bf CALIFORNIA SCHOOL EQUALIZATION}
\end{center}

In 1960s-1970s, California used to have one of the best K-12 public school systems in the nation,
now it has one of the worst

California used to have no school finance equalization and hence big disparities across areas

1976: Serrano vs. Priest case: California Supreme court ruled that disparities above a threshold were unconstitutional

$\Rightarrow$ Wealthy districts forced to give all their tax revenue above the threshold to the common pool
to fund poor districts

$\Rightarrow$ local government has
no incentive to raise taxes $\Rightarrow$ taxes and school funding fall in rich districts

$\Rightarrow$ Property taxes no longer able to fund schools adequately
\end{slide}



\begin{slide}
\begin{center}
{\bf CALIFORNIA PROPOSITION 13}
\end{center}

In 1970s, discontent among the public about growing property taxes in CA due to (1) fast housing price increases and (2) local property taxes no longer funded local schools due to school equalization
(prop tax not capitalized into local prices)

Proposition 13 was voted in 1978 and imposed strong limits on property taxes (and required super majority 2/3 vote in state legislature to increase ANY tax):

Assessed value of real estate property can only grow at most by 2\% per year (instead of following price increases which are around 4-5\% on average)

\small

$\Rightarrow$ Property owners no longer face big increases in prop tax (helps retirees on fixed income)

$\Rightarrow$ New owners end up paying much more than old owners (e.g., house assessed at \$200K that sells for \$1m will see a 5-fold increase in property taxes). Creates a lock-in effect (Ferreira 2010)

\end{slide}


%\begin{slide}
%\includepdf[pages={11}]{local_ch10_new_attach.pdf}
%\end{slide}

%\begin{slide}
%\begin{center}
%{\bf CONCLUSION}
%\end{center}
%
%In every country, the central government collects only part of the total national tax revenues and does only part of the national public spending.
%
%When spending is on goods for which local preferences are relatively similar, and where most residents can benefit from those goods, the Tiebout model suggests that the spending should be done locally.
%\end{slide}

%\begin{slide}
%\begin{center}
%{\bf CONCLUSION}
%\end{center}
%
%%When spending is for goods that benefit only a minority of the population, the Tiebout model suggests that it might be difficult to do this spending locally because the majority of people who do not benefit will ``vote with their feet'' and move elsewhere.
%
%Higher levels of government are able to implement redistribution across lower levels of government
%either directly with means-tested programs for individuals (such as Food stamps) or through grants
%to local governments (such as Medicaid)
%
%In the same way that 100\% tax on individuals is counterproductive, 100\% confiscation of local taxes (as in the most drastic school equalization schemes) is counterproductive (leads local governments to abandon their taxes)
%
%Higher level government should have primary responsibility for redistribution
%
%\end{slide}


\begin{slide}
\begin{center}
{\bf REFERENCES}
\end{center}
{\small

Jonathan Gruber, Public Finance and Public Policy, Fifth Edition, 2016 Worth Publishers, Chapter 10

Cellini, Stephanie, Fernando Ferreira and Jesse Rothstein. 2010. ``The Value of School Facility Investments: Evidence from a Dynamic Regression Discontinuity Design'',
Quarterly Journal of Economics, 125(1), 215-261.
\href{https://www.jstor.org/stable/pdf/40506281.pdf}{(web)}

Ferreira, Fernando. 2010. ``You can take it with you: Proposition 13 tax benefits, residential mobility, and willingness to pay for housing amenities.'' Journal of Public Economics 94(9-10), 661-673.
\href{http://elsa.berkeley.edu/~saez/course131/ferreira10.pdf}{(web)}

Hines, James R., and Richard H. Thaler. ``Anomalies: The flypaper effect.'' The Journal of Economic Perspectives 9.4 (1995): 217-226.\href{http://www.jstor.org/stable/pdfplus/2138399.pdf}{(web)}

Hoxby, Caroline M. ``Does Competition among Public Schools Benefit Students and Taxpayers?.'' The American Economic Review 90.5 (2000): 1209-1238.\href{http://elsa.berkeley.edu/~saez/course131/hoxby00.pdf}{(web)}

%Knight, Brian. ``Endogenous federal grants and crowd-out of state government spending: Theory and evidence from the federal highway aid program.'' American Economic Review (2002): 71-92.\href{http://elsa.berkeley.edu/~saez/course131/Knight02.pdf}{(web)}


Oates, Wallace E. 1969. ``The Effects of Property Taxes and Local Public Spending on Property Values: An Empirical Study of Tax Capitalization and the Tiebout Hypothesis''
Journal of Political Economy 77(6), 957-971.
\href{https://www.jstor.org/stable/pdf/1837209.pdf}{(web)}



Rothstein, Jesse. "Does Competition Among Public Schools Benefit Students and Taxpayers? Comment." The American Economic Review 97.5 (2007): 2026-2037.\href{http://www.jstor.org/stable/pdfplus/10.2307/30034599.pdf}{(web)}

Tiebout, Charles M. ``A pure theory of local expenditures.'' The Journal of Political Economy 64.5 (1956): 416-424.\href{http://www.jstor.org/stable/pdfplus/1826343.pdf}{(web)}

}

\end{slide}






\end{document}
